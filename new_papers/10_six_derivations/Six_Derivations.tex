\documentclass[12pt,a4paper]{article}

% Packages
\usepackage[utf8]{inputenc}
\usepackage[T1]{fontenc}
\usepackage{amsmath,amssymb,amsthm}
\usepackage{mathtools}
\usepackage{booktabs}
\usepackage{makecell}
\usepackage{multirow}
\usepackage{graphicx}
\usepackage{float}
\usepackage{geometry}
\usepackage{setspace}
\usepackage{natbib}
\usepackage{enumitem}
\usepackage[colorlinks=true,citecolor=blue,linkcolor=blue,urlcolor=blue]{hyperref}
\usepackage[capitalise,noabbrev]{cleveref}

\geometry{margin=1in}
\onehalfspacing

% Theorem environments
\newtheorem{theorem}{Theorem}[section]
\newtheorem{proposition}[theorem]{Proposition}
\newtheorem{lemma}[theorem]{Lemma}
\newtheorem{corollary}[theorem]{Corollary}
\newtheorem{definition}[theorem]{Definition}
\newtheorem{remark}[theorem]{Remark}
\newtheorem{example}[theorem]{Example}
\newtheorem{observation}[theorem]{Observation}

% Operators
\DeclareMathOperator{\Var}{Var}
\DeclareMathOperator{\Cov}{Cov}
\DeclareMathOperator{\KL}{KL}
\DeclareMathOperator{\Tr}{Tr}
\DeclareMathOperator{\argmax}{arg\,max}
\DeclareMathOperator{\argmin}{arg\,min}

% Notation shortcuts
\newcommand{\R}{\mathbb{R}}
\newcommand{\E}{\mathbb{E}}
\newcommand{\calF}{\mathcal{F}}
\newcommand{\calL}{\mathcal{L}}
\newcommand{\bone}{\mathbf{1}}

% Tsallis-specific macros
\newcommand{\calFq}{\mathcal{F}_q}
\newcommand{\Sq}{S_q}
\newcommand{\expq}{\exp_q}
\newcommand{\logq}{\ln_q}

\title{\textbf{Six Derivations from the CES Potential}}

\author{Jon Smirl\thanks{Email: \texttt{jonsmirl@gmail.com}.}}

\date{February 2026}

\begin{document}

\maketitle

\begin{abstract}
Six areas of economic theory---information economics, mechanism design, social choice, search and matching, contract theory, and behavioral economics---developed independently, each with its own mathematical apparatus. This paper shows they are six measurements of a single parameter. Each is derived from the CES potential $\calFq = \Phi_{\mathrm{CES}}(\rho) - T \cdot \Sq$ with $q = \rho$, where $\rho$ is the CES curvature and $T$ is information friction. The derivations yield six independent $\rho$-dependent predictions absent from the original theories: Akerlof's breakdown threshold $T^*(\rho)$, Myerson's price of incentive compatibility $\mathrm{PoIC}(\rho)$, Arrow's democratic robustness, DMP search duration $n^* \propto K/T$, Hart-Moore hold-up distortion $D(\rho, \tau)$, and the Kahneman-Tversky probability weighting function $w(p) = p^\rho/(p^\rho + (1-p)^\rho)$. The core empirical implication is \emph{overidentification}: the same $\rho$ estimated from any one domain must predict outcomes in all five others. For a given industry, adverse selection severity, information rents, search duration, hold-up distortion, and behavioral anomaly magnitude are all controlled by one parameter. Any cross-domain inconsistency falsifies the framework. Two within-domain tests confirm the $\rho \times T$ interaction: a banking regulation test predicts 2007--2009 crisis severity ($p = 0.016$), and a procurement test confirms $\rho$-dependent information rents ($p = 0.003$). A cross-domain test design specifies the data, regression, and rejection criteria for the overidentification test.
\end{abstract}

\medskip
\noindent\textbf{JEL Classification:} B41, C60, D80, D82, D83, D90, J64\\
\textbf{Keywords:} CES aggregation, Tsallis entropy, CES potential, information friction, mechanism design, behavioral economics, adverse selection, search and matching

\newpage
\tableofcontents
\newpage

%=============================================================================
\section{Introduction}\label{sec:intro}
%=============================================================================

Consider an industry with CES elasticity of substitution $\sigma = 3$ (equivalently, $\rho = 2/3$). Standard economics uses this parameter to characterize production and trade. This paper shows that the same $\rho$ simultaneously determines: how severely adverse selection afflicts the industry's input markets, what information rents its procurement mechanisms must pay, how robust its governance institutions are to noise, how long its workers search before accepting jobs, how large the hold-up distortion is in its supply relationships, and how prone its participants are to behavioral anomalies. These six predictions come from six different fields of economics that developed independently. They are all controlled by the same parameter because they are all special cases of a single object: the CES potential
\begin{equation}\label{eq:free_energy}
    \boxed{\calFq = \Phi_{\mathrm{CES}}(\rho) - T \cdot \Sq, \qquad q = \rho}
\end{equation}
where $\Phi_{\mathrm{CES}}(\rho)$ is the CES aggregate with curvature parameter $\rho$, $T \geq 0$ is the information friction, $\Sq$ is the Tsallis entropy with $q = \rho$, and $K = (1-\rho)(J-1)/J$ is the CES curvature at symmetric equilibrium. Standard economics is the $T = 0$ limit of this framework; the rich structure of information economics, mechanism design, social choice, search theory, contract theory, and behavioral economics emerges at $T > 0$.

\subsection{The Overidentification Argument}

The paper's contribution is not six re-derivations of known results. It is the demonstration that $\rho$ is \emph{overidentified}: it can be estimated independently from six different domains, and the estimates must agree. For a given industry or market:
\begin{itemize}
    \item Estimate $\rho$ from \textbf{production data} (standard CES estimation, e.g., \citealt{oberfield2021}).
    \item That same $\rho$ predicts \textbf{adverse selection severity}: the breakdown threshold $T^*(\rho)$ (\Cref{sec:akerlof}).
    \item It predicts \textbf{information rents}: the price of incentive compatibility $\mathrm{PoIC}(\rho)$ (\Cref{sec:mechanism}).
    \item It predicts \textbf{democratic robustness}: the noise tolerance of the sector's governance (\Cref{sec:arrow}).
    \item It predicts \textbf{search duration}: $n^* = (K/T) \cdot Q(\rho)$ for the sector's labor market (\Cref{sec:search}).
    \item It predicts \textbf{hold-up distortion}: $D(\rho, \tau) = \tau(1 - R(\rho))/2$ in supply relationships (\Cref{sec:contracts}).
    \item It predicts \textbf{behavioral anomaly magnitude}: the probability weighting parameter and choice overload threshold (\Cref{sec:behavioral}).
\end{itemize}
No existing theory generates cross-domain predictions of this kind. Akerlof's model has no parameter that connects to search duration. Myerson's framework has no parameter that connects to behavioral anomalies. The CES potential connects all six through $\rho$.

The overidentification is falsifiable in a strong sense: estimate $\rho$ from production data, then test whether the predicted adverse selection severity, information rents, search duration, hold-up distortion, and behavioral anomaly magnitude match cross-sectional data from those other domains. Any systematic disagreement rejects the framework---not just one derivation, but the entire CES potential structure.

\subsection{Logical Status of the Derivations}

The CES aggregate in each derivation is not an assumption---it is forced by the emergence theorem of the companion paper \citep{smirl2026ces}. The logical chain is:
\begin{enumerate}
    \item Every economic setting that aggregates heterogeneous agents or goods must use an aggregator satisfying consistent sub-aggregation plus homogeneity, or optimal selection under taste heterogeneity, or inequality-averse social evaluation. Each axiom set independently forces CES (\Cref{prop:ces_forced}).
    \item Therefore CES structure with some $\rho$ is present in each of the six settings: market quality aggregation in Akerlof (via Fr\'{e}chet selection), the principal's objective in Myerson (via consistent sub-aggregation), social welfare in Arrow (via the Atkinson axioms), multi-dimensional skill--task matching in DMP (via consistent sub-aggregation), joint production in Hart (via consistent sub-aggregation), and the logit--CES duality in behavioral economics (via Fr\'{e}chet selection, \citealt{anderson1992}).
    \item The CES potential $\calFq = \Phi - T \cdot \Sq$ then generates all $\rho$-dependent consequences as $T$ moves away from zero. The second companion paper \citep{smirl2026potential} constructs the potential, derives the effective curvature theorem $K_{\mathrm{eff}} = K \cdot (1 - T/T^*)^+$, and classifies dynamical results under the Tsallis generalization.
\end{enumerate}
The derivations are therefore deductive consequences of the emergence theorem combined with positive information friction---not modeling choices.

\subsection{Summary of Results}

The six derivations, with their new $\rho$-dependent predictions:

\begin{enumerate}
    \item \textbf{Information Economics} (\Cref{sec:akerlof}): Akerlof's market-for-lemons result. \emph{New:} the breakdown threshold $T^*$ is increasing in $\rho$---complementary markets are fragile because a single lemon in a complementary bundle is catastrophic.

    \item \textbf{Mechanism Design} (\Cref{sec:mechanism}): Myerson's virtual valuation as the CES potential gradient. \emph{New:} $\mathrm{PoIC}(\rho)$ is decreasing in $\rho$; mechanism format is determined by $\rho$.

    \item \textbf{Social Choice} (\Cref{sec:arrow}): Arrow's impossibility, softened at $T > 0$. \emph{New:} majoritarian systems (high $\rho$) tolerate noise; consensus systems (low $\rho$) are fragile.

    \item \textbf{Search and Matching} (\Cref{sec:search}): DMP with CES matching. \emph{New:} $n^* = (K/T) \cdot Q(\rho)$---complementary occupations have longer search and steeper Beveridge curves.

    \item \textbf{Contract Theory} (\Cref{sec:contracts}): Hart-Moore hold-up with $\rho$ as asset specificity. \emph{New:} $D(\rho, \tau) = \tau(1-R(\rho))/2$ yields a sharp integration boundary.

    \item \textbf{Behavioral Economics} (\Cref{sec:behavioral}): The behavioral catalog at $T > 0$. \emph{New:} loss aversion $\lambda \approx T_{\mathrm{gain}}/T_{\mathrm{loss}}$; probability weighting $w(p) = p^\rho/(p^\rho + (1-p)^\rho)$.
\end{enumerate}

Two within-domain empirical tests confirm the $\rho \times T$ interaction: a banking regulation test ($p = 0.016$) and a procurement test ($p = 0.003$). The cross-domain overidentification test---estimating $\rho$ from production data and predicting outcomes in all six domains simultaneously---is specified in \Cref{sec:cross_domain}.

\subsection{Organization}

\Cref{sec:prelim} states the CES framework and the results from the companion papers needed for the derivations. Sections~\ref{sec:akerlof}--\ref{sec:behavioral} present the six derivations. \Cref{sec:six_revisited} revisits the six derivations under Tsallis. \Cref{sec:empirical} presents empirical tests. \Cref{sec:discussion} discusses what the unification demonstrates. \Cref{sec:conclusion} concludes.

\subsection{Related Literature}

\paragraph{CES across fields.}
The CES aggregate has appeared independently in production theory \citep{arrow1961}, trade \citep{armington1969,ethier1982}, industrial organization \citep{dixit1977}, and economic geography \citep{krugman1991}. \citet{anderson1992} established the formal duality between the CES demand system and the logit choice model, providing the micro-foundation connecting the two generating functions of this paper.

\paragraph{Entropy in economics.}
\citet{theil1967} first imported Shannon entropy into economics. The rational inattention program \citep{sims2003} derives logit choice from Shannon capacity constraints. \citet{matejka2015} proved the logit form is the unique solution to rational inattention problems with unrestricted prior. \citet{caplin2015} provided axiomatic foundations for posterior-separable stochastic choice.

\paragraph{Closest antecedents.}
\citet{costinot2009} showed that a single log-supermodularity condition---mathematically equivalent to CES complementarity $\rho < 1$---unifies Ricardian and Heckscher-Ohlin trade theory. \citet{kremer1993} established that complementary production creates skill sorting and development traps. The CES potential framework generalizes both: the log-supermodularity driving Costinot's sorting \emph{is} the CES curvature $K$, and the stochastic choice axiomatized by Caplin--Dean \emph{is} the $T > 0$ departure from deterministic optimization.

%=============================================================================
\section{Preliminaries: The CES Framework}\label{sec:prelim}
%=============================================================================

This section states the definitions and results from the companion papers \citep{smirl2026ces,smirl2026potential} used in the six derivations. All proofs are in those papers.

\subsection{CES Production Technology}\label{sec:ces}

Consider $J \geq 2$ inputs $x_1, \ldots, x_J > 0$. The CES aggregate is:
\begin{equation}\label{eq:ces}
    F = \left(\frac{1}{J}\sum_{j=1}^{J} x_j^\rho\right)^{1/\rho}, \qquad \rho \in (-\infty, 1], \; \rho \neq 0
\end{equation}
with $F = (\prod x_j)^{1/J}$ at $\rho = 0$ (Cobb-Douglas) and $F = \min_j x_j$ as $\rho \to -\infty$ (Leontief). The elasticity of substitution is $\sigma = 1/(1-\rho)$. The CES curvature at symmetric equilibrium is:
\begin{equation}\label{eq:curvature}
    K = \frac{(1-\rho)(J-1)}{J}
\end{equation}

\subsection{The Quadruple Role of $\rho$}\label{sec:quadruple}

\begin{theorem}[CES Quadruple Role, {\citealt{smirl2026ces}}]\label{thm:quadruple}
At symmetric equilibrium $\bar{x}_j = c$ for all $j$, the CES curvature $K$ simultaneously determines:
\begin{enumerate}
    \item[\emph{(a)}] \textbf{Superadditivity}: the gap between balanced and unbalanced allocation is $\Omega(K) \cdot \|\delta\|^2/c$.
    \item[\emph{(b)}] \textbf{Correlation robustness}: the variance reduction from CES nonlinearity is $O(K^2)$.
    \item[\emph{(c)}] \textbf{Strategic independence}: the coalition deviation penalty is $\Omega(K) \cdot \|\delta\|^2/c^2$.
    \item[\emph{(d)}] \textbf{Network scaling}: the scaling exponent is $1/\rho$.
\end{enumerate}
Superadditivity and strategic independence are first-order in $K$; correlation robustness is second-order.
\end{theorem}

\subsection{Why CES Is Forced}\label{sec:micro_foundations}

\begin{proposition}[CES from primitives, {\citealt{smirl2026ces}}]\label{prop:ces_forced}
CES aggregation is uniquely forced by each of three independent axiom sets:
\begin{enumerate}
    \item[\emph{(a)}] \textbf{Consistent aggregation (Kolmogorov--Nagumo).} Consistent sub-aggregation + homogeneity forces the CES mean \citep{aczel1966,hardy1952}.
    \item[\emph{(b)}] \textbf{Optimal selection (Fr\'{e}chet).} Buyers with i.i.d.\ Fr\'{e}chet taste shocks choosing quality-maximizing varieties produce CES market shares \citep{anderson1992,eaton2002}.
    \item[\emph{(c)}] \textbf{Inequality-averse social evaluation (Atkinson).} Weak Pareto + anonymity + homotheticity + Pigou--Dalton uniquely selects the Atkinson--CES family \citep{atkinson1970}.
\end{enumerate}
\end{proposition}

\subsection{The CES Potential}\label{sec:ces_potential_summary}

The companion paper \citep{smirl2026potential} constructs the CES potential from Lagrangian duality:
\begin{equation}\label{eq:ces_potential_def}
    \calFq = \Phi_{\mathrm{CES}}(\rho) - T \cdot \Sq, \qquad q = \rho
\end{equation}
where $\Phi = -\log F$ is the CES generating function, $T \geq 0$ is the information friction (shadow price of information capacity), and $\Sq = (1 - \sum p_j^q)/(q-1)$ is the Tsallis entropy with $q = \rho$ forced by the emergence theorem. Shannon entropy is the $q \to 1$ limit.

The \emph{effective curvature theorem} (\citealt{smirl2026potential}, Theorem~3) establishes:
\begin{equation}\label{eq:K_eff}
    K_{\mathrm{eff}}(\rho, T) = K \cdot \left(1 - \frac{T}{T^*(\rho)}\right)^{\!+}
\end{equation}
where $T^*(\rho) = 2(J-1)c^2 d^2/K$ is the critical information friction and $(z)^+ = \max(z,0)$. Above $T^*$, no complementarity benefit remains.

\paragraph{Logical status of the six derivations.}
The CES aggregate in each derivation is not an assumption---it is forced by the emergence theorem \citep{smirl2026ces}. The logical chain is:
\begin{enumerate}
    \item Every economic setting that aggregates heterogeneous agents or goods must use an aggregator satisfying one of the three axiom sets in \Cref{prop:ces_forced}. Each independently forces CES.
    \item Therefore CES structure with some $\rho$ is present in each of the six settings: market quality aggregation in Akerlof (via (b)), the principal's CES objective in Myerson (via (a)), social welfare aggregation in Arrow (via (c)), multi-dimensional skill--task matching in DMP (via (a)), CES joint production in Hart (via (a)), and the logit--CES duality in behavioral economics (via (b) through \citet{anderson1992}).
    \item The CES potential $\calFq = \Phi - T \cdot \Sq$ then generates all $\rho$-dependent consequences as $T$ moves away from zero.
\end{enumerate}
The derivations below are therefore not ``CES-flavored restatements'' of known results. They are deductive consequences of the emergence theorem combined with positive information friction. The new $\rho$-dependent predictions---absent from the original theories---are the empirical signature that the CES potential, not the individual models, is the generating object.

\subsection{Why One $\rho$ Governs All Six Domains}\label{sec:same_rho}

The emergence theorem forces CES structure in each domain, but does not by itself force the \emph{same} $\rho$ across domains. Three arguments close this gap.

\paragraph{Same inputs, same complementarity.}
If a firm produces with CES technology over $J$ differentiated inputs, then the dimensions of differentiation that make inputs complementary in production also make them complementary in matching, screening, and contracting. A semiconductor fab requires photolithography AND metrology AND cleanroom protocols; deficiency in any one is catastrophic (low $\rho$). Workers match to tasks along these same dimensions, so the DMP matching function inherits $\rho$ from the production function (\Cref{sec:search}). Quality varies along the same dimensions, so the Akerlof quality aggregate inherits $\rho$ because buyers evaluate quality on the production-relevant dimensions (\Cref{sec:akerlof}). Asset specificity IS low $\rho$: relationship-specific investments are investments in complementary capabilities, and the redeployability ratio $R(\rho)$ in Hart-Moore (\Cref{sec:contracts}) is directly controlled by production $\rho$.

\paragraph{Fr\'{e}chet duality locks choice-side $\rho$ to production-side $\rho$.}
The \citet{anderson1992} duality proves that when buyers with i.i.d.\ Fr\'{e}chet taste shocks choose quality-maximizing varieties, the resulting demand shares are CES with the same shape parameter as the Fr\'{e}chet distribution. The Fr\'{e}chet shape parameter is identified from the dispersion of quality across varieties---which is determined by the production technology. Therefore the behavioral probability weighting $w(p) = p^\rho/(p^\rho + (1-p)^\rho)$ in \Cref{sec:behavioral} uses the $\rho$ from the production-side quality distribution, and the principal's CES objective in \Cref{sec:mechanism} inherits $\rho$ because the principal aggregates the same inputs.

\paragraph{Atkinson axioms lock welfare-side $\rho$.}
The Atkinson index uses $\rho$ as the social aversion to inequality. For Arrow's social choice (\Cref{sec:arrow}), the CES social welfare function evaluates utilities with the same $\rho$ that governs the complementarity of social contributions---which, for economic alternatives, reflects the production technology.

\medskip
The identification is therefore not a coincidence: the same physical complementarity structure that makes inputs hard to substitute in production also makes them hard to match, hard to screen, hard to redeploy, and hard to evaluate. One $\rho$ governs all six domains because all six are different views of the same underlying complementarity.

%=============================================================================
% Sections 3--5 (CES Potential construction, Effective Curvature, q-Dynamics)
% are developed in the companion paper \citep{smirl2026potential}.
% The six derivations below use the results stated in Section 2.
%=============================================================================

%REMOVED_SECTIONS_3_4_5_START

%=============================================================================
\section{Derivation I: Akerlof's Market for Lemons}\label{sec:akerlof}
%=============================================================================

The market-for-lemons result \citep{akerlof1970} is derived from the CES potential framework.

\subsection{Setup}

Consider $J$ sellers with qualities $q_j$ drawn from a distribution $F$ on $[0, \bar{q}]$. Each seller knows their own quality. A representative buyer values quality at rate $v > 1$ but cannot observe quality directly. The CES aggregate of market quality is $\Phi(S) = (|S|^{-1}\sum_{j \in S} q_j^\rho)^{1/\rho}$. By \Cref{prop:ces_forced}(b), this CES form is derived from Fr\'{e}chet taste heterogeneity.

Under rational inattention, the buyer acquires a binary signal (accept/reject) at information cost $T \cdot h(q^*/\bar{q})$, where $h(\cdot)$ is binary entropy. Define $\psi(\rho) = (\rho+1)^{1/\rho}$ and $A(\rho) = v/\psi(\rho) - 1$ (the CES net value rate). With participation fraction $\pi = q^*/\bar{q}$, the CES potential landscape is:
\begin{equation}\label{eq:akerlof_free_energy}
    \calF(\pi) = A(\rho)\,\bar{q}\,\pi^2 - T \cdot h(\pi)
\end{equation}
The first term is the expected surplus from trade: with uniform quality on $[0, q^*]$, the CES aggregate of accepted qualities is $q^*/\psi(\rho)$, the seller's reservation is $q^*/2$, and the volume of trade is $\pi$. The second term is the information cost of the screening signal. Since $h$ is concave and $\pi^2$ is convex, $\calF$ is strictly convex for all $T > 0$: any interior critical point is a \emph{minimum}---a barrier between the no-trade boundary ($\pi = 0$, $\calF = 0$) and the full-trade boundary ($\pi = 1$, $\calF = A(\rho)\bar{q} > 0$). The market sustains participation iff the barrier is surmountable.

\subsection{The $\rho$-Dependent Robustness Result}

\begin{proposition}[Market Robustness]\label{prop:market_robustness}
The depth of the adverse selection barrier scales as
\begin{equation}\label{eq:tstar_closed}
    B(T, \rho) \equiv -\min_{\pi \in (0,1)} \calF(\pi) \sim T \cdot h\!\left(\pi_0(T)\right), \qquad \text{where } \pi_0(T) = O\!\left(\frac{T}{A(\rho)\bar{q}}\right)
\end{equation}
and $A(\rho) = v/\psi(\rho) - 1$ with $\psi(\rho) = (\rho+1)^{1/\rho}$. The market sustains participation when the barrier is small relative to the trade surplus $A(\rho)\bar{q}$. The characteristic scale of information friction at which the barrier becomes macroscopic is:
\begin{equation}\label{eq:Tstar_scale}
    T^* \sim A(\rho)\,\bar{q}
\end{equation}
Since $\psi(\rho)$ is strictly decreasing in $\rho$, $1/\psi(\rho)$ is strictly increasing in $\rho$, and hence $A(\rho)$ and $T^*$ are strictly increasing in $\rho$: markets with higher $\rho$ (more substitutable goods) are more robust to adverse selection. Complementary markets ($\rho < 0$) are more fragile because the CES penalty for low-quality items in a complementary bundle is severe: a single lemon destroys disproportionate value.
\end{proposition}

\begin{proof}
The proof proceeds in four steps: (1)~characterize the convex landscape; (2)~derive the barrier scaling; (3)~connect to CES curvature $K$; (4)~prove monotonicity.

\medskip
\textit{Step 1: The convex landscape.}
With $\pi = q^*/\bar{q}$, the CES potential is:
\begin{equation}\label{eq:akerlof_landscape}
    \calF(\pi) = A(\rho)\,\bar{q}\,\pi^2 - T\,h(\pi)
\end{equation}
Provided $A(\rho) > 0$ (i.e., $v > \psi(\rho)$, so gains from trade exceed the CES aggregation cost), the second derivative $d^2\calF/d\pi^2 = 2A(\rho)\bar{q} + T/(\pi(1-\pi)) > 0$ for all $\pi \in (0,1)$: the landscape is strictly convex. Any interior critical point is a minimum. The boundary values $\calF(0) = 0$ and $\calF(1) = A(\rho)\bar{q} > 0$ confirm that full participation always yields positive surplus.

For any $T > 0$, the entropy cost dominates near $\pi = 0$ (since $dh/d\pi|_{\pi \to 0^+} = +\infty$), creating a negative barrier: $\calF$ dips below zero in a neighborhood of $\pi = 0$ before rising to $\calF(1) > 0$. The barrier depth $B(T) = -\min_\pi \calF(\pi)$ determines whether the market can sustain participation.

\medskip
\textit{Step 2: Barrier scaling.}
Define $\tau = T/(A(\rho)\bar{q})$. The dimensionless CES potential $\Psi(\pi, \tau) = \pi^2 - \tau\,h(\pi)$ has its interior minimum at $\pi_0$ satisfying $2\pi_0 = \tau\log[(1-\pi_0)/\pi_0]$. For small $\tau$: $\pi_0 = O(\tau)$ and $\Psi(\pi_0) = -\tau \cdot h(\pi_0) + O(\tau^2)$, which is $O(\tau^2 \log(1/\tau))$---negligible. For large $\tau$: $\pi_0 \to 1/2$ and $\Psi(\pi_0) \approx 1/4 - \tau \log 2$, which diverges negatively. The barrier grows from negligible to macroscopic as $\tau$ passes through $O(1)$, confirming the characteristic scale $T^* \sim A(\rho)\bar{q}$.

The market functions when the barrier is small relative to the full-participation surplus $A(\rho)\bar{q}$. In the ``lemons'' regime ($T \gg T^*$), the barrier is deep: the cost of screening low-quality sellers exceeds the surplus from trade, and only the no-trade boundary ($\pi = 0$) is dynamically accessible. In the ``functioning'' regime ($T \ll T^*$), the barrier is negligible and the market operates near full participation.

\medskip
\textit{Step 3: Connection to CES curvature $K$.}
Since $T^* \sim A(\rho)\bar{q}$ and $A(\rho) = v/\psi(\rho) - 1$ with $\psi(\rho) = (\rho+1)^{1/\rho}$, the dependence on $\rho$ enters through $1/\psi(\rho)$. Expanding $\log\psi$ around $\rho = 1$ (the substitutes limit, $K = 0$):
\begin{align}
    \log\psi(1) &= \log 2, \qquad
    \frac{d}{d\rho}\log\psi\bigg|_{\rho=1} = \frac{1}{2} - \log 2 \approx -0.193 \notag
\end{align}
Therefore $\psi(\rho) = 2\exp[(\log 2 - \frac{1}{2})(1-\rho) + O((1-\rho)^2)]$ and since $1-\rho = KJ/(J-1)$:
\[
    T^* \sim \bar{q}\left[\left(\frac{v}{2}-1\right) - \frac{v}{2}\left(\log 2 - \tfrac{1}{2}\right)\frac{KJ}{J-1} + O(K^2)\right]
\]
The leading term $(v/2 - 1)$ is the gains from trade at $\rho = 1$ (the standard Akerlof case). The correction proportional to $K$ is \emph{negative}: CES curvature reduces the market's tolerance for information friction, because the CES aggregator penalizes low-quality items more severely.

\medskip
\textit{Step 4: Monotonicity of $T^*$ in $(1-\rho)$.}
It suffices to show that $\psi(\rho) = (\rho+1)^{1/\rho}$ is strictly decreasing in $\rho$ on $(-1,1)$, since $A(\rho) = v/\psi(\rho) - 1$ is strictly increasing in $1/\psi$ for $v > \psi > 0$.

The sign of $d\log\psi/d\rho$ equals the sign of $g(\rho) \equiv \rho/(\rho+1) - \log(\rho+1)$. Observe $g(0) = 0$, and:
\[
    g'(\rho) = \frac{1}{(\rho+1)^2} - \frac{1}{\rho+1} = \frac{-\rho}{(\rho+1)^2}
\]
For $\rho \in (-1, 0)$: $g'(\rho) > 0$, so $g$ is strictly increasing toward $g(0) = 0$, hence $g(\rho) < 0$. For $\rho \in (0, 1)$: $g'(\rho) < 0$, so $g$ is strictly decreasing from $g(0) = 0$, hence $g(\rho) < 0$.

Therefore $\psi(\rho)$ is strictly decreasing in $\rho$, so $1/\psi(\rho)$ is strictly increasing in $\rho$, and hence $A(\rho) = v/\psi(\rho) - 1$ is strictly increasing in $\rho$. Higher $\rho$ (more substitutable goods, lower CES curvature $K$) implies a higher barrier scale $T^*$, and hence greater robustness to adverse selection. Complementary markets are more fragile because the CES aggregator severely penalizes the low-quality items in the pool: a lemon in a complementary bundle is catastrophic.
\end{proof}

%=============================================================================
\section{Derivation II: Myerson's Optimal Mechanism}\label{sec:mechanism}
%=============================================================================

Myerson's virtual valuation \citep{myerson1981} is shown to be the CES potential gradient. The principal's CES objective $\Phi(x(\theta_1), \ldots, x(\theta_J))$ is forced by consistent sub-aggregation of heterogeneous allocations (\Cref{prop:ces_forced}(a)).

\subsection{The Virtual Valuation as CES Potential Gradient}

\begin{proposition}[Virtual Valuation Identification]\label{prop:virtual}
The virtual valuation $\varphi(\theta) = \theta - (1-F(\theta))/f(\theta)$ is the CES potential gradient: the marginal CES contribution $\theta$ minus the marginal entropy cost $(1-F(\theta))/f(\theta)$ (the inverse Mills ratio, measuring residual uncertainty per type).
\end{proposition}

\begin{proof}
The proof proceeds in four steps: formulating the mechanism design problem, deriving the information rent via the IC constraint, identifying the rent as an entropy gradient, and assembling the virtual valuation as the CES potential gradient.

\medskip
\emph{Step 1: The mechanism design problem.}
By the revelation principle, restrict attention to direct mechanisms $(x(\cdot), t(\cdot))$ where each agent reports a type and receives allocation $x_j = x(\hat{\theta}_j)$ and transfer $t_j = t(\hat{\theta}_j)$. Agent $j$ with true type $\theta_j$ values receiving allocation $x$ at $\theta_j \cdot x$ (quasi-linear utility). Agent $j$'s payoff from reporting $\hat{\theta}$ when the true type is $\theta$ is:
\begin{equation}\label{eq:agent_payoff}
    U(\hat{\theta}, \theta) = \theta \cdot x(\hat{\theta}) - t(\hat{\theta})
\end{equation}
Define the equilibrium rent function $U(\theta) \equiv U(\theta, \theta) = \theta \cdot x(\theta) - t(\theta)$.

The principal maximizes expected CES welfare net of transfers:
\begin{equation}\label{eq:principal_obj}
    \max_{x(\cdot), t(\cdot)} \; \E_\theta\!\left[\Phi\bigl(x(\theta_1), \ldots, x(\theta_J)\bigr) + \sum_{j=1}^J t(\theta_j)\right]
\end{equation}
subject to incentive compatibility (IC: $U(\theta) \geq U(\hat{\theta}, \theta)$ for all $\theta, \hat{\theta}$) and individual rationality (IR: $U(\theta) \geq 0$ for all $\theta$).

\medskip
\emph{Step 2: IC implies the information rent is the integral of the allocation rule.}
By standard arguments \citep{myerson1981}, IC requires that $x(\theta)$ is non-decreasing in $\theta$ and the envelope condition holds. From \eqref{eq:agent_payoff}, truth-telling requires $\theta \in \argmax_{\hat{\theta}} \theta \cdot x(\hat{\theta}) - t(\hat{\theta})$. Taking the first-order condition and evaluating at $\hat{\theta} = \theta$: $dU/d\theta = x(\theta)$. Integrating from the lowest type $\underline{\theta}$ and using IR binding at $\underline{\theta}$:
\begin{equation}\label{eq:rent}
    U(\theta) = \int_{\underline{\theta}}^{\theta} x(s) \, ds
\end{equation}
Substituting $t(\theta) = \theta \cdot x(\theta) - U(\theta)$ into \eqref{eq:principal_obj} and applying Fubini's theorem to reverse the order of integration:
\begin{equation}\label{eq:virtual_transfer}
    \E[t(\theta)] = \int_{\underline{\theta}}^{\bar{\theta}} \left[\theta - \frac{1 - F(\theta)}{f(\theta)}\right] x(\theta) f(\theta) \, d\theta = \int_{\underline{\theta}}^{\bar{\theta}} \varphi(\theta) \cdot x(\theta) \cdot f(\theta) \, d\theta
\end{equation}

\medskip
\emph{Step 3: The inverse Mills ratio as entropy gradient.}
Define the cumulative residual entropy $\mathcal{H}(\theta) \equiv -\int_{\underline{\theta}}^{\bar{\theta}} [1 - F(s)] \log[1 - F(s)] \, ds$---the continuous analogue of Shannon entropy applied to the survival function $\bar{F}(\theta) = 1 - F(\theta)$. The information rent from \eqref{eq:rent} can be written as $\E[U(\theta)] = \int x(\theta) [1 - F(\theta)] \, d\theta$. The integrand $[1-F(\theta)]$ is the hazard weight: the mass of types above $\theta$ who benefit from the information rent generated at $\theta$. Dividing by $f(\theta)$ gives the per-unit-density rent $(1-F(\theta))/f(\theta)$---the inverse of the hazard rate $h(\theta) = f(\theta)/(1-F(\theta))$. This is the entropy contribution per agent at type $\theta$: low-hazard types contribute more residual uncertainty and therefore command higher information rents.

\medskip
\emph{Step 4: Assembly as the CES potential gradient.}
From Step 2, the principal's pointwise objective is:
\[
    \calF(\theta) = \underbrace{\theta \cdot x(\theta)}_{\text{CES marginal contribution }\partial\Phi/\partial\theta} - \underbrace{\frac{1-F(\theta)}{f(\theta)} \cdot x(\theta)}_{\text{entropy cost }T \cdot \partial\mathcal{H}/\partial\theta}
\]
The CES potential gradient with respect to $\theta$ is $\partial \calF / \partial \theta = \theta - (1 - F(\theta))/f(\theta) = \varphi(\theta)$, which is Myerson's virtual valuation.

\medskip
\emph{Explicit computation for the uniform distribution.}
Let $F(\theta) = (\theta - \underline{\theta})/(\bar{\theta} - \underline{\theta})$ on $[\underline{\theta}, \bar{\theta}]$. Then $(1-F(\theta))/f(\theta) = \bar{\theta} - \theta$, giving $\varphi(\theta) = 2\theta - \bar{\theta}$. This is non-negative iff $\theta \geq \bar{\theta}/2$---so the optimal mechanism excludes the bottom half of the type distribution, which is Myerson's classical result.

\end{proof}

\subsection{The Optimal Mechanism Format}

The CES curvature determines not only the magnitude of information rents but the optimal \emph{format} of the mechanism.

\begin{observation}[Mechanism Format Determination]\label{obs:format}
The CES curvature parameter suggests a natural taxonomy of optimal mechanism formats as $\rho$ decreases:
\begin{enumerate}[label=(\roman*)]
\item $\rho$ near $1$ (perfect substitutes): \textbf{Simple auction.} The principal posts a price and agents self-select. The auction format is natural because agents are interchangeable---the principal needs only to identify which agent has the lowest cost, not to learn detailed type information.

\item $\rho$ moderate (differentiated inputs): \textbf{Scored evaluation.} The principal evaluates multiple dimensions with explicit weights. The scoring reflects the CES production structure: each dimension receives weight proportional to its marginal contribution to the CES aggregate.

\item $\rho$ near $-\infty$ (perfect complements): \textbf{Sequential revelation with holdback.} The principal reveals partial allocation information to each agent sequentially, exploiting the Leontief structure where each agent's optimal report depends on the reports of others. The mechanism must prevent each agent from free-riding on others' revelations.
\end{enumerate}
\end{observation}

The mechanism format prediction is testable in procurement data: defense contracts (low $\rho$) should use scored evaluations and negotiated awards more frequently than commodity contracts (high $\rho$), which should use sealed-bid auctions. The USAspending.gov data in \Cref{sec:poic} confirm this: 67\% of LOW-substitutability contracts use negotiated awards vs.\ 23\% of HIGH-substitutability contracts.

\subsection{The Revelation Principle as Variational Equilibrium}

\begin{remark}[Revelation Principle as CES Potential Minimization]\label{rem:revelation}
The revelation principle admits a natural interpretation in the CES potential framework. Under the rational inattention formulation of \citet{caplin2015}, an agent's cost of misreporting type $\theta$ as $\theta'$ is proportional to the mutual information $I(\theta; \theta')$. In a direct mechanism with transfers $t(\cdot)$, each agent's effective CES potential for reporting $\theta'$ when the true type is $\theta$ is:
\[
    \calF_j(\theta'; \theta) = -u_j(\theta, x(\theta')) + t(\theta') + T \cdot I(\theta; \theta')
\]
The VCG mechanism sets transfers $t(\cdot)$ so that truthful reporting ($\theta' = \theta$, $I = 0$) minimizes this expression. The revelation principle---that any implementable outcome can be achieved through truthful direct revelation---is thus the statement that equilibrium occurs at the point of maximum order (zero entropy of misrepresentation) in the CES potential landscape.
\end{remark}

\subsection{$\rho$-Dependent Price of Incentive Compatibility}

\begin{proposition}[Price of Incentive Compatibility]\label{prop:poic}
The welfare ratio $\mathrm{PoIC}(\rho) = (W^{\mathrm{FB}} - W^{\mathrm{SB}})/W^{\mathrm{FB}}$ is decreasing in $\rho$:
\begin{enumerate}
    \item[\emph{(a)}] As $\rho \to 1$ (perfect substitutes): $\mathrm{PoIC} \to 0$. Agents are replaceable; information rents vanish.
    \item[\emph{(b)}] As $\rho \to -\infty$ (perfect complements): $\mathrm{PoIC} \to 1$. Each agent is essential; information rents consume the entire surplus.
\end{enumerate}
\end{proposition}

\begin{proof}
The proof proceeds in four steps: solving the first-best, solving the second-best, computing the welfare ratio, and proving monotonicity with the two limits. For closed forms, work with the uniform distribution $F(\theta) = (\theta - \underline{\theta})/(\bar{\theta} - \underline{\theta})$ on $[\underline{\theta}, \bar{\theta}]$ with $\underline{\theta} > 0$.

\medskip
\emph{Step 1: First-best allocation ($T = 0$, no IC constraint).}
With observable types, the principal solves $\max_{x(\theta) \geq 0} \E_\theta[\theta \cdot x(\theta)^\rho - c \cdot x(\theta)]$. Normalizing $c = \rho$, the first-order condition gives:
\begin{equation}\label{eq:fb_alloc}
    x^{\mathrm{FB}}(\theta) = \theta^{1/(1-\rho)} = \theta^{\sigma}
\end{equation}
where $\sigma = 1/(1-\rho)$. The first-best per-type surplus is $(1-\rho)\theta^\sigma$, giving:
\begin{equation}\label{eq:WFB}
    W^{\mathrm{FB}} = (1-\rho) \cdot \E[\theta^{\sigma}]
\end{equation}

\medskip
\emph{Step 2: Second-best allocation (with IC constraint).}
From the virtual valuation identification, the principal maximizes virtual surplus. With the uniform distribution, $\varphi(\theta) = 2\theta - \bar{\theta}$. The second-best allocation for types with $\varphi(\theta) > 0$ (i.e., $\theta > \bar{\theta}/2 \equiv \theta^*$) is:
\begin{equation}\label{eq:sb_alloc}
    x^{\mathrm{SB}}(\theta) = \varphi(\theta)^{\sigma} = (2\theta - \bar{\theta})^{\sigma}, \qquad \theta \geq \theta^*
\end{equation}
and $x^{\mathrm{SB}}(\theta) = 0$ for $\theta < \theta^*$ (exclusion of types with negative virtual valuation).

\medskip
\emph{Step 3: The welfare ratio.}
The welfare loss decomposes into two sources: (i) \emph{exclusion loss}---types $\theta \in [\underline{\theta}, \theta^*)$ are excluded, destroying surplus $(1-\rho)\theta^\sigma$; (ii) \emph{distortion loss}---participating types receive $x^{\mathrm{SB}}(\theta) = \varphi(\theta)^\sigma < \theta^\sigma = x^{\mathrm{FB}}(\theta)$. Both losses depend on $\sigma = 1/(1-\rho)$, and their magnitude relative to $W^{\mathrm{FB}}$ is controlled by how much the mapping $\theta \mapsto \varphi(\theta)$ distorts the allocation rule.

\medskip
\emph{Step 4: Monotonicity in $\rho$ and the two limits.}

\emph{(a) Limit $\rho \to 1$ ($\sigma \to \infty$): $\mathrm{PoIC} \to 0$.}
As $\sigma \to \infty$, both $x^{\mathrm{FB}}(\theta) = \theta^\sigma$ and $x^{\mathrm{SB}}(\theta) = \varphi(\theta)^\sigma$ concentrate on the highest type $\bar{\theta}$, where $\varphi(\bar{\theta}) = \bar{\theta}$ (no distortion). The welfare ratio $W^{\mathrm{SB}}/W^{\mathrm{FB}} \to 1$. Economically: when agents are perfect substitutes, the principal gives everything to the best agent regardless, so screening is unnecessary and information rents vanish.

\emph{(b) Limit $\rho \to -\infty$ ($\sigma \to 0^+$): $\mathrm{PoIC} \to 1$.}
As $\sigma \to 0^+$, the Leontief aggregate $\Phi = \min_j x_j$ means every agent is essential. The mechanism must either exclude low types (losing essential inputs) or pay enormous rents to include them. The bottleneck agent is always near the exclusion threshold, so $W^{\mathrm{SB}}/W^{\mathrm{FB}} \to 0$ and $\mathrm{PoIC} \to 1$.

\emph{(c) Monotonicity.}
The distortion at each type is $r(\theta;\sigma) = (\varphi(\theta)/\theta)^\sigma$. Since $\varphi(\theta) < \theta$ for interior types:
\[
    \frac{\partial r}{\partial\sigma} = r \cdot \log\frac{\varphi(\theta)}{\theta} < 0
\]
Higher $\sigma$ (higher $\rho$) reduces distortion at every interior type. Additionally, the CES welfare weight of excluded types $\int_{\underline{\theta}}^{\theta^*} \theta^\sigma f(\theta)\,d\theta / \int_{\underline{\theta}}^{\bar{\theta}} \theta^\sigma f(\theta)\,d\theta$ is decreasing in $\sigma$---as $\sigma$ increases, the aggregate is increasingly dominated by high types where there is no distortion. Both effects yield $d(W^{\mathrm{SB}}/W^{\mathrm{FB}})/d\rho > 0$, hence $d\,\mathrm{PoIC}/d\rho < 0$.

The qualitative structure holds for all regular type distributions: (i) the distortion ratio is decreasing in $\sigma$ at every interior type; (ii) the welfare weight of excluded types decreases in $\sigma$; (iii) at $\sigma \to \infty$ the allocation concentrates on $\bar{\theta}$ where no distortion occurs; (iv) at $\sigma \to 0^+$ the Leontief bottleneck makes exclusion catastrophic.
\end{proof}

%=============================================================================
\section{Derivation III: Arrow's Impossibility Theorem}\label{sec:arrow}
%=============================================================================

Arrow's impossibility \citep{arrow1951} is an ordinal result: no aggregation of ordinal preferences can satisfy Pareto, IIA, and non-dictatorship simultaneously. The CES social welfare function $W(A) = (J^{-1}\sum_j u_j(A)^\rho)^{1/\rho}$ is forced as the unique inequality-averse aggregator satisfying anonymity, weak Pareto, and homotheticity (\Cref{prop:ces_forced}(c)). The CES potential framework characterizes what becomes possible when this cardinal aggregator operates under finite processing capacity ($T > 0$).

\begin{proposition}[Democratic Feasibility at Positive Information Friction]\label{prop:arrow}
For $\rho \in (0, 1]$ and any $T > 0$, there exists a non-dictatorial social welfare function $W_T$ satisfying:
\begin{enumerate}
    \item \textbf{Pareto efficiency} (exactly): unanimous preference is respected;
    \item \textbf{Menu independence} (exactly): the social ranking of $A$ vs.\ $B$ is independent of utilities for any irrelevant alternative $C$;
    \item \textbf{Ordinal IIA} (approximately): sensitivity to cardinal re-parameterization is bounded by $(1-\rho) \cdot \eta(T)$;
    \item \textbf{Concentration}: the social ordering agrees with the expected CES-optimal ordering with probability at least $1 - \delta(T, \varepsilon, J)$.
\end{enumerate}
There exists an optimal $T^*$ minimizing democratic error.
\end{proposition}

\begin{proof}
The proof constructs an explicit non-dictatorial social welfare function at $T > 0$ and verifies each property.

\medskip
\emph{Step 1: Construction of $W_T$.}
Fix $\rho \in (0, 1]$, $T > 0$, and $J$ agents with utilities $u_j(A) \in [\underline{u}, \bar{u}]$ for alternatives $A \in \mathcal{A}$, where $0 < \underline{u} < \bar{u}$. Each agent submits a noisy report: for each alternative $A$ independently draw $\xi_j^A \sim \mathrm{Logistic}(0, T)$ and set $\tilde{u}_j(A) = u_j(A) + \xi_j^A$. Since $\Pr(\xi_j^A < -\underline{u}) \leq e^{-\underline{u}/T}$, the event $\mathcal{E} = \{\tilde{u}_j(A) > 0 \text{ for all } j, A\}$ satisfies $\Pr(\mathcal{E}) \geq 1 - J|\mathcal{A}| \cdot e^{-\underline{u}/T}$.

On $\mathcal{E}$, define the noisy CES social welfare:
\begin{equation}\label{eq:noisy_ces}
    W_T(A) = \left(\frac{1}{J}\sum_{j=1}^{J} \tilde{u}_j(A)^\rho\right)^{1/\rho}
\end{equation}
The social ranking orders alternatives by $S(A) \equiv \E[W_T(A) \cdot \mathbf{1}_{\mathcal{E}}]$. Since all agents enter symmetrically (equal weight $1/J$), no agent is a dictator.

\medskip
\emph{Step 2: Pareto efficiency.}
Suppose $u_j(A) > u_j(B)$ for all $j$. On $\mathcal{E}$, the CES aggregate is strictly increasing in each $\tilde{u}_j(A)$:
\[
    \frac{\partial W_T(A)}{\partial \tilde{u}_k(A)} = \frac{1}{J}\left(\frac{\tilde{u}_k(A)}{W_T(A)}\right)^{\rho - 1} > 0
\]
Since $u_j(A) > u_j(B)$ for all $j$ and $\tilde{u}_j(A) = u_j(A) + \xi_j^A$ with the same noise distribution, stochastic dominance gives $S(A) > S(B)$: Pareto efficiency holds exactly.

\medskip
\emph{Step 3: Menu independence and approximate ordinal IIA.}
\emph{Menu independence (exact).} Since noise is drawn independently per alternative, $S(A) = \E[W_T(A) \cdot \mathbf{1}_{\mathcal{E}}]$ depends only on the utility profile $(u_1(A), \ldots, u_J(A))$, not on utilities for any other alternative $C$. The social ranking of $A$ vs.\ $B$ is completely independent of irrelevant alternatives in the menu sense.

\emph{Ordinal IIA (approximate).} The CES aggregate with $\rho \neq 1$ depends on cardinal utility levels, not just ordinal rankings. Let $u'_j(X) = u_j(X) + c_j$ for agent-specific constants $c_j$ (a re-cardinalization preserving all ordinal rankings). By the mean value theorem applied to $S(X; u + tc)$ on $\mathcal{E}$:
\[
    \left|\frac{\partial}{\partial t}\bigl[S(A; u + tc) - S(B; u + tc)\bigr]\right| = \frac{1}{J}\left|\sum_{j=1}^{J} c_j \cdot \E\!\left[\left(\frac{\tilde{u}_j(A)}{W_T(A)}\right)^{\!\rho-1} - \left(\frac{\tilde{u}_j(B)}{W_T(B)}\right)^{\!\rho-1}\right]\right|
\]
For $\rho = 1$ (utilitarian), each term equals $1 - 1 = 0$: ordinal IIA holds exactly. For $\rho < 1$, as $T \to \infty$ the noise dominates so $\tilde{u}_j/W_T \to 1$ in probability, and each term converges to zero. An explicit bound:
\begin{equation}\label{eq:iia_bound}
    \eta(\rho, T) \leq (1-\rho) \cdot \frac{(\bar{u} - \underline{u})^2}{3T^2} \quad \text{for } T \gg \bar{u}
\end{equation}
The IIA violation is controlled by $(1-\rho)$: it vanishes at $\rho = 1$ and grows as $\rho \to 0$.

\medskip
\emph{Step 4: Concentration of the social ordering.}
By McDiarmid's bounded differences inequality (each agent's noisy utility affects $W_T$ by at most $c/J$ for $c = O(\bar{u})$):
\begin{equation}\label{eq:mcdiarmid}
    \Pr\bigl[|W_T(A) - \E[W_T(A)]| > \varepsilon\bigr] \leq 2\exp\!\left(-\frac{2J\varepsilon^2}{c^2}\right)
\end{equation}
Taking a union bound over all pairwise comparisons, the probability that the noisy social ordering differs from the expected CES ordering is at most $\delta(T, \varepsilon) = |\mathcal{A}|^2 \exp(-2J\varepsilon^2/c^2)$.

\medskip
\emph{Step 5: Existence of optimal $T^*$.}
Define the democratic error as $E(T) = \E[\|W_T - W_0\|^2] + \lambda \cdot V(T)$, where $V(T)$ measures ordinal IIA violation. The aggregation error is zero at $T = 0$ and grows as $O(T^2)$. The ordinal IIA cost $V(T)$ is positive at $T = 0$ (the deterministic CES violation for $\rho \neq 1$) and decreases toward zero for $T \gg \bar{u}$. Since $E(T) \to \lambda V(0) > 0$ as $T \to 0$ and $E(T) \to \infty$ as $T \to \infty$, by the extreme value theorem there exists $T^* \in (0, \infty)$ minimizing $E(T)$.
\end{proof}

\subsection{$\rho$-Dependent Democratic Robustness}

\begin{proposition}[Robustness of Political Institutions]\label{prop:democracy}
The noise-robustness threshold $T^*_{\mathrm{democracy}}(\rho)$ depends on the aggregation rule in a non-trivial way. Low-$\rho$ (consensus) systems tolerate more IID noise because influence concentrates on fewer agents, but are more vulnerable to strategic manipulation by those agents.
\end{proposition}

\begin{proof}
\emph{Step 1: CES sensitivity to individual perturbations.}
The CES partial derivative with respect to a single agent's utility is:
\[
    \frac{\partial W}{\partial u_k} = \frac{1}{J}\left(\frac{u_k}{W}\right)^{\rho - 1}
\]
For $\rho = 1$: equal influence $1/J$ regardless of utility level. For $\rho < 1$: the weight amplifies agents with low utilities ($\rho - 1 < 0$). As $\rho \to -\infty$: $\partial W/\partial u_k \to \mathbf{1}_{k = \argmin_j u_j}$---the minimum-utility agent has all influence.

\medskip
\emph{Step 2: Noise amplification at low $\rho$.}
At information friction $T$, with $\tilde{u}_j = u_j + \xi_j$ and $\xi_j \sim \mathrm{Logistic}(0, T)$, the variance of the noisy CES aggregate is:
\[
    \Var(W_T) \approx \frac{T^2\pi^2}{3} \cdot \frac{1}{J^2}\sum_{k=1}^{J}\left(\frac{u_k}{W}\right)^{2(\rho-1)} = \frac{T^2\pi^2}{3} \cdot \frac{\mathcal{C}(\rho)}{J}
\]
where $\mathcal{C}(\rho) = J^{-1}\sum_k(u_k/W)^{2(\rho-1)}$ is the influence concentration. For $\rho = 1$: $\mathcal{C}(1) = 1$ (all agents equally influential). For $\rho \to -\infty$ with heterogeneous utilities: $W \to \min_k u_k$, so only the worst-off agent contributes; $\mathcal{C} \to 1/J$ and the variance of $W_T$ scales as $T^2/J^2$. The effective sample size is $J_{\mathrm{eff}}(\rho) = J/\mathcal{C}(\rho)$.

\medskip
\emph{Step 3: Critical information friction.}
Democratic aggregation fails when noise variance exceeds the squared gap between the top two alternatives. Setting the reversal probability equal to a failure threshold $\alpha$:
\begin{equation}\label{eq:T_crit_democracy}
    T^*_{\mathrm{democracy}}(\rho) = \frac{\Delta \cdot \sqrt{3}}{\pi \cdot z_\alpha} \cdot \sqrt{J_{\mathrm{eff}}(\rho)}
\end{equation}
where $z_\alpha = \Phi^{-1}(1-\alpha)$ and $\Delta = |W(A) - W(B)|$.

\medskip
\emph{Step 4: Verification at the limits.}
At $\rho = 1$ (utilitarian): $\mathcal{C} = 1$, $J_{\mathrm{eff}} = J$, threshold scales as $\sqrt{J}$---standard $\sqrt{n}$ convergence. At $\rho \to -\infty$ (Rawlsian): $\mathcal{C} \to 1/J$, $J_{\mathrm{eff}} \to J^2$, threshold scales as $J$---\emph{faster} convergence because only one agent's noise matters. However, this improved noise tolerance comes at a cost: the single influential agent (the worst-off) can strategically manipulate the outcome, and the system is paralyzed if that agent's information is unavailable. The noise-robustness and strategic-vulnerability tradeoffs move in opposite directions as $\rho$ decreases.
\end{proof}

%=============================================================================
\section{Derivation IV: Search and Matching}\label{sec:search}
%=============================================================================

The Diamond-Mortensen-Pissarides search framework \citep{diamond1982,mortensen1982,pissarides1985} is derived from the CES potential principle. Match quality aggregates multiple skill--task dimensions; consistent sub-aggregation forces the CES form (\Cref{prop:ces_forced}(a)). Cobb-Douglas matching ($\rho = 0$) is one special case. The curvature parameter $\rho$ of the skill-task fit determines search intensity, match quality, and labor market dynamics.

\subsection{Setup}

Match quality between worker $i$ and firm $j$ is the CES aggregate of skill-task fit across $L$ dimensions:
\begin{equation}\label{eq:match_quality}
    m(i,j) = \left(\frac{1}{L}\sum_{\ell=1}^{L} (s_{i\ell} \cdot t_{j\ell})^\rho\right)^{1/\rho}
\end{equation}
where $s_{i\ell}$ is worker $i$'s proficiency in skill dimension $\ell$, $t_{j\ell}$ is firm $j$'s requirement in dimension $\ell$, and $\rho$ controls complementarity between worker skills and firm requirements. The interpretation:
\begin{itemize}
\item Low $\rho$ (complements): every skill dimension matters. A surgeon needs anatomy knowledge AND manual dexterity AND clinical judgment---deficiency in any one dimension is catastrophic. The match quality is bottlenecked by the weakest dimension.
\item High $\rho$ (substitutes): skill dimensions are fungible. A cashier needs speed OR accuracy OR friendliness---excellence in one compensates for mediocrity in another.
\end{itemize}

The distribution of match qualities across potential pairings depends on the joint distribution of skills and tasks. Under the assumption that $s_{i\ell}$ and $t_{j\ell}$ are drawn independently from log-normal distributions, the match quality $m(i,j)$ has a $q$-exponential distribution with $q = \rho$ (\citealt{smirl2026potential}, Proposition~2), connecting the CES skill structure to the equilibrium matching distribution.

The worker's search problem is a CES potential optimization: at each meeting, the worker evaluates match quality $m(i,j)$ at information cost $T \cdot \Delta H$ and decides whether to accept. The accept/reject threshold is the \emph{reservation quality} $q^*$, analogous to the reservation wage in standard search theory but generalized to the multi-dimensional CES skill space.

\subsection{The $\rho$-Dependent Search Duration}

\begin{proposition}[Search Duration]\label{prop:search_duration}
The expected number of meetings before acceptance is:
\begin{equation}\label{eq:search_duration}
    n^*(\rho, T) = \frac{K}{T} \cdot Q(\rho)
\end{equation}
where $K = (1-\rho)(L-1)/L$ and $Q(\rho)$ is increasing in $(1-\rho)$. Search duration is increasing in $K$ and decreasing in $T$.
\end{proposition}

\begin{proof}
The proof proceeds in three steps.

\medskip
\emph{Step 1: Reservation quality.}
From the CES potential first-order condition, the reservation quality $q^*$ equates the marginal benefit of continued search to the marginal information cost. For match qualities drawn from the CES distribution \eqref{eq:match_quality}, the probability of exceeding $q^*$ in a single meeting is $P(m > q^*) = 1 - G(q^*)$, where $G$ is the CDF of match quality. The FOC becomes:
\[
    [1 - G(q^*)] \cdot \E[m - q^* \mid m > q^*] = T \cdot \Delta H
\]

\medskip
\emph{Step 2: CES curvature raises the reservation quality.}
By \Cref{thm:quadruple}(a), the CES aggregate of skill--task fit satisfies, for any deviation $\delta$ from the balanced profile:
\[
    m(\bar{s} + \delta) \leq m(\bar{s}) - \frac{K}{2(L-1)} \cdot \frac{\|\delta\|^2}{\bar{s}}
\]
A well-matched pairing ($\|\delta\|$ small) is disproportionately more productive than a poorly-matched one. The option value of continued search increases in $K$ because the right tail of the match quality distribution contributes more surplus when complementarity is high. Substituting into the FOC: the marginal benefit of one more meeting grows with $K$, while the marginal cost $T \cdot \Delta H$ is $K$-independent. The reservation quality satisfies:
\[
    q^*(\rho, T) = q^*_0(T) + K \cdot Q_1(\rho, T) + O(K^2)
\]
where $q^*_0(T)$ is the reservation quality at $K = 0$ (standard DMP) and $Q_1 > 0$.

\medskip
\emph{Step 3: Duration from acceptance probability.}
The expected search duration is $n^* = 1/P(m > q^*)$. Taylor-expanding $1 - G(q^*)$ around $q^*_0$ and inverting:
\[
    n^* = n^*_0 + \frac{K Q_1 g(q^*_0)}{[1-G(q^*_0)]^2} + O(K^2)
\]
where $n^*_0 = 1/[1-G(q^*_0)]$ is the standard DMP duration. Under log-concave match quality distributions, the standard reservation wage equation gives $q^*_0 \propto 1/T$, so $n^*_0 \sim T/c(G)$. Defining $Q(\rho) \equiv Q_1 \cdot g(q^*_0)/[1-G(q^*_0)]^2$ absorbs the distributional constants, yielding the leading behavior $n^* \propto KQ(\rho)/T$.
\end{proof}


\subsection{The Beveridge Curve}

\begin{proposition}[Beveridge Curve Slope]\label{prop:beveridge}
The equilibrium Beveridge curve slope receives a CES correction from the acceptance channel:
\begin{equation}\label{eq:beveridge_slope}
    \left.\frac{dV}{dU}\right|_{\mathrm{BC}} = -1 - \frac{2s}{f(1)\cdot p_a}
\end{equation}
where $p_a = T/(K \cdot Q)$. The slope steepens as $K/T$ increases, and the Beveridge curve shifts outward as $\rho$ decreases, amplifying unemployment-vacancy comovement and addressing the \citet{shimer2005} volatility puzzle.
\end{proposition}

\begin{proof}
\emph{Step 1: CES matching rate.}
The matching function is generalized from Cobb-Douglas to CES: $M(\theta) = A(\alpha\theta^\rho + 1 - \alpha)^{1/\rho}$, where $\theta = V/U$ is market tightness.

\emph{Step 2: Steady-state balance.}
Unemployment evolves as $\dot{u} = s(1-u) - f(\theta) \cdot u$. At steady state, $u = s/(s + f(\theta))$ and $v = \theta \cdot u$.

\emph{Step 3: Acceptance probability and effective matching.}
From \Cref{prop:search_duration}, the effective job-finding rate is $f_{\mathrm{eff}}(\theta) = f(\theta) \cdot p_a(\rho, T)$, where $p_a = T/(K \cdot Q(\rho))$. For $K \to 0$: $p_a \to 1$ (every meeting accepted). For $K \gg T$: $p_a \to 0$ (almost all rejected).

\emph{Step 4: Slope computation.}
Replacing $f(\theta)$ with $f_{\mathrm{eff}}$ in the steady-state condition and differentiating $u(\theta)$ and $v(\theta) = \theta u(\theta)$ implicitly through $\theta$, the Beveridge slope at $\theta = 1$ with $\alpha = 1/2$ is:
\[
    \frac{dV}{dU}\bigg|_{\theta=1} = -1 - \frac{2s}{f(1) \cdot p_a}
\]
The CES correction enters entirely through $p_a$: as $K$ increases, $p_a$ falls, the effective matching rate drops, and the slope steepens. The outward shift with decreasing $\rho$ follows: for fixed $(s, T)$, increasing $K$ reduces $p_a$ at every $\theta$, raising steady-state $u$ at every $v$.
\end{proof}

\section{Derivation V: Contract Theory}\label{sec:contracts}
%=============================================================================

The hold-up problem \citep{grossman1986,hart1990} and vertical integration boundary \citep{williamson1979} are derived from the CES potential. Joint production aggregates the two parties' inputs; consistent sub-aggregation forces the CES form (\Cref{prop:ces_forced}(a)).

\subsection{Setup}

Two parties jointly produce via CES technology $y = (\alpha x_A^\rho + (1-\alpha)x_B^\rho)^{1/\rho}$. The parameter $\rho$ is asset specificity: low $\rho$ means inputs are relationship-specific. A fraction $\tau \in [0,1]$ of output dimensions are non-contractible. The marginal redeployability ratio $R(\rho) \in [0,1]$ measures outside-option quality, with $R(1) = 1$ and $R(\rho) \to 0$ as $\rho \to -\infty$.

\subsection{The Hold-Up Problem}

\begin{proposition}[Hold-Up Distortion]\label{prop:holdup}
Under Nash bargaining over the non-contractible fraction $\tau$, the equilibrium investment distortion is:
\begin{equation}\label{eq:distortion}
    D(\rho, \tau) = \frac{\tau(1 - R(\rho))}{2}
\end{equation}
The distortion is increasing in $\tau$ and decreasing in $\rho$. At $\rho = 1$: $D = 0$ (no hold-up). As $\rho \to -\infty$: $D \to \tau/2$ (maximum hold-up).
\end{proposition}

\begin{proof}
The proof proceeds in three steps: solving for equilibrium investment, computing the distortion, and establishing comparative statics.

\medskip
\emph{Step 1: Equilibrium investment under incomplete contracts.}
Under Nash bargaining over the non-contractible fraction $\tau$, party $A$ receives payoff:
\[
    \pi_A = (1-\tau)\cdot \frac{\partial y}{\partial x_A}\cdot x_A + \tau\!\left[\bar{y}_A + \tfrac{1}{2}(y - \bar{y}_A - \bar{y}_B)\right] - \frac{x_A^2}{2}
\]
The first term provides efficient incentives on the contractible portion. The second is the Nash bargaining outcome: $A$'s outside option plus half the gains from trade. The FOC gives:
\[
    \left[1 - \frac{\tau(1 - R)}{2}\right]\frac{\partial y}{\partial x_A} = x_A
\]
using the redeployability ratio $R(\rho) = \bar{y}'_A / (\partial y/\partial x_A)$. At symmetric equilibrium: $x^* = x^{\mathrm{FB}}[1 - \tau(1-R)/2]$, so $D = \tau(1-R(\rho))/2$.

\medskip
\emph{Step 2: Comparative statics in $\tau$.}
$\partial D/\partial \tau = (1-R(\rho))/2 > 0$ since $R < 1$ for $\rho < 1$. More contractual incompleteness increases the distortion.

\medskip
\emph{Step 3: Comparative statics in $\rho$.}
$\partial D/\partial \rho = -\tau R'(\rho)/2 < 0$ since $R'(\rho) > 0$ (more substitutable investments have better outside options). More complementary inputs suffer worse hold-up because their outside options deteriorate faster than their inside value.
\end{proof}

\section{Derivation VI: Behavioral Economics}\label{sec:behavioral}
%=============================================================================

The behavioral catalog \citep{kahneman1979,tversky1992,thaler2008} emerges as the complete characterization of economic behavior at $T > 0$. The micro-foundation is the logit--CES duality: buyers with i.i.d.\ Fr\'{e}chet taste shocks choosing optimally produce CES demand shares (\Cref{prop:ces_forced}(b); \citealt{anderson1992}). The same structure arises from rational inattention \citep{sims2003,matejka2015}, establishing CES as the forced form on both the production and choice sides.

\subsection{The Logit-CES Duality}

Under rational inattention \citep{sims2003,matejka2015}, optimal choice probabilities are multinomial logit:
\begin{equation}\label{eq:logit_choice}
    P(j) = \frac{\exp(u_j / T)}{\sum_{k=1}^{J} \exp(u_k / T)}
\end{equation}
This is \emph{identical} to CES demand shares in log-utility space. The formal correspondence:
\begin{itemize}
\item The information friction $T$ maps to the inverse of the CES elasticity parameter in the demand system.
\item The inclusive value $\log\sum_k \exp(u_k/T)$ is the CES welfare index.
\item Adding an alternative changes the inclusive value by $\Delta IV = \log(1 + \exp(u_{\mathrm{new}}/T)/Z)$, which is exactly the consumer surplus from adding a variety in the Dixit-Stiglitz framework.
\end{itemize}

This duality was established by \citet{anderson1992} and provides the micro-foundation connecting the two generating functions of this paper: the CES aggregate (production side) and the logit choice model (information side). The CES potential $\calFq = \Phi - T \cdot \Sq$ unifies both sides: $\Phi$ is the CES production value, and $T \cdot \Sq$ is the information cost that generates the logit choice probabilities.

\subsection{The Behavioral Catalog}

\begin{proposition}[Behavioral Catalog from Positive Information Friction]\label{prop:behavioral_catalog}
At $T > 0$, the following behavioral phenomena emerge as necessary consequences of finite information processing capacity:
\begin{enumerate}[label=(\roman*)]
    \item \textbf{Choice stochasticity}: $P(j)$ is smooth, not a step function.
    \item \textbf{Context dependence}: Adding any alternative changes all absolute choice probabilities.
    \item \textbf{Probability weighting}: $w(p) = p^{\rho}/[p^{\rho} + (1-p)^{\rho}]$---the Tversky-Kahneman probability weighting function, derived here as the CES share function applied to binary outcomes.
    \item \textbf{Status quo bias}: Switching requires $\Delta H > 0$ bits; the cost $T \cdot \Delta H > 0$ creates a wedge within which objectively better alternatives are not adopted.
    \item \textbf{Choice overload}: When $J > J^* = \exp((u_{\max} - u_0)/T)$, the entropy cost exceeds the utility gain.
\end{enumerate}
\end{proposition}

\begin{proof}
Each phenomenon is derived from the CES potential \eqref{eq:logit_choice}.

\medskip
\emph{Step 1: Choice stochasticity.}
From \eqref{eq:logit_choice}, $P(j) = \exp(u_j/T) / Z$ where $Z = \sum_k \exp(u_k/T)$. For any $T > 0$, $P(j) \in (0,1)$ for all $j$---no alternative is chosen with certainty. The variance of choice around the best alternative scales as $\Var(j^*) \sim T$.

\medskip
\emph{Step 2: Context dependence.}
Adding alternative $k$ changes the inclusive value from $Z$ to $Z' = Z + \exp(u_k/T) > Z$. For every existing alternative $j$: $P'(j) = \exp(u_j/T)/Z' < \exp(u_j/T)/Z = P(j)$. Adding any alternative---even a dominated one---reduces the absolute choice probability of every existing option. While the logit preserves relative odds $P(i)/P(j)$, absolute probabilities shift. At $T = 0$, this effect vanishes.

\medskip
\emph{Step 3: Probability weighting.}
Consider a binary gamble with objective probability $p$. The agent processes this through the CES aggregator with parameter $\rho$. The CES-weighted share of attention to the high-probability branch is:
\begin{equation}\label{eq:tk_weighting}
    w(p) = \frac{p^{\rho}}{p^{\rho} + (1-p)^{\rho}}
\end{equation}
At $\rho = 1$: $w(p) = p$ (no distortion). At $p = 0$: $w(0) = 0$; at $p = 1$: $w(1) = 1$ (boundary preservation). By construction $w(p) + w(1-p) = 1$. For $\rho < 1$: $w'(0^+) = +\infty$, so $w(p) > p$ for small $p$---overweighting of rare events---and by symmetry $w(p) < p$ for $p$ near 1 (underweighting of likely events). The crossover occurs at $p = 1/2$ where $w(1/2) = 1/2$ for all $\rho$. This is the one-parameter \citet{tversky1992} probability weighting function, derived here as the CES share function applied to binary probability processing. \citet{tversky1992} estimated $\gamma \approx 0.65$ from experimental data. In the CES framework, $\gamma = \rho$, so the estimated probability weighting parameter corresponds to $\sigma = 1/(1-0.65) \approx 2.86$---a CES elasticity squarely in the range of standard estimates for differentiated goods. This is not a coincidence: it reflects the fact that human probability processing has the same complementarity structure as the evaluation of differentiated product bundles.

\medskip
\emph{Step 4: Status quo bias.}
Let $u_0$ be the status quo utility and $u_1 > u_0$ the alternative. Choosing the status quo requires no information processing: $H_{\mathrm{stay}} = 0$ bits. Evaluating the switch requires processing $\Delta H = h(p^*)$ bits. The net benefit of switching is $\Delta \calF = (u_1 - u_0) - T \cdot \Delta H$. The agent maintains the status quo whenever $u_1 - u_0 < T \cdot \Delta H$---status quo bias.

\medskip
\emph{Step 5: Choice overload.}
With $J$ alternatives, the CES potential of the menu is $\calF_J = T \log \sum_j \exp(u_j/T)$. The net value of menu engagement is $V_{\mathrm{menu}} = \calF_J - u_0 - T \cdot H_{\mathrm{menu}}$ where $H_{\mathrm{menu}} \leq \log J$. When $J > J^* \equiv \exp((u_{\max} - u_0)/T)$, the entropy cost exceeds the utility gain and the agent rationally declines to choose.
\end{proof}

\section{Empirical Tests}\label{sec:empirical}
%=============================================================================

\subsection{Banking Regulation and the Global Financial Crisis}\label{sec:gfc}

The framework predicts that markets with higher substitutability ($\rho$ close to 1, low $K$) are more vulnerable to information degradation, and that this vulnerability is \emph{mediated} by information quality. The 2007--2009 GFC provides a natural experiment.

\paragraph{The BCL Puzzle.} \citet{barth2006} documented that countries with stricter activity restrictions experienced worse financial outcomes. The CES potential framework resolves this: activity restrictions force banks into commodity lending (high $\rho$, low $K$), lowering the breakdown threshold $T^*$.

\paragraph{Mechanism.} Activity restrictions (ARI) force banks into commodity lending (high $\rho$, low $K_{\mathrm{eff}}$), lowering $T^*$. When crisis raises $T$, high-ARI countries cross their lower threshold first. The prediction is an interaction: ARI alone should not predict crisis severity, but ARI$\times$PMI (where PMI proxies for $T^{-1}$) should be significant.

\paragraph{Data.} Pre-crisis BCL indices from BRSS Wave~2 (2003): ARI (proxy for $\rho$, range 4--16), PMI (proxy for $T^{-1}$, range 3--12). Outcome: cumulative GDP per capita change 2007--2009. Sample: 147 countries.

\begin{table}[htbp]
\centering
\caption{Cross-Country GFC Severity: Activity Restrictions $\times$ Private Monitoring}
\label{tab:gfc}
\small
\begin{tabular}{lccc}
\toprule
 & (1) & (2) & (3) \\
 & Baseline & +PMI & Interaction \\
\midrule
  ARI (Activity Restrictions) & \makecell{0.324 \\ (0.212)} & \makecell{0.317 \\ (0.210)} & \makecell{$-$1.928$^{*}$ \\ (0.950)} \\[6pt]
  PMI (Private Monitoring) &  & \makecell{$-$0.391 \\ (0.304)} & \makecell{$-$3.310$^{**}$ \\ (1.246)} \\[6pt]
  ARI $\times$ PMI &  &  & \makecell{0.307$^{*}$ \\ (0.127)} \\[6pt]
  $\log$ GDP per capita & \makecell{$-$1.828$^{***}$ \\ (0.496)} & \makecell{$-$1.718$^{***}$ \\ (0.483)} & \makecell{$-$1.797$^{***}$ \\ (0.487)} \\[6pt]
\midrule
  $N$ & 147 & 147 & 147 \\
  $R^2$ & 0.110 & 0.117 & 0.138 \\
\bottomrule
\end{tabular}
\vspace{4pt}
\parbox{0.90\textwidth}{\footnotesize
  Dependent variable: cumulative GDP per capita change 2007--2009 (\%).
  Robust (HC1) standard errors in parentheses.
  Placebo interactions (all insignificant): CSI$\times$PMI $\beta=-0.09$, $p=0.51$; SPI$\times$PMI $\beta=-0.26$, $p=0.07$; EBI$\times$PMI $\beta=-0.24$, $p=0.31$.
  $^{***}p<0.001$; $^{**}p<0.01$; $^{*}p<0.05$.
}
\end{table}

The interaction ARI~$\times$~PMI enters with $\beta = +0.307$ ($p = 0.016$), with both constituent terms significant and correctly signed. At the median PMI of 7, the marginal effect of ARI is near zero; at low PMI ($\sim$4), it is $-0.69$ pp per ARI unit. This is the framework's prediction: $T^*(\sigma)$ is low when $\sigma$ is high (high ARI), so the system collapses unless $T$ is also low (high PMI). Placebo tests using CSI, SPI, and EBI instead of ARI show no significant interactions, confirming specificity to the product substitutability dimension.

The interaction confirms the framework: ARI alone is insignificant (column 1), but ARI$\times$PMI enters with $\beta = +0.307$ ($p = 0.016$) with both constituent terms correctly signed. Countries with high ARI (commodity banking, high $\rho$) and low PMI (high $T$) experienced worse crises. Placebo interactions using CSI, SPI, and EBI instead of ARI are all insignificant, confirming specificity to the product substitutability dimension.

\subsection{Manufacturing Tail Distributions (Tsallis Test)}\label{sec:tsallis_empirical}

The Shannon framework predicts exponential distributions for economic fluctuations; the Tsallis framework predicts $q$-exponential distributions. The distinguishing feature is tail behavior: exponential tails decay as $e^{-\beta|r|}$, while $q$-exponential tails decay as $|r|^{-1/(q-1)}$ for $q > 1$ or have compact support for $q < 1$.

\paragraph{Data.} Monthly industrial production (IP) indices for 17 manufacturing sectors from FRED. For each sector $s$, absolute log-returns $|r_{s,t}| = |\log(\text{IP}_{s,t}/\text{IP}_{s,t-1})|$ are computed and fit with:
\begin{enumerate}[label=(\alph*)]
\item \textbf{Exponential} (Shannon): $f(r) = \lambda e^{-\lambda r}$, one parameter ($\lambda = 1/\bar{r}$).
\item \textbf{$q$-exponential} (Tsallis): $f_q(r) = C_q [1 - (1-q)\beta r]_+^{1/(1-q)}$, two parameters ($q, \beta$).
\end{enumerate}
The exponential is nested within the $q$-exponential ($q = 1$), enabling a likelihood ratio test with 1 degree of freedom.

\paragraph{Identification strategy.} Two tests provide complementary evidence:

\emph{Within-sector}: For each sector, the likelihood ratio statistic $\Lambda_s = -2(\ell_{\exp} - \ell_{q\text{-exp}})$ is asymptotically $\chi^2(1)$ under the null $H_0: q = 1$. The Anderson-Darling statistic against the exponential null provides a complementary test.

\emph{Cross-sector}: If $q = \rho$ (the CES parameter), then estimated $\hat{q}_s$ should correlate with independent estimates of $\hat{\sigma}_s$ (elasticity of substitution) from \citet{oberfield2021}. The test regresses $\hat{q}_s$ on $\hat{\rho}_s = 1 - 1/\hat{\sigma}_s$ and tests $\beta_1 = 1$.

\paragraph{Results.} The $q$-exponential provides a significantly better fit (LR test $p < 0.05$) in 12 of 17 sectors; Anderson-Darling rejects the exponential null in 15 of 17. After GARCH(1,1) standardization, 9 of 17 retain significance, confirming the tail behavior is not purely a volatility artifact. The cross-sector regression of $\hat{q}$ on $\hat{\rho}$ yields $R^2 = 0.03$---correct direction but lacking power in this small cross-section.

\subsection{Price of Incentive Compatibility in Procurement}\label{sec:poic}

The Myerson derivation predicts that information rents increase with product differentiation. Using 1{,}172 federal contract awards from USAspending.gov, stratified by NAICS sector into HIGH (commodity), MEDIUM, and LOW (differentiated) substitutability:

\begin{table}[htbp]
\centering
\caption{Price of Incentive Compatibility: Procurement Markup Regressions}
\label{tab:poic}
\small
\begin{tabular}{lcccc}
\toprule
 & (1) & (2) & (3) & (4) \\
 & Bivariate & +\,Competition & +\,Full Controls & Category Dummies \\
\midrule
  Substitutability ($\rho$ proxy) & \makecell{0.031$^{***}$ \\ (0.009)} & \makecell{0.031$^{***}$ \\ (0.009)} & \makecell{$-$0.038$^{**}$ \\ (0.013)} & \\[6pt]
  LOW (ref = HIGH) & & & & \makecell{0.233$^{***}$ \\ (0.030)} \\[6pt]
  MEDIUM (ref = HIGH) & & & & \makecell{0.373$^{***}$ \\ (0.033)} \\[6pt]
  $\log$(ceiling) & & & \makecell{$-$0.032$^{***}$ \\ (0.003)} & \makecell{$-$0.069$^{***}$ \\ (0.005)} \\[6pt]
\midrule
  $N$ & 1{,}172 & 1{,}172 & 1{,}172 & 1{,}172 \\
  $R^2$ & 0.011 & 0.016 & 0.099 & 0.227 \\
\bottomrule
\end{tabular}
\vspace{4pt}
\parbox{0.90\textwidth}{\footnotesize
  Dependent variable: markup ratio (actual obligation / contract ceiling).
  HC1 robust standard errors in parentheses.
  $^{***}p<0.001$; $^{**}p<0.01$; $^{*}p<0.05$.
}
\end{table}

Conditional on contract size, more substitutable goods have lower markups ($\beta = -0.038$, $p = 0.003$). The category dummy specification ($R^2 = 0.23$) shows LOW contracts have 23.3~pp higher markup than HIGH contracts ($p < 10^{-14}$).

Contracts are classified by NAICS sector into HIGH (commodity: agriculture, mining, food), MEDIUM (differentiated: chemicals, metals, machinery), and LOW (complementary: electronics, aircraft, professional services) substitutability. Results are robust to outlier exclusion, competition controls, and restriction to competitive procurement vehicles.

\subsection{Cross-Domain Test Design}\label{sec:cross_domain}

The two within-domain tests above confirm the $\rho \times T$ interaction within single domains. The decisive test is \emph{cross-domain}: estimate $\rho$ from production data and test whether the same $\hat{\rho}$ predicts outcomes in other domains. This subsection specifies the test.

\paragraph{Data.}
\emph{Production $\rho$}: The NBER-CES Manufacturing Industry Database provides value added, capital, and labor for 364 six-digit NAICS industries (1977--2018). Estimate $\hat{\rho}_s$ per industry via nonlinear least squares on the CES production function, then aggregate to two-digit NAICS (21 manufacturing sectors) using the within-sector median.
\emph{Search duration}: BLS JOLTS reports job openings, hires, and separations by two-digit NAICS. The ratio Openings/Hires proxies for the equilibrium search duration $n^*$ predicted in \Cref{sec:search}.
\emph{Procurement markups}: USAspending.gov contracts coded by NAICS. The markup ratio (obligation/ceiling) proxies for the price of incentive compatibility $\mathrm{PoIC}$ predicted in \Cref{sec:mechanism}.
\emph{Adverse selection}: Product return rates or warranty claim rates by NAICS sector (Consumer Product Safety Commission or industry surveys) proxy for market breakdown predicted in \Cref{sec:akerlof}.

\paragraph{Regression.}
The cross-domain test estimates:
\begin{equation}\label{eq:cross_domain_reg}
    y_{d,s} = \alpha_d + \beta_d \hat{\rho}_s + \gamma_d \hat{T}_s + \delta_d (\hat{\rho}_s \times \hat{T}_s) + \varepsilon_{d,s}
\end{equation}
where $d$ indexes domain (search, procurement, returns), $s$ indexes sector, $\hat{\rho}_s$ is the production-side estimate, and $\hat{T}_s$ is proxied by management practice scores \citep{bloom2013}. The overidentification restriction is that $\beta_d / \delta_d$ is constant across domains: the ratio of the $\rho$ effect to the $\rho \times T$ interaction is determined by the CES potential, not by domain-specific parameters.

\paragraph{Rejection criterion.}
The framework is rejected if: (a)~$\hat{\beta}_d$ has inconsistent signs across domains; (b)~a Hansen $J$-test rejects the restriction that $\beta_d/\delta_d$ is constant across $d$; or (c)~cross-domain $R^2$ from the single $\hat{\rho}_s$ is not significantly above zero.

\paragraph{Power.}
With approximately 20 two-digit sectors, the test has modest power. The NBER-CES data provides 364 six-digit industries, but outcome variables (JOLTS, procurement) are available only at two-digit. The binding constraint is cross-domain outcome data, not production estimates. A Bonferroni-corrected significance level of $\alpha = 0.05/3 \approx 0.017$ per domain accounts for the three simultaneous tests.

%=============================================================================
\section{Tsallis Corrections ($q = \rho$)}\label{sec:six_revisited}
%=============================================================================

The six derivations used Shannon entropy for clarity. The Tsallis generalization ($q = \rho$, established in the companion paper \citealt{smirl2026potential}) modifies each result through the $1/(2-q)$ correction from the $q$-variance-response identity. \Cref{tab:tsallis_corrections} summarizes the corrections. Category~A results (Arrow, Hart-Moore) depend on the CES aggregate's geometry, not the entropy functional, and survive exactly. Category~B results acquire the $1/(2-q)$ factor, which for typical manufacturing elasticities ($\sigma \in [2,5]$) represents a 17--33\% correction.

\begin{table}[htbp]
\centering
\caption{Tsallis corrections to the six derivations ($q = \rho$)}
\label{tab:tsallis_corrections}
\small
\begin{tabular}{@{}llll@{}}
\toprule
Derivation & Shannon result & Tsallis correction & Category \\
\midrule
Akerlof & $\tau^*$ & $\tau^*_q = \tau^*(1)/(2-q)$ & B \\
Myerson & $\mathrm{PoIC}$ & $\mathrm{PoIC}_q = \mathrm{PoIC}(1)/(2-q)$ & B \\
Arrow & $T^*_{\mathrm{dem}}(\rho)$ & Exact (McDiarmid bound) & A \\
DMP & $n^*$ & $n^*_q = n^*(1)/(2-q)$ & B \\
Hart-Moore & $D(\rho, \tau)$ & Exact; hold-up $\leq T/(1-q)$ & A \\
Behavioral & $w(p) = p^\rho/[\cdots]$ & Already $q$-deformed & A \\
\bottomrule
\end{tabular}
\vspace{4pt}
\parbox{0.90\textwidth}{\footnotesize
Category A: depends on CES geometry, survives exactly. Category B: acquires $1/(2-q)$ from $q$-variance-response identity.}
\end{table}

%=============================================================================
\section{Discussion}\label{sec:discussion}
%=============================================================================

\subsection{Overidentification as the Core Test}

The six derivations are not the contribution---they are the setup. The contribution is the \emph{overidentification}: the same $\rho$ estimated from production data must predict outcomes in all six domains simultaneously. This generates cross-field predictions that no existing theory can make:
\begin{itemize}
\item Industries with low $\rho$ (e.g., semiconductor fabrication, surgical teams) should exhibit \emph{simultaneously}: severe adverse selection in input markets, high information rents in procurement, fragile governance institutions, long search durations in labor markets, large hold-up distortions in supply chains, and pronounced behavioral anomalies in decision-making.
\item Industries with high $\rho$ (e.g., commodity agriculture, retail) should exhibit the opposite pattern across all six domains.
\item The cross-sectional correlation between any pair of these six outcomes, after controlling for $T$, should be explained entirely by $\rho$.
\end{itemize}
Existing theory predicts none of these cross-domain correlations. Akerlof's adverse selection severity has no theoretical connection to DMP search duration in any existing framework. The CES potential creates the connection: both are controlled by $\rho$ because both are instances of the same optimization problem---allocating across heterogeneous alternatives under incomplete information.

The apparent fragmentation of economic theory into six traditions is an artifact of treating each domain's parameters as independent. When the parameters are identified---adverse selection's market structure parameter IS the production elasticity IS the mechanism design curvature IS the search complementarity IS the asset specificity IS the behavioral weighting exponent---the six traditions collapse to two parameters: $(\rho, T)$.

\subsection{Falsifiability and Limitations}

The framework's falsifiability is concentrated in the $\rho \times T$ interaction: any empirical finding where $\rho$ and $T$ affect outcomes independently (additively rather than multiplicatively) would reject the CES potential structure. Two within-domain tests confirm the interaction: banking regulation ($p = 0.016$) and procurement ($p = 0.003$). The cross-domain test design in \Cref{sec:cross_domain} specifies the data, regression, and rejection criteria: estimate $\rho$ from NBER-CES production data, then test whether the same $\hat{\rho}$ predicts search duration, procurement markups, and adverse selection severity across NAICS sectors.

Two limitations deserve emphasis. First, $\rho$ is treated as exogenous; endogenizing technology choice creates a feedback loop developed in \citet{smirl2026potential}. Second, the cross-domain tests require a joint $(\hat{\rho}, \hat{T})$ panel combining production estimates with management quality data \citep{bloom2013}.

%=============================================================================
\section{Conclusion}\label{sec:conclusion}
%=============================================================================

Six areas of economic theory---adverse selection, mechanism design, social choice, search and matching, contract theory, and behavioral economics---have been derived from a single variational object: the CES potential $\calFq = \Phi_{\mathrm{CES}}(\rho) - T \cdot \Sq$ with $q = \rho$.

The central result is overidentification. The CES curvature $\rho$ can be estimated from production data, and that single estimate predicts outcomes in all six domains: adverse selection severity, information rents, democratic robustness, search duration, hold-up distortion, and behavioral anomaly magnitude. No existing theory generates these cross-domain predictions because no existing theory recognizes that the parameters governing these six phenomena are the same parameter.

The framework is falsifiable in two ways. \emph{Within-domain}: the $\rho \times T$ interaction predicts that $\rho$ and $T$ affect outcomes multiplicatively, not additively. Two tests confirm this: a banking regulation test ($p = 0.016$) and a procurement test ($p = 0.003$). \emph{Cross-domain}: \Cref{sec:cross_domain} specifies the test design---estimate $\rho$ from NBER-CES production data and test whether the same $\hat{\rho}$ predicts search duration, procurement markups, and adverse selection severity across NAICS sectors. Any systematic disagreement rejects not just one derivation but the entire CES potential structure.

The derivations are not modeling choices. The CES aggregate in each setting is forced by axioms (\citealt{smirl2026ces}): consistent sub-aggregation, Fr\'{e}chet taste heterogeneity, or inequality-averse social evaluation. The CES potential is constructed from Lagrangian duality in the companion paper \citep{smirl2026potential}. The six results derived here are therefore deductive consequences of axioms combined with positive information friction.

\newpage
%=============================================================================
\appendix
\section{Technical Lemmas}\label{app:lemmas}
%=============================================================================

This appendix collects technical results used in the main text.

\begin{lemma}[CES Hessian Eigenstructure]\label{lem:hessian}
Let $F(\mathbf{x}) = (J^{-1}\sum x_j^\rho)^{1/\rho}$ with $\rho \in (-\infty, 1] \setminus \{0\}$. At the symmetric point $\bar{\mathbf{x}} = c \cdot \bone$, the Hessian $\mathbf{H} = \nabla^2 \log F$ has:
\begin{enumerate}[label=(\roman*)]
\item Eigenvalue $\lambda_1 = -1/(Jc^2)$ on $\mathrm{span}\{\bone\}$ (multiplicity 1).
\item Eigenvalue $\lambda_\perp = -(1-\rho)/(Jc^2)$ on $\bone^\perp$ (multiplicity $J-1$).
\item The spectral decomposition: $\mathbf{H} = \lambda_\perp \mathbf{I} + (\lambda_1 - \lambda_\perp) J^{-1}\bone\bone^\top = -(1-\rho)/(Jc^2) \cdot \mathbf{I} - \rho/(J^2 c^2) \cdot \bone\bone^\top$.
\end{enumerate}
Under a budget constraint $\sum x_j = C$, the direction $\bone$ is constrained out, and only the $(J-1)$-dimensional eigenvalue $\lambda_\perp$ governs the curvature of production on the feasible set.
\end{lemma}

\begin{proof}
At $\bar{\mathbf{x}} = c\bone$: $F = c$, $\partial F/\partial x_j = c^{1-\rho} x_j^{\rho-1}/J|_{\mathrm{sym}} = 1/J$. Then $\partial \log F/\partial x_j = (1/F)(\partial F/\partial x_j) = 1/(Jc)$. By Euler's theorem for degree-1 homogeneous functions: $\sum x_j \partial \log F/\partial x_j = 1$, confirming $\bone \cdot \nabla \log F = 1/c$.

For the second derivatives:
\[
\frac{\partial^2 \log F}{\partial x_j \partial x_k} = \frac{1}{F}\frac{\partial^2 F}{\partial x_j \partial x_k} - \frac{1}{F^2}\frac{\partial F}{\partial x_j}\frac{\partial F}{\partial x_k}
\]
At symmetry: $\partial^2 F/\partial x_j^2|_{\mathrm{sym}} = -(1-\rho)(J-1)/(J^2 c)$ and $\partial^2 F/\partial x_j \partial x_k|_{\mathrm{sym}} = (1-\rho)/(J^2 c)$ for $j \neq k$ (both computed from the chain rule applied to $F = S^{1/\rho}$, $S = J^{-1}\sum x_i^\rho$). The Hessian of $\log F$ is:
\begin{align*}
H_{jj} &= \frac{1}{c}\cdot\frac{-(1-\rho)(J-1)}{J^2 c} - \frac{1}{J^2 c^2} = -\frac{J(1-\rho) + \rho}{J^2 c^2} \\
H_{jk} &= \frac{1}{c}\cdot\frac{(1-\rho)}{J^2 c} - \frac{1}{J^2 c^2} = -\frac{\rho}{J^2 c^2} \quad (j \neq k)
\end{align*}
Writing $\mathbf{H} = (H_{jj} - H_{jk})\mathbf{I} + H_{jk}\bone\bone^\top$: the coefficient of $\mathbf{I}$ is $-(1-\rho)/(Jc^2)$ and of $\bone\bone^\top$ is $-\rho/(J^2 c^2)$. On $\bone^\perp$: eigenvalue $-(1-\rho)/(Jc^2)$. On $\bone$: eigenvalue $-(1-\rho)/(Jc^2) + J\cdot(-\rho/(J^2 c^2)) = -(1-\rho)/(Jc^2) - \rho/(Jc^2) = -1/(Jc^2)$.
\end{proof}

\begin{lemma}[$q$-Exponential Normalization]\label{lem:q_normalization}
For $q \in (0, 2)$, $q \neq 1$, and cost levels $\varepsilon_1 \leq \cdots \leq \varepsilon_J$ with $\varepsilon_1 = 0$, the $q$-inclusive value $Z_q = \sum_{j=1}^{J_q} [1 - (1-q)\beta\varepsilon_j]_+^{1/(1-q)}$ satisfies:
\begin{enumerate}[label=(\roman*)]
\item For $q < 1$: $J_q = |\{j : \varepsilon_j < T/(1-q)\}| \leq J$ (only inputs below the cost threshold contribute). $Z_q$ is finite for finite $J$.
\item For $q > 1$: $J_q = J$ (all inputs contribute). For large $\varepsilon_J$: $Z_q \sim \varepsilon_J^{1/(q-1)} \cdot C(q)$ (power-law growth).
\item For $q \to 1$: $Z_q \to Z = \sum_j e^{-\beta\varepsilon_j}$ (standard inclusive value recovery).
\end{enumerate}
\end{lemma}

\begin{proof}
(i) When $q < 1$: $1-q > 0$, so the bracket $[1-(1-q)\beta\varepsilon_j]_+$ vanishes for $\varepsilon_j > T/(1-q)$. The number of contributing terms is $J_q = |\{j : (1-q)\beta\varepsilon_j < 1\}|$. Each surviving term is bounded above by $1$, so $Z_q \leq J_q \leq J$.

(ii) When $q > 1$: $1-q < 0$, so $[1-(1-q)\beta\varepsilon_j]_+ = 1 + (q-1)\beta\varepsilon_j > 0$ for all $\varepsilon_j \geq 0$. For large $\varepsilon_j$: $[1+(q-1)\beta\varepsilon_j]^{1/(1-q)} = [(q-1)\beta\varepsilon_j]^{-1/(q-1)} \cdot (1 + o(1))$, which decays as a power law. The sum converges if and only if $1/(q-1) > 1$, i.e., $q < 2$.

(iii) The limit $\lim_{q \to 1} [1+(1-q)x]^{1/(1-q)} = e^x$ is standard (defining the exponential as a limit of powers).
\end{proof}

\begin{lemma}[Tsallis Entropy Hessian]\label{lem:tsallis_hessian}
The Hessian of Tsallis entropy $\Sq(p) = (1 - \sum p_j^q)/(q-1)$ at the uniform distribution $p_j = 1/J$ is:
\begin{equation}\label{eq:tsallis_hessian}
\frac{\partial^2 \Sq}{\partial p_j \partial p_k}\bigg|_{\mathrm{unif}} = -q J^{2-q} \delta_{jk}
\end{equation}
Compared to Shannon: $\partial^2 H/\partial p_j \partial p_k|_{\mathrm{unif}} = -J \delta_{jk}$. The ratio is $q J^{2-q}/J = q J^{1-q}$. At $q \to 1$: $qJ^{1-q} \to J^0 = 1$, recovering the Shannon case. The effective curvature at symmetric equilibrium, entering the variance-response identity through the inverse Hessian, gives the $1/(2-q)$ factor as stated in the companion paper (\citealt{smirl2026potential}, Theorem~4).
\end{lemma}

\begin{proof}
Direct computation: $\partial \Sq/\partial p_j = -q p_j^{q-1}/(q-1)$ and $\partial^2 \Sq/\partial p_j^2 = -q(q-1) p_j^{q-2}/(q-1) = -q p_j^{q-2}$. At $p_j = 1/J$: $\partial^2 \Sq/\partial p_j^2 = -q (1/J)^{q-2} = -q J^{2-q}$. At $q = 1$: $-1 \cdot J^1 = -J$, matching Shannon. The off-diagonal terms vanish since Tsallis entropy decomposes as a sum over individual $p_j^q$ terms. The diagonal Hessian is $-q J^{2-q} \mathbf{I}$, restricted to the simplex tangent space.
\end{proof}

\newpage
%=============================================================================
% References
%=============================================================================
\bibliographystyle{aer}

\begin{thebibliography}{99}

\bibitem[Acz\'{e}l(1966)]{aczel1966}
Acz\'{e}l, J\'{a}nos. 1966. \textit{Lectures on Functional Equations and Their Applications}. New York: Academic Press.

\bibitem[Akerlof(1970)]{akerlof1970}
Akerlof, George A. 1970. ``The Market for `Lemons': Quality Uncertainty and the Market Mechanism.'' \textit{Quarterly Journal of Economics} 84(3): 488--500.

\bibitem[Anderson, de Palma, and Thisse(1992)]{anderson1992}
Anderson, Simon P., Andr\'{e} de Palma, and Jacques-Fran\c{c}ois Thisse. 1992. \textit{Discrete Choice Theory of Product Differentiation}. Cambridge, MA: MIT Press.

\bibitem[Armington(1969)]{armington1969}
Armington, Paul S. 1969. ``A Theory of Demand for Products Distinguished by Place of Production.'' \textit{IMF Staff Papers} 16(1): 159--178.

\bibitem[Arrow(1951)]{arrow1951}
Arrow, Kenneth J. 1951. \textit{Social Choice and Individual Values}. New York: Wiley.

\bibitem[Arrow et~al.(1961)]{arrow1961}
Arrow, Kenneth J., Hollis B. Chenery, Bagicha S. Minhas, and Robert M. Solow. 1961. ``Capital-Labor Substitution and Economic Efficiency.'' \textit{Review of Economics and Statistics} 43(3): 225--250.

\bibitem[Atkinson(1970)]{atkinson1970}
Atkinson, Anthony B. 1970. ``On the Measurement of Inequality.'' \textit{Journal of Economic Theory} 2(3): 244--263.

\bibitem[Barth, Caprio, and Levine(2006)]{barth2006}
Barth, James R., Gerard Caprio Jr., and Ross Levine. 2006. \textit{Rethinking Bank Regulation: Till Angels Govern}. Cambridge: Cambridge University Press.

\bibitem[Bloom et~al.(2013)]{bloom2013}
Bloom, Nicholas, Benn Eifert, Aprajit Mahajan, David McKenzie, and John Roberts. 2013. ``Does Management Matter? Evidence from India.'' \textit{Quarterly Journal of Economics} 128(1): 1--51.

\bibitem[Caplin and Dean(2015)]{caplin2015}
Caplin, Andrew, and Mark Dean. 2015. ``Revealed Preference, Rational Inattention, and Costly Information Acquisition.'' \textit{American Economic Review} 105(7): 2183--2203.

\bibitem[Costinot(2009)]{costinot2009}
Costinot, Arnaud. 2009. ``An Elementary Theory of Comparative Advantage.'' \textit{Econometrica} 77(4): 1165--1192.

\bibitem[Diamond(1982)]{diamond1982}
Diamond, Peter A. 1982. ``Aggregate Demand Management in Search Equilibrium.'' \textit{Journal of Political Economy} 90(5): 881--894.

\bibitem[Dixit and Stiglitz(1977)]{dixit1977}
Dixit, Avinash K., and Joseph E. Stiglitz. 1977. ``Monopolistic Competition and Optimum Product Diversity.'' \textit{American Economic Review} 67(3): 297--308.

\bibitem[Eaton and Kortum(2002)]{eaton2002}
Eaton, Jonathan, and Samuel Kortum. 2002. ``Technology, Geography, and Trade.'' \textit{Econometrica} 70(5): 1741--1779.

\bibitem[Ethier(1982)]{ethier1982}
Ethier, Wilfred J. 1982. ``National and International Returns to Scale in the Modern Theory of International Trade.'' \textit{American Economic Review} 72(3): 389--405.

\bibitem[Grossman and Hart(1986)]{grossman1986}
Grossman, Sanford J., and Oliver D. Hart. 1986. ``The Costs and Benefits of Ownership: A Theory of Vertical and Lateral Integration.'' \textit{Journal of Political Economy} 94(4): 691--719.

\bibitem[Hardy, Littlewood, and P\'{o}lya(1952)]{hardy1952}
Hardy, G.~H., J.~E. Littlewood, and G. P\'{o}lya. 1952. \textit{Inequalities}. 2nd ed. Cambridge: Cambridge University Press.

\bibitem[Hart and Moore(1990)]{hart1990}
Hart, Oliver, and John Moore. 1990. ``Property Rights and the Nature of the Firm.'' \textit{Journal of Political Economy} 98(6): 1119--1158.

\bibitem[Kahneman and Tversky(1979)]{kahneman1979}
Kahneman, Daniel, and Amos Tversky. 1979. ``Prospect Theory: An Analysis of Decision under Risk.'' \textit{Econometrica} 47(2): 263--291.

\bibitem[Kremer(1993)]{kremer1993}
Kremer, Michael. 1993. ``The O-Ring Theory of Economic Development.'' \textit{Quarterly Journal of Economics} 108(3): 551--575.

\bibitem[Krugman(1991)]{krugman1991}
Krugman, Paul. 1991. ``Increasing Returns and Economic Geography.'' \textit{Journal of Political Economy} 99(3): 483--499.

\bibitem[Mat\v{e}jka and McKay(2015)]{matejka2015}
Mat\v{e}jka, Filip, and Alisdair McKay. 2015. ``Rational Inattention to Discrete Choices: A New Foundation for the Multinomial Logit Model.'' \textit{American Economic Review} 105(1): 272--298.

\bibitem[Mortensen(1982)]{mortensen1982}
Mortensen, Dale T. 1982. ``The Matching Process as a Noncooperative Bargaining Game.'' In \textit{The Economics of Information and Uncertainty}, ed. John J. McCall, 233--258.

\bibitem[Myerson(1981)]{myerson1981}
Myerson, Roger B. 1981. ``Optimal Auction Design.'' \textit{Mathematics of Operations Research} 6(1): 58--73.

\bibitem[Oberfield and Raval(2021)]{oberfield2021}
Oberfield, Ezra, and Devesh Raval. 2021. ``Micro Data and Macro Technology.'' \textit{Econometrica} 89(2): 703--732.

\bibitem[Pissarides(1985)]{pissarides1985}
Pissarides, Christopher A. 1985. ``Short-Run Equilibrium Dynamics of Unemployment, Vacancies, and Real Wages.'' \textit{American Economic Review} 75(4): 676--690.

\bibitem[Shimer(2005)]{shimer2005}
Shimer, Robert. 2005. ``The Cyclical Behavior of Equilibrium Unemployment and Vacancies.'' \textit{American Economic Review} 95(1): 25--49.

\bibitem[Sims(2003)]{sims2003}
Sims, Christopher A. 2003. ``Implications of Rational Inattention.'' \textit{Journal of Monetary Economics} 50(3): 665--690.

\bibitem[Smirl(2026a)]{smirl2026ces}
Smirl, Jon. 2026a. ``Emergent CES and the Quadruple Role of Curvature.'' Working Paper.

\bibitem[Smirl(2026b)]{smirl2026potential}
Smirl, Jon. 2026b. ``The CES Potential: Information Friction and Complementary Production.'' Working Paper.

\bibitem[Thaler and Sunstein(2008)]{thaler2008}
Thaler, Richard H., and Cass R. Sunstein. 2008. \textit{Nudge: Improving Decisions About Health, Wealth, and Happiness}. New Haven: Yale University Press.

\bibitem[Theil(1967)]{theil1967}
Theil, Henri. 1967. \textit{Economics and Information Theory}. Amsterdam: North-Holland.

\bibitem[Tversky and Kahneman(1992)]{tversky1992}
Tversky, Amos, and Daniel Kahneman. 1992. ``Advances in Prospect Theory: Cumulative Representation of Uncertainty.'' \textit{Journal of Risk and Uncertainty} 5(4): 297--323.

\bibitem[Williamson(1979)]{williamson1979}
Williamson, Oliver E. 1979. ``Transaction-Cost Economics: The Governance of Contractual Relations.'' \textit{Journal of Law and Economics} 22(2): 233--261.

\end{thebibliography}

\end{document}
