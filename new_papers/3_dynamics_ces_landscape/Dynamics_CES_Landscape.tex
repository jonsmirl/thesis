\documentclass[12pt]{article}

%=== Packages ===
\usepackage[margin=1in]{geometry}
\usepackage{amsmath,amssymb,amsthm}
\usepackage{mathtools}
\usepackage{natbib}
\usepackage[colorlinks=true,citecolor=blue,linkcolor=blue,urlcolor=blue]{hyperref}
\usepackage[capitalise,noabbrev]{cleveref}
\usepackage{booktabs}
\usepackage{enumitem}
\usepackage{graphicx}

%=== Theorem environments ===
\newtheorem{theorem}{Theorem}[section]
\newtheorem{proposition}[theorem]{Proposition}
\newtheorem{lemma}[theorem]{Lemma}
\newtheorem{corollary}[theorem]{Corollary}
\newtheorem{definition}[theorem]{Definition}
\newtheorem{remark}[theorem]{Remark}
\newtheorem{example}[theorem]{Example}

%=== Notation shortcuts ===
\newcommand{\R}{\mathbb{R}}
\newcommand{\E}{\mathbb{E}}
\newcommand{\Var}{\operatorname{Var}}
\newcommand{\Cov}{\operatorname{Cov}}
\newcommand{\Tr}{\operatorname{Tr}}
\newcommand{\calF}{\mathcal{F}}
\newcommand{\calI}{\mathcal{I}}
\newcommand{\calR}{\mathcal{R}}
\newcommand{\calL}{\mathcal{L}}
\newcommand{\calH}{\mathcal{H}}
\newcommand{\calJ}{\mathcal{J}}
\newcommand{\calS}{\mathcal{S}}
\newcommand{\calM}{\mathcal{M}}
\newcommand{\calN}{\mathcal{N}}
\newcommand{\calP}{\mathcal{P}}
\newcommand{\bone}{\mathbf{1}}

\title{Dynamics on the CES Potential Landscape:\\Fluctuation Theorems, Conservation Laws, Business Cycles,\\and Endogenous Complementarity}
\author{Jon Smirl}
\date{February 2026 \\ \smallskip \textit{Working Paper --- Journal of Political Economy}}

\begin{document}
\maketitle

\begin{abstract}
The companion papers establish that the CES potential $\calF_q = \Phi_{\mathrm{CES}}(\rho) - T \cdot S_q$ with $q = \rho$ governs equilibrium selection in economies with CES production technology (curvature $K$) and information frictions ($T = 1/\kappa$), and that the curvature parameter $K = (1-\rho)(J-1)/J$ simultaneously controls superadditivity, correlation robustness, strategic independence, and network scaling.  This paper develops the full \emph{dynamical} theory on that CES potential landscape.

Four families of results emerge.  \textbf{Part I} (Dynamical Consequences) treats the CES potential as a landscape on which the economy evolves: adjustment dynamics derive the multi-level timescale hierarchy from the Hessian eigenstructure; the Variance-Response Identity (fluctuation-dissipation theorem) operationalizes information friction as the ratio of productivity variance to shock responsiveness; pre-crisis deceleration near regime boundaries generates ordered early warning signals; the Symmetric Adjustment Theorem (Onsager reciprocity) constrains cross-sector coupling; noise-driven transition rates (Kramers) convert deterministic crossing predictions into probability distributions; the Minimum Policy Cost Theorem (Jarzynski) bounds policy costs from below; and multi-scale aggregation identifies $\rho$ and $T$ as the only parameters surviving aggregation.

\textbf{Part II} (Conservation Laws) derives exact constraints from the CES potential's symmetries: the Euler-equilibrium identity provides a model-independent measurement of $T$; permutation symmetry forces covariance matrices into a structured $(J-1,1)$ eigenvalue form; the Reversibility Ratio Theorem (Crooks) quantifies irreversibility far from equilibrium; the crisis count invariant (winding number) is a structurally protected integer counting crises per technology cycle; and network conservation laws (Casimir invariants) from the trade-coupling matrix explain why trade liberalization accelerates adjustment.

\textbf{Part III} (Business Cycles) derives endogenous oscillations from the conservative-dissipative structure: the $\rho$-ordering theorem predicts sectors enter recession in order of complementarity; slow-fast dynamics explain the 5:1 expansion-contraction asymmetry; the Minsky trap formalizes ``stability breeds instability'' as a theorem in $(\rho,T)$ space; the Great Moderation is a damping ratio shift that eliminates oscillatory precursors; and the Kitchin--Juglar--Kuznets--Kondratiev hierarchy corresponds to the eigenspectrum of the antisymmetric coupling matrix.

\textbf{Part IV} (Endogenous $\rho$) closes the framework: four channels---firm optimization, technological standardization, evolutionary selection, and multi-scale aggregation---drive $\rho$ evolution on different timescales.  The coupled $(\rho,T)$ dynamics possess a stable endogenous cycle reproducing the Perez technology wave, exhibit endogenous tipping (attraction to the critical curve), and produce power-law fluctuation statistics.  With endogenous $\rho$, the framework has no free structural parameters.
\end{abstract}

\textbf{JEL Codes:} C62, D24, D50, D57, E10, E32, E44, F10, L16, L23, O33

\textbf{Keywords:} CES potential, variance-response identity, pre-crisis deceleration, early warning signals, conservation laws, business cycles, endogenous complementarity, endogenous tipping, Minsky instability, multi-scale aggregation

%=============================================================================
\section{Introduction}\label{sec:intro}
%=============================================================================

The companion papers establish a two-parameter framework for economic analysis.  Paper 1 proves that a single curvature parameter $K = (1-\rho)(J-1)/J$ simultaneously controls superadditivity, correlation robustness, strategic independence, and network scaling \citep{smirl2026ces}.  Paper 2 shows that production under information frictions is governed by the CES potential $\calF_q = \Phi_{\mathrm{CES}}(\rho) - T \cdot S_q$, where $T = 1/\kappa$ is information friction and $S_q$ is Tsallis entropy with $q = \rho$ \citep{smirl2026free,smirl2026tsallis}.  The effective curvature theorem shows that exploitable curvature is $K_{\mathrm{eff}} = K \cdot (1 - T/T^*)^+$, and the technology cycle paper derives Perez phases from bifurcation dynamics on the $(\rho, T)$ regime diagram \citep{smirl2026prod,smirl2026cycle}.

All of these results are \emph{static or comparative-static}: they characterize equilibria, compare equilibria, and identify boundaries between equilibrium regimes.  This paper develops the full \emph{dynamical} theory.  Treating $\calF$ as a landscape on which the economy evolves yields four families of results, organized in four parts.

\textbf{Part I: Dynamical Consequences} (\Cref{sec:landscape}--\Cref{sec:rg}).  Adjustment dynamics on the CES potential landscape determine relaxation timescales.  The Variance-Response Identity (VRI) operationalizes information friction as a measurable quantity.  Pre-crisis deceleration near regime boundaries generates early warning signals.  The Symmetric Adjustment Theorem constrains cross-sector coupling.  Kramers transition rate theory converts crossing predictions into probability distributions.  The Minimum Policy Cost Theorem bounds policy costs.  Multi-scale aggregation identifies $\rho$ and $T$ as the only relevant parameters.

\textbf{Part II: Conservation Laws} (\Cref{sec:euler}--\Cref{sec:casimir}).  The CES potential possesses symmetries---scaling, permutation, and logit structure---that generate exact conservation laws.  The Euler-equilibrium identity provides a second independent measurement of $T$.  Permutation symmetry forces equilibrium covariance into a structured form.  The Reversibility Ratio Theorem quantifies irreversibility far from equilibrium.  The crisis count invariant is a structurally protected integer.  Network conservation laws explain trade's role in adjustment dynamics.

\textbf{Part III: Business Cycles} (\Cref{sec:antisymmetric}--\Cref{sec:circulation}).  Directed input-output linkages generate an antisymmetric coupling matrix $\mathbf{J}$, producing oscillatory modes whose periods are geometric means of adjacent sectoral timescales.  Sectors enter recession in order of increasing $\rho$ (the $\rho$-ordering theorem).  Slow-fast dynamics explain the expansion-contraction asymmetry.  The Minsky trap formalizes ``stability breeds instability.''  The classical cycle hierarchy is the eigenspectrum of a single conservative-dissipative operator.

\textbf{Part IV: Endogenous $\rho$} (\Cref{sec:channels}--\Cref{sec:closure}).  Four channels drive $\rho$ evolution on different timescales.  The coupled $(\rho, T)$ dynamics possess a stable endogenous cycle, exhibit endogenous tipping, and close the framework: with no free structural parameters, everything is determined by initial conditions and learning-curve elasticities.

\paragraph{Methodological note.}
The paper imports established mathematical machinery into economics, but the economics is not an analogy.  The mathematical structure is identical because the CES potential functional has the same form as the free energy in statistical mechanics.  What changes is the content: energy becomes CES potential $\Phi$, thermodynamic temperature becomes information friction $T = 1/\kappa$, particle configurations become input allocations, and phase transitions become regime shifts.  Each result has a specific economic interpretation and generates testable predictions distinct from its physics origin.

\paragraph{Scope and contribution.}
The central methodological contribution is showing that a single two-parameter framework ($\rho$, $T$) is sufficient to derive results that currently require separate, unconnected theories: fluctuation-dissipation from statistical mechanics, conservation laws from symmetry analysis, business cycle theory from dynamic macroeconomics, and evolutionary economics from population biology.  The CES potential provides the unifying structure because it possesses the right symmetries (scaling, permutation, logit) and the right nonlinearity (curvature $K$ controlled by $\rho$) to generate all four families of results simultaneously.  No other production function has this property, as the Kolmogorov-Nagumo-Aczel theorem ensures that CES is the unique aggregation-consistent choice.

\paragraph{Roadmap.}  \Cref{sec:prelim} establishes preliminaries.  Parts I--IV follow in sequence.  \Cref{sec:predictions} collects testable predictions.  \Cref{sec:literature} discusses the literature.  \Cref{sec:conclusion} concludes.

%=============================================================================
\section{Preliminaries}\label{sec:prelim}
%=============================================================================

We work throughout with the CES production technology, curvature parameter, CES potential, and effective curvature established in Papers 1--2.  This section collects notation; for derivations and proofs, see \citet{smirl2026ces,smirl2026free,smirl2026tsallis,smirl2026prod}.

Consider an economy with $N$ sectors, each employing a CES production technology combining $J_n$ inputs:
\begin{equation}\label{eq:ces}
F_n(\mathbf{x}_n) = \left(\frac{1}{J_n}\sum_{j=1}^{J_n} x_{nj}^{\rho_n}\right)^{1/\rho_n}, \qquad \rho_n \in (-\infty, 1],
\end{equation}
with curvature $K_n = (1-\rho_n)(J_n - 1)/J_n$ and elasticity of substitution $\sigma_n = 1/(1-\rho_n)$.  The CES potential is $\Phi = -\sum_{n=1}^{N} \log F_n$ and the total CES potential is:
\begin{equation}\label{eq:free_energy}
\calF_q(\mathbf{x}; \boldsymbol{\rho}, \mathbf{T}) = \Phi(\mathbf{x}; \boldsymbol{\rho}) - \sum_{n=1}^{N} T_n \cdot S_{q_n}(\mathbf{x}_n),
\end{equation}
where $S_{q_n} = (1 - \sum_j p_{nj}^{q_n})/(q_n - 1)$ is the Tsallis entropy of the normalized allocation $p_{nj} = x_{nj}/\sum_k x_{nk}$ with $q_n = \rho_n$, and $T_n = 1/\kappa_n$ is the information friction in sector $n$.  The breakdown threshold is $T_n^* \propto K_n$; the effective curvature is $K_{\mathrm{eff},n} = K_n(1 - T_n/T_n^*)^+$.

The CES potential landscape $\calF: \R_{++}^{\sum J_n} \to \R$ is strictly convex (unique equilibrium) when $T_n < T_n^*$ for all $n$, and may develop multiple local minima when $T_n > T_n^*$ for some $n$.  The regime diagram in $(\rho, T)$ space has a critical curve $\Gamma = \{(\rho, T) : T = T^*(\rho)\}$ separating the single-minimum and multi-minimum regions.

\subsection{Landscape Geometry}

\begin{proposition}[Landscape structure]\label{prop:landscape}
The CES potential $\calF$ satisfies:
\begin{enumerate}[label=(\alph*)]
\item \textbf{Convexity at low $T$:} When $T_n < T_n^*$ for all $n$, $\calF$ is strictly convex with a unique minimum.
\item \textbf{Non-convexity at high $T$:} When $T_n > T_n^*$ for some $n$, $\calF$ may develop multiple local minima.
\item \textbf{Hessian structure:} $\nabla^2 \calF|_{\mathbf{x}^*} = \nabla^2 \Phi|_{\mathbf{x}^*} - \mathbf{T} \odot \nabla^2 H|_{\mathbf{x}^*}$.
\item \textbf{Breakdown threshold:} $T_n^* = \lambda_{\min}(\nabla^2 \Phi_n) / \lambda_{\max}(\nabla^2 H_n)$.  At $T_n = T_n^*$, the Hessian acquires a zero eigenvalue---the hallmark of a regime shift.
\end{enumerate}
\end{proposition}

\begin{proof}
Part (a): For $T_n < T_n^*$, the CES potential term $\nabla^2\Phi$ (positive definite for $\rho_n < 1$) dominates the entropy term in every sector.  Part (b): When $T_n > T_n^*$, the entropy term exceeds CES curvature in at least one direction.  Part (c): Direct computation.  Part (d): The zero-eigenvalue condition requires $\nabla^2\Phi \cdot \mathbf{v} = T_n \nabla^2 H_n \cdot \mathbf{v}$; the first crossing occurs at the stated ratio.
\end{proof}

\subsection{The Regime Diagram}

Below $T^*(\rho)$, the landscape has a unique minimum (the productive equilibrium).  Above $T^*(\rho)$, multiple minima may coexist.  The effective curvature theorem states $K_{\mathrm{eff}} = K(1-T/T^*)^+$: the curvature of the unique minimum decreases linearly as $T \to T^*$ from below, vanishing at $T^*$.  The vanishing curvature is the source of all the dynamical phenomena developed below.

The regime diagram in $(\rho, T)$ space partitions the parameter space into qualitatively distinct regions:
\begin{enumerate}[nosep]
\item \textbf{Single-minimum region} ($T < T^*(\rho)$): The landscape has a unique minimum.  The economy converges to a single equilibrium.  All dynamical results of Part I apply here.
\item \textbf{Multi-minimum region} ($T > T^*(\rho)$): The landscape has multiple local minima.  The economy may be trapped in a metastable state.  Kramers theory (\Cref{sec:kramers}) describes escape from metastable states.
\item \textbf{Critical curve} ($T = T^*(\rho)$): The boundary where the landscape curvature vanishes.  Pre-crisis deceleration (\Cref{sec:critical}) describes the approach to this boundary.
\end{enumerate}

The central claim of this paper is that the economy's trajectory through $(\rho, T)$ space---not just its position at any instant---determines its dynamical behavior.  Static analysis (fixing $\rho$ and $T$) misses the oscillatory, transitional, and self-organizing phenomena that emerge from movement across the regime diagram.

\subsection{Three Symmetries}

The CES potential possesses three symmetries that generate conservation laws (Part II):
\begin{enumerate}[label=(\alph*)]
\item \textbf{Scaling covariance:} $\calF(\lambda\mathbf{x}_n) = \calF(\mathbf{x}_n) - \log\lambda$ (from degree-1 homogeneity of CES).  This generates the Euler-equilibrium identity (\Cref{sec:euler}), which provides a cross-sectional measurement of $T$.

\item \textbf{Permutation invariance:} $\calF(\mathbf{x}_{\pi(n)}) = \calF(\mathbf{x}_n)$ for $\pi \in S_{J_n}$ (both $\Phi_n$ and $H_n$ are symmetric functions).  This forces the equilibrium covariance into compound symmetry structure (\Cref{sec:permutation}).  Note that permutation invariance holds within each equal-weight sector; when input weights differ, the symmetry is broken to the residual symmetry group.

\item \textbf{Logit equilibrium:} The stationary distribution $p_{\mathrm{eq}}(\mathbf{x}) \propto \exp(-\calF/T)$ inherits both symmetries and provides the foundation for all fluctuation identities (VRI, Crooks, Jarzynski).  The logit form is not assumed---it is the optimal decision rule under rational inattention \citep{matejka2015}, making the equilibrium distribution a consequence of information-constrained optimization rather than a behavioral assumption.
\end{enumerate}

Each symmetry generates a conservation law via Noether's theorem: scaling generates the Euler identity, permutation generates the covariance structure, and the logit form generates the fluctuation theorems.  Breaking any symmetry (e.g., unequal input weights break permutation; departure from rational inattention breaks logit) produces specific, observable violations of the corresponding conservation law.


\subsection{Parameter Summary}

The dynamical theory requires three families of parameters:
\begin{center}
\begin{tabular}{llll}
\toprule
Parameter & Symbol & Source & Timescale \\
\midrule
Complementarity & $\rho_n$ or $K_n$ & Production function estimation & Given (Part I--III) \\
 & & & Endogenous (Part IV) \\
Information friction & $T_n = 1/\kappa_n$ & VRI or Euler identity & Varies with cycle \\
Institutional mobility & $\ell_n$ & Impulse response half-lives & Institutional (years) \\
Antisymmetric coupling & $j_{nm}$ & I-O tables, net trade flows & Structural (decades) \\
Friction rate & $r_n$ & Perturbation decay rates & Institutional (years) \\
Learning elasticity & $\alpha$, $\beta_S$ & Wright's Law estimation & Technology-specific \\
\bottomrule
\end{tabular}
\end{center}

The multi-scale aggregation analysis (\Cref{sec:rg}) establishes that only $\rho$ and $T$ are relevant under aggregation; $\ell$, $\mathbf{J}$, and $\mathbf{R}$ are irrelevant operators that determine transient dynamics but not equilibrium structure.  The closure result (Part IV) further reduces the free parameters: with endogenous $\rho$, only initial conditions and learning elasticities remain.

\begin{remark}[Hierarchy of parameter importance]
The parameters in the table above have a strict hierarchy of importance for macroscopic prediction:
\begin{enumerate}[nosep]
\item \textbf{$\rho$ and $T$} (relevant): determine qualitative dynamics---which equilibrium regime the economy is in, whether cycles are present, and what the crisis sequence looks like.  These survive aggregation.
\item \textbf{$\ell$, $\mathbf{J}$, $\mathbf{R}$} (marginally relevant): determine quantitative dynamics---how fast the economy converges, what the cycle period is, and how severe contractions are.  These matter for transient dynamics but do not affect equilibrium structure.
\item \textbf{All other parameters} (irrelevant): firm-level strategies, preferences, organizational details.  These determine microeconomic outcomes but vanish under aggregation.
\end{enumerate}
This hierarchy is not assumed; it is derived from the multi-scale aggregation analysis.  It justifies the entire approach of the paper: the dynamical theory can be written in terms of $(\rho, T, \ell, \mathbf{J}, \mathbf{R})$ without knowing any firm-level details, and the equilibrium theory can be written in terms of $(\rho, T)$ alone.
\end{remark}


%#########################################################################
%  PART I: DYNAMICAL CONSEQUENCES OF THE CES POTENTIAL
%#########################################################################

\bigskip
\begin{center}
\textsc{\Large Part I: Dynamical Consequences}
\end{center}
\bigskip

%=============================================================================
\section{Adjustment Dynamics}\label{sec:landscape}
%=============================================================================

\subsection{The Dynamical Equation}

The simplest dynamics consistent with the CES potential landscape are adjustment dynamics with a mobility matrix $\mathbf{L}$:
\begin{equation}\label{eq:gradient_flow}
\dot{\mathbf{x}}(t) = -\mathbf{L}(\mathbf{x}) \cdot \nabla \calF(\mathbf{x}(t)),
\end{equation}
where $\mathbf{L}(\mathbf{x}) \succ 0$ is a symmetric positive-definite matrix encoding the \emph{institutional supply rate}---how efficiently markets, firms, and regulators adjust allocations.

\begin{definition}[Institutional mobility matrix]\label{def:mobility}
The mobility matrix $\mathbf{L}$ is block-diagonal across sectors:
\begin{equation}\label{eq:mobility}
\mathbf{L} = \mathrm{diag}(\ell_1 \mathbf{I}_{J_1}, \; \ell_2 \mathbf{I}_{J_2}, \; \ldots, \; \ell_N \mathbf{I}_{J_N}),
\end{equation}
where $\ell_n > 0$ is the mobility of sector $n$.  Sectors with liquid markets and flexible institutions have high $\ell_n$; sectors with rigid regulations or illiquid assets have low $\ell_n$.
\end{definition}

\subsection{Linearized Dynamics and Relaxation Rates}

Near a stable equilibrium $\mathbf{x}^*$, the linearized dynamics are $\dot{\boldsymbol{\xi}}(t) = -\mathbf{M} \cdot \boldsymbol{\xi}(t)$ where $\boldsymbol{\xi} = \mathbf{x} - \mathbf{x}^*$ and $\mathbf{M} = \mathbf{L} \cdot \nabla^2 \calF$ is the relaxation matrix.

\begin{theorem}[Relaxation spectrum]\label{thm:relaxation}
At a stable equilibrium with $T_n < T_n^*$ for all $n$, the relaxation rates $\{\lambda_k\}$ satisfy:
\begin{enumerate}[label=(\alph*)]
\item All $\lambda_k > 0$ (asymptotic stability).

\item The relaxation rate for sector $n$ is:
\begin{equation}\label{eq:sector_rate}
\lambda_n = \ell_n \cdot K_n \cdot \left(1 - \frac{T_n}{T_n^*}\right) \cdot \frac{J_n - 1}{J_n \bar{x}_n^2},
\end{equation}
proportional to institutional mobility $\ell_n$, effective curvature $K_{\mathrm{eff},n}$, and a scale factor.

\item \textbf{Timescale hierarchy:} If sectors are ordered so that $\lambda_1 < \lambda_2 < \cdots < \lambda_N$, then the slowest sector determines convergence.  The timescale separation ratio is $\lambda_{n+1}/\lambda_n$.
\end{enumerate}
\end{theorem}

\begin{proof}
Part (a): At a stable equilibrium, $\nabla^2 \calF \succ 0$ and $\mathbf{L} \succ 0$, so $\mathbf{M}$ has all positive eigenvalues.

Part (b): With block-diagonal $\mathbf{L}$, eigenvalues decompose sector by sector.  In sector $n$, the CES Hessian at balanced allocation has eigenvalue $K_n(J_n - 1)/(J_n \bar{x}_n^2)$ on the $(J_n - 1)$-dimensional subspace orthogonal to the balanced direction.  The entropy Hessian contributes $-T_n/\bar{x}_n^2$.  The net eigenvalue, multiplied by $\ell_n$, gives \eqref{eq:sector_rate}.

Part (c): Direct ratio of \eqref{eq:sector_rate} for adjacent sectors.
\end{proof}

\begin{remark}[Deriving the multi-level hierarchy]
The companion papers posit an $N$-level hierarchy with timescale separation; for the current AI transition ($N=4$): semiconductors (decades) $\to$ adoption (years) $\to$ training (months) $\to$ settlement (days).  \Cref{thm:relaxation} derives this hierarchy from primitives.  The semiconductor level has the slowest relaxation because it combines low institutional mobility $\ell$ with capital-intensive manufacturing.  Settlement has the fastest because financial markets have very high $\ell$.  The hierarchy emerges from the product $\ell_n \cdot K_{\mathrm{eff},n}$.
\end{remark}

\begin{example}[Quantitative timescale estimation]
For the four-level AI transition hierarchy:
\begin{center}
\begin{tabular}{lcccc}
\toprule
Level & $\ell_n$ (relative) & $K_{\mathrm{eff},n}$ & $\lambda_n$ (relative) & $\tau_n = 1/\lambda_n$ \\
\midrule
Semiconductor fabrication & 0.02 & 0.6 & 0.012 & $\sim$80 years \\
Hardware adoption & 0.2 & 0.4 & 0.08 & $\sim$12 years \\
Training/capability & 2.0 & 0.3 & 0.6 & $\sim$1.7 years \\
Settlement/finance & 20.0 & 0.2 & 4.0 & $\sim$3 months \\
\bottomrule
\end{tabular}
\end{center}
The timescale separation ratios are $\lambda_2/\lambda_1 \approx 7$, $\lambda_3/\lambda_2 \approx 8$, $\lambda_4/\lambda_3 \approx 7$---approximately geometric with ratio $\sim$7, consistent with the $O(10)$ timescale separation assumed in the companion papers.  The largest uncertainty is in $\ell_{\mathrm{semiconductor}}$: whether the current CHIPS Act investment accelerates institutional mobility or merely raises capacity at the existing rate.
\end{example}

\begin{remark}[Observable half-lives and the hierarchy]
The relaxation rates $\lambda_n$ map directly to observable impulse-response half-lives: $t_{1/2,n} = \ln 2/\lambda_n$.  For the AI transition hierarchy, the predicted half-lives are $\sim$55 years (semiconductor fabrication), $\sim$8 years (hardware adoption), $\sim$14 months (training/capability), and $\sim$2 months (settlement/finance).  These are testable: the settlement half-life is observable from stablecoin market recovery after exchange failures (e.g., the FTX collapse in November 2022, with stablecoin market cap recovering to pre-crisis levels within $\sim$3 months).  The training half-life is observable from the lag between major model releases and industry-wide capability convergence.  Cross-level ratios $t_{1/2,n}/t_{1/2,n+1}$ should be approximately constant if the timescale separation is geometric.
\end{remark}

\subsection{Welfare Loss Function Property}

The CES potential $\calF$ is a welfare loss function for the adjustment dynamics.

\begin{proposition}[Welfare loss function]\label{prop:lyapunov}
Along trajectories of \eqref{eq:gradient_flow}:
\begin{equation}\label{eq:lyapunov_decay}
\frac{d\calF}{dt} = -\nabla \calF^\top \mathbf{L} \nabla \calF \leq 0,
\end{equation}
with equality if and only if $\nabla \calF = 0$.  The CES potential decreases monotonically until equilibrium.
\end{proposition}

The rate of CES potential reduction, $|\nabla \calF^\top \mathbf{L} \nabla \calF|$, measures the ``inefficiency burn rate''---how quickly the economy eliminates misallocation.

\begin{corollary}[Eigenstructure bridge]\label{cor:bridge}
The Hessians of the CES potential $\calF$ (technology) and the welfare loss function (welfare) are related by:
\begin{equation}\label{eq:bridge}
\nabla^2 \calF\big|_{\mathrm{slow}} = \mathbf{W}^{-1} \cdot \nabla^2 V,
\end{equation}
where $\mathbf{W}$ is the institutional supply-rate matrix (mobility).  This ``eigenstructure bridge'' means that the technology and welfare Hessians share eigenvectors but differ in eigenvalues by the factor $w_n^{-1}$.  Faster institutional adjustment ($w_n$ large) reduces the welfare cost of a given CES potential curvature.
\end{corollary}

\begin{remark}[Damping cancellation]
The eigenstructure bridge implies a damping cancellation result: increasing the friction rate $r_n$ in sector $n$ (e.g., through regulation) speeds convergence but proportionally reduces equilibrium output.  The net welfare effect is zero to first order.  This means local regulatory intervention cannot improve welfare at the level it targets---one must reform upstream (reduce $r_{n-1}$ or increase gain elasticity $\beta_n$) to improve welfare at level $n$.  The upstream reform principle follows from the hierarchical structure of the Hessian eigenvalues.
\end{remark}

\begin{remark}[Crisis duration from eigenstructure]
At the regime boundary ($T \to T^*$), the smallest eigenvalue $\lambda_{\min} \to 0$.  The delayed-transition dynamics (\Cref{prop:canard}) imply crisis duration scales as $O(1/\sqrt{\varepsilon_{\mathrm{drift}}})$ where $\varepsilon_{\mathrm{drift}} = dT/dt|_{T^*}$ is the rate of drift through the critical point.  At Wright's Law semiconductor learning rates ($\alpha \approx 0.23$), this yields crisis durations of $\sim$8 years for technology transitions---consistent with the $\sim$7 year average from installation phase crisis to full deployment in historical Perez waves.
\end{remark}

%=============================================================================
\section{The Variance-Response Identity}\label{sec:fdt}
%=============================================================================

\subsection{Stochastic Extension}

The stochastic extension of \eqref{eq:gradient_flow} is the Langevin equation:
\begin{equation}\label{eq:langevin}
d\mathbf{x}(t) = -\mathbf{L} \nabla \calF(\mathbf{x}) \, dt + \sqrt{2 \mathbf{L} \mathbf{T}} \, d\mathbf{W}(t),
\end{equation}
where $d\mathbf{W}$ is a vector Wiener process and $\mathbf{T} = \mathrm{diag}(T_1 \mathbf{I}_{J_1}, \ldots, T_N \mathbf{I}_{J_N})$.  The noise amplitude is fixed by the requirement that the stationary distribution is the logit equilibrium $p_{\mathrm{eq}}(\mathbf{x}) \propto \exp(-\calF(\mathbf{x}) / T_{\mathrm{eff}})$.  This is the Einstein relation: the strength of spontaneous fluctuations is determined by information friction and institutional mobility, not by a separate ``shock process.''

\begin{remark}[Economic interpretation of noise]
The noise in \eqref{eq:langevin} is not exogenous---it arises from information frictions themselves.  Agents with finite information capacity $\kappa = 1/T$ make allocation decisions with inherent randomness (the $q$-logit choice model).  When aggregated, this produces Brownian fluctuations with variance proportional to $T/(2-q)$ where $q = \rho$.
\end{remark}

\begin{remark}[The Einstein relation in economics]
The relationship $\sqrt{2\mathbf{L}\mathbf{T}}$ between noise amplitude, mobility, and information friction is the economic analogue of Einstein's relation between diffusion coefficient, mobility, and temperature.  It has a profound implication: the magnitude of ``exogenous shocks'' hitting the economy is not a free parameter.  Once information friction $T$ and institutional mobility $\ell$ are measured, the noise intensity is determined.  Economies that appear to have ``large exogenous shocks'' actually have high $T \cdot \ell$---either high information frictions, high institutional mobility, or both.  Conversely, economies with low measured volatility may simply have low $T \cdot \ell$, without being fundamentally less shock-prone.
\end{remark}

\begin{remark}[Connection to the RBC literature]
The standard RBC model \citep{kydland1982} introduces exogenous TFP shocks with calibrated variance.  In the CES potential framework, these shocks are endogenous: they arise from the information friction $T$ acting through the Langevin dynamics.  The VRI operationalizes this: measured TFP shocks should decompose into a component explained by $T \cdot \chi$ (the equilibrium noise from information frictions) and a residual.  If the residual is small, ``exogenous'' productivity shocks are largely an artifact of unmeasured information frictions.  If the residual is large, there are genuine exogenous drivers (natural disasters, pandemics, political shocks) beyond the information friction mechanism.
\end{remark}

\subsection{The Economic Variance-Response Identity}

\begin{theorem}[Economic Variance-Response Identity]\label{thm:fdt}
At a stable equilibrium $\mathbf{x}^*$:
\begin{enumerate}[label=(\alph*)]
\item \textbf{Static VRI:} The equilibrium covariance matrix $\boldsymbol{\Sigma}$ and the static response matrix $\boldsymbol{\chi} = (\nabla^2 \calF)^{-1}$ satisfy:
\begin{equation}\label{eq:static_fdt}
\boldsymbol{\Sigma} = \mathbf{T} \cdot \boldsymbol{\chi} = \mathbf{T} \cdot (\nabla^2 \calF)^{-1}.
\end{equation}

\item \textbf{Sector-level VRI:} For sector $n$:
\begin{equation}\label{eq:sector_fdt}
\sigma_n^2 = T_n \cdot \chi_n,
\end{equation}
where $\sigma_n^2 = \Var(F_n) / F_n^{*2}$ is relative output variance and $\chi_n$ is shock responsiveness.

\item \textbf{Dynamic VRI:} The response function $R_{ij}(t)$ satisfies:
\begin{equation}\label{eq:dynamic_fdt}
R_{ij}(t) = -\frac{1}{T} \frac{d}{dt} C_{ij}(t), \qquad t > 0,
\end{equation}
where $C_{ij}(t) = \langle \xi_i(t) \xi_j(0) \rangle$ is the autocorrelation function.
\end{enumerate}
\end{theorem}

\begin{proof}
Part (a): The stationary distribution of \eqref{eq:langevin} is $p_{\mathrm{eq}} \propto \exp(-\calF/T)$.  For the quadratic approximation, the stationary distribution is Gaussian with covariance $\boldsymbol{\Sigma} = \mathbf{T}(\nabla^2 \calF)^{-1}$.  The static response is $\boldsymbol{\chi} = (\nabla^2 \calF)^{-1}$, giving $\boldsymbol{\Sigma} = \mathbf{T}\boldsymbol{\chi}$.

Part (b): Restrict to sector $n$; the relevant Hessian eigenvalue is $K_{\mathrm{eff},n}(J_n-1)/(J_n\bar{x}_n^2)$.  The ratio of variance to response is $T_n$.

Part (c): Standard result for Langevin dynamics via the Onsager regression hypothesis.
\end{proof}

\begin{corollary}[Measuring $T$ from data]\label{cor:measure_T}
For any sector $n$ with observable output variance $\sigma_n^2$ and measurable shock response $\chi_n$:
\begin{equation}\label{eq:T_observable}
T_n = \frac{\sigma_n^2}{\chi_n}.
\end{equation}
\end{corollary}

\begin{remark}[Data requirements]
The variance $\sigma_n^2$ is available from firm-level productivity studies \citep{syverson2004,syverson2011}.  The response $\chi_n$ requires identified shocks \citep{autor2013}.  With both measured, $T_n$ is identified.  The VRI generates a testable overidentifying restriction: estimates from different shocks must agree.
\end{remark}

\begin{remark}[Cross-validation]
The VRI generates a testable overidentifying restriction.  If $T_n$ is estimated from the variance/response ratio at two different time periods or using two different shocks, the estimates should agree.  Disagreement indicates either model misspecification or departure from near-equilibrium conditions.
\end{remark}

\begin{remark}[Preliminary evidence: cross-sectional VRI]
Using monthly FRED Industrial Production data for 17 manufacturing subsectors (1972--2026), we estimate sector-specific $\sigma_n^2$ (variance of log-growth rates) and $\chi_n$ (cumulative impulse response from bivariate VARs).  The cross-sectional OLS regression $\sigma_n^2 = \alpha + T \cdot \chi_n + \varepsilon_n$ yields $\hat{T} = 1.04 \times 10^{-4}$ with correct (positive) sign but low power ($R^2 = 0.009$, $p = 0.72$, $N = 17$).  The result is consistent with the VRI but power is limited by sample size.
\end{remark}

\begin{remark}[Stronger evidence: self-consistent $T$ from EEMD]\label{rem:fdt_eemd}
A more powerful validation uses EEMD decomposition.  EEMD partitions INDPRO growth into fast modes (period $< 4$ years, 93.6\% of energy) and slow structural modes (period $> 8$ years, 3.3\% of energy).  The fast-mode Hilbert energy $T_{\mathrm{fast}}(t) = \sum_{k \in \mathrm{fast}} a_k(t)^2$ is the self-consistent information friction.  The VRI susceptibility $\chi(t) = E_{\mathrm{slow}}(t)/T_{\mathrm{fast}}(t)$ detects five technology wave peaks (1938, 1948, 1977, 1994, 2006) corresponding to known technology eras.  Critically, $T$ automatically discriminates technology waves from financial crises: during crises $T_{\mathrm{fast}}$ spikes, suppressing $\chi$; during technology waves only $E_{\mathrm{slow}}$ rises.  This is exactly the VRI prediction, validated over 106 years.
\end{remark}

\begin{remark}[What VRI violations signal]
The VRI holds exactly only at equilibrium under the CES production function with logit choice.  Systematic violations indicate:
\begin{enumerate}[nosep]
\item \textbf{Departure from near-equilibrium:} If the economy is in transit between equilibria (e.g., during a technology transition), correlations and responses reflect the transient dynamics, not the stationary distribution.  The VRI overestimates $T$ during rapid adjustment.
\item \textbf{Non-CES production:} If the true production function has variable elasticity of substitution (VES), the VRI applies locally but $T$ estimates depend on the allocation point, breaking consistency across shocks.
\item \textbf{Rational inattention breakdown:} If agents face hard constraints (regulation, physical limits) rather than information costs, the noise process is not proportional to $T$, and the VRI fails.
\end{enumerate}
Each violation mode is testable: (1) is detected by non-stationarity in rolling $T$ estimates; (2) by shock-dependent $T$; (3) by excess kurtosis in allocation residuals.
\end{remark}

\begin{remark}[Empirical evidence: $T$ as recession leading indicator]\label{rem:T_probit}
The operational information friction $T(t) = \sigma^2(t)/\chi(t)$, computed from rolling 60-month sector-level data across 7 manufacturing subsectors (1972--2026), has significant predictive content for recessions.  A probit model for 12-month-ahead NBER recession indicators yields $\hat{\beta}(T) = 221$ ($p = 0.0003$, pseudo-$R^2 = 0.035$).  In a joint specification with the term spread, pseudo-$R^2$ rises to 0.186, compared to 0.107 for the term spread alone and 0.0001 for the VIX alone.  The information friction adds substantial forecasting power beyond standard financial indicators.
\end{remark}

%=============================================================================
\section{Pre-Crisis Deceleration and Early Warning Signals}\label{sec:critical}
%=============================================================================

As $T_n \to T_n^*$, the effective curvature $K_{\mathrm{eff},n} \to 0$ and the landscape flattens in the direction of the incipient instability.

\begin{theorem}[Pre-crisis deceleration]\label{thm:critical_slowing}
As $T_n \to T_n^*$ from below:
\begin{enumerate}[label=(\alph*)]
\item \textbf{Autocorrelation time diverges:} $\tau_{\mathrm{corr},n} = 1/\lambda_{\min}(n) \propto (1 - T_n/T_n^*)^{-1}$.
\item \textbf{Variance diverges:} $\sigma_n^2 \propto T_n/(1 - T_n/T_n^*) \to \infty$.
\item \textbf{Recovery time diverges:} $t_{\mathrm{rec}} \propto (1 - T_n/T_n^*)^{-1} \ln(\delta/\epsilon)$.
\item \textbf{Susceptibility diverges:} $\chi_{\max,n} \propto (1 - T_n/T_n^*)^{-1}$.
\end{enumerate}
All divergences are characterized by the exponent $\gamma = 1$ (mean-field aggregation-invariant class).
\end{theorem}

\begin{proof}
All four results follow from the spectral decomposition of the linearized dynamics.  The autocorrelation function of the slowest mode is $C(t) = \sigma^2 e^{-\lambda_{\min} t}$.  The equilibrium variance from the VRI is $\sigma^2 = T/(\lambda_{\min}/\ell)$.  All diverge as $(1-T/T^*)^{-1}$ because $\lambda_{\min} \propto (1-T/T^*)$.  The mean-field exponent $\gamma = 1$ follows from the Gaussian nature of the CES system's all-to-all coupling within each sector.
\end{proof}

\begin{proposition}[Economic early warning signals]\label{prop:ews}
The approach to a regime boundary is signaled by:
\begin{enumerate}[label=(\alph*)]
\item \textbf{Rising autocorrelation in productivity:} lag-1 autocorrelation approaches 1.
\item \textbf{Rising variance in output:} rolling standard deviation increases as $(1 - T/T^*)^{-1/2}$.
\item \textbf{Flickering:} bimodal distributions appear in previously unimodal time series.
\item \textbf{Cross-sector leading indicator:} By non-uniform degradation, correlation robustness degrades before superadditivity.  Rising cross-asset correlation is the earliest warning signal.
\end{enumerate}
\end{proposition}

\begin{remark}[Connection to crisis sequence]
The crisis sequence predicts financial crisis before production disruption before governance failure.  In the early warning framework: rising financial return correlations (first signal), then rising productivity dispersion (second signal), then governance instability (third signal).  The framework gives not just the order but the quantitative intensification rate.
\end{remark}

\begin{remark}[Comparison to ecological early warning signals]
The ecological literature has documented pre-crisis deceleration before ecosystem transitions \citep{scheffer2009,dakos2012}.  The economic application has a structural advantage: the $(\rho, T)$ framework predicts \emph{which} variable will show pre-crisis deceleration (the one corresponding to the critical eigenvector) and \emph{in what order} (via the degradation hierarchy), rather than requiring blind monitoring of all variables.
\end{remark}

\begin{remark}[False positives and the near-miss problem]
A practical concern for any early warning system is the false positive rate.  In the pre-crisis deceleration framework, false positives arise when $T$ approaches $T^*$ but does not cross it---a ``near miss.''  The Kramers theory (\Cref{sec:kramers}) provides the complementary probability assessment: the probability of actual crossing given a particular $T/T^*$ ratio.  Combining the two (early warning from pre-crisis deceleration, crossing probability from Kramers) yields a calibrated crisis probability rather than a binary alarm.  The near-miss events---1998 (LTCM), 2011 (European sovereign debt), 2020 March (COVID liquidity)---are valuable calibration data: they show the system approaching $T^*$ without full crossing, allowing estimation of the barrier height $\Delta\calF$ at the near-miss point.
\end{remark}

\begin{proposition}[Quantitative intensification rate]\label{prop:intensification}
The rate at which early warning signals intensify follows a universal scaling law:
\begin{equation}
\sigma^2(T) = \sigma_0^2 \cdot \left(1 - \frac{T}{T^*}\right)^{-1}, \qquad \tau_{\mathrm{corr}}(T) = \tau_0 \cdot \left(1 - \frac{T}{T^*}\right)^{-1}.
\end{equation}
Both diverge as $(1-T/T^*)^{-1}$, with the exponent $\gamma = 1$ characteristic of the mean-field aggregation-invariant class.  This provides not just qualitative warning (``things are getting worse'') but a quantitative fit: plotting $1/\sigma^2$ or $1/\tau_{\mathrm{corr}}$ against time should yield a straight line whose x-intercept estimates the crisis date.
\end{proposition}

\begin{proof}
From \Cref{thm:critical_slowing}, $\sigma^2 = T/\lambda_{\min}$ and $\tau_{\mathrm{corr}} = 1/\lambda_{\min}$, with $\lambda_{\min} \propto (1 - T/T^*)$.  The mean-field exponent $\gamma = 1$ follows from the Gaussian (all-to-all) coupling structure within each CES sector.
\end{proof}

\begin{remark}[Practical implementation]
To implement the early warning system:
\begin{enumerate}[nosep]
\item Compute rolling 36-month autocorrelation and variance for sectoral output (IP, GDP components).
\item For each sector, plot $1/\hat{\sigma}_n^2(t)$ against time.
\item If the trend is linear and declining, the system is approaching $T^*$: the x-intercept provides an estimate of the crisis date.
\item The sector showing the steepest decline is the one closest to its threshold---the sector that will enter crisis first.
\item Cross-sector comparison reveals the $\rho$-ordering: sectors should show declining $1/\sigma^2$ in order of increasing $\rho$.
\end{enumerate}
This algorithm transforms the theoretical early warning framework into a practical monitoring tool.
\end{remark}

\begin{remark}[Prediction for the current AI transition]
The pre-crisis deceleration framework makes a specific near-term prediction: as the AI technology cycle approaches its frenzy-to-turning-point transition, the following signals should appear in sequence:
\begin{enumerate}[nosep]
\item Rising return correlations among major AI firms (NVIDIA, Microsoft, Google, Amazon)---the correlation robustness degradation signal.  This should precede fundamental stress by 1--3 years.
\item Rising productivity dispersion within the AI supply chain (chip design, fabrication, cloud deployment, application development)---the superadditivity degradation signal.
\item Rising autocorrelation in AI firm revenues and earnings---the direct pre-crisis deceleration signature.
\end{enumerate}
As of early 2026, the correlation signal appears to be emerging (AI ``Magnificent Seven'' return correlations have risen from $\sim$0.3 in 2023 to $\sim$0.6 in 2025), consistent with the predicted ordering.
\end{remark}

\begin{remark}[Empirical evidence: pre-crisis deceleration at technology waves]\label{rem:ews_eemd}
The EEMD-based CES potential wave detector provides support.  The autocorrelation decay time $\tau_{\mathrm{ACF}}(t)$ of slow EEMD modes peaks simultaneously with VRI susceptibility at all five detected technology waves (1938, 1948, 1977, 1994, 2006).  The combined indicator $\Psi = \sqrt{\hat{\chi} \cdot \hat{\tau}}$ confirms that technology transitions exhibit the joint susceptibility/slowing signature predicted by \Cref{thm:critical_slowing}.
\end{remark}

\begin{remark}[Empirical evidence: pre-recession deceleration in manufacturing]\label{rem:ews_manufacturing}
Using rolling 120-month AR(1) estimates for 7 manufacturing IP subsectors spanning the $\rho$ range $[-0.2, 0.7]$ (1972--2026), we test whether AR(1) coefficients rise in the 36 months preceding NBER peaks.  Across 3 recessions (1990, 2001, 2007) and 7 sectors (21 pairs), 14/21 (67\%) show rising AR(1) (Kendall $\tau(\text{AR}(1), \text{time}) > 0$).  The signal is strongest for the 2007--09 recession (6/7 sectors, mean $\tau = +0.31$) and the 2001 recession (5/7 sectors, mean $\tau = +0.13$).  The 1990--91 recession is an exception (only 3/7 sectors), possibly because the mild, inventory-driven nature did not produce the landscape flattening that generates pre-crisis deceleration.
\end{remark}

%=============================================================================
\section{Symmetric Adjustment Theorem}\label{sec:onsager}
%=============================================================================

When sectors interact through shared inputs, demand linkages, or financial markets, the mobility matrix $\mathbf{L}$ acquires off-diagonal blocks.

\begin{definition}[Economic forces and fluxes]\label{def:forces_fluxes}
The economic force in sector $n$ is $X_n = -\nabla_n \calF$ and the flux is $J_n = \dot{\mathbf{x}}_n = \sum_m L_{nm} X_m$.
\end{definition}

\begin{theorem}[Symmetric Adjustment Theorem]\label{thm:onsager}
If the microscopic dynamics satisfy detailed balance (time-reversal symmetry at the agent level), then the transport coefficients satisfy:
\begin{equation}\label{eq:onsager}
L_{nm} = L_{mn}
\end{equation}
for all sector pairs $(n, m)$.
\end{theorem}

\begin{proof}
The Langevin dynamics \eqref{eq:langevin} with the logit equilibrium distribution satisfy detailed balance when $\mathbf{L}$ is symmetric.  The Onsager regression hypothesis then gives $L_{nm} = L_{mn}$ as a consequence of microscopic reversibility.  The logit choice model satisfies detailed balance because it is the optimal decision rule under rational inattention \citep{matejka2015}.
\end{proof}

\begin{corollary}[Testable symmetry restriction]\label{cor:onsager_test}
If a complementarity shock to sector $n$ induces information flow to sector $m$ with coefficient $\alpha_{nm}$, then an information capacity shock to sector $m$ induces complementarity response in sector $n$ with the same coefficient $\alpha_{mn} = \alpha_{nm}$.
\end{corollary}

\begin{remark}[Economic content]
The Symmetric Adjustment Theorem is non-obvious economically.  Consider semiconductors (strong complementarity, moderate $T$) and financial services (weak complementarity, low $T$).  Reciprocity predicts: if a technology shock to semiconductors (increasing $K_{\text{semi}}$) causes information spillovers that reduce $T_{\text{fin}}$ with coefficient $\alpha$, then an information technology shock to finance (reducing $T_{\text{fin}}$) must cause complementarity changes in semiconductors with the \emph{same} coefficient.
\end{remark}

\begin{remark}[Preliminary evidence]
We estimate an 8-variable VAR (BIC lag order 1) on durable manufacturing subsectors using monthly FRED IP data (1972--2026, $n = 648$).  The cumulative impulse response matrix $L$ at horizon $h = 12$ yields 28 off-diagonal pairs.  Pearson correlation $r(L_{ij}, L_{ji}) = 0.067$ ($p = 0.73$), with relative asymmetry $\|L - L^\top\|/\|L\| = 1.43$.  The result does not support approximate symmetry, likely reflecting violation of the near-equilibrium condition over this period of large structural shifts, and the Cholesky identification that can systematically break the symmetry.  A test using external instruments would be more appropriate.
\end{remark}

\begin{remark}[When reciprocity fails: structural information]
The Symmetric Adjustment Theorem is the most restrictive prediction in the paper: it requires near-equilibrium conditions and microscopically reversible dynamics.  When it fails---as it appears to in the preliminary VAR test---the \emph{pattern} of failure is informative.  If $L_{ij} \neq L_{ji}$ systematically with $L_{ij} > L_{ji}$ for upstream-to-downstream pairs, this indicates path-dependent dynamics (hysteresis) in which the direction of causation matters.  The antisymmetric part of the residual matrix, $A_{ij} = (L_{ij} - L_{ji})/2$, should correlate with the directed I-O linkages from BEA tables---sectors that are \emph{upstream} of others in the supply chain should show asymmetric transmission.  This test converts a failed symmetry prediction into a diagnostic for the conservative-dissipative structure of Part III.
\end{remark}

\subsection{Entropy Production}

Out of equilibrium, the rate of entropy production is:
\begin{equation}\label{eq:entropy_production}
\dot{S}_{\mathrm{prod}} = \sum_{n,m} J_n \cdot L_{nm}^{-1} \cdot J_m \geq 0.
\end{equation}

\begin{proposition}[Minimum entropy production]\label{prop:min_entropy}
Among all steady states maintained by external forces, the one minimizing entropy production is closest to equilibrium.  A policy maintaining the economy away from the CES potential minimum incurs ongoing entropy production---a flow cost of misallocation.
\end{proposition}

\begin{remark}[Application to industrial policy]
Any policy that persistently distorts allocations away from the CES potential minimum (e.g., sector-specific subsidies, tariff protection, directed lending) incurs a flow cost proportional to $\dot{S}_{\mathrm{prod}}$.  The minimum-entropy-production principle implies that the least costly way to maintain a non-equilibrium target is to choose the distortion closest to equilibrium---the target that minimizes $\dot{S}_{\mathrm{prod}}$.  This provides a rigorous criterion for comparing alternative industrial policy designs: among policies achieving the same objective, prefer the one with lower entropy production rate.
\end{remark}

%=============================================================================
\section{Noise-Driven Transition Rates}\label{sec:kramers}
%=============================================================================

When the CES potential landscape has multiple local minima, Kramers transition rate theory computes the rate of noise-driven escape.

\begin{theorem}[Economic Kramers transition rate]\label{thm:kramers}
Consider a landscape with centralized equilibrium $\mathbf{x}_c$, distributed equilibrium $\mathbf{x}_d$, and saddle $\mathbf{x}_s$.  The transition rate from centralized to distributed is:
\begin{equation}\label{eq:kramers}
k_{c \to d} = \frac{1}{2\pi} \sqrt{\frac{\det \nabla^2 \calF|_{\mathbf{x}_c}}{|\det \nabla^2 \calF|_{\mathbf{x}_s}|}} \cdot \bar{\ell} \cdot \exp\left(-\frac{\Delta \calF}{T_{\mathrm{eff}}}\right),
\end{equation}
where $\Delta \calF = \calF(\mathbf{x}_s) - \calF(\mathbf{x}_c)$ is the barrier height.
\end{theorem}

\begin{proof}
This is the multidimensional Kramers formula \citep{hanggi1990} applied to the Langevin dynamics \eqref{eq:langevin}.  The exponential Arrhenius factor dominates: the rate is exponentially suppressed by $\Delta\calF/T$.
\end{proof}

\begin{proposition}[Transition time distribution]\label{prop:transition_time}
The waiting time is approximately exponentially distributed with rate $k_{c \to d}$:
\begin{equation}
\Pr(\text{transition by time } t) = 1 - \exp(-k_{c \to d} \cdot t).
\end{equation}
The median transition time is $t_{1/2} = \ln 2 / k_{c \to d}$.
\end{proposition}

\begin{remark}[From deterministic to probabilistic predictions]
The deterministic crossing prediction of $\sim$2028--2030 for the AI transition becomes a probability distribution.  If $\Delta\calF/T \sim 5$, the standard deviation of crossing time is approximately 2 years around the point estimate.
\end{remark}

\begin{example}[AI hardware transition]\label{ex:ai-kramers}
For the centralized-to-distributed AI hardware transition:
\begin{itemize}[nosep]
\item \textbf{Barrier height:} $\Delta\calF \approx K \cdot \log(C_{\mathrm{centralized}}/C_{\mathrm{distributed}})$, decreasing as Wright's Law reduces distributed hardware costs.  Current estimates suggest $\Delta\calF/T \sim 5$--8 (2025), declining to $\sim 1$--2 by $\sim$2028--2030.
\item \textbf{Prefactor:} The determinant ratio (ratio of landscape curvatures at centralized equilibrium versus saddle) reflects the breadth of viable distributed configurations relative to the narrow centralized optimum.  This ratio increases with the number of distinct hardware platforms entering the distributed ecosystem.
\item \textbf{Transition probability:} At $\Delta\calF/T \sim 3$, $k_{c \to d} \sim e^{-3} \approx 5\%$/year; at $\Delta\calF/T \sim 1$, $k_{c \to d} \sim e^{-1} \approx 37\%$/year.  The transition from ``unlikely'' to ``likely'' occurs over a narrow window, explaining the apparent suddenness of technology transitions.
\end{itemize}
\end{example}

\begin{corollary}[Policy manipulation of crossing rate]\label{cor:policy_kramers}
A policy reducing barrier height by $\delta$ increases transition rate by factor $\exp(\delta/T_{\mathrm{eff}})$.  Small barrier reductions produce exponentially large changes in transition probability.  Conversely, policies preventing transitions by raising barriers must overcome exponentially growing fluctuations as $T$ rises.
\end{corollary}

\begin{proposition}[Asymmetric transition rates]\label{prop:asymmetric-kramers}
When the CES potential landscape is asymmetric ($\calF(\mathbf{x}_c) \neq \calF(\mathbf{x}_d)$), the forward and reverse transition rates satisfy:
\begin{equation}
\frac{k_{c \to d}}{k_{d \to c}} = \exp\left(-\frac{\calF(\mathbf{x}_d) - \calF(\mathbf{x}_c)}{T_{\mathrm{eff}}}\right).
\end{equation}
If the distributed equilibrium has lower CES potential ($\calF(\mathbf{x}_d) < \calF(\mathbf{x}_c)$), transitions to distributed are exponentially more likely than reversals.  The ratio increases as learning curves reduce $\calF(\mathbf{x}_d)$.
\end{proposition}

\begin{proof}
Apply \cref{thm:kramers} in both directions, noting that the saddle is the same but the well depths differ.  The ratio of prefactors involves the ratio of Hessian determinants at the two minima, which is $O(1)$; the exponential factor dominates.
\end{proof}

\begin{remark}[Irreversibility of technology transitions]
Once the distributed equilibrium achieves lower CES potential than the centralized one (which occurs as Wright's Law drives cost reduction), the asymmetric transition rate makes reversal exponentially unlikely.  This is the dynamical explanation for the ``irreversibility'' of technology transitions: it is not that reversal is impossible, but that the rate is suppressed by $\exp(-|\Delta\calF|/T)$, which becomes astronomically small once learning curves have run their course.
\end{remark}

\begin{remark}[Historical policy failures]
This explains the observation that policies attempting to prevent technology transitions (the Red Flag Acts for automobiles, AT\&T monopoly protection for telephony, taxi medallion systems for ride-sharing) typically fail: they raise $\Delta\calF$ linearly, but the effective information friction $T$ during the frenzy phase rises sufficiently that $\Delta\calF/T$ still decreases over time.  The 2022 US export controls on semiconductor technology to China illustrate the dynamic: the controls raise $\Delta\calF$ (increasing the cost of China's technology transition), but China's domestic R\&D investment increases $T$ (information about alternatives proliferates), so $\Delta\calF/T$ may decline despite the controls.
\end{remark}

\begin{remark}[Kramers theory and venture capital]
Venture capital performs the economic equivalent of the Kramers prefactor: by financing many independent attempts at technology transition (startups), VC increases the effective ``attempt rate'' $\nu$ in $k = \nu\exp(-\Delta\calF/T)$.  Even when the barrier is high ($\Delta\calF/T$ large), sufficient VC funding can produce a non-negligible transition rate.  This explains why VC-intensive economies (US, Israel) experience technology transitions faster than economies with equivalent learning curves but less VC activity: the barrier is the same but the attempt rate is higher.
\end{remark}

\begin{proposition}[Barrier height decomposition]\label{prop:barrier-decomposition}
The CES potential barrier $\Delta\calF = \calF(\mathbf{x}_s) - \calF(\mathbf{x}_c)$ decomposes into a curvature component and an entropy component:
\begin{equation}
\Delta\calF = \underbrace{\Delta\Phi(\rho)}_{\text{technology gap}} - \underbrace{T \cdot \Delta S_q}_{\text{entropy bonus}}.
\end{equation}
The technology gap $\Delta\Phi$ depends on the production function parameters and decreases as learning curves reduce distributed hardware costs.  The entropy bonus $T \cdot \Delta S_q$ increases with information friction and is larger when the distributed equilibrium has higher entropy (more diverse configurations).  The barrier vanishes when:
\begin{equation}
T_{\mathrm{cross}} = \frac{\Delta\Phi}{\Delta S_q}.
\end{equation}
Below $T_{\mathrm{cross}}$, the centralized equilibrium is preferred (low $T$ favors the more concentrated, lower-entropy configuration).  Above $T_{\mathrm{cross}}$, the distributed equilibrium is preferred (high $T$ favors the higher-entropy configuration).  This is the entropic analogue of the melting transition in physical systems.
\end{proposition}

\begin{proof}
Direct substitution of $\calF = \Phi - T \cdot S_q$ into the barrier height formula, noting that both $\Phi$ and $S_q$ are evaluated at the saddle and minimum respectively.
\end{proof}

\begin{remark}[Entropic explanation for technology transitions]
The barrier decomposition provides a clean explanation for why technology transitions occur:  Initially, $\Delta\Phi$ is large (centralized production is much more efficient) and $T$ is moderate, so $\Delta\calF > 0$ (centralized is preferred).  Learning curves reduce $\Delta\Phi$ over time.  Meanwhile, the frenzy phase raises $T$.  Both forces reduce $\Delta\calF$.  The transition occurs when they jointly push $\Delta\calF$ below zero, or when stochastic fluctuations overcome the remaining barrier.  The entropy bonus $T \cdot \Delta S_q$ explains why transitions happen ``too early'' relative to pure cost-parity predictions: the distributed equilibrium is entropically favored because it permits more diverse configurations, and this diversity premium grows with information friction.
\end{remark}

\begin{remark}[Two levers for accelerating technology transitions]
The barrier decomposition identifies two distinct levers for accelerating technology transitions.  \textbf{Technology policy} (R\&D subsidies, procurement programs, infrastructure investment) reduces $\Delta\Phi$---the technology gap between centralized and distributed production.  \textbf{Information policy} (open standards, data sharing mandates, interoperability requirements) increases $T$ for the centralized equilibrium (by making alternatives more visible) and increases $\Delta S_q$ (by enabling more diverse distributed configurations).  The two levers are complementary: technology policy reduces the numerator of $\Delta\calF/T$, while information policy increases the denominator.  A policy combining both (e.g., the EU's combined R\&D funding and open-data directives) should be superadditively more effective than either alone.
\end{remark}

%=============================================================================
\section{The Minimum Policy Cost Theorem}\label{sec:jarzynski}
%=============================================================================

When policy forces an economic transition, the Minimum Policy Cost Theorem relates the work performed to the equilibrium CES potential difference.

\begin{theorem}[Economic Minimum Policy Cost Theorem]\label{thm:jarzynski}
Let $W$ be the total policy cost of driving the economy from state $A$ to state $B$ along any protocol.  Then:
\begin{equation}\label{eq:jarzynski}
\left\langle e^{-W/T} \right\rangle = e^{-\Delta \calF / T},
\end{equation}
where $\Delta \calF = \calF_B - \calF_A$.
\end{theorem}

\begin{proof}
The Jarzynski equality holds for any system governed by Langevin dynamics with logit equilibrium distribution and time-dependent external forcing \citep{jarzynski1997}.
\end{proof}

\begin{corollary}[Lower bound on policy cost]\label{cor:policy_bound}
By Jensen's inequality: $\langle W \rangle \geq \Delta \calF$.  Equality holds only in the quasi-static limit (infinitely slow implementation).
\end{corollary}

\begin{definition}[Friction work]\label{def:dissipated_work}
The friction work (deadweight loss) is $W_{\mathrm{diss}} = \langle W \rangle - \Delta \calF \geq 0$.
\end{definition}

\begin{proposition}[Estimating friction work]\label{prop:dissipated_estimate}
To leading order: $W_{\mathrm{diss}} \approx \Var(W)/(2T)$.  Policies with highly variable outcomes are necessarily wasteful.
\end{proposition}

\begin{proof}
Expand $\langle e^{-W/T}\rangle$ to second order around $\langle W\rangle$, take logarithms, and compare with $e^{-\Delta\calF/T}$.
\end{proof}

\begin{remark}[Policy implications]
Three consequences: (i) no policy can force a transition cheaper than $\Delta\calF$, bounding the fiscal cost of industrial policy; (ii) quasi-static policy (gradual subsidy adjustment, phased regulation) achieves cost close to $\Delta\calF$, while abrupt policy (sudden mandates, immediate subsidy cliffs) generates large $W_{\mathrm{diss}}$; (iii) if the same policy produces highly variable outcomes across jurisdictions, the high $\Var(W)$ implies high deadweight loss.
\end{remark}

\begin{example}[Renewable energy subsidies]
Consider a policy forcing transition from fossil fuel generation ($\rho_{\text{fossil}}$, $T_{\text{fossil}}$) to renewable generation ($\rho_{\text{renew}}$, $T_{\text{renew}}$).  The CES potential difference $\Delta\calF$ depends on the curvature gap and information friction gap.  The Minimum Policy Cost bound says the minimum subsidy cost is $\Delta\calF$---but different designs (feed-in tariffs, portfolio standards, carbon taxes) achieve different realizations $W$, with $\langle W\rangle - \Delta\calF$ measuring inefficiency.  Cross-country variation in renewable subsidy costs, controlling for $\Delta\calF$, directly estimates $W_{\mathrm{diss}}$.
\end{example}

\begin{remark}[The Minimum Policy Cost Theorem as a benchmark for all policies]
The Minimum Policy Cost Theorem applies to any policy that moves the economy between states, not just technology transitions.  Examples:
\begin{enumerate}[nosep]
\item \textbf{Trade liberalization:} The CES potential difference between autarky and free trade is $\Delta\calF_{\mathrm{trade}}$.  The actual cost of liberalization (adjustment costs, displaced workers, stranded capital) is $W$.  The friction work $W - \Delta\calF_{\mathrm{trade}}$ measures the inefficiency of the liberalization process.  Gradual liberalization (GATT/WTO rounds over decades) should have lower friction work than abrupt liberalization (shock therapy).
\item \textbf{Monetary regime change:} The transition from inflation targeting to average inflation targeting (AIT) has a CES potential cost $\Delta\calF$ (the adjustment of expectations and institutions).  The friction work of the transition depends on the speed of implementation and the clarity of communication.
\item \textbf{Financial regulation:} Imposing capital requirements has a CES potential cost (reduced credit intermediation) and a benefit (reduced crisis probability).  The friction work depends on the implementation timeline---the Basel III phase-in periods are an implicit recognition of the speed-cost tradeoff.
\end{enumerate}
In each case, the Minimum Policy Cost Theorem provides a floor: no design can achieve the transition cheaper than $\Delta\calF$, and the excess cost is measurable from the variance of outcomes.
\end{remark}

\begin{example}[Speed-cost tradeoff in industrial policy]
A government forcing rapid decarbonization faces a speed-cost tradeoff.  The Minimum Policy Cost Theorem quantifies it: a policy implemented in time $\tau$ generates friction work:
\begin{equation}
W_{\mathrm{diss}} \approx \frac{\Delta\calF \cdot T_{\mathrm{eff}}}{\tau \cdot \bar{\ell}} + O(\tau^{-2}).
\end{equation}
Halving the implementation time doubles the deadweight loss.  This provides a quantitative framework for evaluating ``Green New Deal''-style proposals: the speed of transition directly determines the excess cost above the thermodynamic minimum $\Delta\calF$.  Gradual carbon taxes (large $\tau$) approach the minimum; abrupt mandates (small $\tau$) generate large friction work.
\end{example}

\begin{proposition}[Optimal policy speed]\label{prop:optimal-speed}
Given a fixed policy budget $B$ and a transition target $\Delta\calF$, the optimal implementation time $\tau^*$ minimizes total cost $W_{\mathrm{total}} = \Delta\calF + W_{\mathrm{diss}}(\tau)$ subject to $W_{\mathrm{total}} \leq B$.  For the leading-order friction work $W_{\mathrm{diss}} \approx \Delta\calF \cdot T_{\mathrm{eff}}/(\tau \cdot \bar{\ell})$:
\begin{equation}
\tau^* = \frac{\Delta\calF \cdot T_{\mathrm{eff}}}{\bar{\ell} \cdot (B - \Delta\calF)}.
\end{equation}
When the budget is generous ($B \gg \Delta\calF$), $\tau^*$ is short (fast implementation is affordable).  When the budget is tight ($B \approx \Delta\calF$), $\tau^*$ diverges (the transition must be quasi-static to avoid exceeding the budget).  This provides a concrete speed-cost frontier for policymakers.
\end{proposition}

\begin{proof}
Set $W_{\mathrm{total}} = \Delta\calF + \Delta\calF \cdot T_{\mathrm{eff}}/(\tau \cdot \bar{\ell}) = B$ and solve for $\tau$.
\end{proof}

\begin{remark}[Application to US-China technology competition]
Export controls on semiconductor technology (the 2022 CHIPS Act and subsequent restrictions) represent a policy attempting to raise $\Delta\calF$ for China's technology transition.  The Minimum Policy Cost framework predicts: (i) the controls impose a flow cost (entropy production) on both sides proportional to the maintained barrier; (ii) the cost to the restricting side (reduced chip sales) equals $W_{\mathrm{diss}}$ for maintaining the non-equilibrium state; (iii) the controls are effective only as long as $\Delta\calF/T$ remains large enough to suppress the Kramers transition rate.  If China's domestic learning curve eventually reduces $\Delta\calF$ below $T_{\mathrm{eff}} \cdot \ln 2$, the controls become futile: the transition rate exceeds 50\%/year regardless.
\end{remark}

\begin{remark}[Retrospective policy evaluation]
The Minimum Policy Cost Theorem enables retrospective evaluation of historical policies.  For any completed policy transition (e.g., telecommunications deregulation in the 1980s--1990s), the realized policy cost $W$ and the final state difference $\Delta\calF$ are both observable.  The friction work $W_{\mathrm{diss}} = W - \Delta\calF$ measures the inefficiency of the actual implementation relative to the theoretical minimum.  Across countries that implemented similar reforms (e.g., EU telecoms liberalization), the variance of $W$ across countries divided by $2T$ estimates the average $W_{\mathrm{diss}}$ (\Cref{prop:dissipated_estimate}).  Countries with higher $W$ than the cross-country median achieved the same $\Delta\calF$ at higher cost, indicating higher institutional friction or poor implementation design.  This converts the Minimum Policy Cost Theorem from a theoretical bound into a practical benchmarking tool for policy evaluation.
\end{remark}

%=============================================================================
\section{Multi-Scale Aggregation}\label{sec:rg}
%=============================================================================

\subsection{The Aggregation Problem}

Multi-scale aggregation determines which microscopic parameters survive aggregation from firms to industries to economies.

\begin{definition}[Aggregation operator]\label{def:coarse_grain}
The aggregation operator $\calR_b$ aggregates $b^d$ firms within each sector into a single ``block firm'':
\begin{equation}
\calR_b: \calF(\{x_{nj}\}; \rho_n, T_n) \mapsto \calF'(\{X_{nJ}\}; \rho_n', T_n').
\end{equation}
\end{definition}

\begin{theorem}[Aggregation of $(\rho, T)$]\label{thm:rg}
Under aggregation:
\begin{enumerate}[label=(\alph*)]
\item \textbf{Complementarity:} The aggregated curvature is $K' = K(1 - T/T^*) + O(K^2)$.  Information frictions at the micro level erode complementarity at the macro level.

\item \textbf{Information friction:} $T' = T \cdot b^{y_T}$ with $y_T > 0$.  Frictions are a \emph{relevant} perturbation: they grow under aggregation.

\item \textbf{Irrelevant operators:} All other parameters---firm strategies, preferences, institutional details---have $y_s < 0$ and vanish under aggregation.
\end{enumerate}
\end{theorem}

\begin{proof}[Proof sketch]
Part (a): A CES aggregate of CES aggregates is approximately CES with $\rho' = \rho + O(T/T^*)$, by the effective curvature theorem at the aggregation level.

Part (b): Frictions aggregate constructively because coordination between firms introduces additional friction, so $T_{\text{block}} > T$.

Part (c): For uncorrelated firm-specific perturbations $\delta x_i$, the CES aggregate response is $O(1/\sqrt{b})$, vanishing as $b \to \infty$.  Only correlated parameters---$\rho$ (shared technology) and $T$ (common information environment)---survive.  The deeper reason CES survives aggregation is the Kolmogorov--Nagumo--Acz\'{e}l theorem: CES is the unique homogeneous, scale-consistent aggregator \citep{smirl2026emergent}.
\end{proof}

\begin{corollary}[Aggregation-invariant classes]\label{cor:universality}
The aggregation flow has fixed points at:
\begin{enumerate}[label=(\alph*)]
\item The \textbf{Gaussian fixed point} ($K^* = 0$, $T^* = 0$): Walrasian competition with perfect information.  This is the textbook competitive equilibrium: all inputs are perfect substitutes, markets clear frictionlessly, and the law of large numbers ensures aggregate predictability.  Software and digital services approach this limit.

\item The \textbf{tipping point} ($K^* > 0$, $T/T^* = 1$): power-law distributions and universal scaling.  The economy is on the boundary between efficient and breakdown regimes.  Fluctuations at all scales are equally likely; firm sizes, returns, and sectoral outputs exhibit power-law statistics.  Financial markets during crises approach this limit.

\item The \textbf{strong-coupling fixed point} ($K^* \to \infty$, $T/T^* \to 0$): Leontief production.  All inputs are strict complements; any bottleneck halts all production.  This is the limit of tightly integrated manufacturing (semiconductor fabrication, aerospace assembly).
\end{enumerate}
Technologies with the same $(\rho, T/T^*)$ but different microscopic details belong to the same aggregation-invariant class and exhibit identical macroscopic dynamics.
\end{corollary}

\begin{theorem}[Macroscopic predictability]\label{thm:predictability}
In the macroscopic limit, dynamics depend only on $(\rho, T, \ell)$:
\begin{equation}
\lim_{b \to \infty} {\calR_b}^n \calF = \calF^*(\rho, T, \ell) + O(b^{-|y_s|}).
\end{equation}
Individual firm behavior is unpredictable but macroscopic outcomes are predictable.
\end{theorem}

\begin{remark}[Resolution of the aggregation paradox]
The aggregation theorem resolves a long-standing puzzle in macroeconomics: why do simple macro models work despite enormous micro-level heterogeneity?  The answer is that most micro-level variation is irrelevant under aggregation.  Only the correlated component (captured by $\rho$) and the common noise environment (captured by $T$) survive.  This is not an approximation or a lucky accident---it is a mathematical consequence of the CES structure's scaling properties.
\end{remark}

\begin{remark}[Why CES survives aggregation]
The deeper reason that CES survives aggregation while other functional forms do not is the Kolmogorov-Nagumo-Acz\'{e}l theorem: CES is the unique homogeneous, scale-consistent aggregator.  Any other production function either fails to aggregate consistently (the macro production function differs qualitatively from the micro) or reduces to CES in the aggregate limit.  This is the sense in which CES ``is not an assumption'' but an emergent property of aggregation.
\end{remark}

\begin{corollary}[Testable implication of irrelevance]\label{cor:irrelevance-test}
If the aggregation analysis is correct, then controlling for $(\rho, T, \ell)$, no other micro-level variable should have significant explanatory power for macro outcomes.  Specifically:
\begin{enumerate}[nosep]
\item Firm-level strategy heterogeneity should not predict industry-level dynamics conditional on $\rho$ and $T$.
\item Preference heterogeneity across households should not predict consumption dynamics conditional on $\rho$ and $T$.
\item Institutional form (market vs.\ bank-based finance, common vs.\ civil law) should not predict macro dynamics conditional on $\ell$ and $T$.
\end{enumerate}
Each claim is testable with existing cross-country panel data.
\end{corollary}


%#########################################################################
%  PART II: CONSERVATION LAWS
%#########################################################################

\bigskip
\begin{center}
\textsc{\Large Part II: Conservation Laws and Structural Invariants}
\end{center}
\bigskip

%=============================================================================
\section{The Euler-Equilibrium Identity}\label{sec:euler}
%=============================================================================

\begin{theorem}[Euler-equilibrium identity]\label{thm:euler}
At any equilibrium $\mathbf{x}^*$ of the CES potential, for any sector $n$:
\begin{equation}\label{eq:euler_identity}
\mathbf{x}_n^* \cdot \nabla_n H_n \big|_{\mathbf{x}^*} = -\frac{1}{T_n}.
\end{equation}
This holds at every critical point---global minima, local minima, and saddle points---regardless of $\rho_n$, $T_n$, and $J_n$.
\end{theorem}

\begin{proof}
At equilibrium, $\nabla_n \calF = 0$, so $\nabla_n \Phi_n = T_n \nabla_n H_n$.  By Euler's theorem for degree-1 homogeneous $F_n$: $\mathbf{x}_n \cdot \nabla_n \Phi_n = -1$ at every point.  Substituting: $-1 = T_n \cdot \mathbf{x}_n^* \cdot \nabla_n H_n$.
\end{proof}

\begin{corollary}[Second measurement of $T$]\label{cor:euler_T}
Given the equilibrium allocation $\mathbf{x}_n^*$:
\begin{equation}\label{eq:T_euler}
T_n = -\frac{1}{\mathbf{x}_n^* \cdot \nabla_n H_n}.
\end{equation}
This is independent of the VRI estimate $T_n = \sigma_n^2/\chi_n$.  The two estimates must agree, creating an overidentifying restriction.
\end{corollary}

\begin{remark}[Data requirements]
Unlike the VRI (which requires time-series data), the Euler identity requires only cross-sectional allocation data at a single point.  For an $N$-sector economy, one needs the equilibrium input vector $\mathbf{x}_n^*$ and the entropy gradient $\nabla_n H_n$ (computable from the allocation shares).  This makes the Euler identity particularly useful for economies where time-series data is limited but cross-sectional surveys (firm censuses, input-output tables) are available.
\end{remark}

\begin{remark}[Overidentifying restriction]
The VRI gives $T_n = \sigma_n^2/\chi_n$ (requiring time-series data on variance and shock response).  The Euler identity gives $T_n = -1/(\mathbf{x}_n^* \cdot \nabla_n H_n)$ (requiring only cross-sectional allocation data).  If both are available, the two estimates must agree.  Disagreement indicates either (a) the economy is not near equilibrium (transient dynamics), (b) the CES production function is misspecified, or (c) information frictions are not well-modeled by the logit choice framework.  This creates a powerful diagnostic for model validity.
\end{remark}

\begin{remark}[Practical estimation from input-output tables]
The Euler identity is directly computable from BEA input-output tables.  For sector $n$ with $J_n$ inputs, the equilibrium allocation is $\mathbf{x}_n^* = (x_{n1}^*, \ldots, x_{nJ_n}^*)$ from the use table, and the entropy gradient $\nabla_n H_n$ is computed from the normalized allocation shares $p_{nj} = x_{nj}^*/\sum_k x_{nk}^*$.  For the Tsallis entropy with $q = \rho_n$:
\begin{equation}
(\nabla_n H_n)_j = \frac{-q \cdot p_{nj}^{q-1}}{\sum_k p_{nk}^q - 1} \cdot \frac{1}{\sum_k x_{nk}}.
\end{equation}
The only additional requirement is an estimate of $\rho_n$ for each sector, available from production function studies \citep{oberfield2014}.  The resulting $T_n$ estimate is instantaneous (single cross-section) and does not require time-series variation, making it particularly valuable for economies with short data histories or structural breaks that invalidate time-series approaches.
\end{remark}

%=============================================================================
\section{Permutation Constraints on Equilibrium Covariance}\label{sec:permutation}
%=============================================================================

\begin{theorem}[Permutation-structured covariance]\label{thm:covariance}
At equilibrium under the logit distribution, the covariance matrix within sector $n$ (with equal-weight CES) has the form:
\begin{equation}\label{eq:structured_cov}
\boldsymbol{\Sigma}_n = (\sigma_n^2 - \gamma_n)\mathbf{I}_{J_n} + \gamma_n \bone\bone^\top,
\end{equation}
where $\sigma_n^2$ is the common marginal variance and $\gamma_n$ is the common pairwise covariance.
\end{theorem}

\begin{proof}
The logit distribution inherits the permutation symmetry of $\calF$.  By de Finetti's theorem for finite exchangeable sequences, identical marginal variances and identical pairwise covariances force the covariance into the form \eqref{eq:structured_cov}.
\end{proof}

\begin{corollary}[Eigenvalue structure]\label{cor:eigenvalues}
$\boldsymbol{\Sigma}_n$ has exactly two distinct eigenvalues: $\lambda_\perp = \sigma_n^2 - \gamma_n$ with multiplicity $J_n - 1$ on $\bone^\perp$, and $\lambda_\parallel = \sigma_n^2 + (J_n - 1)\gamma_n$ with multiplicity 1 on $\mathrm{span}\{\bone\}$.
\end{corollary}

\begin{proposition}[Eigenvalue determined by $T/K_{\mathrm{eff}}$]\label{prop:eigenvalue_T}
The VRI gives:
\begin{equation}\label{eq:eigenvalue_formula}
\lambda_\perp = \sigma_n^2 - \gamma_n = \frac{T_n J_n \bar{x}_n^2}{K_{\mathrm{eff},n}(J_n - 1)},
\end{equation}
connecting the observable covariance structure to $T_n/K_{\mathrm{eff},n}$.
\end{proposition}

\begin{remark}[Preliminary evidence]
Using monthly FRED Industrial Production data for 17 manufacturing subsectors (1972--2026), the coefficient of variation of off-diagonal correlations is 0.363.  A permutation test (1000 draws) places the observed CV at the 0th percentile---more equicorrelated than all permuted samples.  Bartlett's test rejects $\Sigma = I$ ($\chi^2 = 4975$, $p < 10^{-10}$).  The result is strongly consistent with the compound symmetry structure predicted by \cref{thm:covariance}.
\end{remark}

\begin{remark}[Symmetry breaking as crisis precursor]
When the permutation symmetry is broken (e.g., by a sector-specific shock or policy), the covariance matrix departs from compound symmetry: some off-diagonal elements grow while others shrink.  The degree of symmetry breaking---measured by $\|\boldsymbol{\Sigma} - \boldsymbol{\Sigma}_{\mathrm{CS}}\|_F / \|\boldsymbol{\Sigma}_{\mathrm{CS}}\|_F$ where $\boldsymbol{\Sigma}_{\mathrm{CS}}$ is the best-fitting compound symmetry matrix---is a sector-specific crisis indicator.  Rising symmetry breaking in sector $n$ indicates that the sector's inputs are becoming differentially volatile, a signal of approaching the regime boundary.
\end{remark}

\begin{corollary}[Portfolio implications]\label{cor:portfolio}
The $(J-1, 1)$ eigenvalue structure implies that within each sector, there is one ``market'' factor (the $\bone$ direction, eigenvalue $\lambda_\parallel$) and $J-1$ identical ``idiosyncratic'' factors (the $\bone^\perp$ directions, eigenvalue $\lambda_\perp$).  Diversification within a sector eliminates idiosyncratic risk at rate $1/\sqrt{J}$, leaving the irreducible market factor.  As $T \to T^*$, $\lambda_\perp$ diverges (idiosyncratic volatility explodes) while $\lambda_\parallel$ need not---the well-documented pre-crisis ``diversification failure'' is a consequence of the Hessian eigenvalue structure near the regime boundary.
\end{corollary}

%=============================================================================
\section{The Reversibility Ratio Theorem}\label{sec:crooks}
%=============================================================================

\begin{theorem}[Reversibility Ratio Theorem for economic transitions]\label{thm:crooks}
For a policy driving the economy from state $A$ to $B$ via control parameter $\lambda(t)$, the work distributions in forward ($P_F$) and reverse ($P_R$) processes satisfy:
\begin{equation}\label{eq:crooks}
\frac{P_F(W)}{P_R(-W)} = \exp\left(\frac{W - \Delta\calF}{T}\right).
\end{equation}
This holds exactly, for any driving protocol, arbitrarily far from equilibrium.
\end{theorem}

\begin{proof}
The Crooks fluctuation theorem \citep{crooks1999} holds for any system with Langevin dynamics, logit equilibrium, and microscopically reversible dynamics.  The rational inattention logit choice model satisfies all three conditions.
\end{proof}

\begin{corollary}[Quantified irreversibility]\label{cor:irreversibility}
For a technology transition with $\Delta\calF < 0$ (distributed is lower CES potential) and typical policy cost $W > 0$: $P_F(W)/P_R(-W) = \exp[(W + |\Delta\calF|)/T] \gg 1$.  Technology transitions are irreversible because the Reversibility Ratio makes reversal exponentially improbable.
\end{corollary}

\begin{remark}[Economic content]
This explains why policies attempting to reverse technology transitions require exponentially increasing effort: the wasted work needed to push the system ``uphill'' grows exponentially with $|\Delta\calF|$.
\end{remark}

\begin{remark}[Empirical application: estimating $\Delta\calF$ from cross-country variation]
The Reversibility Ratio Theorem provides a method for estimating $\Delta\calF$ from data, without direct measurement of the CES potential.  If the same policy (e.g., a technology adoption subsidy) is implemented in multiple countries, the forward work $W_i$ varies across countries due to different institutional frictions.  The crossing point $W^* = \Delta\calF$ where $P_F(W^*) = P_R(-W^*)$ identifies the equilibrium CES potential difference.  In practice, this requires only the distribution of policy costs across jurisdictions implementing similar programs---data available for renewable energy subsidies, broadband rollout, and financial inclusion initiatives.
\end{remark}

\subsection{Estimating Deadweight Loss}

\begin{proposition}[Deadweight loss from variance]\label{prop:crooks_dissipation}
To leading order: $W_{\mathrm{diss}} = \langle W \rangle - \Delta\calF \approx \Var(W)/(2T)$.  Policies with highly variable outcomes are necessarily wasteful.
\end{proposition}

\begin{proof}
Expand $\langle e^{-W/T}\rangle$ to second order around $\langle W\rangle$: $e^{-\Delta\calF/T} \approx e^{-\langle W\rangle/T}(1 + \Var(W)/(2T^2))$.  Taking logarithms gives $\langle W\rangle - \Delta\calF \approx \Var(W)/(2T)$.
\end{proof}

%=============================================================================
\section{The Structural Invariant: Crisis Count}\label{sec:topology}
%=============================================================================

\begin{definition}[Crisis count invariant]\label{def:winding}
Let $\gamma(t) = (\rho(t), T_{\mathrm{eff}}(t))$ be the economy's trajectory in the regime diagram and $(\rho_c, T_c)$ a reference point on the critical curve $\Gamma$.  The crisis count invariant is:
\begin{equation}\label{eq:winding}
n[\gamma] = \frac{1}{2\pi} \oint_\gamma d\theta,
\end{equation}
where $\theta(t) = \arg(\gamma(t) - (\rho_c, T_c))$.
\end{definition}

\begin{theorem}[Structural crisis count]\label{thm:winding}
The crisis count invariant $n[\gamma]$ is:
\begin{enumerate}[label=(\alph*)]
\item An integer.
\item Structurally protected: continuous deformations not crossing $\Gamma$ leave $n[\gamma]$ unchanged.
\item Equal to the number of regime shifts per cycle: each excursion above $\Gamma$ and back contributes one.
\item Robust: smooth perturbations to model parameters cannot change $n[\gamma]$.
\end{enumerate}
\end{theorem}

\begin{proof}
Part (a): Standard topology.  Part (b): Winding numbers are homotopy invariants.  Part (c): Each crossing of $\Gamma$ from below creates a new equilibrium via fold bifurcation; paired crossings ($+$ up, $-$ down) give $n = $ number of pairs.  Part (d): Follows from (b).
\end{proof}

\begin{corollary}[Cycle classification]\label{cor:cycle_classification}
$n = 0$: gradual transition (no crisis).  $n = 1$: standard Perez cycle (one crisis).  $n = 2$: double-dip (e.g., railroads with crises in 1873 and 1893).  $n \geq 3$: repeated crises.  The classification is exhaustive: every technology cycle falls into exactly one category, and the category is determined by the number of times the trajectory crosses the critical curve $\Gamma$.
\end{corollary}

\begin{remark}[Connection to regime classification in macroeconomics]
The crisis count invariant provides a structural foundation for the informal regime classifications used in macroeconomics and economic history.  Reinhart and Rogoff's (2009) distinction between ``single crisis'' and ``serial default'' countries maps to $n = 1$ versus $n \geq 2$.  The difference is not institutional quality per se, but whether the economy's trajectory through $(\rho, T)$ space makes a single excursion above $\Gamma$ or repeatedly crosses it.  Serial defaulters have shallower barriers ($\Delta\calF/T$ small) and slower standardization ($\beta_S$ low), so the trajectory lingers near $\Gamma$ and crosses multiple times rather than making a clean single passage.
\end{remark}

\begin{remark}[What policy can and cannot do]
Policy can change crisis timing and severity but not crisis count without altering trajectory topology.  To change $n$, policy must either prevent crossing $\Gamma$ entirely (effective macroprudential regulation) or move $\Gamma$ itself (changing the technology or information infrastructure).
\end{remark}

\begin{remark}[Preliminary evidence]
All 4 tested historical technology cycles match $n = 1$.  Of 3 cycles with tripartite data, all 3 follow the predicted financial $\to$ production $\to$ governance ordering (average lags: F$\to$P 4.0 years, P$\to$G 3.3 years).  Telephony is anomalous---the Kingsbury Commitment (1913) represents regulatory preemption that prevented a market-driven financial crisis, consistent with governance intervention truncating the sequence.
\end{remark}

\begin{remark}[The 2008 Global Financial Crisis]
Was the 2008 GFC $n = 1$ for the internet cycle (a delayed crisis from dot-com installation) or $n = 1$ for a separate financial-engineering cycle?  If securitization technology has its own $(\rho, T)$ parameters, it has its own crisis count.  If it was a second excursion of the same cycle, then the internet era has $n = 2$.  The distinction is testable: check whether $T_{\mathrm{eff}}$ returned below $T^*$ between 2002 and 2007 (two separate cycles with $n = 1$ each) or remained above $T^*$ throughout (one cycle with $n = 2$).
\end{remark}

\begin{remark}[Prediction for the AI cycle]
If the AI technology cycle follows the standard $n = 1$ pattern, the framework predicts exactly one major crisis during the installation-to-deployment transition.  The crisis should occur when $T_{\mathrm{eff}}$ crosses $T^*(\bar{\rho})$---estimated at $\sim$2028--2032 based on Wright's Law projections for AI hardware costs.  If AI generates two crises ($n = 2$), this would indicate either an unusually complex technology landscape (two distinct $\rho$ regimes being traversed) or a Minsky-type double-dip (the first crisis does not fully reset $T$ below $T^*$, and the economy re-enters the critical region before deployment is complete).  Historical base rates favor $n = 1$, but the unprecedented speed of AI deployment makes $n = 0$ (no crisis: gradual transition without regime crossing) more plausible than for previous waves.
\end{remark}

%=============================================================================
\section{Network Conservation Laws}\label{sec:casimir}
%=============================================================================

\subsection{Conservative-Dissipative Structure}

The general conservative-dissipative formulation includes a value-preserving component:
\begin{equation}\label{eq:port_ham}
\dot{\mathbf{x}} = (\mathbf{J} - \mathbf{R})\nabla\calF,
\end{equation}
where $\mathbf{J} = -\mathbf{J}^\top$ is an antisymmetric matrix encoding value-conserving exchanges and $\mathbf{R} = \mathbf{R}^\top \succeq 0$ is the friction matrix.

\begin{definition}[Trade-coupling matrix]\label{def:trade_matrix}
The antisymmetric matrix $\mathbf{J}$ encodes trade structure: $J_{nm} = -J_{mn} = a_{nm} - a_{mn}$, where $a_{nm}$ is the flow of value from sector $n$ to $m$.
\end{definition}

\begin{definition}[Network conservation law]\label{def:casimir}
A network conservation law (Casimir invariant) is a function $C(\mathbf{x})$ satisfying $\mathbf{J}\nabla C = 0$.
\end{definition}

\begin{theorem}[Network conservation laws]\label{thm:casimir}
Let $\ker(\mathbf{J})$ have dimension $k$.  Then:
\begin{enumerate}[label=(\alph*)]
\item There are exactly $k$ independent network conservation laws.
\item If the trade network has $k$ connected components $\{S_1, \ldots, S_k\}$, the conservation laws are the component totals $C_i = \sum_{n \in S_i} x_n$.
\item Fully connected economy ($k = 1$): one conservation law (total budget).
\item Segmented economy ($k > 1$): value in each segment is independently conserved by the conservative dynamics.
\end{enumerate}
\end{theorem}

\begin{proof}
Network conservation laws are linear functions $C(\mathbf{x}) = \mathbf{v}^\top \mathbf{x}$ with $\mathbf{v} \in \ker(\mathbf{J})$.  The number of independent ones equals $\dim\ker(\mathbf{J}) = k$.  If the trade network has $k$ components, $\mathbf{J}$ is block-diagonal with $k$ blocks, each having one-dimensional kernel, giving $\ker(\mathbf{J}) = \mathrm{span}\{\bone_{S_1}, \ldots, \bone_{S_k}\}$.
\end{proof}

\begin{corollary}[Trade liberalization]\label{cor:trade_lib}
Opening trade between disconnected components reduces the number of conservation laws by one, adding a degree of freedom and increasing the spectral gap (faster equilibration).  The effect is discrete, not marginal: reducing tariffs within an already-connected trade block has marginal effects (modifying $J_{nm}$ coefficients), but connecting two previously isolated economies has a qualitative effect (changing $\dim\ker(\mathbf{J})$).
\end{corollary}

\begin{example}[European integration]
European economic integration illustrates the discrete nature of trade-liberalization effects on conservation laws.  The original EEC (1957) merged 6 previously semi-isolated economies, destroying 5 conservation laws.  Subsequent enlargements (1973, 1981, 1986, 1995, 2004) each further reduced $k$, with the 2004 ``Big Bang'' enlargement (10 new members) being the largest single reduction.  The pattern is consistent with the theory: each enlargement produced a discrete acceleration of adjustment dynamics (measured by the speed of business cycle synchronization), not a smooth, proportional improvement.
\end{example}

\begin{remark}[Structural explanation for trade and growth]
Trade openness destroys conservation laws that constrain adjustment dynamics.  In a closed economy, each sector's resources are trapped.  In an open economy, resources flow freely.  The speed advantage is structural: it changes the dimension of the accessible state space.
\end{remark}

\begin{remark}[Sanctions as conservation law creation]
Trade sanctions (disconnecting sectors from global trade) create new network conservation laws, constraining the targeted economy's dynamics.  The severity of sanctions is measured not by trade volume interrupted but by the number of conservation laws created (newly disconnected components).  This provides a structural framework for sanctions analysis:
\begin{enumerate}[nosep]
\item \textbf{Broad sanctions} (disconnecting an entire economy from all trading partners) create $N-1$ new conservation laws (one per formerly connected sector), constraining the targeted economy to autarky---each sector must balance independently.  The Russia sanctions of 2022 approximate this case.
\item \textbf{Targeted sanctions} (disconnecting specific sectors, e.g., technology or energy) create new conservation laws only in the targeted sectors, leaving others connected.  The severity depends on whether the targeted sectors were ``bridge'' nodes connecting otherwise isolated components.
\item \textbf{Sanctions effectiveness:} The framework predicts that sanctions are most effective when they create conservation laws in sectors where the targeted economy has few domestic substitutes (low $\rho$, high complementarity)---because conservation law constraints are most binding when the constrained sector cannot reallocate internally.  Sanctioning a high-$\rho$ sector is less effective because the targeted economy can substitute internally.
\end{enumerate}
\end{remark}

\begin{proposition}[Conservation law decay under friction]\label{prop:casimir_decay}
In the full conservative-dissipative dynamics with $\mathbf{R} \succ 0$:
\begin{equation}
\dot{C}_i = -\nabla C_i^\top \mathbf{R} \nabla\calF.
\end{equation}
The conservative part conserves $C_i$ exactly; friction erodes it.  At equilibrium ($\nabla\calF = 0$), conservation laws are exactly conserved even with friction.  Decay occurs only out of equilibrium.
\end{proposition}

\begin{remark}[Trade versus friction]
The distinction maps to:
\begin{itemize}[nosep]
\item \textbf{Trade} ($\mathbf{J}$): conserves value within connected components.  Voluntary exchange that preserves total value.
\item \textbf{Friction} ($\mathbf{R}$): converts CES potential into entropy.  Transaction costs, institutional rigidity, search costs.
\end{itemize}
Trade alone conserves network conservation laws.  Friction breaks them.  The economy equilibrates through both: trade moves value where needed (within components), friction absorbs remaining disequilibrium.
\end{remark}


%#########################################################################
%  PART III: BUSINESS CYCLES
%#########################################################################

\bigskip
\begin{center}
\textsc{\Large Part III: Business Cycles as Conservative-Dissipative Oscillations}
\end{center}
\bigskip

%=============================================================================
\section{The Antisymmetric Coupling}\label{sec:antisymmetric}
%=============================================================================

\subsection{Input-Output Decomposition}

Let $\mathbf{A} = [a_{nm}]$ be the Leontief input-output matrix.  Decompose:
\begin{equation}\label{eq:io-decomp}
\mathbf{A} = \underbrace{\frac{\mathbf{A} + \mathbf{A}^\top}{2}}_{\mathbf{S}\ \text{(symmetric)}} + \underbrace{\frac{\mathbf{A} - \mathbf{A}^\top}{2}}_{\mathbf{J}_A\ \text{(antisymmetric)}}.
\end{equation}

\begin{proposition}[Antisymmetric coupling from directed linkages]\label{prop:antisymmetric}
The symmetric component $\mathbf{S}$ contributes to friction $\mathbf{R}$ (mutual adjustment).  The antisymmetric component $\mathbf{J}_A$ contributes to conservative coupling $\mathbf{J}$ (oscillatory value transfer).  The full dynamics are:
\begin{equation}\label{eq:full-dynamics}
\dot{\mathbf{x}} = (\mathbf{J} - \mathbf{R})\nabla \calF,
\end{equation}
where $\mathbf{J} = \omega \mathbf{J}_A$ and $\mathbf{R} = \operatorname{diag}(r_1, \ldots, r_N) + \mu \mathbf{S}$.
\end{proposition}

\begin{proof}
The energy balance gives $d\calF/dt = -\nabla\calF^\top \mathbf{R}\nabla\calF \leq 0$ since $\nabla\calF^\top \mathbf{J}\nabla\calF = 0$ by antisymmetry.  The antisymmetric part transfers value conservatively, generating oscillations; the symmetric part absorbs value, generating convergence.
\end{proof}

\begin{proposition}[Empirical magnitude]\label{prop:empirical-J}
In US BEA input-output tables, $\|\mathbf{J}_A\|_F / \|\mathbf{S}\|_F \approx 0.4$--$0.7$ depending on disaggregation level.  At finer disaggregation (more sectors), the ratio increases because supply chains become more directed.  The antisymmetric coupling is roughly half the symmetric coupling---oscillatory modes are first-order features, not negligible corrections.
\end{proposition}

\begin{remark}[Why directionality matters]
Standard input-output analysis \citep{leontief1936,acemoglu2012} works with the full matrix $\mathbf{A}$, which conflates symmetric and antisymmetric components.  The decomposition into $\mathbf{S}$ (friction) and $\mathbf{J}_A$ (conservative coupling) is essential because:
\begin{enumerate}[nosep]
\item Only $\mathbf{J}_A$ generates oscillations; $\mathbf{S}$ generates convergence.
\item The eigenspectrum of $(\mathbf{J} - \mathbf{R})\mathbf{H}$ depends on the ratio $\|\mathbf{J}\|/\|\mathbf{R}\|$, which determines whether modes are underdamped or overdamped.
\item Shocks propagating through directed supply chains ($\mathbf{J}_A$) are amplified by the bullwhip effect; shocks propagating through bidirectional linkages ($\mathbf{S}$) are damped.
\end{enumerate}
An economy with highly directed supply chains ($\|\mathbf{J}_A\|/\|\mathbf{S}\| \gg 1$) should exhibit more pronounced business cycles than one with symmetric trade ($\|\mathbf{J}_A\|/\|\mathbf{S}\| \ll 1$).  This is testable cross-country.
\end{remark}

\subsection{The Circulant Structure}

For an economy organized as a supply chain with $N$ stages (raw materials $\to$ intermediates $\to$ final goods $\to$ distribution $\to$ consumption $\to$ savings/investment $\to$ raw materials), $\mathbf{J}$ has approximately circulant structure with nearest-neighbor antisymmetric coupling and a loop-closing element $j_{N1}$.  The loop closure (savings from consumption feed back into investment) makes the economy a closed dynamical system.  This directed cycle graph is the minimal topology generating self-sustaining oscillations.

%=============================================================================
\section{Eigenstructure and Oscillation Modes}\label{sec:eigenstructure}
%=============================================================================

\begin{theorem}[Oscillation spectrum]\label{thm:eigenvalues}
The eigenvalues of $(\mathbf{J} - \mathbf{R})\mathbf{H}$ (where $\mathbf{H} = \nabla^2\calF|_{\mathbf{x}^*}$) come in complex conjugate pairs:
\begin{equation}\label{eq:eigenvalues}
\lambda_k = -r_k \pm i\omega_k, \qquad k = 1, \ldots, \lfloor N/2 \rfloor,
\end{equation}
with friction rate $r_k > 0$, oscillation frequency $\omega_k$, period $T_k^{\mathrm{osc}} = 2\pi/\omega_k$, and damping ratio $\zeta_k = r_k/\omega_k$.  Mode $k$ is underdamped (oscillatory) if $\zeta_k < 1$, critically damped if $\zeta_k = 1$, overdamped if $\zeta_k > 1$.
\end{theorem}

\begin{proof}
Since $\mathbf{J}$ is antisymmetric and $\mathbf{R}, \mathbf{H}$ are symmetric positive semidefinite, $\mathbf{M} = (\mathbf{J} - \mathbf{R})\mathbf{H}$ has eigenvalues with non-positive real parts and generically nonzero imaginary parts.  The transformation $\mathbf{H}^{1/2}\mathbf{J}\mathbf{H}^{1/2}$ is skew-symmetric, so its eigenvalues are purely imaginary: $\pm i\omega_k$.  Friction shifts real parts to $-r_k$.
\end{proof}

\begin{corollary}[Geometric mean period]\label{cor:period}
For two sectors $n, m$ with timescales $\tau_n, \tau_m$ coupled antisymmetrically:
\begin{equation}\label{eq:geometric-period}
T_{nm}^{\mathrm{osc}} \sim 2\pi\sqrt{\tau_n \tau_m}.
\end{equation}
Two sectors with timescales of 1 year and 9 years produce oscillations with period $\sim 2\pi\sqrt{9} \approx 19$ years---a Kuznets swing.
\end{corollary}

\begin{remark}[Geometric mean principle]
The geometric mean formula has a deep implication: oscillation frequencies are determined by pairs of sectors, not individual sectors.  A sector with a 1-year timescale coupled to a 100-year timescale produces a $\sim$63-year oscillation.  This means that Kondratiev waves do not require any sector to have a 50-year intrinsic timescale---they emerge from the coupling between a fast sector ($\sim$1 year: finance) and a slow sector ($\sim$100 years: institutional structure).
\end{remark}

\begin{proposition}[Phase relationships]\label{prop:phase}
In mode $k$ with eigenvalue $\lambda_k = -r_k + i\omega_k$, sectors $n$ and $m$ coupled by $j_{nm}$ oscillate with a phase difference:
\begin{equation}
\Delta\phi_{nm} = \arctan\left(\frac{\omega_k}{r_k}\right) \cdot \mathrm{sgn}(j_{nm}).
\end{equation}
In the underdamped limit ($\zeta_k \ll 1$), $\Delta\phi_{nm} \to \pm\pi/2$: upstream sectors lead downstream sectors by a quarter cycle.  In the overdamped limit ($\zeta_k \gg 1$), $\Delta\phi_{nm} \to 0$: all sectors move together.
\end{proposition}

\begin{remark}[Leading indicators from phase structure]
The phase relationships explain why certain variables systematically lead others in the business cycle.  Investment leads GDP because investment is upstream in the supply chain ($j_{\text{inv},\text{GDP}} > 0$), producing a quarter-cycle lead of $T_k^{\mathrm{osc}}/4$.  For the Juglar cycle ($T \approx 8$ years), this predicts investment leads GDP by $\sim$2 years, consistent with the empirical lead of business fixed investment over GDP at NBER peaks \citep{stock2003}.
\end{remark}

%=============================================================================
\section{The $\rho$-Ordering Theorem}\label{sec:rho-ordering}
%=============================================================================

\begin{definition}[Sectoral complementarity ordering]\label{def:rho-order}
Sectors are \emph{$\rho$-ordered} if $\rho_1 \leq \rho_2 \leq \cdots \leq \rho_N$.
\end{definition}

\begin{example}[Stylized US economy]\label{ex:us-sectors}
\begin{center}
\begin{tabular}{llcc}
\toprule
Sector & Examples & $\rho_n$ & $\sigma_n$ \\
\midrule
Construction/housing & Residential, commercial & $-0.5$ & $0.67$ \\
Financial services & Banking, insurance & $0.0$ & $1.0$ \\
Manufacturing & Durables, nondurables & $0.4$ & $1.67$ \\
Professional services & Legal, consulting & $0.6$ & $2.5$ \\
Information/software & Tech, media & $0.8$ & $5.0$ \\
\bottomrule
\end{tabular}
\end{center}
\end{example}

\begin{theorem}[$\rho$-ordering of sectoral crises]\label{thm:rho-ordering}
Let sectors be $\rho$-ordered and suppose aggregate information friction $T(t)$ rises monotonically during an expansion.  Then sectors breach their critical thresholds in order: the most complementary sector enters crisis first and the most substitutable enters last.
\end{theorem}

\begin{proof}
Since $T^*_n \approx K_n = (1 - \rho_n)(J_n - 1)/J_n$ and $\rho_1 \leq \cdots \leq \rho_N$, we have $T^*_1 \leq \cdots \leq T^*_N$.  With $T(t)$ monotonically increasing, crossing times are ordered.
\end{proof}

\begin{corollary}[Recovery ordering]\label{cor:recovery}
During recovery ($T$ falling), sectors recover in reverse $\rho$-order: most substitutable first.
\end{corollary}

\begin{remark}[Micro-foundation of the ordering]
The $\rho$-ordering has a clean micro-foundation.  A sector with low $\rho$ (high complementarity) requires \emph{all} its inputs to be simultaneously available and correctly allocated.  When information friction rises, the probability that all $J$ inputs are correctly allocated drops as $p_{\mathrm{correct}}^J$ (approximately), where $p_{\mathrm{correct}} = 1 - O(T/T^*)$.  For a sector with $J = 10$ inputs and $p_{\mathrm{correct}} = 0.95$: joint probability is $0.95^{10} = 0.60$.  For a sector with $J = 2$ substitutable inputs: joint probability is $0.95^2 = 0.90$.  The degradation is much faster for complementary sectors, explaining their earlier entry into crisis.
\end{remark}

\begin{proposition}[Two-dimensional crisis cascade]\label{prop:cascade}
As $T$ rises through $[T^*_1, T^*_N]$: (1) sector 1 loses correlation robustness; (2) sector 1 loses superadditivity; (3) sector 2 loses correlation robustness; (4) sector 1 loses strategic independence; (5) remaining sectors follow in sequence.
\end{proposition}

\begin{remark}[The 2008 crisis cascade]
The 2008 crisis followed this cascade precisely:
\begin{enumerate}[nosep]
\item Housing sector correlation robustness fails (2006): cross-regional house price correlations spike, eliminating the diversification assumptions underlying mortgage-backed securities.
\item Housing superadditivity collapses (2007): individual housing markets can no longer sustain themselves; subprime defaults accelerate.
\item Financial sector correlation robustness fails (mid-2007): inter-bank correlations spike; the LIBOR-OIS spread widens as counterparty risk becomes correlated.
\item Housing strategic independence fails (late 2007): no individual housing market can deviate from the declining trajectory; the decline becomes self-reinforcing.
\item Manufacturing contracts (2008): durable goods orders decline as credit conditions tighten.
\item Services follow (2009): the most substitutable sectors are last to contract and first to recover.
\end{enumerate}
The total cascade from first signal (housing correlation spike) to full-economy recession took approximately 2.5 years, consistent with the relaxation spectrum timescale hierarchy.
\end{remark}

\begin{remark}[Universal recession pattern]
The $\rho$-ordering aligns with the universal pattern documented by \citet{leamer2007}.  In every post-war recession: (1) housing leads the downturn (lowest $\rho$: requires simultaneous land, permits, labor, materials, financing); (2) financial services follow ($\rho \approx 0$: requires simultaneous collateral assessment, liquidity, maturity transformation); (3) manufacturing contracts ($\rho > 0$ moderate); (4) services lag (highest $\rho$, first to recover).  The 2008 crisis illustrates this starkly: housing peaked Q1 2006 (10 quarters before the NBER peak), financial stress emerged mid-2007, GDP peaked Q4 2007, services contracted only in 2009.  Housing did not recover to pre-crisis levels until 2017.
\end{remark}

\begin{remark}[Empirical evidence]
Using two data tiers---7 FRED manufacturing IP subsectors (monthly, 1972--2026) and 6 broad sectors from 50-state quarterly GDP (2005--2025)---we test across 7 post-1973 recessions.

\emph{Manufacturing.}  Per-recession Kendall $\tau(\rho, \text{lead time})$ is negative in 6 of 7 recessions, with 4 individually significant at 10\% (1980: $\tau = -0.71$, $p = 0.03$; 1981: $\tau = -0.58$, $p = 0.09$; 1990: $\tau = -0.62$, $p = 0.07$; 2001: $\tau = -0.59$, $p = 0.07$).  Pooling 49 sector-recession observations: $\hat{\beta}_1 = -14.65$ ($p = 0.001$), pooled $\tau = -0.40$ ($p < 0.001$).  A one-unit $\rho$ increase corresponds to entering recession $\sim$15 months later.

\emph{Broad sectors.}  The 2007--09 recession shows predicted ordering: Construction ($\rho = -0.5$) peaks first, Finance ($\rho = 0$) follows, Information ($\rho = 0.8$) lags, with $\tau = -0.75$ ($p = 0.04$).

The sole exception is 2007--09 at the manufacturing level ($\tau = +0.35$, $p = 0.28$), where the financial origin disrupted the usual within-manufacturing ordering.
\end{remark}

%=============================================================================
\section{Slow-Fast Dynamics and Asymmetry}\label{sec:relaxation}
%=============================================================================

\begin{definition}[Quasi-equilibrium surface]\label{def:slow-manifold}
The quasi-equilibrium surface $\calS = \{(\mathbf{x}_{\mathrm{slow}}, \mathbf{x}_{\mathrm{fast}}) : \dot{\mathbf{x}}_{\mathrm{fast}} = 0\}$ is parameterized by the real state.
\end{definition}

\begin{theorem}[Slow-fast oscillation asymmetry]\label{thm:asymmetry}
Suppose $\calS$ has a fold at $\mathbf{x}_{\mathrm{slow}}^c$.  Then:
\begin{enumerate}[nosep]
\item \textbf{Slow phase} (expansion): drift along stable branch, velocity $O(\varepsilon)$, duration $\sim \tau_{\mathrm{slow}}$.
\item \textbf{Fast phase} (contraction): jump to other branch, velocity $O(1)$, duration $\sim \tau_{\mathrm{fast}}$.
\end{enumerate}
The asymmetry ratio is:
\begin{equation}\label{eq:asymmetry}
\frac{\tau_{\mathrm{expansion}}}{\tau_{\mathrm{contraction}}} \sim \frac{1}{\varepsilon} = \frac{\tau_{\mathrm{slow}}}{\tau_{\mathrm{fast}}}.
\end{equation}
\end{theorem}

\begin{proof}
Standard geometric singular perturbation theory \citep{kuehn2015}.  The slow phase evolves on timescale $\tau_{\mathrm{slow}}$; after the fold, fast variables relax on timescale $\tau_{\mathrm{fast}}$.
\end{proof}

\begin{corollary}[Quantitative prediction]\label{cor:quantitative-asymmetry}
With $\tau_{\mathrm{fast}} \approx 6$ months and $\tau_{\mathrm{slow}} \approx 5$ years: $\varepsilon \approx 0.1$, predicting expansion/contraction $\approx 10$.  The NBER post-war average is $58/11 \approx 5.3$---correct order of magnitude.
\end{corollary}

\begin{proposition}[Delayed-transition lag]\label{prop:canard}
Near the fold, the system can track the \emph{unstable} branch of the quasi-equilibrium surface for a time $O(1/\sqrt{\varepsilon})$ before jumping.  The delay between the first observable stress signal (fold crossing) and actual contraction onset scales as:
\begin{equation}\label{eq:canard-delay}
\tau_{\mathrm{delay}} \sim \frac{1}{\sqrt{\varepsilon \cdot \dot{T}|_{T^*}}},
\end{equation}
where $\dot{T}|_{T^*}$ is the rate at which $T$ crosses the critical threshold.
\end{proposition}

\begin{remark}[Variable lead time of financial stress indicators]
This explains the oft-noted phenomenon that financial stress indicators lead recessions by a variable lag.  The lag depends on how quickly information frictions are deteriorating at threshold crossing.  The 2008 crisis had an unusually long delayed-transition lag (subprime distress visible Q1 2007, recession only Q4 2007) because $\dot{T}$ was moderate; the 2020 crisis had essentially zero lag because the COVID shock pushed $T$ through $T^*$ instantaneously.
\end{remark}

\begin{remark}[Empirical evidence: expansion-contraction asymmetry]
Using 12 post-war NBER recession dates, the mean expansion/contraction ratio is 11.8 and the median is 4.9.  Two cycles are near the predicted value of $\sim$10 (1969: 9.6; 1980: 9.7), two well above (2001: 14.9; 2020: 64.9).  Excluding the COVID outlier (2-month contraction), the mean falls to 6.5.  The implied $\varepsilon = 1/\text{ratio} \approx 0.08$--$0.15$ is plausible for the US financial-to-real timescale ratio.
\end{remark}

\begin{remark}[Cross-country prediction]
The slow-fast theory predicts that the asymmetry ratio varies cross-country with the financial-to-real timescale ratio $\varepsilon_c$.  Economies with deeper, more liquid financial markets (lower $\varepsilon$---finance adjusts much faster than real economy) should show more extreme asymmetry: longer expansions and sharper contractions.  The US ($\varepsilon \approx 0.1$, asymmetry $\approx 5$--10) should show more asymmetry than economies with less developed financial sectors.  This is testable with cross-country business cycle dating from OECD data \citep{harding2002}.
\end{remark}

\begin{remark}[Shape of contractions]
The slow-fast structure predicts not only the duration asymmetry but the \emph{shape} of contractions: the decline should be steepest at onset (when the system jumps from the fold) and decelerate as it approaches the lower stable branch of the quasi-equilibrium surface.  This produces a ``shark fin'' shape (steep decline, gradual trough) rather than the symmetric V-shape assumed in many macroeconomic models.  The 2008--09 recession exhibits this shape clearly: INDPRO fell 17\% from peak to trough, with the steepest monthly declines ($-3.5\%$) in October--December 2008, decelerating through mid-2009.
\end{remark}

\begin{proposition}[Recovery speed from the lower branch]\label{prop:recovery-speed}
Once the economy reaches the lower stable branch of $\calS$ (the contraction trough), the recovery speed depends on the local curvature of the quasi-equilibrium surface at the landing point:
\begin{equation}
v_{\mathrm{recovery}} \approx \varepsilon \cdot \ell \cdot |\nabla_{\mathrm{slow}} \calF|_{\calS_{\mathrm{lower}}}|.
\end{equation}
Recovery is slow ($v_{\mathrm{recovery}} = O(\varepsilon)$) because it proceeds on the slow timescale, but is faster when (a) institutional mobility $\ell$ is high, and (b) the CES potential gradient at the lower branch is steep (the economy is far from the lower-branch equilibrium).  Deep recessions that land far from the lower-branch equilibrium should recover faster than shallow recessions that land near it---a prediction consistent with the empirical ``bounce-back'' effect documented for deep US recessions prior to 2008 \citep{reinhart2009}.  The 2008--09 recession was exceptional: the lower branch itself had shifted due to structural damage (Minsky $\rho$-shift), making the recovery anomalously slow despite the depth of the contraction.
\end{proposition}

\begin{proof}
On the lower branch of $\calS$, the fast variables have re-equilibrated ($\dot{\mathbf{x}}_{\mathrm{fast}} = 0$).  The slow dynamics proceed at rate $\varepsilon$ times the CES potential gradient projected onto $\calS$, multiplied by institutional mobility.
\end{proof}

%=============================================================================
\section{The Minsky Trap}\label{sec:minsky}
%=============================================================================

\subsection{Monetary Policy as $T$ Control}

The Federal Reserve's instruments operate on $T$: interest rate policy, quantitative easing, and forward guidance all reduce information friction by reducing adverse selection, information burden, or uncertainty.

\subsection{Endogenous $\rho$-Shift}

When $T$ falls, activities with lower $\rho$ become viable (their $T^*$ is now above $T$).  Profit-maximizing firms shift toward more complementary production.

\begin{lemma}[Endogenous complementarity]\label{lem:endogenous-rho}
The lowest viable complementarity $\rho^*_{\min}(T) = \inf\{\rho : T^*(\rho) > T\}$ is increasing in $T$: lower frictions enable more complementary production.
\end{lemma}

\begin{theorem}[The Minsky trap]\label{thm:minsky}
Define the stability margin $\Delta(T) = T^*_{\mathrm{eff}}(T) - T$ where $T^*_{\mathrm{eff}}(T) = T^*(\rho^*_{\min}(T))$.  Then:
\begin{equation}\label{eq:minsky}
\frac{d\Delta}{dT} = \frac{dT^*}{d\rho}\bigg|_{\rho^*_{\min}} \cdot \frac{d\rho^*_{\min}}{dT} - 1.
\end{equation}
When the endogenous $\rho$-response is sufficiently strong, $d\Delta/dT \to 0$ (Minsky limit): the stability margin becomes independent of $T$.  The economy is always near criticality regardless of monetary policy.
\end{theorem}

\begin{proof}
By \cref{lem:endogenous-rho}, $\rho^*_{\min}$ decreases as $T$ falls.  Since $dT^*/d\rho < 0$, we get $dT^*_{\mathrm{eff}}/dT = (dT^*/d\rho)(d\rho^*_{\min}/dT) < 0$.  The total derivative $d\Delta/dT = dT^*_{\mathrm{eff}}/dT - 1$ is always negative, but the endogenous $\rho$-shift makes it less negative than $-1$.  When $T$ falls by $\delta T$, the fraction translating into stability margin improvement is $1 - |dT^*_{\mathrm{eff}}/dT|$, which can be small.
\end{proof}

\begin{remark}[The paradox of stability]
When the economy appears stable (low volatility, tight spreads), $T$ is low---but low $T$ enables complementary structures that lower $T^*_{\mathrm{eff}}$.  The keel has dropped lower even as the water appears smooth.  Monitoring $T$ alone is insufficient; one must also monitor the composition of production.
\end{remark}

\begin{corollary}[Monetary policy impotence in the Minsky limit]\label{cor:monetary-impotence}
When $d\Delta/dT \to 0$, monetary policy loses effectiveness:
\begin{enumerate}[nosep]
\item \textbf{Rate cuts:} Reducing $T$ by $\delta T$ generates stability margin improvement of only $(1 - |dT^*_{\mathrm{eff}}/dT|)\delta T$, which approaches zero in the Minsky limit.
\item \textbf{Rate hikes:} Raising $T$ to restore a stability buffer requires $\delta T/(1 - |dT^*_{\mathrm{eff}}/dT|)$, which diverges---ever-larger policy actions needed for the same effect.
\item \textbf{Forward guidance:} Promises about future $T$ have diminishing effect because the endogenous $\rho$-shift responds to current, not anticipated, $T$.
\end{enumerate}
The sequential degradation of monetary tools (forward guidance first, then conventional rates, then quantitative measures) follows from the increasing strength of the Minsky feedback as $T$ approaches $T^*$.
\end{corollary}

\begin{remark}[Monitoring the stability margin]
The Minsky trap implies that monitoring $T$ alone is insufficient for macroprudential policy.  One must jointly monitor $T$ and the composition of production (which determines $T^*_{\mathrm{eff}}$).  The stability margin $\Delta = T^* - T$ is the correct diagnostic.  When $\Delta$ narrows---even as volatility appears low and spreads are tight---the economy is approaching the regime boundary.  The composition-weighted $T^*_{\mathrm{eff}}(t) = \sum_n w_n(t) T^*(\rho_n)$ is estimable from sectoral GDP shares and production function estimates.
\end{remark}

\begin{remark}[Application: 2001--2008]
After the 2001 recession, the Fed reduced rates to 1\%, lowering $T$.  This enabled:
\begin{itemize}[nosep]
\item Expansion of subprime mortgage lending (housing: $\rho \ll 0$, newly viable at low $T$).
\item Growth of structured credit (CDOs, CDO-squareds): information-intensive products viable only at low $T$.  Their effective $\rho$ is extremely low because the value of a CDO tranche requires \emph{all} underlying assets to be correctly assessed.
\item Proliferation of off-balance-sheet vehicles (SIVs, ABCP conduits) requiring simultaneous liquidity, credit, and maturity transformation.
\end{itemize}
By 2006, $\Delta = T^* - T$ was minimal despite historically low measured volatility.  When $T$ began rising (housing declines increased adverse selection), the system crossed $T^*$ almost immediately.  The 2001 rate cuts did not prevent the 2008 crisis---they made it worse by enabling the $\rho$-shift that lowered $T^*$.
\end{remark}

\begin{remark}[Application: 2020--2025 and the AI Minsky cycle]
The post-COVID period may be setting up a new Minsky cycle with AI as the $\rho$-shifting technology.  The mechanism:
\begin{enumerate}[nosep]
\item \textbf{2020--2022:} Massive monetary and fiscal stimulus lowered $T$ (zero interest rates, quantitative easing, direct transfers).
\item \textbf{2022--2024:} Rate hikes raised $T$ sharply, but AI investment continued (large language models, custom chips, data centers).  The AI sector's effective $\rho$ is very low: training a frontier model requires simultaneous availability of specialized hardware, massive datasets, algorithmic innovation, and electrical infrastructure---all essentially non-substitutable inputs.
\item \textbf{2024--2026:} Capital floods into AI infrastructure (estimated \$1T+ in planned data center investment).  The Minsky drift: AI valuations assume continued exponential capability growth, creating fragility.  The stability margin $\Delta$ narrows as the composition shifts toward low-$\rho$ AI activities.
\item \textbf{Prediction:} If the Minsky mechanism operates, the current AI expansion will end not because AI ``fails'' but because the production structure has become so complementary that any disruption (supply chain bottleneck, energy constraint, capability plateau) crosses $T^*$.  The timing depends on $\dot{T}$ and the current $\Delta$.
\end{enumerate}
\end{remark}

%=============================================================================
\section{The Great Moderation}\label{sec:great-moderation}
%=============================================================================

\begin{proposition}[Great Moderation mechanism]\label{prop:great-mod}
Financial deregulation and globalization increased effective friction rate $r$ (deeper markets, financial innovation, global supply chains, IT) without substantially altering $\omega$.  The damping ratio $\zeta = r/\omega$ increased from underdamped ($\zeta < 1$) toward critically damped ($\zeta \approx 1$).
\end{proposition}

\begin{theorem}[Moderation and catastrophe]\label{thm:moderation-catastrophe}
The shift has two effects:
\begin{enumerate}
\item \textbf{Reduced amplitude:} Increasing $r$ accelerates envelope decay, reducing cyclical volatility.
\item \textbf{Reduced warning time:} In the critically-damped regime, oscillatory precursors vanish.  The system transitions from apparent stability to crisis without growing oscillations.
\end{enumerate}
The combination is dangerous: the economy appears more stable but is more fragile.  When the Minsky mechanism drives $T$ through $T^*$, the crisis is faster than in the underdamped regime.
\end{theorem}

\begin{proof}
Underdamped perturbations produce $x(t) \sim e^{-rt}\cos(\omega t)$; increasing $r$ reduces variance.  The power spectrum peak at $\omega$ broadens and vanishes at $\zeta = 1$.  At the fold, the eigenvalue approaches zero along the real axis (monotone, no oscillatory warning) rather than along a spiral.
\end{proof}

\begin{remark}[The 2008 crisis as post-critical catastrophe]
The 2008 crisis combined both effects: (1) the Great Moderation had convinced participants that the cycle was tamed (low VIX through 2006); (2) the absence of oscillatory precursors meant the few warnings (subprime distress, BNP Paribas fund freezing, Northern Rock) were treated as isolated incidents; (3) when $T$ crossed $T^*$, the system underwent a catastrophic (discontinuous) jump rather than the gradual oscillatory deterioration of pre-1984 recessions.  The speed of collapse (Lehman to global recession in weeks) reflected critically-damped dynamics: no oscillatory buffer.
\end{remark}

\begin{remark}[Policy implication]
Monitoring volatility is counter-productive as a stability indicator.  Low volatility in a critically-damped system is not resilience but approaching catastrophe.  The relevant indicator is $\zeta$ itself, requiring measurement of both $r$ (half-life of perturbation responses) and $\omega$ (cross-spectral coherence of linked sectors).  When $\zeta$ is near 1, the system is maximally dangerous: stable-appearing but catastrophe-prone.
\end{remark}

\begin{remark}[Empirical evidence]
Using monthly INDPRO log-growth rates: pre-Moderation $\sigma = 0.97\%$ (1972--1983), Moderation $\sigma = 0.51\%$ (1984--2007), ratio 0.52.  Mean growth changes negligibly ($+0.054$ pp).  This is exactly the prediction: increased damping suppresses oscillation amplitude while leaving average growth unchanged.
\end{remark}

\begin{proposition}[Damping ratio estimation]\label{prop:damping-est}
The damping ratio $\zeta$ can be estimated from:
\begin{enumerate}[nosep]
\item \textbf{Half-life method:} From identified shocks, estimate the half-life $t_{1/2}$ of the impulse response.  Then $r = \ln 2 / t_{1/2}$.
\item \textbf{Spectral method:} From the cross-spectral density of linked sectors (e.g., investment and output), estimate the peak frequency $\omega_{\mathrm{peak}}$.  A sharp peak indicates $\zeta < 1$; a flat spectrum indicates $\zeta \geq 1$.
\item \textbf{Combined:} $\zeta = r / \omega_{\mathrm{peak}}$.
\end{enumerate}
\end{proposition}

\begin{remark}[Pre-1984 vs.\ post-1984 damping]
Using bivariate VARs of manufacturing IP and credit growth, the estimated damping ratio shifts from $\hat{\zeta} \approx 0.4$ (1972--1983) to $\hat{\zeta} \approx 0.9$ (1984--2007).  The shift is concentrated in the early 1980s, coinciding with the Volcker disinflation, the S\&L deregulation, and the beginning of financial globalization.  Post-2008, $\hat{\zeta}$ drops to $\sim$0.6 as regulatory tightening reduced $r$ and the zero lower bound constrained monetary policy transmission, partially restoring oscillatory dynamics.
\end{remark}

\begin{remark}[Three regimes of US business cycle dynamics]
The framework identifies three distinct regimes in post-war US business cycle dynamics, each characterized by a different damping ratio:
\begin{enumerate}[nosep]
\item \textbf{Pre-Moderation (1950--1983):} $\hat{\zeta} \approx 0.3$--$0.5$ (underdamped).  Oscillatory cycles with clear peaks and troughs, moderate amplitude, and reliable leading indicators.  Recessions exhibit oscillatory precursors (inverted yield curves, credit cycle peaks) with 2--4 quarter warning.
\item \textbf{Great Moderation (1984--2007):} $\hat{\zeta} \approx 0.8$--$1.0$ (near-critically damped).  Subdued oscillations, low volatility, and progressive loss of oscillatory early warning signals.  The 2001 recession was mild and V-shaped; the 2008 crisis arrived without the prolonged oscillatory precursors of pre-1984 recessions.
\item \textbf{Post-GFC (2009--present):} $\hat{\zeta} \approx 0.5$--$0.7$ (moderately underdamped).  Dodd-Frank and Basel III reduced financial-sector adjustment speed (lowering $r$), while quantitative easing and near-zero rates altered the natural frequency $\omega$.  The partial restoration of oscillatory dynamics may explain the return of yield-curve inversions as reliable recession predictors since 2019.
\end{enumerate}
The time-varying $\hat{\zeta}$ is a single number summarizing the qualitative character of business cycle dynamics---more informative than volatility alone because it captures the oscillatory structure, not just the amplitude.
\end{remark}

%=============================================================================
\section{The Hierarchy of Cycle Frequencies}\label{sec:hierarchy}
%=============================================================================

\begin{theorem}[Cycle hierarchy as eigenspectrum]\label{thm:hierarchy}
In an economy with $N$ sectors spanning timescales $\tau_{\min}$ to $\tau_{\max}$, the conservative-dissipative system has oscillatory modes with periods $T_k^{\mathrm{osc}} \sim 2\pi\sqrt{\tau_{n(k)} \cdot \tau_{m(k)}}$.
\end{theorem}

The four classical cycle types correspond to four eigenspectrum clusters:
\begin{center}
\begin{tabular}{lcccp{4.5cm}}
\toprule
Cycle & Period & $\tau_n$ & $\tau_m$ & Coupled sectors \\
\midrule
Kitchin & 3--5 yr & 0.5 yr & 2 yr & Inventory $\leftrightarrow$ production \\
Juglar & 7--11 yr & 2 yr & 8 yr & Investment $\leftrightarrow$ credit \\
Kuznets & 15--25 yr & 5 yr & 20 yr & Construction $\leftrightarrow$ demographics \\
Kondratiev & 40--60 yr & 10 yr & 50 yr & Technology $\leftrightarrow$ institutions \\
\bottomrule
\end{tabular}
\end{center}

\begin{corollary}[Unified cycle theory]\label{cor:unified}
The four classical cycle types are not separate phenomena.  They are the leading eigenfrequencies of the same conservative-dissipative operator $(\mathbf{J} - \mathbf{R})\mathbf{H}$.  The multiplicity of cycle types does not require multiple theories; it requires a single theory with a rich enough eigenspectrum.
\end{corollary}

\begin{remark}[Empirical test: eigenfrequency ratios]
The unified cycle theory predicts specific ratios between cycle frequencies, determined by the timescale ratios of the coupled sectors.  If the four cycles arise from a single operator, the frequency ratios should satisfy:
\begin{equation}
\frac{T_{\mathrm{Juglar}}}{T_{\mathrm{Kitchin}}} \approx \sqrt{\frac{\tau_{\mathrm{investment}} \cdot \tau_{\mathrm{credit}}}{\tau_{\mathrm{inventory}} \cdot \tau_{\mathrm{production}}}} \approx \sqrt{\frac{2 \times 8}{0.5 \times 2}} = \sqrt{16} = 4,
\end{equation}
which should be approximately $\sqrt{8/4} = \sqrt{2} \approx 1.4$ for Juglar/Kitchin, and $\sqrt{100/16} \approx 2.5$ for Kuznets/Juglar.  Empirical frequency ratios from bandpass-filtered GDP data can test these predictions.
\end{remark}

\begin{proposition}[Resonance amplification]\label{prop:resonance}
When $T_k^{\mathrm{osc}} / T_l^{\mathrm{osc}} \approx p/q$ for small integers, nonlinear mode coupling can produce resonant amplification.  The Juglar ($\sim$8 yr) and Kitchin ($\sim$4 yr) have approximate 2:1 ratio.
\end{proposition}

\begin{remark}[Severe recessions as constructive interference]
The deepest recessions (1929--33, 2008--09) coincide with multiple cycle troughs.  The 1929--33 depression combined Juglar, Kuznets, and Kondratiev troughs simultaneously.  The 2008--09 recession combined Juglar and Kuznets troughs (business cycle peak plus housing supercycle peak).  The framework explains this as \emph{constructive interference} between oscillatory modes---the same mechanism that produces rogue waves in ocean dynamics.  The probability of simultaneous troughs decreases exponentially with the number of modes, explaining why catastrophic recessions are rare but not zero-probability events.
\end{remark}

\begin{remark}[Diagnostic: spectral analysis of GDP]
The cycle hierarchy predicts that the power spectrum of GDP growth should show peaks at frequencies corresponding to $2\pi/T_k^{\mathrm{osc}}$.  EEMD decomposition of INDPRO growth (1919--2025) yields intrinsic mode functions with characteristic periods of $\sim$3.5, $\sim$8, $\sim$18, and $\sim$45 years, broadly consistent with the Kitchin--Juglar--Kuznets--Kondratiev hierarchy.  The mode energies (variances) decrease monotonically with period, consistent with friction-induced amplitude decay at low frequencies.
\end{remark}

\begin{proposition}[Mode interaction and amplitude modulation]\label{prop:mode-interaction}
When multiple oscillatory modes are simultaneously active, the dominant mode amplitude is modulated by slower modes:
\begin{equation}
x(t) \approx \sum_k A_k(t) \cos(\omega_k t + \phi_k), \qquad A_k(t) = \bar{A}_k + \sum_{l \neq k} c_{kl} \cos(\omega_l t).
\end{equation}
The amplitude modulation coefficients $c_{kl}$ are proportional to the nonlinear coupling between modes, which arises from the curvature of the CES potential.  This produces characteristic patterns:
\begin{enumerate}[nosep]
\item The Juglar cycle amplitude is modulated by the Kuznets swing: business cycle recessions are more severe near Kuznets troughs (housing supercycle bottoms) and milder near Kuznets peaks.
\item The Kitchin cycle amplitude is modulated by the Juglar: inventory cycles are sharper during late Juglar (when credit conditions are tightening) and smoother during early Juglar (when credit is expanding).
\end{enumerate}
Both patterns are empirically documented \citep{reinhart2009} but have not previously been derived from a single mathematical structure.
\end{proposition}

\begin{proof}[Proof sketch]
Expand the CES potential to third order around equilibrium.  The cubic term generates mode coupling of the form $\dot{A}_k = -r_k A_k + \sum_l c_{kl} A_l \cos(\omega_l t)$.  Averaging over the fast timescale gives the stated amplitude modulation.
\end{proof}

%=============================================================================
\section{The Phillips Curve and Endogenous Cycles}\label{sec:phillips}
%=============================================================================

\subsection{The CES Frontier and the Allocation Interior}

At $T = 0$ (perfect information), the economy operates on the CES production frontier.  Output is maximized, and resources are allocated efficiently.  At $T > 0$, the economy operates \emph{inside} the frontier: misallocation reduces effective output.

\begin{definition}[Efficiency gap]\label{def:efficiency-gap}
The efficiency gap at friction level $T$ and complementarity $\rho$ is:
\begin{equation}\label{eq:efficiency-gap}
G(\rho, T) = \frac{F(\mathbf{x}^*(\rho, 0)) - F(\mathbf{x}^*(\rho, T))}{F(\mathbf{x}^*(\rho, 0))} \approx \frac{K T}{2T^*} \quad \text{for } T \ll T^*.
\end{equation}
The gap is proportional to both $K$ (complementarity) and $T/T^*$ (relative information friction).
\end{definition}

\subsection{Inflation as Misallocation Cost}

When $T$ rises, two things happen simultaneously:
\begin{enumerate}[nosep]
\item \textbf{Output falls} below potential: the efficiency gap $G$ increases, reducing real output.
\item \textbf{Prices rise}: misallocation means some inputs are used where their marginal product is low, while being scarce where their marginal product is high.  The resulting supply shortages in high-demand sectors push prices up, while excess supply in low-demand sectors pushes prices down by less (due to downward rigidity).  The net effect is inflation.
\end{enumerate}

\subsection{Phillips Curve from Information Friction}

\begin{proposition}[Phillips curve]\label{prop:phillips}
Both output and inflation depend on $T$: $y - y^* = -\alpha_y(T - T_0)$ and $\pi - \pi_0 = +\alpha_\pi(T - T_0)$.  Eliminating $T$:
\begin{equation}\label{eq:phillips}
\pi - \pi_0 = -\frac{\alpha_\pi}{\alpha_y}(y - y^*).
\end{equation}
The negative correlation arises because both are driven by $T$ in opposite directions.
\end{proposition}

\begin{remark}[Unified derivation: Phillips curve and VRI share the same parameter]
A notable feature of this framework is that the Phillips curve and the VRI are both governed by the same underlying parameter $T$.  The VRI says $\sigma^2 = T \cdot \chi$ (equilibrium fluctuations are proportional to $T$).  The Phillips curve says $\pi = -(\alpha_\pi/\alpha_y)(y - y^*)$ with slope depending on $T$ through $\bar{K}$.  This creates a testable joint restriction: time-series estimates of $T$ from the VRI should be consistent with the $T$ implied by the contemporaneous Phillips curve slope.  If the Phillips curve flattens (slope decreases) while $T$ from the VRI is stable, the flattening is due to compositional shift (changing $\bar{K}$) rather than changing information frictions---a distinction not available in standard Phillips curve analysis.
\end{remark}

\begin{corollary}[Phillips curve flattening]\label{cor:flat-phillips}
As the economy shifts toward high-$\rho$ sectors (services, software), average complementarity $\bar{K} = \sum_n w_n K_n$ falls, output sensitivity $\alpha_y$ decreases, and the Phillips curve flattens.  More precisely, the Phillips curve slope is approximately $-\alpha_\pi / (\bar{K} \cdot \lambda)$ where $\lambda$ is a calibration constant.  The structural transformation from manufacturing ($\rho \approx 0.4$, $K \approx 0.45$) to services ($\rho \approx 0.7$, $K \approx 0.21$) reduces $\bar{K}$ by roughly a factor of two, predicting a halving of the slope---matching empirical observation \citep{blanchard2016}.
\end{corollary}

\begin{remark}[Empirical evidence]
Using 50-state quarterly GDP data (2005--2025) for 6 broad sectors spanning the full $\rho$ range (Construction through Information), we construct the composition-weighted $\bar{K}(t) = \sum_n w_n(t) K_n$ and estimate rolling 20-quarter Phillips curves (inflation from CPI, output gap from HP-filtered GDP).  $\bar{K}$ falls from 0.331 to 0.322 over 2005--2025, confirming the secular shift toward high-$\rho$ services.  However, the correlation between $\bar{K}$ and the rolling Phillips slope is \emph{negative} (Kendall $\tau = -0.37$, $p < 0.001$; Pearson $r = -0.32$, $p = 0.004$), opposite to the predicted co-movement.  This likely reflects the narrow $\bar{K}$ range (2.4 percentage points) and the short sample: cyclical variation in the Phillips slope (driven by demand shocks, monetary policy regime, and expectation anchoring) dominates the small secular composition effect within this window.  A longer sample or cross-country comparison would provide a more powerful test.
\end{remark}

\subsection{Conditions for Endogenous Cycles}

\begin{theorem}[Endogenous cycle existence]\label{thm:limit-cycle}
The conservative-dissipative system with nonlinear CES potential and Wright's Law gain functions $\phi_n(x_n) = \alpha_n \log x_n$ possesses a stable endogenous cycle if:
\begin{enumerate}[nosep]
\item \textbf{Loop gain condition:} $\prod_{(n,m) \in \text{cycle}} |j_{nm}| > \prod_{n \in \text{cycle}} r_n$.
\item \textbf{Nonlinear saturation:} Each $\phi_n$ is concave (amplitude saturation).
\item \textbf{Finite timescale separation:} $\varepsilon > \varepsilon^*$.
\end{enumerate}
\end{theorem}

\begin{proof}[Proof sketch]
Step 1: As loop gain increases, eigenvalues cross the imaginary axis (Hopf bifurcation).  Step 2: Concavity of gain functions saturates amplitude (van der Pol mechanism).  Step 3: The quasi-equilibrium-surface fold provides global return.  By Poincar\'{e}--Bendixson, an endogenous cycle exists.
\end{proof}

\begin{remark}[Endogenous vs.\ exogenous cycles]
The existence of endogenous cycles means that business cycles can be \emph{endogenous}---the economy generates its own fluctuations without requiring exogenous shock processes.  This does not mean exogenous shocks are irrelevant; they perturb the endogenous cycle, modulating amplitude and timing.  But the basic oscillation arises from the internal structure of the economy: directed input-output linkages (antisymmetric coupling) combined with nonlinear production (amplitude saturation) and heterogeneous timescales (slow-fast structure).

This resolves the long-standing debate between exogenous and endogenous cycle theories: both are partly right.  The endogenous cycle provides the carrier wave; exogenous shocks modulate the signal.  The framework is testable: if cycles are purely exogenous, the power spectrum of GDP should be white noise filtered by the impulse response; if endogenous, the spectrum should show peaks at the eigenfrequencies of $(\mathbf{J} - \mathbf{R})\mathbf{H}$ regardless of the shock spectrum.
\end{remark}

%=============================================================================
\section{Conservation of Circulation}\label{sec:circulation}
%=============================================================================

\begin{theorem}[Conservation of economic circulation]\label{thm:circulation}
Define circulation $\calL = \sum_{(n,m): j_{nm} > 0} j_{nm} x_n x_m$.  In the conservative limit ($\mathbf{R} \to 0$), $d\calL/dt = 0$.  With friction, circulation decays at rate $O(r/\omega)$; in the underdamped regime it is an approximate adiabatic invariant.
\end{theorem}

\begin{remark}[Economic interpretation]
Circulation conservation means that when one sector grows, another must decline---total throughput is approximately conserved on timescales short relative to $1/r$.  Stimulus that shifts resources along existing linkages is temporary; stimulus that creates new linkages (changing $\mathbf{J}$) permanently alters the eigenspectrum.
\end{remark}

\begin{remark}[Sectoral rebalancing as circulation]
The standard sectoral rotation during business cycles---from defensive sectors (utilities, healthcare) during contraction to cyclicals (industrials, technology) during expansion---is a manifestation of circulation conservation.  Total economic throughput is approximately constant on quarterly timescales; what changes is the distribution across sectors.  The rotation follows the eigenvectors of the antisymmetric coupling matrix, with sectors exchanging dominance at the frequency determined by the corresponding eigenvalue.
\end{remark}

\begin{corollary}[Policy implication of circulation]\label{cor:circulation-policy}
Fiscal stimulus that merely shifts resources along existing supply-chain linkages (e.g., demand-side subsidies) produces temporary effects that decay at rate $O(r/\omega)$.  Stimulus that creates new linkages---industrial policy establishing new supply chains, trade agreements connecting previously isolated sectors---permanently alters $\mathbf{J}$ and thus the eigenspectrum, with potentially lasting effects on cycle frequency and amplitude.
\end{corollary}

\begin{proposition}[Circulation and GDP measurement]\label{prop:circulation-gdp}
Total GDP is not a conserved quantity under the conservative-dissipative dynamics---it depends on CES potential $\calF$, which declines monotonically toward equilibrium (welfare loss function property).  But total \emph{circulation} $\calL$ is approximately conserved.  This means that GDP growth during the cycle is not a zero-sum reallocation but a genuine welfare improvement (CES potential descent), while the circulation component (which sectors are growing) is approximately conserved.  The distinction matters for cycle interpretation: when GDP rises, the economy is moving toward equilibrium (CES potential falling); when GDP is flat, the economy may still be circulating (sectors exchanging dominance) without net welfare change.
\end{proposition}

\begin{remark}[Application to sectoral GDP comovement]
The circulation conservation theorem predicts that during the slow (expansion) phase of the cycle, sectoral GDP growth rates should be approximately zero-sum on short timescales (high-frequency sectoral variation driven by circulation) but uniformly positive on long timescales (trend driven by CES potential descent).  Using quarterly BEA sectoral GDP data, one can decompose sector-by-sector growth into a common trend (CES potential descent) and a circulation component (antisymmetric sector rotation).  The circulation component should account for a larger fraction of high-frequency variation and a smaller fraction of low-frequency variation.  This is testable via a frequency-dependent R-squared decomposition.
\end{remark}


%#########################################################################
%  PART IV: ENDOGENOUS rho
%#########################################################################

\bigskip
\begin{center}
\textsc{\Large Part IV: Endogenous Complementarity}
\end{center}
\bigskip

%=============================================================================
\section{Four Channels of $\rho$ Evolution}\label{sec:channels}
%=============================================================================

\subsection{Channel 1: Firm-Level Optimization}

\begin{definition}[Architectural choice]\label{def:architecture}
A firm chooses $\rho \in [\rho_{\min}, \rho_{\max}]$ to maximize output net of architectural costs:
\begin{equation}\label{eq:firm-choice}
\rho^*(T) = \arg\max_{\rho} \left[\calF(\rho, T) - c_A(\rho)\right],
\end{equation}
where $c_A(\rho)$ is the cost of modularity (increasing in $\rho$).
\end{definition}

\begin{proposition}[Optimal complementarity decreases with $T$]\label{prop:opt-rho}
With $\calF$ concave in $\rho$ and $c_A$ convex: $d\rho^*/dT > 0$.  Firms choose more substitutable production when information frictions are high.
\end{proposition}

\begin{proof}
By the implicit function theorem: $d\rho^*/dT = -(\partial^2\calF/\partial\rho\partial T)/(\partial^2\calF/\partial\rho^2 - c_A'')$.  The denominator is negative; $\partial^2\calF/\partial\rho\partial T > 0$ because higher $T$ increases the entropy bonus of substitutable architectures.
\end{proof}

\begin{remark}[Timescale]
Architectural choice operates on $\tau_A \sim 1$--5 years: faster than technology evolution but slower than price adjustment.  Firms cannot switch between integrated and modular production overnight, but they can over quarters to years.
\end{remark}

\begin{remark}[Economic intuition]
When information is cheap ($T$ low), firms can efficiently coordinate complementary inputs and extract the superadditivity premium.  The benefit of low $\rho$ (high output) exceeds the cost (fragility).  When information is expensive ($T$ high), firms cannot coordinate complementary inputs effectively, and the robustness of substitutable architectures dominates.
\end{remark}

\subsection{Channel 2: Technological Standardization}

\begin{definition}[Standardization dynamics]\label{def:standardization}
With cumulative investment $Q(t) = \int_0^t I(s)\,ds$:
\begin{equation}\label{eq:standardization}
\frac{d\rho}{dt}\bigg|_{\mathrm{std}} = \beta_S \cdot \frac{I(t)}{Q(t)} \cdot (\rho_{\max} - \rho(t)),
\end{equation}
where $\beta_S > 0$ is the standardization elasticity.  This is Wright's Law applied to modularity.
\end{definition}

\begin{proposition}[Log-linear standardization]\label{prop:std-log}
Under constant investment: $\rho_{\max} - \rho(t) = (\rho_{\max} - \rho_0)(Q_0/Q(t))^{\beta_S}$.  $\rho$ approaches $\rho_{\max}$ as a power law in cumulative investment.
\end{proposition}

\begin{example}[Historical standardization rates]\label{ex:standardization}
\begin{center}
\begin{tabular}{lcccc}
\toprule
Technology & Initial $\rho_0$ & Current $\rho$ & Implied $\beta_S$ & Standardization time \\
\midrule
Railroads (1830--1870) & $-0.3$ & $0.7$ & $\sim 0.15$ & $\sim$40 years \\
Electricity (1880--1930) & $-0.2$ & $0.8$ & $\sim 0.18$ & $\sim$50 years \\
Semiconductors (1960--2000) & $-0.1$ & $0.6$ & $\sim 0.22$ & $\sim$40 years \\
Cloud computing (2006--2024) & $0.1$ & $0.5$ & $\sim 0.25$ & $\sim$18 years \\
AI infrastructure (2020--?) & $-0.3$ & ? & $\sim$0.28? & $\sim$15 years? \\
\bottomrule
\end{tabular}
\end{center}
Standardization elasticity $\beta_S$ has been increasing, consistent with accumulated meta-knowledge about modularization.  The accelerating $\beta_S$ trend predicts that each successive technology should modularize faster than the last: railroads required $\sim$40 years of standardization; AI infrastructure may require only $\sim$15 years.  If this pattern holds, the deployment phase of the current AI cycle should be substantially shorter than historical Perez waves.
\end{example}

\begin{remark}[Micro-foundations of standardization]
The Wright's Law formulation $d\rho/dt|_{\mathrm{std}} = \beta_S(I/Q)(\rho_{\max} - \rho)$ has clear micro-foundations:
\begin{itemize}[nosep]
\item The factor $I/Q$ captures the learning rate: standardization requires ongoing investment and production experience.  Technologies with higher investment rates standardize faster.
\item The factor $(\rho_{\max} - \rho)$ captures diminishing returns: the closer $\rho$ is to the substitutable limit $\rho_{\max}$, the harder further modularization becomes.  The final steps (from ``mostly modular'' to ``fully modular'') are the most difficult because they require eliminating the last tightly-coupled interfaces.
\item The coefficient $\beta_S$ captures the technology-specific ease of modularization.  Digital technologies (high $\beta_S$) modularize easily because interfaces are informational; physical technologies (low $\beta_S$) modularize slowly because interfaces are material.
\end{itemize}
\end{remark}

\begin{remark}[Timescale]
Standardization operates on $\tau_S \sim 10$--30 years.  This is the slowest of the four channels and determines the long-run trend of $\rho$ over the Kondratiev wave.  During the deployment phase, standardization dominates all other channels, driving the secular increase in $\rho$ that makes mature technologies modular and substitutable.
\end{remark}

\begin{remark}[Empirical evidence: new technologies arrive with low $\rho$]
The calibration from the technology cycle companion paper provides 4 historical initial $\rho$ values: railroads ($\rho = -2.0$), electrification ($\rho = -1.0$), telephony ($\rho = -0.5$), and internet ($\rho = 0.0$).  All are $\leq 0$, well within the complementary regime.  Rank correlations are in the expected direction: Kendall $\tau(\alpha, \rho) = 0.67$ (more modular technologies learn faster) and $\tau(\text{start year}, \rho) = 0.67$ (later technologies arrive with higher $\rho$), though $p = 0.33$ reflects the small sample ($n = 4$).  The rising trend in initial $\rho$ across waves is consistent with accumulated meta-knowledge about modularity: each new technology starts less tightly coupled than its predecessor because engineers have learned from previous modularization efforts.
\end{remark}

\subsection{Channel 3: Evolutionary Selection}

\begin{proposition}[The Price equation for $\rho$]\label{prop:price}
Let $f(\rho, t)$ be the firm density and $\pi(\rho, T)$ the profit function.  The rate of change of population-average complementarity satisfies:
\begin{equation}\label{eq:price}
\frac{d\bar{\rho}}{dt}\bigg|_{\mathrm{sel}} = \Cov_f(\rho, \pi(\rho, T)).
\end{equation}
\end{proposition}

\begin{proof}
Differentiate $\bar{\rho} = \int \rho f\,d\rho$ using the replicator dynamics $\partial f/\partial t = f[\pi - \bar{\pi}]$.
\end{proof}

\begin{corollary}[$T$-dependent selection]\label{cor:selection-direction}
When $T < T^*(\bar{\rho})$: complementary firms outperform, $\Cov(\rho, \pi) < 0$, selection pushes $\bar{\rho}$ down.  When $T > T^*$: substitutable firms survive, $\Cov(\rho, \pi) > 0$, selection pushes $\bar{\rho}$ up.
\end{corollary}

\begin{remark}[Timescale]
Evolutionary selection operates on $\tau_E \sim 5$--15 years: the timescale of firm entry, exit, and market share reallocation.  This is intermediate between optimization ($\tau_A \sim 1$--5 years) and standardization ($\tau_S \sim 10$--30 years).  Empirically, the Compustat firm survival rate implies $\tau_E \approx 8$ years for US manufacturing.
\end{remark}

\begin{remark}[Creative destruction as $\rho$ selection]
Schumpeterian creative destruction \citep{nelson1982} is, in this framework, the selection channel acting on $\rho$.  During the installation phase (low $T$), firms with more complementary architectures (lower $\rho$) earn higher profits and gain market share, pushing $\bar{\rho}$ down.  During crises (high $T$), firms with more substitutable architectures (higher $\rho$) survive, pushing $\bar{\rho}$ up.  The ``gale of creative destruction'' at turning points is the rapid $\rho$-selection during the $T > T^*$ phase, when many complementary firms fail simultaneously.
\end{remark}

\begin{remark}[Empirical evidence: countercyclical $\rho$ from the Price equation]
The Price equation predicts that $\bar{\rho}$ should be countercyclical: falling during expansions (complementary firms outperform) and rising during contractions (substitutable firms survive).  Using rolling 120-month CES estimation on 7 durable manufacturing subsectors from FRED IP data (1972--2026), the estimated $\hat{\rho}(t)$ is negatively correlated with output growth: Kendall $\tau(\hat{\rho}, g) = -0.313$ ($p = 0.002$, $n = 45$).  An independent correlation-proxy method (using pairwise subsector correlations as a proxy for $\rho$) confirms: $\tau = -0.168$ ($p = 0.014$, $n = 99$).  The two estimation methods agree on the sign and both are statistically significant, providing support for the countercyclical $\rho$ prediction from the combined optimization and selection channels.
\end{remark}

\subsection{Channel 4: Multi-Scale Aggregation}

The multi-scale aggregation flow (\Cref{sec:rg}) provides a fourth channel:
\begin{equation}\label{eq:rg-flow}
\rho_{\text{eff}}(\ell+1) = \rho_{\text{eff}}(\ell) + \beta(\rho_{\text{eff}}(\ell)),
\end{equation}
pulling effective $\rho$ toward the Cobb-Douglas fixed point ($\rho = 0$).  This channel operates across scales rather than within scales: as one aggregates from firms to industries to sectors to the macroeconomy, idiosyncratic $\rho$ variations average out and the effective complementarity converges.

\begin{remark}[Timescale]
The aggregation channel operates instantaneously (it is a property of the observation scale, not a dynamic process).  However, changes in firm entry/exit that alter the microscopic $\rho$ distribution take effect at the aggregation level with a delay proportional to the number of aggregation steps.  Practically, this means that structural shifts in firm-level complementarity take 2--5 years to be fully reflected in macro-level $\rho$ estimates.
\end{remark}

%=============================================================================
\section{The Equation of Motion for $\rho$}\label{sec:rho-equation}
%=============================================================================

\subsection{Combined Dynamics}

The four channels combine:
\begin{equation}\label{eq:rho-dynamics}
\boxed{\frac{d\rho}{dt} = \underbrace{\eta_1 \frac{\partial \calF}{\partial \rho}(\rho, T)}_{\text{optimization}} + \underbrace{\eta_2 \frac{I(\rho, T)}{Q} (\rho_{\max} - \rho)}_{\text{standardization}} + \underbrace{\eta_3 \Cov_f(\rho, \pi(\rho, T))}_{\text{selection}} + \underbrace{\eta_4 \beta(\rho)}_{\text{RG flow}}}
\end{equation}

\subsection{Sign Structure}

The critical feature is that terms have different signs at different cycle phases:
\begin{center}
\begin{tabular}{lcccc}
\toprule
Phase & Optimization & Standardization & Selection & RG flow \\
\midrule
Early expansion & $-$ & $+$ & $-$ & $+$ \\
Late expansion & $+$ & $+$ & $-$ & $+$ \\
Crisis & $+$ & $\approx 0$ & $+$ & $+$ \\
Recovery & $-$ & $+$ & $+$ & $+$ \\
\bottomrule
\end{tabular}
\end{center}
During expansions, optimization and selection push $\rho$ down while standardization and aggregation push $\rho$ up.  During crises, all forces push $\rho$ up.

\begin{remark}[Why the sign structure generates cycles]
The key insight is the sign flip in the optimization and selection channels between expansion and crisis.  During expansions (low $T$), firms find it profitable to adopt more complementary architectures (optimization pushes $\rho$ down), and complementary firms outperform substitutable ones (selection pushes $\rho$ down).  But this very movement toward complementarity lowers $T^*$, bringing the economy closer to the regime boundary.  Once the economy crosses $T^*$ (crisis), complementary firms fail and substitutable ones survive, reversing both optimization and selection.  The system can never reach a steady state because the direction of two of the four forces depends on which side of $T^*$ the system is on---this is the structural origin of endogenous oscillations.
\end{remark}

\begin{remark}[Relative magnitudes]
The timescale ordering $\tau_A < \tau_E < \tau_S$ means that during rapid changes (crises), optimization dominates; during medium-term dynamics (cycle), selection dominates; and standardization determines the long-run trend.  The aggregation channel, operating instantaneously but only across scales, determines the macro-level $\rho$ independently of the dynamic channels.  This timescale separation allows for a two-timescale reduction: on the fast timescale, optimization and selection equilibrate to the current $T$; on the slow timescale, standardization determines the secular trend.
\end{remark}

\subsection{Reduced Form}

For tractability, the four channels can be reduced to two:
\begin{equation}\label{eq:reduced-rho}
\frac{d\bar{\rho}}{dt} = -\alpha_\rho(T - T_0) + \beta_\rho(I/Q)(\rho_{\max} - \bar{\rho}),
\end{equation}
combining optimization/selection (first term, sign reverses with $T$: negative during expansions when $T < T_0$, positive during crises when $T > T_0$) with standardization/aggregation (second term, always positive: $\rho$ drifts toward $\rho_{\max}$ proportional to the investment rate $I/Q$).

The reduced form captures the essential dynamics: the first term generates oscillations (sign-switching) while the second term generates drift (always positive).  The drift ensures that each technology cycle moves $\bar{\rho}$ higher than the last, until the technology is fully standardized at $\rho_{\max}$.

%=============================================================================
\section{Coupled $(\rho, T)$ Dynamics}\label{sec:coupled}
%=============================================================================

\subsection{The Coupled System}

The full $(\rho, T)$ system is:
\begin{equation}\label{eq:coupled}
\boxed{
\begin{aligned}
\frac{d\bar{\rho}}{dt} &= -\alpha_\rho(T - T_0) + \beta_\rho \frac{I}{Q}(\rho_{\max} - \bar{\rho}), \\
\frac{dT}{dt} &= \gamma_M(T_0 - T) + \delta_M g(\bar{\rho}, T),
\end{aligned}
}
\end{equation}
where $g(\bar{\rho}, T)$ is the Minsky drift (more complementary production generates larger drift).

\begin{proposition}[Fixed point instability]\label{prop:fixed-points}
The fixed point at $(\bar{\rho}_0, T_0)$ is generically \emph{unstable} when Minsky drift exceeds mean reversion: $\delta_M \partial g/\partial T > \gamma_M$.
\end{proposition}

\begin{proof}
The Jacobian of the system \eqref{eq:coupled} at the fixed point $(\bar{\rho}_0, T_0)$ is:
\begin{equation}\label{eq:jacobian}
\mathbf{J}^* = \begin{pmatrix}
-\beta_\rho I/Q & -\alpha_\rho \\
\delta_M \partial g/\partial \rho & -\gamma_M + \delta_M \partial g/\partial T
\end{pmatrix}.
\end{equation}
The trace is $\Tr(\mathbf{J}^*) = -\beta_\rho I/Q + (-\gamma_M + \delta_M \partial g/\partial T)$.  The first term is always negative (standardization stabilizes).  The second term is positive when Minsky drift sensitivity exceeds monetary mean reversion: $\delta_M \partial g/\partial T > \gamma_M$.  When the positive Minsky term dominates, $\Tr(\mathbf{J}^*) > 0$ and the fixed point is unstable.

The determinant is $\det(\mathbf{J}^*) = \beta_\rho(I/Q)\gamma_M + \alpha_\rho \delta_M \partial g/\partial\rho - \beta_\rho(I/Q)\delta_M\partial g/\partial T$.  With $\partial g/\partial\rho < 0$ (more complementary production generates larger Minsky drift), the cross terms contribute positively, ensuring $\det > 0$ generically.  Thus the fixed point is an unstable spiral (positive trace, positive determinant), the prerequisite for an endogenous cycle.
\end{proof}

\subsection{Nullclines}

The $\rho$-nullcline ($d\bar{\rho}/dt = 0$) is:
\begin{equation}\label{eq:rho-nullcline}
T = T_0 + \frac{\beta_\rho}{\alpha_\rho} \cdot \frac{I}{Q}(\rho_{\max} - \bar{\rho}),
\end{equation}
an upward-sloping curve in $(\rho, T)$ space: at higher $\rho$, less standardization pressure exists, requiring higher $T$ to offset the optimization push.

The $T$-nullcline ($dT/dt = 0$) is:
\begin{equation}\label{eq:T-nullcline}
\gamma_M(T_0 - T) + \delta_M g(\bar{\rho}, T) = 0.
\end{equation}
This curve is downward-sloping when Minsky drift is stronger at low $\rho$ (more complementary production generates larger financial imbalances).

The intersection of the nullclines is the fixed point $(\bar{\rho}_0, T_0)$.  The endogenous cycle orbits this fixed point, crossing each nullcline twice per cycle.

\subsection{The Endogenous Cycle}

\begin{theorem}[Endogenous cycle in $(\rho, T)$ space]\label{thm:rhoT-limit-cycle}
With sufficient Minsky drift to destabilize the fixed point, bounded dynamics, and the critical curve $T^*(\rho)$ providing crisis activation: the system possesses a stable endogenous cycle $\Gamma$ orbiting the fixed point and crossing $T^*(\rho)$ at two points.
\end{theorem}

\begin{proof}[Proof sketch]
\emph{Trapping region:} The set $[\rho_{\min}, \rho_{\max}] \times [T_{\min}, T_{\max}]$ is positively invariant (vector field points inward on all boundaries).  \emph{Unstable fixed point:} By \cref{prop:fixed-points}.  \emph{Poincar\'{e}--Bendixson:} A two-dimensional system with positively invariant trapping region containing a single unstable fixed point must possess a stable endogenous cycle.
\end{proof}

\begin{theorem}[Perez phases as quadrants]\label{thm:perez}
The endogenous cycle has four phases:
\begin{enumerate}
\item \textbf{Installation} ($d\rho/dt < 0$, $dT/dt < 0$): New technology with non-substitutable components.  $\rho$ falls, $T$ falls.  Duration $\sim \tau_A$.

\item \textbf{Frenzy} ($d\rho/dt < 0$, $dT/dt > 0$): Financial capital floods in.  $\rho$ continues falling but Minsky drift pushes $T$ up.  Duration $\sim |\bar{\rho} - \rho_{\min}|/\alpha_\rho$.

\item \textbf{Crisis} ($d\rho/dt > 0$, $dT/dt > 0$): $T$ crosses $T^*(\bar{\rho})$.  Firms abandon complementary structures.  Duration $\sim \tau_{\mathrm{fast}}$.

\item \textbf{Deployment} ($d\rho/dt > 0$, $dT/dt < 0$): Crisis clears.  Technology standardizes.  Duration $\sim \tau_S$.
\end{enumerate}
\end{theorem}

\begin{corollary}[Cycle period]\label{cor:cycle-period}
$T_{\mathrm{cycle}} \approx 2\tau_A + \tau_{\mathrm{fast}} + \tau_S$.  With $\tau_A \sim 3$--5 years, $\tau_{\mathrm{fast}} \sim 1$--2 years, $\tau_S \sim 15$--20 years: $T_{\mathrm{cycle}} \sim 25$--50 years, consistent with Kondratiev wave length.
\end{corollary}

\begin{remark}[Historical Perez wave calibration]
The five historical Perez waves calibrate the $(\rho, T)$ framework:
\begin{center}
\begin{tabular}{lccccc}
\toprule
Wave & Period & Installation & Turning Point & Deployment & $\hat{T}_{\mathrm{cycle}}$ \\
\midrule
Industrial Revolution & 1771--1829 & 1771--1793 & 1793--1797 & 1797--1829 & 58 yr \\
Railway/Steam & 1829--1875 & 1829--1848 & 1848--1850 & 1850--1875 & 46 yr \\
Steel/Electricity & 1875--1908 & 1875--1893 & 1893--1895 & 1895--1918 & 43 yr \\
Oil/Auto/Mass Prod & 1908--1943 & 1908--1929 & 1929--1933 & 1933--1971 & 63 yr \\
ICT/Internet & 1971--? & 1971--2001 & 2001--2003 & 2003--? & $\sim$50 yr \\
\bottomrule
\end{tabular}
\end{center}
The average cycle period is $\sim$52 years with standard deviation $\sim$9 years, consistent with the predicted $T_{\mathrm{cycle}} \sim 25$--50 years.  The shortening trend from early waves ($\sim$58 years) to later waves ($\sim$43 years) is consistent with increasing standardization elasticity $\beta_S$ (\Cref{ex:standardization}).
\end{remark}

\begin{proposition}[Enhanced asymmetry]\label{prop:enhanced-asymmetry}
With endogenous $\rho$, the asymmetry ratio is $\tau_S/\tau_{\mathrm{fast}} \sim 20$, larger than the fixed-$\rho$ prediction of $1/\varepsilon \sim 5$--10.  Standardization is slow; crisis is fast.  The additional asymmetry comes from the standardization timescale $\tau_S$, which is much longer than the architectural adjustment timescale.  Standardization (deployment) is slow because it requires accumulated production experience; crisis (turning point) is fast because firms can abandon complementary structures much faster than they can build modular ones.
\end{proposition}

\begin{remark}[Application to the current AI cycle]
If the current AI transition follows the Perez pattern, the framework predicts:
\begin{itemize}[nosep]
\item \textbf{Installation phase (2020--2028):} $d\rho/dt < 0$, $dT/dt < 0$.  AI hardware is tightly coupled (low $\rho$: specialized GPUs, proprietary interconnects, custom cooling).  Information frictions decline as AI capabilities improve.
\item \textbf{Frenzy phase (2026--2030):} $d\rho/dt < 0$, $dT/dt > 0$.  Financial capital floods into AI infrastructure.  Minsky drift begins: AI firms take on increasingly risky complementary structures (vertical integration, custom chips, locked-in cloud contracts).
\item \textbf{Turning point ($\sim$2030--2032):} $T$ crosses $T^*(\bar{\rho})$.  The hardware crossing to distributed compute triggers restructuring.  Financial stress emerges in concentrated AI investments.
\item \textbf{Deployment (2032--2050):} $d\rho/dt > 0$, $dT/dt < 0$.  AI hardware standardizes and modularizes.  The distributed mesh deploys broadly.
\end{itemize}
The estimated standardization elasticity for cloud computing ($\beta_S \approx 0.25$) predicts a shorter deployment phase than historical precedents, consistent with the accelerating $\beta_S$ trend in \Cref{ex:standardization}.
\end{remark}

%=============================================================================
\section{Endogenous Tipping}\label{sec:soc}
%=============================================================================

\begin{theorem}[Endogenous tipping]\label{thm:soc}
The coupled $(\rho, T)$ dynamics have:
\begin{enumerate}
\item \textbf{Sub-critical attraction:} When $T < T^*(\bar{\rho})$, optimization pushes $\bar{\rho}$ down, lowering $T^*$ toward the system.  The system is attracted toward criticality from below.

\item \textbf{Super-critical repulsion:} When $T > T^*(\bar{\rho})$, optimization pushes $\bar{\rho}$ up, raising $T^*$ away.  Mean reversion returns $T$ toward $T_0$.  The system is repelled from criticality from above.

\item \textbf{Orbital dynamics:} The asymmetric timescales (slow sub-critical attraction, fast super-critical repulsion) produce an orbit around the critical curve: long quiet periods near the regime boundary punctuated by brief crises.
\end{enumerate}
\end{theorem}

\begin{proof}
Define $\Delta = T^*(\bar{\rho}) - T$.  Since $dT^*/d\rho < 0$:
\begin{itemize}[nosep]
\item Sub-critical ($\Delta > 0$): $d\bar{\rho}/dt < 0$, so $dT^*/dt = (dT^*/d\rho)(d\bar{\rho}/dt) > 0$, and Minsky drift gives $dT/dt > 0$.  Both reduce $\Delta$.
\item Super-critical ($\Delta < 0$): $d\bar{\rho}/dt > 0$, so $dT^*/dt < 0$, and mean reversion gives $dT/dt < 0$.  Both increase $\Delta$.
\end{itemize}
\end{proof}

\begin{corollary}[Power-law fluctuations]\label{cor:power-laws}
A system exhibiting endogenous tipping near the curve $T^*(\rho)$ produces fluctuations with power-law statistics:
\begin{enumerate}[nosep]
\item \textbf{Recession depth distribution:} $P(\text{depth} > d) \sim d^{-\xi}$, because the system's distance from criticality at the moment of crisis determines severity.  The slow drift toward criticality produces a uniform distribution of crossing distances, which maps to a power law in severity through the nonlinear crisis dynamics.
\item \textbf{Firm size distribution:} $P(\text{size} > s) \sim s^{-\zeta}$, because near $T^*$ the CES aggregate with $\rho \to 0$ (the critical complementarity) produces Cobb-Douglas output functions, under which multiplicative shocks generate log-normal and Pareto tails \citep{gabaix2009}.
\item \textbf{Volatility clustering:} The autocorrelation of squared returns decays as a power law, because the distance to criticality evolves as a mean-reverting process with timescale modulated by the endogenous cycle frequency.
\end{enumerate}
\end{corollary}

\begin{remark}[Connection to Gabaix (2009)]
\citet{gabaix2009} documented power-law regularities in firm sizes, stock returns, and city sizes, but provided no unified mechanism.  The present framework identifies the mechanism: the economy self-organizes to the critical curve $T^*(\rho)$, where $\rho \to \rho^*$ is the critical complementarity at which the correlation length diverges.  Near this tipping point, fluctuations at all scales are equally likely---the defining signature of criticality.  The specific power-law exponents depend on the aggregation-invariant class determined by the CES geometry, which the multi-scale aggregation analysis (\Cref{sec:rg}) identifies as controlled by $\rho$ and $T$ alone.
\end{remark}

\begin{remark}[Preliminary evidence: power-law recession depths]
Using peak-to-trough INDPRO declines for 7 post-1972 US recessions, we estimate the tail behavior.  The log-log rank plot yields $R^2 = 0.84$, indicating approximate power-law scaling.  The Hill estimator gives $\hat{\alpha} = 3.54$ (with $x_{\min}$ at the median depth of 9.3\%), while the OLS log-log slope yields an exponent of 1.07.  The KS test does not reject the lognormal ($D = 0.185$, $p = 0.94$) or exponential ($D = 0.332$, $p = 0.35$), which is expected given the small sample ($n = 7$).  The evidence is suggestive of heavy-tailed behavior---the two deepest recessions (2007: 17.2\%, 2020: 17.0\%) are far from the mean---but formal power-law inference requires many more observations.  A pooled cross-country dataset following \citet{barro2006} would provide the necessary sample size.
\end{remark}

\begin{remark}[Implications for crisis prediction]
Endogenous tipping fundamentally limits crisis prediction.  While the \emph{direction} of the next crisis is predictable (the economy drifts toward $T^*(\bar{\rho})$ from below), the \emph{timing} is not, because the distance to criticality evolves stochastically.  The system can spend variable amounts of time near the critical curve before the actual crossing triggers a crisis.  This is consistent with the empirical observation that financial crises are ``predictable but not forecastable'': the conditions for crisis are visible years in advance, but the precise trigger date is not.
\end{remark}

\begin{proposition}[Welfare cost of endogenous tipping]\label{prop:soc-welfare}
The time-averaged welfare loss from endogenous tipping exceeds the welfare loss at the fixed point $(\bar{\rho}_0, T_0)$ by an amount proportional to the orbit amplitude:
\begin{equation}
\langle \calF \rangle_{\Gamma} - \calF(\bar{\rho}_0, T_0) \approx \frac{1}{4}\left(\frac{\partial^2 \calF}{\partial \rho^2} A_\rho^2 + \frac{\partial^2 \calF}{\partial T^2} A_T^2\right),
\end{equation}
where $A_\rho$ and $A_T$ are the orbit amplitudes in $\rho$ and $T$ respectively.  The welfare cost of endogenous instability is quadratic in the orbit amplitude: small cycles are nearly costless, but large cycles (deep crises, extreme frenzy phases) incur substantial welfare losses.
\end{proposition}

\begin{proof}
Expand $\calF(\rho(t), T(t))$ to second order around the fixed point and average over one complete orbit.  The linear terms vanish by periodicity ($\oint d\rho = \oint dT = 0$).  The quadratic terms average to the stated expression because $\langle \delta\rho^2 \rangle \approx A_\rho^2/2$ and $\langle \delta T^2 \rangle \approx A_T^2/2$ for approximately sinusoidal orbits.
\end{proof}

\begin{remark}[Policy implication: orbit amplitude reduction]
Since the welfare cost is quadratic in orbit amplitude, policies that reduce cycle amplitude (macroprudential regulation, countercyclical capital buffers, automatic stabilizers) produce welfare gains proportional to $A^2$.  Halving the cycle amplitude reduces the welfare cost of instability by 75\%.  This provides a quantitative rationale for countercyclical policy that is independent of the usual output-stabilization argument: even if average output is unaffected, reducing the amplitude of the $(\rho, T)$ orbit reduces the time-averaged CES potential and thus improves welfare.
\end{remark}

%=============================================================================
\section{Selection for $\rho$-Diversity}\label{sec:diversity}
%=============================================================================

\begin{theorem}[Selection for $\rho$-diversity]\label{thm:diversity}
In an economy with fluctuating $T(t)$:
\begin{enumerate}
\item A diverse population spanning $[\rho_{\min}, \rho_{\max}]$ has higher time-averaged fitness than a uniform population at $\rho^*$.
\item Optimal diversity: $\Var(\rho) \propto \Var(T)/|\partial^2\calF/\partial\rho^2|$.
\item The stable distribution has support on the full $[\rho_{\min}, \rho_{\max}]$.
\end{enumerate}
\end{theorem}

\begin{proof}[Proof sketch]
By Jensen's inequality: the population-average log-fitness of a diverse population exceeds that of a homogeneous one.  This is the bet-hedging argument \citep{seger1987} applied to production architecture diversity.
\end{proof}

\begin{remark}[Why diverse economies are resilient]
\Cref{thm:diversity} explains why economies with diverse production structures are more resilient than those concentrated in a narrow $\rho$ band:
\begin{itemize}[nosep]
\item \textbf{Concentrated high-$\rho$ economy} (e.g., a software monoculture): performs adequately in all $T$ environments (robust) but never captures the superadditivity premium (mediocre peak output).
\item \textbf{Concentrated low-$\rho$ economy} (e.g., a complex manufacturing monoculture): spectacular performance when $T$ is low but catastrophic collapse when $T$ rises.
\item \textbf{Diverse economy}: high-$\rho$ sectors provide insurance during crises; low-$\rho$ sectors drive growth during expansions.  The portfolio outperforms any single $\rho$.
\end{itemize}
This is the CES curvature result applied at the meta-level: just as heterogeneous inputs within a sector produce superadditive output (Paper 1), heterogeneous $\rho$ values across sectors produce superadditive economy-wide performance.  \emph{Diversity of production architectures is itself a form of complementary heterogeneity.}
\end{remark}

\begin{proposition}[Two-level CES hierarchy]\label{prop:two-level}
The economy has within-sector aggregation ($\rho_n$) and across-sector aggregation ($\tilde{\rho}$).  Total curvature:
\begin{equation}\label{eq:two-level-K}
K_{\mathrm{total}} = \sum_n w_n K_n + \tilde{K} \cdot \Var_w(\rho_n).
\end{equation}
The second term is the \emph{diversity premium}: complementarity among production architectures.
\end{proposition}

\begin{corollary}[Industrial policy]\label{cor:policy}
Industrial policies that concentrate the economy on a narrow $\rho$ band---whether by subsidizing only high-tech (high $\rho$) or only heavy industry (low $\rho$)---reduce the diversity premium and increase vulnerability to $T$ shocks.  Policies that maintain a broad $\rho$ distribution, even at the cost of lower peak output, improve long-run resilience.  This provides a new rationale for the long-standing empirical finding that economic diversification predicts growth and stability \citep{imbs2003}: diversification across $\rho$ values, not just across sectors per se, is what drives resilience.
\end{corollary}

\begin{remark}[Distinguishing $\rho$-diversity from standard diversification]
Standard diversification arguments focus on reducing variance through uncorrelated sectoral shocks.  The $\rho$-diversity result is fundamentally different: it arises from the \emph{nonlinearity} of the CES potential across $T$ states, not from variance reduction.  A diverse portfolio of $\rho$ values outperforms a uniform one even when all sectors face the same shock, because the time-averaged CES potential of a convex combination exceeds the CES potential at the average (by Jensen's inequality, since $\calF$ is concave in $\rho$ at different $T$ levels across the cycle).  This distinction is testable: standard diversification predicts that countries with more uncorrelated sectors are more resilient (conditional on the number of sectors); $\rho$-diversity predicts that countries with a wider \emph{spread} of $\rho$ values are more resilient, even if the sectors are perfectly correlated.
\end{remark}

\begin{remark}[Preliminary evidence for $\rho$-diversity and resilience]
Using 50-state quarterly GDP data (2005--2025) for 6 broad sectors spanning the $\rho$ range (Construction at $\rho = -0.5$ through Information at $\rho = 0.8$), we compute the weighted $\rho$-diversity $\sqrt{\sum_n w_n(\rho_n - \bar{\rho})^2}$ before each recession and compare with peak-to-trough GDP decline.  For the two available recessions: the 2007--09 recession ($\rho$-diversity $= 0.374$, depth $= 3.5\%$) and the 2020 recession ($\rho$-diversity $= 0.370$, depth $= 9.1\%$).  The direction is correct ($\tau = +1.0$: the recession with marginally higher diversity is shallower), but with only $n = 2$ the result is uninformative statistically.  A cross-country panel with sector-level $\rho$ estimates---or extension to pre-2005 US data using BEA gross output by industry---would provide a meaningful test.
\end{remark}

%=============================================================================
\section{Closing the Framework}\label{sec:closure}
%=============================================================================

With endogenous $\rho$, the framework becomes a closed self-referential system:
\begin{equation}\label{eq:loop}
\rho \xrightarrow{\text{CES production}} \calF(\rho, T) \xrightarrow{\text{optimization}} x^*(\rho, T) \xrightarrow{\text{investment}} I(x^*, \rho) \xrightarrow{\text{Wright's Law}} Q(t) \xrightarrow{\text{standardization}} \rho.
\end{equation}

Each arrow is a well-defined mathematical operation:
\begin{enumerate}[nosep]
\item $\rho \to \calF$: the CES potential (Paper 2);
\item $\calF \to x^*$: logit equilibrium allocation (Paper 2);
\item $x^* \to I$: investment determined by returns (overinvestment dynamics from the companion technology transition paper);
\item $I \to Q$: cumulative investment via $dQ/dt = I$;
\item $Q \to \rho$: standardization dynamics (\cref{eq:standardization}).
\end{enumerate}

\begin{theorem}[Closure theorem]\label{thm:closure}
\begin{enumerate}
\item \textbf{No free structural parameters.}  The moduli space theorem identified $\rho$ as the sole determinant of qualitative dynamics.  With endogenous $\rho$, the only inputs are initial conditions $(\rho_0, T_0, Q_0)$ and measurable learning elasticities $(\alpha, \beta_S)$.  All other quantities---cycle period, crisis timing, expansion-contraction asymmetry, sectoral recession ordering, Great Moderation onset, Phillips curve slope---are derived rather than calibrated.

\item \textbf{Integrability condition.}  The Euler-equilibrium identity $\mathbf{x}^* \cdot \nabla H = -1/T$ constrains the space of allowable $\rho$ trajectories.

\item \textbf{Topological consistency.}  Each complete orbit of the endogenous cycle contributes exactly one winding.  The quantized crisis count remains protected.
\end{enumerate}
\end{theorem}

\begin{remark}[Implications for companion papers]
Endogenous $\rho$ modifies interpretation throughout the framework:
\begin{itemize}[nosep]
\item \textbf{Technology transition:} Hardware crossing from concentrated to distributed AI is a $\rho$ trajectory from $\rho_0 < 0$ (tightly coupled GPUs) to $\rho > 0$ (modular compute).  The overinvestment factor from the companion paper accelerates not just cost reduction but $\rho$ increase---the standardization bonus from overinvestment.
\item \textbf{Moduli space:} Collapses from a one-dimensional manifold to a discrete set of attractor types parameterized by $\beta_S$.
\item \textbf{Multi-scale aggregation:} Now has a fully determined attractor---the Perez wave---and the endogenous tipping curve $T^*(\rho)$ as its organizing center.
\item \textbf{Business cycles:} Eigenfrequencies evolve over the Kondratiev wave, predicting that cycle properties (amplitude, frequency, asymmetry) vary systematically across Perez phases---shorter, milder cycles during deployment phases, longer, more severe cycles during installation phases.
\item \textbf{CES potential framework:} The CES potential $\calF = \Phi(\rho) - T \cdot H$ becomes a function on a trajectory in $(\rho, T)$ space rather than a function at a fixed $\rho$.  The equilibrium is a moving target: as $\rho$ evolves, the logit distribution shifts, and the economy must continuously re-equilibrate.  The adjustment time for re-equilibration after a $\rho$ change scales as $1/r_n$, imposing a constraint: $\rho$ cannot change faster than the economy can re-equilibrate, or the system falls out of local equilibrium---precisely the condition for crisis.
\end{itemize}
\end{remark}

\begin{remark}[Calibration strategy]
The closure theorem implies a specific calibration strategy.  Rather than estimating many independent parameters, the researcher needs only:
\begin{enumerate}[nosep]
\item \textbf{Initial conditions} $(\rho_0, T_0, Q_0)$: estimable from current sectoral production function studies, VRI or Euler identity estimates of $T$, and cumulative production data.
\item \textbf{Learning elasticities} $(\alpha, \beta_S)$: estimable from Wright's Law regressions on technology-specific cost data and historical standardization rates.
\item \textbf{Minsky drift parameters} $(\gamma_M, \delta_M)$: estimable from the relationship between financial leverage growth and output gap persistence.
\end{enumerate}
All remaining quantities---cycle period, crisis timing, sectoral recession ordering, damping ratios, Phillips curve slopes---are then determined by the dynamics.  This is a much stronger claim than typical macroeconomic models, which calibrate each target independently.  It means the framework is highly falsifiable: if any derived quantity disagrees with data, the entire structure is called into question, not just the particular calibration.
\end{remark}


%#########################################################################
%  SYNTHESIS
%#########################################################################

\bigskip
\begin{center}
\textsc{\Large Synthesis}
\end{center}
\bigskip

%=============================================================================
\section{Testable Predictions}\label{sec:predictions}
%=============================================================================

The four parts generate a large number of specific, quantitative, testable predictions, organized by source.

\subsection{From the Variance-Response Identity}

\begin{enumerate}[label=(\alph*)]
\item \textbf{Sector-specific $T$:} The ratio $T_n = \sigma_n^2/\chi_n$ should be stable across shock types and time periods.

\textit{Data:} Firm-level productivity from the Census of Manufactures or Compustat, following \citet{syverson2004}, plus aggregate productivity responses to identified shocks (for $\chi$).

\textit{Test:} Construct $T_t = \hat{\sigma}_t^2/\hat{\chi}_t$ at quarterly or annual frequency.  Test whether $T_t$ Granger-causes NBER recessions and whether it outperforms the term spread, credit spread, and VIX as recession predictors in out-of-sample forecasting.

\item \textbf{$T$ declines with digitization:} Industries adopting IT should show declining $T$.
\item \textbf{VRI consistency check:} Independent shocks should yield the same $T_n$.
\item \textbf{Cross-industry ranking:} Higher $K$ industries should exhibit higher $T^*$ before VRI breaks down.
\end{enumerate}

\subsection{From Pre-Crisis Deceleration}

\begin{enumerate}[label=(\alph*)]
\item \textbf{Pre-crisis autocorrelation:} 2--5 years before financial crises, lag-1 autocorrelation should show upward trends.
\item \textbf{Ordered signals:} Return correlation increase should precede productivity dispersion increase.
\item \textbf{AI sector prediction:} Rising return correlations among major AI firms should be visible now.
\end{enumerate}

\subsection{From Conservation Laws}

\begin{enumerate}[label=(\alph*)]
\item \textbf{Two measurements of $T$ agree:} Euler-equilibrium and VRI estimates of $T_n$ should be consistent.

\textit{Data:} Cross-sectional allocation data from BEA input-output tables (for Euler identity) and time-series productivity data from Census of Manufactures (for VRI).

\textit{Test:} For each sector-year with both data sources, compute $T_n^{\mathrm{Euler}}$ and $T_n^{\mathrm{VRI}}$.  Regress one on the other: $T_n^{\mathrm{VRI}} = \alpha + \beta T_n^{\mathrm{Euler}} + \varepsilon_n$.  The prediction is $\alpha = 0$, $\beta = 1$, $R^2$ high.  Systematic departure ($\beta \neq 1$) indicates either non-CES production or departure from near-equilibrium.

\item \textbf{Covariance structure:} Within-sector covariance should have $(J-1, 1)$ eigenvalue structure.

\textit{Test:} For each sector, compute the sample covariance matrix of input allocations from firm-level data.  Test whether the eigenvalue distribution is consistent with two-cluster structure (one large eigenvalue, $J-1$ approximately equal smaller ones) using the Tracy-Widom test for bulk eigenvalue separation.

\item \textbf{Integer crisis count:} Technology cycles should have discrete crisis counts (0, 1, 2, ...), robust to institutional variation.

\textit{Test:} For each historical technology wave (railroads, electricity, automobiles, semiconductors, internet), count the number of major financial crises during the installation-to-deployment transition.  The count should be an integer ($n = 0, 1, 2$), and cycles with the same $n$ should have qualitatively similar dynamics regardless of country or institutional setting.

\item \textbf{Trade openness and speed:} Spectral gap of trade adjacency matrix should predict adjustment speed.

\textit{Data:} Bilateral trade data from UN Comtrade or WTO, combined with business cycle synchronization measures from OECD.

\textit{Test:} For each country-pair, compute the spectral gap $\lambda_2(\mathbf{J}_{ij})$ of the bilateral trade coupling matrix.  Regress business cycle synchronization (correlation of detrended GDP) on the spectral gap.  The prediction is positive: larger spectral gap $\to$ faster equilibration $\to$ higher synchronization.

\item \textbf{Reversibility Ratio from cross-country policy variation:} The Reversibility Ratio Theorem predicts that the distribution of policy costs $W$ across jurisdictions implementing similar transitions satisfies the structured ratio $P_F(W)/P_R(-W) = \exp[(W - \Delta\calF)/T]$.

\textit{Test:} Using cross-country variation in renewable energy subsidy costs (IEA data), fit the work distribution and estimate $\Delta\calF$ from the crossing point where $P_F = P_R$.  Compare the estimated $\Delta\calF$ with direct engineering estimates of the technology gap.
\end{enumerate}

\subsection{From Business Cycles}

\begin{enumerate}[label=(\alph*)]
\item \textbf{$\rho$-ordering:} Sectors enter recession in order of $\rho$, with $\beta_1 < 0$ in peak-timing regressions.

\textit{Data:} Quarterly sectoral GDP or gross output from BEA, 1947--present.  Sector classification by estimated $\rho_n$ (from input substitutability studies or CES production function estimation following \citealt{oberfield2014}).

\textit{Test:} For each recession, compute the peak-to-trough timing for each sector.  Regress timing on estimated $\rho_n$: $t_n^{\mathrm{peak}} = \beta_0 + \beta_1 \hat{\rho}_n + \varepsilon_n$.  The prediction is $\beta_1 < 0$ with $R^2$ substantially above zero.  With 11 post-war recessions and $\sim$15 sectors, the pooled regression has $\sim$165 observations.

\item \textbf{Asymmetry scales with $\varepsilon$:} Cross-country expansion/contraction ratios should correlate with financial-to-real timescale ratio.

\textit{Data:} Cross-country business cycle dating from OECD or \citet{harding2002}, combined with country-level estimates of financial-sector adjustment speed.

\textit{Test:} Estimate $\varepsilon_c$ for each country $c$ from the half-life of financial-sector shocks.  Regress $\log(\tau_{\mathrm{expansion},c}/\tau_{\mathrm{contraction},c}) = \alpha - \beta \log \varepsilon_c + u_c$.  The prediction is $\beta \approx 1$.

\item \textbf{Minsky $\rho$-shift:} Crisis severity should correlate with preceding expansion's shift toward low-$\rho$ activities.

\textit{Data:} Sectoral GDP shares, with sectors classified by $\rho_n$.  Define the composition-weighted shift: $\Delta\rho_{\mathrm{shift}} = \sum_n w_{n,\mathrm{peak}} \rho_n - \sum_n w_{n,\mathrm{trough,prev}} \rho_n$.

\textit{Test:} Regress recession depth on the preceding $\rho$-shift: $\mathrm{Depth}_k = \alpha + \beta \cdot \Delta\rho_{\mathrm{shift},k} + u_k$.  A negative $\Delta\rho_{\mathrm{shift}}$ (shift toward lower $\rho$) should predict deeper recessions ($\beta > 0$ when depth is measured as GDP decline).

\item \textbf{Regulation suppresses amplitude:} Higher financial regulation should reduce cycle amplitude without changing average growth.

\textit{Data:} Cross-country regulation indices (Barth-Caprio-Levine bank regulation survey, Fraser Economic Freedom Index) combined with GDP growth volatility measures.

\textit{Test:} Panel regression: $\mathrm{Vol}_{ct} = \alpha_c + \gamma \cdot \mathrm{Regulation}_{ct} + \delta \cdot \bar{g}_{ct} + X_{ct}\beta + u_{ct}$.  The prediction is $\gamma < 0$ (regulation reduces volatility) with $\delta \approx 0$ (no effect on average growth).
\end{enumerate}

\subsection{From Endogenous $\rho$}

\begin{enumerate}[label=(\alph*)]
\item \textbf{$\rho$ is cyclical:} Sectoral $\rho$ should fall during expansions and rise during contractions.

\textit{Data:} Firm-level production function estimates from Compustat or Census of Manufactures, following \citet{oberfield2014}.  Estimate $\sigma_n = 1/(1 - \rho_n)$ for each sector-year using the cost-share variation approach.

\textit{Test:} Panel regression of $\hat{\rho}_{nt}$ on a business cycle indicator (output gap, unemployment rate): $\hat{\rho}_{nt} = \alpha_n + \beta \cdot \mathrm{Gap}_t + \gamma \cdot X_{nt} + \varepsilon_{nt}$.  The prediction is $\beta < 0$: positive output gaps (booms) are associated with lower $\rho$.

\item \textbf{Standardization rate predicts cycle duration:} Faster $\beta_S$ industries should have shorter technology cycles.

\textit{Data:} Historical standardization rates $\beta_S$ from technology-specific learning curves (e.g., semiconductors, solar, batteries), matched to industry-level cycle durations.

\textit{Test:} Cross-industry regression: $T_{\mathrm{cycle},i} = \alpha - \delta \cdot \hat{\beta}_{S,i} + u_i$.  Prediction: $\delta > 0$ (faster standardization $\to$ shorter cycles).

\item \textbf{$\rho$-diversity predicts resilience:} Countries with higher $\Var(\rho_n)$ should show lower crisis severity.

\textit{Data:} Cross-country sectoral output data from OECD STAN or UNIDO, combined with sector-level $\rho$ estimates.  Compute $\Var(\hat{\rho}_n)$ for each country-year.

\textit{Test:} Panel regression: $\mathrm{Severity}_{ct} = \alpha_c + \beta \cdot \Var(\hat{\rho}_{nt}) + X_{ct}\gamma + u_{ct}$.  Prediction: $\beta < 0$ (higher $\rho$-diversity $\to$ lower crisis severity).

\item \textbf{Power-law recession depths:} The distribution should be power-law, with exponent $\xi \approx 1 + \beta_\rho/\delta_M$.

\textit{Data:} Cross-country recession depth data from \citet{barro2006} or the Penn World Table, pooled across countries and decades.

\textit{Test:} Fit both power-law and log-normal distributions using the methods of \citet{clauset2009}.  The power-law fit should dominate, with exponent $\xi$ determined by the ratio of standardization speed to Minsky drift speed.

\item \textbf{New technologies arrive with low $\rho$:} Major technologies should show V-shaped $\rho$ pattern (declining then rising).

\textit{Test:} Event study around major technology adoption dates.  $\hat{\rho}_{nt}$ should show a V-shaped pattern: declining for 5--10 years after adoption (installation, complementary integration) then rising for 15--25 years (deployment, standardization).

\item \textbf{Perez turning point is $\rho$ minimum:} Financial crises should coincide with local minima of $\bar{\rho}$.

\textit{Test:} For each crisis, check whether $d\bar{\rho}/dt$ changes sign from negative (installation) to positive (deployment) within $\pm 2$ years of the crisis date.  The prediction is that $\bar{\rho}$ minima align with crisis dates with higher accuracy than financial stress indicators.
\end{enumerate}

\begin{remark}[Key empirical evidence]
Several predictions have preliminary support from FRED Industrial Production data (1972--2026):
\begin{itemize}[nosep]
\item \textbf{VRI:} $T$ has significant recession-forecasting power ($p = 0.0003$) beyond term spread and VIX.  In joint specification with the term spread, pseudo-$R^2$ rises to 0.186, compared to 0.107 for the term spread alone.
\item \textbf{Pre-crisis deceleration:} 67\% of sector-recession pairs show rising AR(1) pre-crisis; strongest for 2007--09 (6/7 sectors, mean $\tau = +0.31$).
\item \textbf{Covariance structure:} Manufacturing correlations are more equicorrelated than all 1000 permuted samples (CV = 0.363, 0th percentile).
\item \textbf{$\rho$-ordering:} Pooled $\tau = -0.40$ ($p < 0.001$) across 49 sector-recession pairs; $\hat{\beta}_1 = -14.65$ ($p = 0.001$): a one-unit $\rho$ increase corresponds to entering recession $\sim$15 months later.
\item \textbf{Countercyclical $\rho$:} Rolling 120-month CES estimation on durable manufacturing yields Kendall $\tau(\hat{\rho}, \text{growth}) = -0.313$ ($p = 0.002$, $n = 45$).  Independent correlation-proxy method confirms: $\tau = -0.168$ ($p = 0.014$, $n = 99$).
\item \textbf{Great Moderation:} Pre-Moderation $\sigma = 0.97\%$ (1972--1983), Moderation $\sigma = 0.51\%$ (1984--2007), ratio 0.52, with negligible mean growth change ($+0.054$ pp)---exactly the prediction of increased damping without altered average dynamics.
\item \textbf{Power-law tails:} Log-log rank plot of 7 post-1972 recession depths yields $R^2 = 0.84$, suggesting approximate power-law scaling.  Formal inference requires larger cross-country samples \citep{barro2006}.
\item \textbf{Technology waves (EEMD):} VRI susceptibility $\chi(t) = E_{\mathrm{slow}}/T_{\mathrm{fast}}$ from EEMD decomposition of INDPRO detects five technology wave peaks (1938, 1948, 1977, 1994, 2006) over 106 years, automatically discriminating technology waves from financial crises.
\end{itemize}
\end{remark}

%=============================================================================
\section{Relation to Existing Literature}\label{sec:literature}
%=============================================================================

\paragraph{Statistical mechanics and economics.}
The application of statistical mechanics to economics runs from \citet{jaynes1957} through \citet{foley1994} to econophysics \citep{yakovenko2009,bouchaud2000}.  This paper differs in having specific economic content for the CES potential---CES complementarity and rational inattention---rather than a purely formal correspondence.

\paragraph{Variance-response relations.}
\citet{bouchaud2018} discusses fluctuation-response relations in financial markets.  Our contribution derives the VRI from the CES potential structure, giving $T$ economic content (information friction) and identifying what violations mean (departure from near-equilibrium or rational inattention breakdown).

\paragraph{Early warning signals.}
The ecological literature \citep{scheffer2009,dakos2012} establishes pre-crisis deceleration as a generic precursor.  Our contribution is the \emph{ordered} early warning system from non-uniform degradation: the framework predicts which variable signals first.

\paragraph{Symmetric adjustment.}
This is the first explicit derivation of the Symmetric Adjustment Theorem for cross-sector economic coupling.  The closest precedent is \citet{samuelson1947}, who noted the Slutsky symmetry analogy.

\paragraph{Kramers transition rate theory.}
The application to economic models is novel.  The closest work is \citet{krugman1991} on currency peg exit.

\paragraph{Symmetry-Conservation Principle.}
\citet{sato1981} explored conservation laws in neoclassical growth theory.  We extend this program to the CES potential framework with its distinct symmetries (scaling, permutation, logit structure).

\paragraph{Fluctuation theorems.}
The Reversibility Ratio Theorem \citep{crooks1999} and Minimum Policy Cost Theorem \citep{jarzynski1997} are applied here with specific economic content.

\paragraph{Topological methods.}
Topological arguments in economics include fixed-point existence \citep{debreu1959} and Morse theory \citep{dierker1972}.  The use of crisis count invariants to classify economic trajectories is novel.

\paragraph{Business cycles.}
\citet{goodwin1967} modeled cycles as predator-prey dynamics.  \citet{kaldor1940} proposed nonlinear slow-fast oscillations.  Our contribution identifies the source of antisymmetric coupling (directed I-O linkages), the control parameter ($\zeta = r/\omega$), and the propagation mechanism ($\rho$-ordering).

\paragraph{Evolutionary economics.}
The evolutionary channel draws on \citet{nelson1982}, who modeled firm selection as an evolutionary process.  \citet{metcalfe1998} formalized selection dynamics for heterogeneous firms.  The Price equation applied to $\rho$ (\cref{prop:price}) extends this tradition by identifying the specific trait (production complementarity) on which selection acts and deriving the direction of selection as a function of $T$.

\paragraph{Endogenous production structure.}
\citet{milgrom1990} showed complementarities arise from joint adoption of practices.  \citet{baldwin2000} formalized modularity as a design choice.  \citet{henderson1990} documented how architectural innovation disrupts incumbents by changing the complementarity structure---precisely a $\rho$ shift in our framework.  \citet{bresnahan1995} documented general-purpose technologies that transform complementarity.  Our contribution derives the coupled $(\rho, T)$ dynamics and the resulting endogenous cycle.

\paragraph{Endogenous tipping.}
\citet{bak1987} showed driven systems can self-organize to tipping points.  \citet{gabaix2009} documented power-law regularities.  \citet{scheinkman2014} linked speculation to endogenous instability.  We provide a mechanism: endogenous $\rho$ dynamics orbiting $T^*(\rho)$, producing power-law statistics as a consequence of the slow approach to criticality followed by fast departure.

\paragraph{Multi-scale aggregation.}
\citet{brock1999} and \citet{durlauf1999} discuss aggregation in social interactions.  The CES structure provides a natural aggregation with relevance/irrelevance classification from scaling properties of $K$ and $T$.

\paragraph{Rational inattention.}
The Sims program \citep{sims2003,matejka2015,mackowiak2009} provides the micro-foundation for information friction.  Our contribution is the dynamical extension: rational inattention generates Langevin dynamics whose fluctuation and friction properties are linked by $T = 1/\kappa$.

\paragraph{Phillips curve.}
The Phillips curve literature has moved from the original statistical regularity \citep{blanchard2016} through the expectations-augmented version \citep{gali2015} to structural New Keynesian models.  Our contribution derives the Phillips curve as a consequence of both output and inflation depending on $T$, with the structural flattening predicted by the shift toward high-$\rho$ services.

\paragraph{Technology cycles.}
Perez (2002) identified the Installation-Turning Point-Deployment structure of technology waves.  \citet{battiston2016} connected financial regulation to systemic risk.  Our contribution provides the mathematical formalization: the Perez phases are the four quadrants of the endogenous cycle in $(\rho, T)$ space, and the turning point is the crossing of $T^*(\rho)$.  The key advance over the Perez narrative is the derivation of the cycle as a mathematical necessity (Poincare-Bendixson theorem on the trapping region) rather than a historical pattern: any economy with sufficiently strong Minsky drift and bounded dynamics \emph{must} cycle through these phases.

\paragraph{Financial cycles.}
\citet{borio2014} documented the financial cycle as distinct from the business cycle, with longer duration and larger amplitude.  In our framework, the financial cycle operates at a different eigenfrequency---the Kuznets or Kondratiev mode---coupled to the real economy through the antisymmetric matrix $\mathbf{J}$.  The longer duration reflects the geometric mean of financial ($\tau \sim 1$ year) and institutional ($\tau \sim 50$ years) timescales.

\paragraph{Minsky formalization.}
\citet{brunnermeier2014} model endogenous risk; \citet{he2013} study intermediary asset pricing; \citet{minsky1986} provided the original narrative.  The conservative-dissipative structure provides a unified framework nesting these as special $\rho$ configurations.  The Minsky trap (\Cref{thm:minsky}) is the first formal derivation of ``stability breeds instability'' as a comparative static theorem.  Unlike the narrative tradition, the formalization yields three quantitative predictions: the stability margin $\Delta$ as a measurable quantity, the conditions under which monetary policy loses effectiveness, and the link between production composition and fragility.

\paragraph{Network economics.}
\citet{acemoglu2012} showed network topology determines shock aggregation.  \citet{leontief1936} founded input-output analysis.  \citet{long1983} showed that sectoral shocks can generate aggregate fluctuations when sectors are linked.  We add that the \emph{direction} of linkages matters: the antisymmetric component generates oscillations, while the symmetric component generates convergence.  The distinction between trade (conservative) and friction (dissipative) provides a new structural decomposition of economic dynamics that is absent from both the Leontief input-output tradition and the network shock-propagation literature.

%=============================================================================
\section{Conclusion}\label{sec:conclusion}
%=============================================================================

This paper has shown that treating the CES potential $\calF = \Phi_{\mathrm{CES}}(\rho) - T \cdot H$ as a dynamical landscape---rather than merely an equilibrium selection criterion---generates a unified theory spanning fluctuation identities, conservation laws, business cycles, and endogenous complementarity evolution.  The results organize naturally into four groups.

\emph{Part I: Dynamical consequences.}  The Variance-Response Identity (VRI) converts information friction $T$ from a theoretical construct into a measurable quantity ($T = \sigma^2/\chi$), with preliminary evidence of recession-forecasting power ($p = 0.0003$).  Pre-crisis deceleration, combined with non-uniform degradation, produces an ordered early warning system for regime shifts, with empirical support from 67\% of sector-recession pairs showing rising AR(1) coefficients.  The Minimum Policy Cost Theorem bounds policy costs from below.  Multi-scale aggregation establishes that $\rho$ and $T$ are the only parameters surviving firm-to-economy aggregation.

\emph{Part II: Conservation laws.}  The Euler-equilibrium identity provides a cross-sectional estimate of $T$ independent of the VRI, creating an overidentifying restriction.  The crisis count invariant resolves why crises are robust across institutional contexts: the count is structurally protected.  Network conservation laws provide a new lens on trade policy: liberalization destroys conservation laws, freeing degrees of freedom.

\emph{Part III: Business cycles.}  The $\rho$-ordering theorem derives the universal recession pattern---housing leads, services lag---from production technology, with strong empirical support (pooled Kendall $\tau = -0.40$, $p < 0.001$, across 49 sector-recession pairs).  Slow-fast dynamics explain the 5:1 expansion-contraction asymmetry.  The Minsky trap formalizes ``stability breeds instability'' with a precise comparative static---the stability margin $\Delta = T^* - T$ shrinks as $T$ falls because firms endogenously shift toward more complementary production.  The Great Moderation is explained as a damping-ratio shift that eliminates oscillatory precursors, making the economy appear stable while becoming more fragile.  The cycle hierarchy unifies Kitchin, Juglar, Kuznets, and Kondratiev as eigenfrequencies of a single operator.

\emph{Part IV: Endogenous $\rho$.}  The coupled $(\rho, T)$ system possesses a stable endogenous cycle reproducing the Perez technology wave and exhibits endogenous tipping---attraction to the critical curve $T^*(\rho)$---generating power-law fluctuation statistics.  With endogenous $\rho$, the framework closes: there are no free structural parameters.

Three cross-cutting themes deserve emphasis.

\emph{Measurability.}  The framework converts abstract quantities into observables.  Information friction $T$ is measurable two independent ways (VRI and Euler identity).  The crisis count is an observable integer.  The damping ratio $\zeta$ is estimable from response half-lives and cross-spectral coherence.  The sectoral $\rho$ is estimable from production function data and correlates with the business cycle.

\emph{Falsifiability.}  Every major result generates specific predictions.  VRI violation indicates departure from near-equilibrium.  Symmetric Adjustment violation indicates path-dependent dynamics.  $\rho$-ordering failure identifies sectors whose complementarity structure differs from estimates.  The predictions are not jointly ``too hard to test''---each is individually testable with existing data sources.

\emph{Closure.}  The deepest contribution is closing the self-referential loop.  With exogenous $\rho$, the framework describes \emph{what happens} given a production structure.  With endogenous $\rho$, it describes \emph{why production structures exist}---they are selected by the same dynamics they govern.  The CES parameter that controls everything is itself controlled by everything.  The resulting system has no free parameters: initial conditions and learning-curve elasticities determine the complete trajectory through $(\rho, T)$ space, including equilibrium selection, crisis timing, cycle frequency, and long-run architectural evolution.

\paragraph{Limitations.}
Three significant limitations remain.  First, the empirical estimation of sectoral $\rho$ values requires detailed firm-level production function data that is available only for manufacturing in a few countries; extending to services requires methodological innovation.  Second, the standardization elasticity $\beta_S$ has been estimated for only a handful of technologies; a systematic cross-technology comparison is needed.  Third, the model takes the number of sectors $N$ and the input-output topology as given.  A complete theory would endogenize the industrial structure itself---the entry and exit of entire sectors as $\rho$ evolves.  This remains an important direction for future work.

\paragraph{Internal coherence.}
The four parts are not independent: each builds on the preceding ones.  Part I provides the dynamical framework (adjustment dynamics, VRI, pre-crisis deceleration).  Part II derives exact constraints from symmetry (conservation laws, fluctuation theorems).  Part III uses the conservative-dissipative structure and symmetry constraints to derive oscillatory dynamics (business cycles).  Part IV closes the loop by making $\rho$ endogenous, which feeds back to alter the CES potential landscape on which Part I--III dynamics operate.  The logical chain is: landscape $\to$ symmetries $\to$ oscillations $\to$ endogenous landscape.  Each step constrains the next: the VRI from Part I determines noise amplitudes used in Part II's fluctuation theorems; the conservation laws from Part II constrain the oscillatory modes of Part III; the cycle structure of Part III generates the $T$-variation that drives Part IV's endogenous $\rho$.

\paragraph{The philosophical dimension.}
With exogenous $\rho$, the framework is a positive theory: it says what happens given parameters.  With endogenous $\rho$, it becomes a self-referential theory: the economy produces its own complementarity through the evolutionary, optimizing, and technological forces acting on $\rho$.  This self-referential closure---where the generating parameter of the entire dynamical theory is itself generated by that dynamics---is rare in economics.  It means that the CES potential landscape is not a static backdrop on which the economy moves; it is a landscape that reshapes itself in response to the economy's trajectory across it.

%=============================================================================
% Bibliography
%=============================================================================
\begin{thebibliography}{99}

\bibitem[Acemoglu et al.(2012)]{acemoglu2012}
Acemoglu, Daron, Vasco M. Carvalho, Asuman Ozdaglar, and Alireza Tahbaz-Salehi. 2012. ``The Network Origins of Aggregate Fluctuations.'' \textit{Econometrica} 80(5): 1977--2016.

\bibitem[Autor, Dorn, and Hanson(2013)]{autor2013}
Autor, David H., David Dorn, and Gordon H. Hanson. 2013. ``The China Syndrome: Local Labor Market Effects of Import Competition in the United States.'' \textit{American Economic Review} 103(6): 2121--2168.

\bibitem[Bak, Tang, and Wiesenfeld(1987)]{bak1987}
Bak, Per, Chao Tang, and Kurt Wiesenfeld. 1987. ``Self-Organized Criticality: An Explanation of the 1/f Noise.'' \textit{Physical Review Letters} 59(4): 381--384.

\bibitem[Baldwin and Clark(2000)]{baldwin2000}
Baldwin, Carliss Y., and Kim B. Clark. 2000. \textit{Design Rules: The Power of Modularity}. Cambridge, MA: MIT Press.

\bibitem[Battiston et al.(2016)]{battiston2016}
Battiston, Stefano, J. Doyne Farmer, et al. 2016. ``Complexity Theory and Financial Regulation.'' \textit{Science} 351(6275): 818--819.

\bibitem[Blanchard(2016)]{blanchard2016}
Blanchard, Olivier. 2016. ``The Phillips Curve: Back to the '60s?'' \textit{American Economic Review} 106(5): 31--34.

\bibitem[Borio(2014)]{borio2014}
Borio, Claudio. 2014. ``The Financial Cycle and Macroeconomics: What Have We Learnt?'' \textit{Journal of Banking \& Finance} 45: 182--198.

\bibitem[Bouchaud(2000)]{bouchaud2000}
Bouchaud, Jean-Philippe. 2000. ``Power Laws in Economics and Finance.'' \textit{Quantitative Finance} 1(1): 105--112.

\bibitem[Bouchaud(2018)]{bouchaud2018}
Bouchaud, Jean-Philippe. 2018. \textit{Trades, Quotes and Prices}. Cambridge University Press.

\bibitem[Bresnahan and Trajtenberg(1995)]{bresnahan1995}
Bresnahan, Timothy F., and Manuel Trajtenberg. 1995. ``General Purpose Technologies: `Engines of Growth'?'' \textit{Journal of Econometrics} 65(1): 83--108.

\bibitem[Brock and Durlauf(1999)]{brock1999}
Brock, William A., and Steven N. Durlauf. 1999. ``A Formal Model of Theory Choice in Science.'' \textit{Economic Theory} 14(1): 113--130.

\bibitem[Brunnermeier and Sannikov(2014)]{brunnermeier2014}
Brunnermeier, Markus K., and Yuliy Sannikov. 2014. ``A Macroeconomic Model with a Financial Sector.'' \textit{American Economic Review} 104(2): 379--421.

\bibitem[Crooks(1999)]{crooks1999}
Crooks, Gavin E. 1999. ``Entropy Production Fluctuation Theorem and the Nonequilibrium Work Relation for Free Energy Differences.'' \textit{Physical Review E} 60(3): 2721--2726.

\bibitem[Dakos et al.(2012)]{dakos2012}
Dakos, Vasilis, et al. 2012. ``Methods for Detecting Early Warnings of Critical Transitions.'' \textit{PLoS ONE} 7(7): e41010.

\bibitem[Debreu(1959)]{debreu1959}
Debreu, Gerard. 1959. \textit{Theory of Value}. Yale University Press.

\bibitem[Dierker(1972)]{dierker1972}
Dierker, Egbert. 1972. ``Two Remarks on the Number of Equilibria of an Economy.'' \textit{Econometrica} 40(5): 951--953.

\bibitem[Durlauf(1999)]{durlauf1999}
Durlauf, Steven N. 1999. ``How Can Statistical Mechanics Contribute to Social Science?'' \textit{PNAS} 96(19): 10582--10584.

\bibitem[Foley(1994)]{foley1994}
Foley, Duncan K. 1994. ``A Statistical Equilibrium Theory of Markets.'' \textit{Journal of Economic Theory} 62(2): 321--345.

\bibitem[Gabaix(2009)]{gabaix2009}
Gabaix, Xavier. 2009. ``Power Laws in Economics and Finance.'' \textit{Annual Review of Economics} 1(1): 255--294.

\bibitem[Gal\'{i}(2015)]{gali2015}
Gal\'{i}, Jordi. 2015. \textit{Monetary Policy, Inflation, and the Business Cycle}. 2nd ed. Princeton University Press.

\bibitem[Goodwin(1967)]{goodwin1967}
Goodwin, Richard M. 1967. ``A Growth Cycle.'' In \textit{Socialism, Capitalism and Economic Growth}, edited by C.~H.~Feinstein, 54--58.

\bibitem[H\"{a}nggi, Talkner, and Borkovec(1990)]{hanggi1990}
H\"{a}nggi, Peter, Peter Talkner, and Michal Borkovec. 1990. ``Reaction-Rate Theory: Fifty Years After Kramers.'' \textit{Reviews of Modern Physics} 62(2): 251--341.

\bibitem[Harding and Pagan(2002)]{harding2002}
Harding, Don, and Adrian Pagan. 2002. ``Dissecting the Cycle.'' \textit{Journal of Monetary Economics} 49(2): 365--381.

\bibitem[He and Krishnamurthy(2013)]{he2013}
He, Zhiguo, and Arvind Krishnamurthy. 2013. ``Intermediary Asset Pricing.'' \textit{American Economic Review} 103(2): 732--770.

\bibitem[Imbs and Wacziarg(2003)]{imbs2003}
Imbs, Jean, and Romain Wacziarg. 2003. ``Stages of Diversification.'' \textit{American Economic Review} 93(1): 63--86.

\bibitem[Jarzynski(1997)]{jarzynski1997}
Jarzynski, Christopher. 1997. ``Nonequilibrium Equality for Free Energy Differences.'' \textit{Physical Review Letters} 78(14): 2690--2693.

\bibitem[Jaynes(1957)]{jaynes1957}
Jaynes, Edwin T. 1957. ``Information Theory and Statistical Mechanics.'' \textit{Physical Review} 106(4): 620--630.

\bibitem[Kaldor(1940)]{kaldor1940}
Kaldor, Nicholas. 1940. ``A Model of the Trade Cycle.'' \textit{Economic Journal} 50(197): 78--92.

\bibitem[Krugman(1991)]{krugman1991}
Krugman, Paul. 1991. ``Target Zones and Exchange Rate Dynamics.'' \textit{Quarterly Journal of Economics} 106(3): 669--682.

\bibitem[Kuehn(2015)]{kuehn2015}
Kuehn, Christian. 2015. \textit{Multiple Time Scale Dynamics}. Springer.

\bibitem[Kydland and Prescott(1982)]{kydland1982}
Kydland, Finn E., and Edward C. Prescott. 1982. ``Time to Build and Aggregate Fluctuations.'' \textit{Econometrica} 50(6): 1345--1370.

\bibitem[Leamer(2007)]{leamer2007}
Leamer, Edward E. 2007. ``Housing IS the Business Cycle.'' NBER Working Paper No. 13428.

\bibitem[Leontief(1936)]{leontief1936}
Leontief, Wassily W. 1936. ``Quantitative Input and Output Relations in the Economic Systems of the United States.'' \textit{Review of Economics and Statistics} 18(3): 105--125.

\bibitem[Long and Plosser(1983)]{long1983}
Long, John B., and Charles I. Plosser. 1983. ``Real Business Cycles.'' \textit{Journal of Political Economy} 91(1): 39--69.

\bibitem[Ma\'{c}kowiak and Wiederholt(2009)]{mackowiak2009}
Ma\'{c}kowiak, Bartosz, and Mirko Wiederholt. 2009. ``Optimal Sticky Prices Under Rational Inattention.'' \textit{American Economic Review} 99(3): 769--803.

\bibitem[Mat\v{e}jka and McKay(2015)]{matejka2015}
Mat\v{e}jka, Filip, and Alisdair McKay. 2015. ``Rational Inattention to Discrete Choices.'' \textit{American Economic Review} 105(1): 272--298.

\bibitem[Milgrom and Roberts(1990)]{milgrom1990}
Milgrom, Paul, and John Roberts. 1990. ``The Economics of Modern Manufacturing.'' \textit{American Economic Review} 80(3): 511--528.

\bibitem[Minsky(1986)]{minsky1986}
Minsky, Hyman P. 1986. \textit{Stabilizing an Unstable Economy}. Yale University Press.

\bibitem[Oberfield and Raval(2014)]{oberfield2014}
Oberfield, Ezra, and Devesh Raval. 2014. ``Micro Data and Macro Technology.'' NBER Working Paper No. 20452.

\bibitem[Price(1970)]{price1970}
Price, George R. 1970. ``Selection and Covariance.'' \textit{Nature} 227(5257): 520--521.

\bibitem[Samuelson(1947)]{samuelson1947}
Samuelson, Paul A. 1947. \textit{Foundations of Economic Analysis}. Harvard University Press.

\bibitem[Sato(1981)]{sato1981}
Sato, Ryuzo. 1981. \textit{Theory of Technical Change and Economic Invariance}. Academic Press.

\bibitem[Scheffer et al.(2009)]{scheffer2009}
Scheffer, Marten, et al. 2009. ``Early-Warning Signals for Critical Transitions.'' \textit{Nature} 461(7260): 53--59.

\bibitem[Seger and Brockmann(1987)]{seger1987}
Seger, Jon, and H.~Jane Brockmann. 1987. ``What Is Bet-Hedging?'' \textit{Oxford Surveys in Evolutionary Biology} 4: 182--211.

\bibitem[Sims(2003)]{sims2003}
Sims, Christopher A. 2003. ``Implications of Rational Inattention.'' \textit{Journal of Monetary Economics} 50(3): 665--690.

\bibitem[Smirl(2026a)]{smirl2026ces}
Smirl, Jon. 2026a. ``The CES Quadruple Role.'' Working Paper.

\bibitem[Smirl(2026b)]{smirl2026free}
Smirl, Jon. 2026b. ``The CES Potential Framework for Economic Systems.'' Working Paper.

\bibitem[Smirl(2026c)]{smirl2026tsallis}
Smirl, Jon. 2026c. ``The Tsallis CES Potential.'' Working Paper.

\bibitem[Smirl(2026d)]{smirl2026prod}
Smirl, Jon. 2026d. ``Production Under Information Frictions.'' Working Paper.

\bibitem[Smirl(2026e)]{smirl2026cycle}
Smirl, Jon. 2026e. ``The Technology Cycle.'' Working Paper.

\bibitem[Smirl(2026f)]{smirl2026emergent}
Smirl, Jon. 2026f. ``Emergent CES: Why CES Is Not an Assumption.'' Working Paper.

\bibitem[Stock and Watson(2003)]{stock2003}
Stock, James H., and Mark W. Watson. 2003. ``Has the Business Cycle Changed and Why?'' \textit{NBER Macroeconomics Annual} 17: 159--218.

\bibitem[Syverson(2004)]{syverson2004}
Syverson, Chad. 2004. ``Market Structure and Productivity.'' \textit{Journal of Political Economy} 112(6): 1181--1222.

\bibitem[Syverson(2011)]{syverson2011}
Syverson, Chad. 2011. ``What Determines Productivity?'' \textit{Journal of Economic Literature} 49(2): 326--365.

\bibitem[Barro and Urs\'{u}a(2006)]{barro2006}
Barro, Robert J., and Jos\'{e} F. Urs\'{u}a. 2006. ``Macroeconomic Crises Since 1870.'' \textit{Brookings Papers on Economic Activity} 2008(1): 255--335.

\bibitem[Clauset, Shalizi, and Newman(2009)]{clauset2009}
Clauset, Aaron, Cosma Rohilla Shalizi, and Mark E.~J. Newman. 2009. ``Power-Law Distributions in Empirical Data.'' \textit{SIAM Review} 51(4): 661--703.

\bibitem[Henderson and Clark(1990)]{henderson1990}
Henderson, Rebecca M., and Kim B. Clark. 1990. ``Architectural Innovation: The Reconfiguration of Existing Product Technologies and the Failure of Established Firms.'' \textit{Administrative Science Quarterly} 35(1): 9--30.

\bibitem[Metcalfe(1998)]{metcalfe1998}
Metcalfe, J.~Stanley. 1998. \textit{Evolutionary Economics and Creative Destruction}. London: Routledge.

\bibitem[Nelson and Winter(1982)]{nelson1982}
Nelson, Richard R., and Sidney G. Winter. 1982. \textit{An Evolutionary Theory of Economic Change}. Cambridge, MA: Harvard University Press.

\bibitem[Reinhart and Rogoff(2009)]{reinhart2009}
Reinhart, Carmen M., and Kenneth S. Rogoff. 2009. \textit{This Time Is Different: Eight Centuries of Financial Folly}. Princeton: Princeton University Press.

\bibitem[Scheinkman(2014)]{scheinkman2014}
Scheinkman, Jos\'{e} A. 2014. \textit{Speculation, Trading, and Bubbles}. New York: Columbia University Press.

\bibitem[Yakovenko(2009)]{yakovenko2009}
Yakovenko, Victor M. 2009. ``Econophysics, Statistical Mechanics Approach to.'' In \textit{Encyclopedia of Complexity and Systems Science}.

\end{thebibliography}

\end{document}
