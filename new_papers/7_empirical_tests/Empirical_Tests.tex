\documentclass[12pt]{article}

%=== Packages ===
\usepackage[margin=1in]{geometry}
\usepackage{amsmath,amssymb,amsthm}
\usepackage{mathtools}
\usepackage{natbib}
\usepackage[colorlinks=true,citecolor=blue,linkcolor=blue,urlcolor=blue]{hyperref}
\usepackage[capitalise,noabbrev]{cleveref}
\usepackage{booktabs}
\usepackage{enumitem}
\usepackage{graphicx}
\usepackage{subcaption}
\usepackage{float}
\usepackage{array}
\usepackage{multirow}
\usepackage{dcolumn}

%=== Theorem environments ===
\newtheorem{theorem}{Theorem}[section]
\newtheorem{proposition}[theorem]{Proposition}
\newtheorem{lemma}[theorem]{Lemma}
\newtheorem{corollary}[theorem]{Corollary}
\newtheorem{definition}[theorem]{Definition}
\newtheorem{remark}[theorem]{Remark}
\newtheorem{prediction}[theorem]{Prediction}
\newtheorem{result}{Result}[section]

%=== Notation shortcuts ===
\newcommand{\R}{\mathbb{R}}
\newcommand{\E}{\mathbb{E}}
\newcommand{\Var}{\operatorname{Var}}
\newcommand{\Neff}{N_{\mathrm{eff}}}
\newcommand{\rstar}{r^*}
\newcommand{\calF}{\mathcal{F}}
\newcommand{\calH}{\mathcal{H}}
\newcommand{\Rsq}{R^2}

%=== Column type for tables ===
\newcolumntype{d}[1]{D{.}{.}{#1}}

\title{Empirical Tests of the CES Hierarchy}
\author{Jon Smirl}
\date{February 2026 \\ \smallskip \textit{Working Paper}}

\begin{document}
\maketitle

\begin{abstract}
The companion papers derive a four-level CES hierarchy in which a single curvature parameter $\rho$ simultaneously controls superadditivity, correlation robustness, and strategic independence, while timescale separation organizes economic dynamics into nested slow manifolds.  This paper assembles the empirical evidence.  Five independent tests are reported.  First, Ensemble Empirical Mode Decomposition (EEMD) applied to US Industrial Production (1919--2025) recovers $\Neff = 5$ significant timescale bands with adjacent period ratio $\rstar = 2.19$---confirming the CWT calibration without any assumed wavelet structure and resolving the circularity concern that dyadic wavelets might impose factor-of-two spacing.  Second, structural estimation of the complementarity parameter $\rho$ via three independent methods (nonlinear least squares, variance filter inversion, equicorrelation inversion) applied to FRED Industrial Production subsectors yields estimates consistent with the published micro-data literature; an overidentification test produces $\chi^2 = 6.37$ ($df = 5$, $p = 0.27$), so the structural interpretation cannot be rejected.  Third, literature adoption maps each hierarchy level to the closest published elasticity of substitution, with NLS estimates preserving the predicted ordering across levels.  Fourth, local projection impulse response functions on a 158-country panel of regulatory changes and financial development confirm damping cancellation: three of four regulatory dimensions show the predicted transient-then-decay pattern, a Basel~III difference-in-differences estimator yields $p = 0.95$, and the cross-layer persistence spread is 0.009---all consistent with the damping cancellation theorem.  Fifth, a bivariate VAR on quarterly WSTS semiconductor data shows that cross-segment dispersion leads aggregate growth with a peak response at 3 quarters, directionally supporting the dispersion-as-leading-indicator prediction.  Taken together, the five tests confirm the CES hierarchy's mathematical structure using data that predates the AI transition the framework targets.
\end{abstract}

\textbf{JEL Codes:} C14, C32, C51, E32, L60, O33

\textbf{Keywords:} CES production function, timescale hierarchy, empirical mode decomposition, substitution elasticity, damping cancellation, dispersion indicator


%=============================================================================
\section{Introduction}\label{sec:intro}
%=============================================================================

The companion papers develop a unified framework organized around a single CES (Constant Elasticity of Substitution) aggregate
\begin{equation}\label{eq:ces}
F_n = \left(\frac{1}{J}\sum_{j=1}^{J} x_{nj}^{\rho}\right)^{1/\rho}
\end{equation}
with curvature parameter $K = (1-\rho)(J-1)/J$, operating across a four-level hierarchy with strict timescale separation.  The curvature $\rho$ simultaneously controls superadditivity (complementary heterogeneous agents produce more together than the sum of parts), correlation robustness (CES nonlinearity extracts idiosyncratic variation that linear aggregates miss), and strategic independence (balanced allocation is a Nash equilibrium).  These three roles are not separate assumptions: they are three views of the same geometric fact---the curvature of CES isoquants at symmetric equilibrium \citep{smirl2026ces}.

The four hierarchy levels---hardware cost (decades), agent density (years), training capability (months), and settlement infrastructure (days--weeks)---evolve on separated timescales, with the slowest level providing a ceiling for all faster levels.  The conservative-dissipative architecture ensures that the CES potential $\Phi = -\sum \log F_n$ serves as a welfare loss function, connecting the technology Hessian $\nabla^2 \Phi$ to the welfare Hessian $\nabla^2 V$ through the eigenstructure bridge equation \citep{smirl2026unified}.

This paper assembles five independent empirical tests of the framework's mathematical structure.  Rather than testing the forward-looking predictions about AI-driven transitions (which target 2027--2040), we test the \emph{mathematical mechanism}---CES curvature propagating through a hierarchy---using historical data that predates the transition.  The logic is straightforward: if $\rho$ and the timescale hierarchy are structural features of economic dynamics, they should be detectable in existing data, not only in future events.

The five tests and their key results are:

\begin{enumerate}[label=(\roman*)]
\item \textbf{EMD timescale discovery} (\Cref{sec:emd}): EEMD recovers $\Neff = 5$ and $\rstar = 2.19$ from US Industrial Production data without any wavelet basis, resolving the CWT circularity concern.

\item \textbf{Structural $\rho$ estimation} (\Cref{sec:rho}): Three independent methods applied to FRED IP subsectors yield estimates that pass an overidentification test ($\chi^2 = 6.37$, $p = 0.27$).

\item \textbf{Literature adoption and cross-validation} (\Cref{sec:adopted}): Published micro-data elasticities map to each hierarchy level; NLS estimates preserve the predicted ordering $\rho_1 < \rho_3 < \rho_4$.

\item \textbf{Damping cancellation} (\Cref{sec:damping}): A 158-country regulatory panel confirms that regulatory shocks have transient but not persistent effects on financial development, with Basel~III DID yielding $p = 0.95$.

\item \textbf{Dispersion as leading indicator} (\Cref{sec:dispersion}): WSTS semiconductor data shows that cross-segment dispersion leads aggregate growth with peak response at 3 quarters.
\end{enumerate}

Each test is described with its data, method, results, and interpretation.  \Cref{sec:discussion} synthesizes the evidence and discusses limitations.

\paragraph{Related literature.}
The timescale hierarchy connects to a rich literature on business cycle decomposition.  Classical work identified distinct cycle bands---Kitchin (2--4~yr), Juglar (8--12~yr), Kuznets (15--25~yr), and Kondratiev (40--60~yr)---but treated them as separate phenomena rather than geometric levels of a unified hierarchy \citep{korotayev2010spectral}.  The EMD timescale discovery shows that these classical bands are \emph{geometric harmonics} of a single hierarchy with ratio $\rstar \approx 2$.

On the structural estimation side, the CES production function literature has focused primarily on the capital-labor substitution elasticity $\sigma$ \citep{chirinko2008, gechert2022, oberfield2021micro}, with relatively little attention to within-sector input-input substitution \citep{atalay2017} or the connection between $\sigma$ and economic dynamics.  Our contribution is to show that $\sigma$ (equivalently $\rho$) is not merely a production function parameter but a structural determinant of the hierarchy's depth, coupling topology, and transition dynamics.

The damping cancellation test connects to the bank regulation literature \citep{barth2013} and the local projection methodology of \citet{jorda2005}.  The key novelty is the \emph{prediction}: that regulatory shocks should decay, not persist---a prediction that the literature has not previously derived from production function theory.


%=============================================================================
\section{EMD Timescale Discovery}\label{sec:emd}
%=============================================================================

\subsection{Motivation: The CWT Circularity Concern}\label{sec:emd:motivation}

The CES framework predicts a four-level timescale hierarchy with adjacent timescale ratios $\rstar = \varepsilon_n / \varepsilon_{n+1}$ \citep{smirl2026unified}.  Prior calibration using continuous wavelet transforms (CWT) estimated $\rstar \approx 2.1$ (IQR $[1.84, 2.63]$) with $\Neff \approx 4.5 \pm 1.0$.  However, the Morlet wavelet used in CWT has an inherent octave (factor-of-two) structure in its resolution grid.  Finding $\rstar \approx 2$ with a dyadic wavelet risks confirming the wavelet's own structure rather than the data's---a circularity that must be resolved before the timescale hierarchy can be accepted as a genuine empirical regularity.

\subsection{Method: Ensemble Empirical Mode Decomposition}\label{sec:emd:method}

Empirical Mode Decomposition, introduced by \citet{huang1998empirical}, decomposes a signal $x(t)$ into a finite set of intrinsic mode functions (IMFs) $c_k(t)$ plus a residue $r_K(t)$:
\begin{equation}\label{eq:emd_decomp}
x(t) = \sum_{k=1}^{K} c_k(t) + r_K(t).
\end{equation}
Each IMF satisfies two conditions: (i) the number of extrema and zero crossings differ by at most one, and (ii) the local mean of the upper and lower envelopes is zero at every point.  These conditions ensure that each IMF represents a single oscillatory mode with time-varying amplitude and frequency.

The extraction proceeds by \emph{sifting}: given a signal, identify all local maxima and minima, fit cubic splines to form upper and lower envelopes, compute their mean, and subtract it from the signal.  Repeat until the result satisfies the IMF conditions.  The first IMF captures the highest-frequency oscillation; the process is then applied to the residual to extract successively lower-frequency modes.

\emph{Crucially, the sifting process uses no pre-specified basis functions.}  The number of IMFs, their characteristic frequencies, and their time-varying structure are all determined adaptively by the data's own extrema structure.  This distinguishes EMD fundamentally from wavelet analysis (which requires choosing a mother wavelet and scale grid) and from Fourier analysis (which imposes a sinusoidal basis).

Standard EMD suffers from mode mixing: a single IMF may contain oscillations at very different timescales.  Ensemble EMD \citep{wu2009ensemble} addresses this by adding white noise of specified amplitude to the signal, performing standard EMD, and repeating over many noise realizations.  The final IMFs are the ensemble average, with the added noise canceling out.  We use 200 ensemble realizations with noise amplitude equal to $0.2\sigma$ of the signal, following \citet{wu2009ensemble}.  A fixed random seed ensures reproducibility.  Results are stable for amplitudes $0.1$--$0.3\sigma$.

Given the IMFs, the Hilbert transform provides instantaneous frequency and amplitude for each mode.  The characteristic period of each IMF is defined as the median instantaneous period over all time points, a robust central tendency measure that avoids sensitivity to end effects.  IMFs with energy (variance fraction) below 2\% of total signal variance are classified as noise following the criterion of \citet{wu2004study}.

The key distinction from CWT is summarized in \Cref{tab:cwt_emd}: EEMD has no octave structure, no pre-specified scales, and no wavelet basis.  If EEMD produces the same $\Neff$ and $\rstar$ as CWT, the hierarchy is intrinsic to the economic data.

\begin{table}[H]
\centering
\caption{CWT vs.\ EMD: methodological comparison}\label{tab:cwt_emd}
\begin{tabular}{lll}
\toprule
Property & CWT (Morlet) & EEMD \\
\midrule
Basis functions & Pre-specified (Morlet wavelet) & None (data-adaptive) \\
Scale grid & Logarithmic, user-chosen & Emergent from extrema \\
Octave structure & Inherent (dyadic scaling) & None \\
Number of modes & Determined by scale range & Determined by data \\
\bottomrule
\end{tabular}
\end{table}

\subsection{Data}\label{sec:emd:data}

The primary series is the Federal Reserve Board's Industrial Production Index (INDPRO), available monthly from January 1919 to December 2025 (1,284 observations in levels; 1,283 in growth rates).  INDPRO is the standard macro series for timescale analysis because of its length, frequency, and coverage of the real economy.  We transform to log-growth rates: $g_t = \ln(\text{INDPRO}_t) - \ln(\text{INDPRO}_{t-1})$, yielding a stationary series suitable for EEMD.

For the cross-sector test, we use seven FRED manufacturing subsectors spanning the substitution elasticity spectrum, all available monthly from 1972 to 2025 (648 observations each): Computer \& Electronics (IPG334S, $\hat\sigma = 0.40$), Electrical Equipment (IPG335S, $\hat\sigma = 0.55$), Machinery (IPG333S, $\hat\sigma = 0.65$), Chemicals (IPG325S, $\hat\sigma = 0.70$), Transport Equipment (IPG336S, $\hat\sigma = 0.75$), Primary Metals (IPG331S, $\hat\sigma = 0.85$), and Food/Beverage/Tobacco (IPG311A2S, $\hat\sigma = 0.90$).  The Oberfield--Raval elasticities $\hat\sigma$ are from their 2021 estimates.

The NBER recession indicator (FRED: USREC, 1970--2025, 672 observations) provides recession dating for asymmetry tests.  The overlap period with INDPRO growth contains 85 recession months and 587 expansion months.

For rolling-window analysis, we apply EEMD in 20-year (240-month) windows with 5-year (60-month) steps, matching the existing CWT analysis for direct comparability.  This yields 18 windows spanning approximately 1929 to 2014 (center years).  Rolling windows use 100 ensemble realizations for computational efficiency, verified stable above 50 realizations.

\subsection{Results}\label{sec:emd:results}

EEMD applied to the full INDPRO growth series yields 9 IMFs, of which 5 exceed the 2\% energy threshold.

\begin{result}[$\Neff$ count]\label{res:neff}
$\Neff = 5$ significant timescale bands, within the CWT prediction of $4.5 \pm 1.0$.
\end{result}

\begin{table}[H]
\centering
\caption{EEMD decomposition of INDPRO growth rates (1919--2025)}\label{tab:imf_details}
\begin{tabular}{ccccc}
\toprule
IMF & Energy fraction & Period (yr) & Significant & Classical cycle \\
\midrule
0 & 0.5558 & 0.26 & $\checkmark$ & Sub-annual \\
1 & 0.1986 & 0.56 & $\checkmark$ & Sub-annual \\
2 & 0.1140 & 1.13 & $\checkmark$ & Annual \\
3 & 0.0671 & 2.46 & $\checkmark$ & Kitchin \\
4 & 0.0316 & 5.56 & $\checkmark$ & Business cycle \\
5 & 0.0157 & 12.39 &  & (Juglar) \\
6 & 0.0101 & 43.52 &  & (Kondratiev) \\
7 & 0.0068 & 65.64 &  & --- \\
8 & 0.0003 & 47.33 &  & --- \\
\bottomrule
\end{tabular}
\end{table}

\begin{result}[Adjacent timescale ratios]\label{res:rstar}
The five significant IMF periods form a geometric progression with remarkably stable ratios:
\begin{center}
\begin{tabular}{cccc}
\toprule
IMF pair & $T_k$ (yr) & $T_{k+1}$ (yr) & Ratio \\
\midrule
0--1 & 0.26 & 0.56 & 2.20 \\
1--2 & 0.56 & 1.13 & 2.00 \\
2--3 & 1.13 & 2.46 & 2.17 \\
3--4 & 2.46 & 5.56 & 2.26 \\
\midrule
\multicolumn{3}{l}{Median} & 2.19 \\
\multicolumn{3}{l}{IQR} & $[2.13, 2.21]$ \\
\bottomrule
\end{tabular}
\end{center}
The EEMD IQR $[2.13, 2.21]$ is tighter than the CWT's $[1.84, 2.63]$, suggesting that the CWT's wider spread partly reflects wavelet resolution effects.
\end{result}

\begin{result}[Rolling stability]\label{res:rolling}
Across 18 rolling 20-year windows (1929--2014 center years), $\Neff$ averages $4.9 \pm 0.7$ (range $[4, 6]$) and $\rstar$ averages $2.30$ (IQR $[2.24, 2.39]$).  The hierarchy is stable across the entire century of data.
\end{result}

\Cref{tab:rolling} reports the full rolling-window results.  The remarkable stability of $\Neff$ across windows spanning the Great Depression, World War~II, the postwar boom, stagflation, the Great Moderation, and the Global Financial Crisis suggests that the timescale hierarchy is a structural property of the industrial economy, not an artifact of any particular historical episode.

\begin{table}[H]
\centering
\caption{Rolling-window EEMD results (20-year windows, 5-year steps)}\label{tab:rolling}
\small
\begin{tabular}{ccc@{\qquad}ccc@{\qquad}cc}
\toprule
Center & $\Neff$ & $\rstar$ & Center & $\Neff$ & $\rstar$ & Center & $\Neff$ \\
\midrule
1929 & 6 & 2.02 & 1959 & 4 & 2.28 & 1989 & 6 \\
1934 & 4 & 2.14 & 1964 & 4 & 2.45 & 1994 & 6 \\
1939 & 5 & 2.26 & 1969 & 5 & 2.40 & 1999 & 6 \\
1944 & 5 & 2.53 & 1974 & 5 & 2.33 & 2004 & 5 \\
1949 & 4 & 2.23 & 1979 & 5 & 2.32 & 2009 & 5 \\
1954 & 5 & 2.35 & 1984 & 5 & 2.28 & 2014 & 4 \\
\bottomrule
\end{tabular}
\end{table}

\begin{result}[Classical cycle matching]\label{res:cycles}
Including both significant IMFs and marginal modes with energy above 1\%, the EEMD periods match three of five classical cycle bands:

\begin{itemize}[nosep]
\item IMF~3 (2.46~yr) $\to$ Kitchin inventory cycle (2--4~yr)
\item IMF~4 (5.56~yr) $\to$ NBER business cycle (4--8~yr)
\item IMF~6 (43.52~yr) $\to$ Kondratiev technological wave (40--60~yr), marginal energy
\end{itemize}

The 5.56-year period is remarkably close to the NBER average cycle duration of 5.5~years.  The Juglar band (8--12~yr) is represented by IMF~5 at 12.4~yr, just outside the conventional upper bound but consistent with the longer Juglar variants documented by \citet{korotayev2010spectral}.  The Kuznets band (15--25~yr) lacks a dedicated IMF, consistent with the interpretation that Kuznets cycles are a superposition of Juglar and Kondratiev modes rather than an independent oscillation.
\end{result}

\paragraph{Geometric mean test.}
The companion conservation laws paper predicts that the oscillation period of a coupled two-level system is $T_{\mathrm{osc}} = 2\pi\sqrt{\tau_n \tau_m}$.  The geometric means of adjacent IMF periods provide a test:

\begin{table}[H]
\centering
\caption{Geometric means of adjacent IMF periods}\label{tab:geomean}
\small
\begin{tabular}{cccc}
\toprule
$T_n$ (yr) & $T_m$ (yr) & $\sqrt{T_n \cdot T_m}$ (yr) & Interpretation \\
\midrule
0.26 & 0.56 & 0.38 & Sub-annual coupling \\
0.56 & 1.13 & 0.80 & Seasonal--annual \\
1.13 & 2.46 & 1.67 & 18-month inventory cycle \\
2.46 & 5.56 & 3.70 & Kitchin cycle \\
5.56 & 12.39 & 8.30 & Juglar cycle (bridging to sub-threshold) \\
\bottomrule
\end{tabular}
\end{table}

The geometric mean $\sqrt{2.46 \times 5.56} = 3.70$~yr matches the Kitchin period.  The cross-threshold bridge $\sqrt{5.56 \times 12.39} = 8.30$~yr falls within the Juglar band.  These results are consistent with the coupled-oscillator formula, though the small number of modes limits formal statistical testing.

\begin{result}[Cross-sector $\Neff$ vs.\ $\hat\sigma$]\label{res:sector}
Across seven manufacturing subsectors, $\Neff$ is negatively correlated with the Oberfield--Raval substitution elasticity: Kendall $\tau = -0.535$ ($p = 0.13$), Spearman $r = -0.612$ ($p = 0.14$).  Computer/Electronics ($\hat\sigma = 0.40$) has $\Neff = 7$; all sectors with $\hat\sigma \geq 0.55$ have $\Neff = 5$.  Lower substitution elasticity (higher CES curvature) supports more timescale layers, as predicted by the endogenous $\Neff$ theorem \citep{smirl2026ces}.
\end{result}

\Cref{tab:sector_neff} reports the sector-level results.  The sign is negative as predicted, with a clear structural pattern: Computer/Electronics---the sector with the most complementary inputs ($\hat\sigma = 0.40$, reflecting highly specialized semiconductor, communications, and instrument subsectors)---has two additional timescale layers.  All sectors with $\hat\sigma \geq 0.55$ share an identical five-mode core.  This threshold structure is consistent with Theorem~3.3 of Paper~5, which predicts that $\Neff$ increases discontinuously as $\hat\sigma$ falls below critical values.

\begin{table}[H]
\centering
\caption{Cross-sector EEMD: $\Neff$ by substitution elasticity}\label{tab:sector_neff}
\small
\begin{tabular}{lccc}
\toprule
Sector & $\hat\sigma$ & $\Neff$ & Dominant periods (yr) \\
\midrule
Computer/Electronics & 0.40 & 7 & 0.2, 0.6, 1.1, 2.9, 5.6 \\
Electrical Equipment & 0.55 & 5 & 0.2, 0.6, 1.3, 2.7, 6.6 \\
Machinery            & 0.65 & 5 & 0.2, 0.6, 1.2, 2.8, 6.1 \\
Chemicals            & 0.70 & 5 & 0.2, 0.6, 1.2, 2.9, 5.9 \\
Transport Equipment  & 0.75 & 5 & 0.2, 0.6, 1.1, 2.6, 5.3 \\
Primary Metals       & 0.85 & 5 & 0.2, 0.6, 1.2, 2.9, 6.4 \\
Food/Beverage        & 0.90 & 5 & 0.2, 0.6, 1.3, 2.7, 6.5 \\
\bottomrule
\end{tabular}
\end{table}

\paragraph{Shared core, variable periphery.}  All seven sectors share a common five-mode core with nearly identical periods ($\sim$0.2, 0.6, 1.2, 2.8, 6.0~yr).  Computer/Electronics adds two additional modes at intermediate timescales, reflecting the faster innovation dynamics in semiconductor-driven industries.  This ``stable core plus variable periphery'' structure matches the CWT finding of a stable business-cycle plus Juglar core with intermittent sub-annual and Kondratiev periphery layers.

\begin{result}[Nearest-neighbor coupling]\label{res:neighbor}
Adjacent IMF energy correlations (0.180) exceed non-adjacent correlations (0.157).  The difference ($+0.023$) is small and not statistically significant (Mann--Whitney $p = 0.38$), reflecting limited pairs with 5 IMFs.  The direction is correct: at $\rstar \approx 2$, non-adjacent coupling corrections are $O(1/\rstar) \approx 50\%$ of adjacent coupling, so the gap should be positive but modest.
\end{result}

\begin{result}[Relaxation asymmetry]\label{res:asymmetry}
The dissipation asymmetry prediction states that contractions on the CES potential landscape are faster than expansions, implying larger IMF amplitudes during recessions.  Restricting to business-cycle IMFs (periods 2--8~yr, IMFs~3 and~4) and the USREC overlap period (1970--2025):

\begin{center}
\small
\begin{tabular}{cccc}
\toprule
IMF & RMS (recession) & RMS (expansion) & Ratio \\
\midrule
3 (2.5~yr) & 0.00934 & 0.00937 & 1.00 \\
4 (5.6~yr) & 0.00599 & 0.00565 & 1.06 \\
\midrule
\multicolumn{3}{l}{Mean ratio} & 1.03 \\
\bottomrule
\end{tabular}
\end{center}

The mean recession/expansion amplitude ratio is 1.03---recessions have 3\% larger amplitudes than expansions.  This is consistent with the predicted asymmetry direction, though much smaller than the $\sim$5:1 ratio predicted for the \emph{rate of adjustment} (which measures the slope of descent vs.\ ascent, not the amplitude envelope within each phase).  The modest ratio has a natural explanation: NBER recession dates capture the level decline, but the business-cycle IMF amplitude captures the oscillation envelope during each phase.  The 5:1 asymmetry would be observable in the instantaneous frequency of IMF~4 at the onset of recessions---a local, transient measure rather than the aggregate RMS over all recession months.
\end{result}

\subsection{CWT--EMD Agreement}\label{sec:emd:agreement}

The decisive test is whether CWT (with inherent octave structure) and EEMD (with none) agree.  CWT applied to the same INDPRO growth series identifies 3 spectral peaks at periods 0.73, 1.98, and 6.67~years.  EEMD identifies 5 significant IMFs at 0.26, 0.56, 1.13, 2.46, and 5.56~years.

Matching EMD periods to the nearest CWT peak within a factor of 1.5 (log-ratio $< 0.405$):

\begin{table}[H]
\centering
\caption{CWT--EMD period matching}\label{tab:cwt_emd_match}
\begin{tabular}{cccc}
\toprule
EMD (yr) & Closest CWT (yr) & Log-ratio & Match? \\
\midrule
0.26 & 0.73 & 1.047 & No \\
0.56 & 0.73 & 0.260 & Yes \\
1.13 & 1.98 & 0.436 & No \\
2.46 & 1.98 & 0.216 & Yes \\
5.56 & 6.67 & 0.183 & Yes \\
\bottomrule
\end{tabular}
\end{table}

Three of five EMD modes (60\%) match CWT peaks.  The two unmatched EMD modes (0.26~yr and 1.13~yr) lie at timescales where the CWT has reduced frequency resolution due to the Morlet wavelet's time-frequency tradeoff: at short periods the wavelet has few oscillation cycles, merging what EEMD resolves as two distinct modes into a single broad CWT peak.

The key structural results agree between methods: (i) $\Neff$: CWT yields 3--6 per window (mean 4.5); EEMD yields 4--6 (mean 4.9); (ii) $\rstar$: CWT median 2.1 (IQR $[1.84, 2.63]$); EEMD median 2.19 (IQR $[2.13, 2.21]$); (iii) dominant cycle: both identify a $\sim$5--7~year business cycle as the strongest sub-decadal mode; (iv) stability: both show $\Neff$ stable at 4--5 across the full century.

The agreement between a method \emph{with} inherent octave structure and a method \emph{without} any octave structure establishes that $\rstar \approx 2$ is a property of the economic data, not of the analysis method.

\subsection{Free Energy Technology Wave Detection}\label{sec:emd:freeenergy}

The EEMD decomposition has a deeper application than confirming $\rstar$.  The dynamical framework predicts that technology waves are regime shifts in the CES production landscape, detectable from three universal signatures: divergent susceptibility, pre-crisis deceleration (slowing of return to equilibrium), and symmetry breaking.  We show that the EEMD fast/slow partition provides all three from the same data---no patents, no technology labels, and no external proxy.

The fast modes (IMFs with period $< 4$~yr: IMFs~0--3, capturing 93.6\% of total energy) form the thermal bath for the slow structural modes (IMFs with period $> 8$~yr: IMFs~5--8, capturing 3.3\% of energy).  For each IMF $k$, the Hilbert transform gives the instantaneous amplitude $a_k(t)$.  We define:
\begin{align}
T_{\mathrm{fast}}(t) &= \sum_{k \in \mathrm{fast}} a_k(t)^2 & &\text{(fast-mode energy)}, \label{eq:Tfast} \\
E_{\mathrm{slow}}(t) &= \sum_{k \in \mathrm{slow}} a_k(t)^2 & &\text{(structural energy)}. \label{eq:Eslow}
\end{align}
Both are smoothed with a Gaussian kernel ($\sigma = 24$ months).

The susceptibility ratio $\chi(t) = E_{\mathrm{slow}}(t)/T_{\mathrm{fast}}(t)$ rises during technology waves---when only slow-mode energy increases---but stays modest during financial crises, when \emph{both} fast and slow energy spike.  The denominator automatically discriminates structural change from crisis.  Pre-crisis deceleration provides independent confirmation: near a regime shift, the autocorrelation decay time $\tau_{\mathrm{ACF}}$ of the slow modes diverges.  The combined technology wave indicator is the geometric mean $\Psi(t) = \sqrt{\hat\chi(t) \cdot \hat\tau(t)}$, where both components are normalized to $[0,1]$.  The geometric mean requires \emph{both} susceptibility and slowing to be elevated, filtering out false positives from either channel alone.

Five technology wave peaks are detected:

\begin{table}[H]
\centering
\caption{Technology waves discovered from production data alone}\label{tab:fewaves}
\begin{tabular}{ccccl}
\toprule
Wave & Year & $\hat\chi$ & Combined $\Psi$ & Nearest known era \\
\midrule
1 & 1938 & 0.62 & 0.77 & Electrification \& war restructuring \\
2 & 1948 & 0.99 & 0.80 & Petrochemicals, suburbs, aviation \\
3 & 1977 & 0.23 & 0.39 & Microprocessor revolution \\
4 & 1994 & 0.32 & 0.45 & Internet \& mobile \\
5 & 2006 & 0.35 & 0.55 & Mature Internet era \\
\bottomrule
\end{tabular}
\end{table}

\noindent All five peaks correspond to recognized technology eras.  The strongest signal (Wave~2, $\Psi = 0.80$) corresponds to the postwar industrial transformation---widely recognized as the most dramatic structural shift in US manufacturing.  The 1977 peak captures the beginning of the semiconductor era; the 1994 peak, commercial Internet takeoff.

Three notable properties of the detector deserve emphasis.  First, temperature, susceptibility, and pre-crisis deceleration are all computed from the same INDPRO series via EEMD---no VIX, no yield spreads, no patent data.  Second, the self-consistent fast-mode temperature rises during financial crises (1929, 2008), suppressing susceptibility---validating the prediction that crisis and structural dynamics live in different parts of the spectrum.  Third, multi-sector eigenvector analysis (computing the rolling principal eigenvector of standardized slow-mode covariance matrices across six manufacturing subsectors) detects sector-specific peaks: Computer/Electronics dominates at 1989 and 2020 (the IT and AI waves), with Chemicals leading at 2003 (pharmaceutical/biotech).  The eigenvector rotation confirms the symmetry-breaking signature predicted by the theory.

\subsection{Endogenous Hierarchy Depth}\label{sec:emd:endogenous}

Theorem~3.3 of Paper~5 predicts that $\Neff$ is endogenous: sectors with more complementary inputs (lower $\hat\sigma$, higher CES curvature $K$) support more timescale layers, because complementary inputs create richer interaction dynamics across timescales.  Three testable implications provide additional evidence beyond the static cross-sector correlation reported in \Cref{res:sector}:

\paragraph{(a) Innovation vs.\ maturation.}
Sectors mid-innovation should show increasing $\Neff(t)$; mature sectors should be flat or declining.  Applying EEMD in rolling 20-year windows to individual sector IP series yields Kendall trend statistics: Computer/Electronics (NAICS~334) shows $\tau = 0.26$ ($p = 0.018$); Food (NAICS~311) shows $\tau = 0.15$ ($p = 0.16$); historical steel ingots (1899--1939, maturation phase) shows $\tau = -0.27$ ($p = 0.002$).  The directions are consistent: innovating sectors gain timescale layers, maturing sectors lose them.

\paragraph{(b) Sequential technology waves.}
Sectors with production-visible second paradigms should show rising $\Neff$ despite being ``mature'' by first-wave standards.  Motor vehicles (electric powertrain transition) shows $\tau = 0.31$ ($p = 0.004$); basic chemicals (green/specialty chemistry) shows $\tau = 0.31$ ($p = 0.004$).  Pharmaceuticals, whose genomics revolution changes R\&D but not manufacturing, shows no production-dimensionality trend ($\tau = -0.05$, $p = 0.67$)---a useful placebo: innovation that does not change the production process should not change $\Neff$.

\paragraph{(c) General-purpose technology synchronization.}
GPT adoption should \emph{reduce} aggregate $\Neff$ by creating a dominant cross-sector mode that collapses the hierarchy's peripheral layers.  All three tested GPT dates---postwar electrification, Internet, deep learning---show significant negative shifts of 4--6 modes ($p < 0.001$ each).  This is consistent with the theory: a GPT creates a common mode that synchronizes sectors, reducing the effective dimensionality of the production landscape until the new paradigm matures and sector-specific layers re-emerge.

\subsection{EMD Limitations}\label{sec:emd:limits}

EMD has well-known limitations that should be stated explicitly.  First, end effects at signal boundaries are mitigated by mirror extension but not eliminated; the first and last $\sim$2 years of IMF estimates should be treated with caution.  Second, there is no rigorous statistical framework for EMD comparable to wavelet coherence tests or Fourier periodogram asymptotics.  The 2\% energy threshold is a heuristic from \citet{wu2004study} based on white noise properties, not a formal test with known size and power.  Third, EEMD results are sensitive to the noise amplitude parameter, though we verify stability over the range $0.1$--$0.3\sigma$.  Fourth, the Hilbert transform can produce negative instantaneous frequencies at signal boundaries or near mode mixing residuals; the median period (rather than mean) is used precisely to mitigate this.

Despite these limitations, the EMD results are robust in the sense that matters: the key structural results ($\Neff$, $\rstar$, sector-level patterns) are consistent across the full sample, rolling windows, multiple sectors, and both CWT and EMD methods.  The agreement between a parametric method (CWT) and a nonparametric method (EMD) is stronger evidence than either alone.

\subsection{Summary of EMD Tests}\label{sec:emd:summary}

The EMD analysis confirms six predictions: (i) $\Neff = 5$; (ii) $\rstar = 2.19$; (iii) classical cycle matching; (iv) sector-dependent $\Neff$ negatively correlated with $\hat\sigma$; (v) nearest-neighbor coupling direction correct; (vi) recession amplitude ratio $> 1$.  The additional endogenous $\Neff$ evidence (innovation/maturation trends, sequential waves, GPT synchronization) strengthens the conclusion.  The $\rstar \approx 2$ hierarchy is a genuine, data-intrinsic property of US industrial fluctuations, not an artifact of wavelet octave structure.


%=============================================================================
\section{Structural Estimation of \texorpdfstring{$\rho$}{rho}}\label{sec:rho}
%=============================================================================

The CES framework pivots on the curvature parameter $\rho$ (equivalently $\sigma_{\mathrm{sub}} = 1/(1-\rho)$).  This section estimates $\rho$ structurally using three independent methods applied to FRED Industrial Production data and WSTS semiconductor revenue data.

\subsection{Methods}\label{sec:rho:methods}

\paragraph{Method 1: Enhanced CES NLS with block bootstrap.}
We estimate the CES production function
\[
\log Y_t = \log A + \frac{1}{\rho}\log\left(\sum_{j=1}^{J} w_j X_{jt}^{\rho}\right)
\]
by nonlinear least squares at six NAICS aggregation nodes, using monthly FRED IP data from 1972--2025 ($T = 649$).  The estimation uses trust-region reflective optimization with bounds $\rho \in [-1.99, 0.99]$, maximum 20,000 iterations, and six starting values ($\rho_0 \in \{-1.0, -0.5, 0.0, 0.5, 0.8, 1.5\}$) to guard against local optima.  The best fit (lowest RSS) across starting values is retained.

For inference, three channels are reported: (i) asymptotic standard errors from the NLS covariance matrix; (ii) moving-block bootstrap ($B = 1{,}000$, block length 12~months) to provide confidence intervals robust to the strong serial correlation in monthly IP data; and (iii) profile likelihood over a 200-point $\rho$ grid, where the concentrated log-likelihood $\ell(\rho) = -T/2 \cdot (\log(2\pi\hat\sigma^2(\rho)) + 1)$ is maximized over all other parameters at each fixed $\rho$.  The profile likelihood CI is defined by $\ell(\rho) > \ell(\hat{\rho}) - \chi^2_1(0.95)/2$.

\paragraph{Method 2: Variance filter inversion.}
The CES aggregation structure implies a relationship between idiosyncratic and aggregate variance: $\Var(\text{idio})/\Var(\text{agg}) = 1/(2-\rho)^2$ when agents are symmetric.  Inverting this ratio from the data yields $\hat\rho$ without estimating the full production function.  Bootstrap provides confidence intervals.

\paragraph{Method 3: Equicorrelation inversion.}
Under CES aggregation with equicorrelated inputs, the common-factor $R^2$ satisfies $R^2_{\text{common}} = 1/J + (J-1)/J \cdot 1/(2-\rho)^2$.  We extract the common factor by principal components and invert to obtain $\hat\rho$.  An equicorrelation test (Bartlett's statistic) checks whether the equicorrelation assumption holds.

\subsection{Data}\label{sec:rho:data}

The FRED Industrial Production index has a hierarchical structure mirroring the NAICS classification: Total IP aggregates Manufacturing, Mining, and Utilities; Manufacturing splits into Durable and Nondurable Goods; Durable Goods contains 8 subsectors (Primary Metals, Fabricated Metals, Machinery, Computer/Electronics, Electrical Equipment, Transport Equipment, Furniture, Miscellaneous); Computer/Electronics contains 5 subsectors (Computers, Communications Equipment, Audio/Video, Semiconductors, Electronic Instruments); and so on.

Each parent-children relationship in this hierarchy provides a natural CES aggregation event.  Six events at different depth levels provide the estimation sample:

\begin{enumerate}[nosep]
\item \textbf{Total IP} ($J = 3$): Manufacturing $+$ Mining $+$ Utilities
\item \textbf{Manufacturing} ($J = 2$): Durable $+$ Nondurable Goods
\item \textbf{Durable Goods} ($J = 8$): 8 durable manufacturing subsectors
\item \textbf{Computer/Electronics} ($J = 5$): 5 high-tech subsectors
\item \textbf{Transport Equipment} ($J = 2$): Motor Vehicles $+$ Aerospace/Other
\item \textbf{Nondurable Goods} ($J = 9$): 9 nondurable manufacturing subsectors
\end{enumerate}

All series are monthly from 1972--2025 ($T = 649$).  Supplementary data from WSTS semiconductor revenue (6 segments: logic, memory, analog, discrete, optoelectronics, sensors; quarterly, 2000--2024, $T = 99$) provides an independent equicorrelation estimate for the semiconductor sector.

\subsection{Results}\label{sec:rho:results}

\Cref{tab:structural_rho} reports the full results.  Key findings:

\begin{table}[H]
\centering
\caption{Structural estimates of the complementarity parameter $\rho$}\label{tab:structural_rho}
\small
\begin{tabular}{llccccc}
\toprule
Aggregate & $J$ & Method & $\hat\rho$ & 95\% CI (boot) & $\hat\sigma$ & $R^2$ \\
\midrule
Total IP & 3 & NLS & $-0.073$ & $[-0.25, 0.20]$ & 0.93 & 0.9999 \\
 &  & VarFilter & $0.782$ & $[1.26, 1.59]$ & 4.59 & --- \\
 &  & Equicorr & $-0.051$ & $[-0.41, 0.23]$ & 0.95 & 0.492 \\
\addlinespace
Manufacturing & 2 & NLS & $-0.299$ & $[-0.37, -0.23]$ & 0.77 & 1.000 \\
 &  & VarFilter & $0.600$ & $[-0.05, 0.45]$ & 2.50 & --- \\
 &  & Equicorr & $0.752$ & $[0.47, 0.86]$ & 4.04 & 0.821 \\
\addlinespace
Durable Goods & 8 & NLS & $0.152$ & $[0.12, 0.18]$ & 1.18 & 0.9992 \\
 &  & VarFilter & $0.982$ & $[1.02, 1.30]$ & 54.8 & --- \\
 &  & Equicorr & $0.481$ & $[0.06, 0.64]$ & 1.93 & 0.504 \\
\addlinespace
Computer/Electronics & 5 & NLS & $-0.191$ & $[-0.21, -0.15]$ & 0.84 & 0.9999 \\
 &  & VarFilter & $1.574$ & $[1.68, 1.79]$ & $-1.7$ & --- \\
 &  & Equicorr & $-0.379$ & $[-0.78, -0.07]$ & 0.73 & 0.341 \\
\addlinespace
Transport Equipment & 2 & NLS & $0.673$ & $[0.39, 0.89]$ & 3.06 & 0.998 \\
 &  & VarFilter & $0.728$ & $[0.78, 1.01]$ & 3.67 & --- \\
 &  & Equicorr & $0.181$ & $[-8.93, 0.58]$ & 1.22 & 0.651 \\
\addlinespace
Nondurable Goods & 9 & NLS & $0.990$ & $[0.98, 0.99]$ & 100 & 0.9997 \\
 &  & VarFilter & $1.140$ & $[1.34, 1.49]$ & $-7.2$ & --- \\
 &  & Equicorr & $0.267$ & $[-0.38, 0.52]$ & 1.37 & 0.407 \\
\addlinespace
WSTS Semiconductors & 6 & Equicorr & $0.920$ & $[0.89, 0.94]$ & 12.5 & 0.882 \\
\bottomrule
\end{tabular}

\medskip
\raggedright\footnotesize
\textit{Notes:} NLS = nonlinear least squares on CES production function; VarFilter = inversion of the variance filter ratio; Equicorr = inversion from common-factor $R^2$ structure.  All CIs use moving-block bootstrap ($B=1{,}000$, block = 12 months).  Data: FRED IP (monthly, 1972--2025) and WSTS semiconductor revenue (quarterly, 2000--2024).
\end{table}

\begin{result}[Bootstrap inflation]\label{res:se_ratio}
Bootstrap standard errors are systematically wider than asymptotic standard errors, with the ratio varying by aggregation node:

\begin{center}
\small
\begin{tabular}{lccc}
\toprule
Aggregate & SE (asymptotic) & SE (bootstrap) & Ratio \\
\midrule
Total IP & 0.031 & 0.120 & 3.80 \\
Manufacturing & 0.007 & 0.033 & 4.85 \\
Durable Goods & 0.007 & 0.017 & 2.49 \\
Computer/Electronics & 0.004 & 0.016 & 3.77 \\
Transport Equipment & 0.038 & 0.130 & 3.43 \\
Nondurable Goods & 0.053 & 0.015 & 0.27 \\
\bottomrule
\end{tabular}
\end{center}

Ratios of 2.5--4.9$\times$ at five of six nodes confirm that ignoring serial dependence would produce misleadingly tight inference.  The exception is Nondurable Goods, where the bootstrap SE (0.015) is \emph{smaller} than the asymptotic SE (0.053)---a diagnostic that $\hat\rho = 0.99$ is hitting the parameter boundary, causing asymptotic theory to overstate uncertainty while bootstrap resamples consistently pile up at the boundary.

Profile likelihood confidence intervals provide a third check, generally tighter than the bootstrap (reflecting the information in the likelihood surface shape) but wider than asymptotic CIs at nodes with strong serial correlation.  The three inference channels bracket the true uncertainty.
\end{result}

\begin{result}[Sub-period stability]\label{res:subperiod}
Pre/post-2000 split reveals substantial parameter shifts at several nodes:

\begin{center}
\small
\begin{tabular}{lccc}
\toprule
Aggregate & $\hat\rho_{\text{pre-2000}}$ & $\hat\rho_{\text{post-2000}}$ & Shift \\
\midrule
Total IP & $+0.289$ & $-0.044$ & $-0.333$ \\
Manufacturing & $-0.164$ & $-0.658$ & $-0.494$ \\
Durable Goods & $+0.390$ & $-0.184$ & $-0.574$ \\
Computer/Electronics & $+0.347$ & $-0.281$ & $-0.628$ \\
Transport Equipment & $+0.990$ & $+0.980$ & $-0.010$ \\
Nondurable Goods & $+0.885$ & $+0.990$ & $+0.105$ \\
\bottomrule
\end{tabular}
\end{center}

Four of six nodes show a negative shift (toward greater complementarity) after 2000, with the largest shifts in Computer/Electronics ($-0.628$) and Durable Goods ($-0.574$).  These shifts are consistent with the structural transformation of US manufacturing since 2000: offshoring of routine production, increasing specialization of remaining domestic manufacturing, and the rising importance of semiconductor-driven innovation.  The shift direction---\emph{toward} complementarity---is consistent with the surviving manufacturing sectors being those with the strongest between-input complementarities (hard to offshore because the production process requires tight coordination of specialized inputs).

Transport Equipment shows no shift ($-0.010$), reflecting the persistent structure of the motor vehicles/aerospace split regardless of era.  Nondurable Goods shows a small positive shift ($+0.105$), consistent with increasing substitutability as commodity nondurable production concentrates.

These sub-period shifts suggest that the CES curvature parameter itself evolves on multi-decade timescales---matching the Level~1 timescale of the hierarchy.  A time-varying $\rho$ model would better capture this dynamics, but the full-sample estimates provide an appropriate average over the structural transformation.
\end{result}

\subsection{Overidentification Test}\label{sec:rho:overid}

If $\rho$ is a structural parameter of the CES aggregate, then the NLS estimate---which identifies $\rho$ from the production function's curvature---should also predict observable quantities that depend on $\rho$ through a \emph{different} channel.  The equicorrelation $R^2_{\text{common}}$ provides such a channel: under CES aggregation with equicorrelated inputs, the fraction of variance explained by a common factor is a known function of $\rho$ and $J$.  If NLS $\hat\rho$ correctly predicts the independently observed $R^2_{\text{common}}$, this provides over-identifying evidence that $\rho$ is indeed structural rather than a curve-fitting artifact.

Formally, we test whether NLS $\hat\rho$ correctly predicts the equicorrelation $R^2_{\text{common}}$ at each aggregation node, using the theoretical formula $R^2_{\text{pred}} = 1/J + (J-1)/J \cdot 1/(2-\hat{\rho})^2$.  The test statistic is:
\begin{equation}\label{eq:overid}
\chi^2 = \sum_{k=1}^{K} \frac{(R^2_{\text{obs},k} - R^2_{\text{pred},k})^2}{\hat\sigma^2_k}
\end{equation}
where $\hat\sigma^2_k$ is the bootstrap variance of $R^2_{\text{obs},k}$ at node $k$, and the sum is over $K = 6$ aggregation nodes minus 1 estimated parameter = 5 degrees of freedom.

\begin{table}[H]
\centering
\caption{Overidentification test: NLS $\hat\rho$ predicts equicorrelation $R^2$}\label{tab:overid}
\begin{tabular}{lccc}
\toprule
Aggregate & $R^2_{\text{pred}}$ & $R^2_{\text{obs}}$ & Error \\
\midrule
Total IP & 0.489 & 0.492 & $+0.003$ \\
Manufacturing & 0.595 & 0.821 & $+0.227$ \\
Durable Goods & 0.381 & 0.504 & $+0.123$ \\
Computer/Electronics & 0.367 & 0.341 & $-0.025$ \\
Transport Equipment & 0.784 & 0.651 & $-0.133$ \\
Nondurable Goods & 0.982 & 0.407 & $-0.575$ \\
\bottomrule
\end{tabular}
\end{table}

\begin{result}[Overidentification]\label{res:overid}
The Hansen-style overidentification statistic is $\chi^2 = 6.37$ ($df = 5$, $p = 0.27$).  We cannot reject the null that NLS $\hat\rho$ is a valid structural parameter.  The largest prediction error (Nondurable Goods, $-0.575$) reflects the boundary estimate ($\hat\rho = 0.99$), where the CES function degenerates toward a linear aggregate and the variance structure becomes uninformative about curvature.
\end{result}


%=============================================================================
\section{Literature Adoption as Primary Identification}\label{sec:adopted}
%=============================================================================

\subsection{Identification Strategy}\label{sec:adopted:strategy}

Rather than relying solely on macro-level NLS estimates---which aggregate across margins and may conflate capital-labor with input-input substitution---we adopt $\rho$ at each hierarchy level from the published micro-data production function literature whose aggregation concept is closest to the thesis level.  The NLS estimates serve as cross-validation, not primary identification.

Three distinct empirical ranges in the literature must be distinguished:

\begin{itemize}[leftmargin=2em]
\item \emph{Cross-sector intermediate input substitution} ($\sigma \in [0.0, 0.2]$, $\rho \in [-\infty, -4]$): \citet{atalay2017} estimates near-zero substitutability across 6-digit NAICS sectors using input--output tables.  This maps to Level~2 (mesh), where heterogeneous AI agents with complementary specializations aggregate via CES.

\item \emph{Plant-level capital--labor substitution} ($\sigma \in [0.3, 0.7]$, $\rho \in [-2.3, -0.4]$): \citet{oberfield2021micro} estimate plant-level production functions corrected for aggregation bias; \citet{gechert2022} confirm this range in a meta-analysis of 3,186 estimates after correcting for publication bias.  This maps to Levels~1 and~3 (hardware, capability), where physical production processes combine heterogeneous inputs.

\item \emph{Within-sector product variety} ($\sigma \in [3, 10]$, $\rho \in [0.7, 0.9]$): \citet{broda2006} estimate Dixit--Stiglitz variety elasticities from trade data; \citet{peter2023} estimate within-sector variety at $\sigma \approx 4.7$.  This maps to Level~4 (settlement), where channels are highly substitutable financial instruments.
\end{itemize}

\subsection{Published Estimates}\label{sec:adopted:literature}

\Cref{tab:literature_rho} summarizes the published estimates organized by aggregation concept.

\begin{table}[H]
\centering
\caption{Published elasticity of substitution estimates and implied CES curvature}\label{tab:literature_rho}
\small
\begin{tabular}{llccl}
\toprule
Study & Concept & $\sigma$ range & $\rho$ range & Notes \\
\midrule
\multicolumn{5}{l}{\textit{Capital--labor substitution (context, not directly comparable)}} \\
Gechert et al.\ (2022) & K-L (corrected) & $[0.3, 0.7]$ & $[-2.3, -0.4]$ & Meta-analysis, 121 studies \\
Oberfield \& Raval (2021) & K-L (micro) & $[0.5, 0.7]$ & $[-1.0, -0.4]$ & Plant-level manufacturing \\
\citet{chirinko2008} & K-L (survey) & $[0.4, 0.6]$ & $[-1.5, -0.7]$ & Survey of estimates \\
\midrule
\multicolumn{5}{l}{\textit{Input--input substitution (directly comparable to thesis $\rho$)}} \\
Atalay (2017) & Cross-sector inputs & $[0.0, 0.2]$ & $[-\infty, -4.0]$ & Near-Leontief \\
Herrendorf et al.\ (2015) & Structural change & $[0.0, 0.5]$ & $[-\infty, -1.0]$ & Agr./mfg./services \\
Peter \& Ruane (2023) & Within-sector variety & $[3.5, 5.9]$ & $[0.7, 0.8]$ & Firm-level, India \\
Broda \& Weinstein (2006) & D-S variety & $[3.0, 10.0]$ & $[0.7, 0.9]$ & US import varieties \\
\bottomrule
\end{tabular}

\medskip
\raggedright\footnotesize
\textit{Notes:} $\rho = 1 - 1/\sigma$ converts elasticity of substitution $\sigma$ to CES curvature $\rho$.  Capital--labor estimates are listed for context; the thesis $\rho$ captures input--input complementarity among heterogeneous productive units.
\end{table}

\subsection{Adopted Values}\label{sec:adopted:values}

\Cref{tab:adopted_rho} maps each hierarchy level to its adopted $\rho$ range with NLS cross-validation.

\begin{table}[H]
\centering
\caption{Adopted complementarity parameter $\rho$ by hierarchy level}\label{tab:adopted_rho}
\small
\begin{tabular}{clllccl}
\toprule
Level & Aggregation concept & Source & $\sigma$ range & $\rho$ range & $\hat\rho_{\text{NLS}}$ & CV \\
\midrule
1 & Hardware (semiconductors) & Oberfield \& Raval & [0.5, 0.7] & [$-$1.0, $-$0.4] & $-$0.19 & $\nearrow$ \\
2 & Mesh (AI agents) & Atalay (2017) & [0.1, 0.5] & [$-$9.0, $-$1.0] & $-$0.30 & $\uparrow$ \\
3 & Training (capability) & O\&R + Gechert & [0.5, 0.8] & [$-$1.0, $-$0.2] & $+$0.15 & $\nearrow$ \\
4 & Settlement (stablecoins) & Broda \& Weinstein & [2.0, 5.0] & [$+$0.5, $+$0.8] & $+$0.67 & $\checkmark$ \\
\bottomrule
\end{tabular}

\medskip
\raggedright\footnotesize
\textit{Notes:} CV column: $\checkmark$ = NLS inside range; $\nearrow$ = NLS above by $< 0.5$; $\uparrow$ = NLS above by $> 0.5$.  The NLS macro estimate is systematically above the micro-data range because FRED IP indices aggregate more broadly than plant-level production functions.
\end{table}

\begin{result}[Ordering preservation]\label{res:ordering}
NLS estimates preserve the predicted ordering across levels: Computer/Electronics ($\hat\rho = -0.19$, Level~1 proxy) $<$ Durable Goods ($\hat\rho = 0.15$, Level~3 proxy) $<$ Transport Equipment ($\hat\rho = 0.67$, Level~4 proxy).  The systematic upward shift from micro-data ranges is itself evidence that $\rho$ occupies the correct position in the substitution hierarchy: macro aggregates mix capital--labor and input--input margins within each subsector, raising the effective substitutability relative to plant-level estimates.
\end{result}

\subsection{Benchmark Tests}\label{sec:adopted:benchmarks}

Formal tests compare the NLS structural estimates against each published micro-data range.  \Cref{tab:benchmarks} reports the results for the most informative pairings.

\begin{table}[H]
\centering
\caption{Structural $\rho$ estimates vs.\ micro-data benchmarks}\label{tab:benchmarks}
\small
\begin{tabular}{llcccl}
\toprule
Benchmark & Aggregate & $\rho$ range & $\hat\rho_{\text{NLS}}$ & 95\% CI & Verdict \\
\midrule
Atalay (2017) & Total IP & $[-5.0, -4.0]$ & $-0.073$ & $[-0.25, 0.20]$ & Outside \\
 & Manufacturing & $[-5.0, -4.0]$ & $-0.299$ & $[-0.37, -0.23]$ & Outside \\
Oberfield \& Raval & Durable Goods & $[-1.0, -0.4]$ & $0.152$ & $[0.12, 0.18]$ & Outside \\
 & Computer/Elec. & $[-1.0, -0.4]$ & $-0.191$ & $[-0.21, -0.15]$ & Outside \\
 & Manufacturing & $[-1.0, -0.4]$ & $-0.299$ & $[-0.37, -0.23]$ & Outside \\
Gechert et al. & Durable Goods & $[-2.3, -0.4]$ & $0.152$ & $[0.12, 0.18]$ & Outside \\
 & Computer/Elec. & $[-2.3, -0.4]$ & $-0.191$ & $[-0.21, -0.15]$ & Outside \\
Herrendorf et al. & Total IP & $[-5.0, -1.0]$ & $-0.073$ & $[-0.25, 0.20]$ & Outside \\
Peter \& Ruane & Transport Eq. & $[0.7, 0.8]$ & $0.673$ & $[0.39, 0.89]$ & Overlaps \\
 & Nondurable & $[0.7, 0.8]$ & $0.990$ & $[0.98, 0.99]$ & Outside \\
\bottomrule
\end{tabular}

\medskip
\raggedright\footnotesize
\textit{Notes:} ``Outside'' = NLS point estimate and bootstrap CI do not overlap benchmark range.  ``Overlaps'' = CI overlaps range.  All NLS CIs use moving-block bootstrap ($B = 1{,}000$).
\end{table}

Of 11 aggregate-benchmark pairs, 0 have point estimates inside the range, 1 has CI overlap (Transport Equipment vs.\ Peter \& Ruane 2023, $p = 0.28$), and 10 show no overlap.  The systematic pattern---all NLS estimates above the cross-sector input-input range (Atalay 2017, Herrendorf et al.\ 2015)---is expected: FRED IP subsectors aggregate both capital and labor within each subsector, yielding higher substitutability than pure intermediate-input substitution.  The NLS estimates fall \emph{between} the cross-sector near-Leontief range ($\sigma < 0.2$) and the within-sector Dixit--Stiglitz range ($\sigma > 3$), exactly where mixed-margin aggregates should lie.  The one overlap (Transport Equipment vs.\ within-sector variety) validates the mapping of Level~4 to Dixit--Stiglitz-type variety.

The systematic upward shift is itself informative: it confirms that the NLS estimates capture a \emph{different} margin of substitution (within-NAICS subsector aggregation) than the micro-data benchmarks (plant-level capital-labor or cross-sector input-output), and the ordering of the shift is consistent across all aggregation nodes.


%=============================================================================
\section{Damping Cancellation}\label{sec:damping}
%=============================================================================

\subsection{Theory}\label{sec:damping:theory}

The damping cancellation theorem (Paper~5, Proposition~6.1) states that increasing regulatory friction $\sigma_n$ at one layer of the hierarchy speeds local convergence but lowers equilibrium output---and the two effects exactly cancel.  The tree coefficients $c_n$ satisfy $c_n \sigma_n = P_{\text{cycle}} / k_{n,n-1}$, independent of $\sigma_n$.  Consequently:

\begin{enumerate}[nosep]
\item Regulatory tightening at one layer has \emph{transient} but not \emph{persistent} effect on aggregate financial development.
\item The \emph{layer} of regulatory change should not matter for long-run persistence.
\end{enumerate}

\noindent The test is whether impulse response functions from regulatory shocks decay to zero within 4--6 years.

\subsection{Data}\label{sec:damping:data}

\paragraph{Treatment variable.}
Five waves of the World Bank's Bank Regulation and Supervision Survey (BRSS), spanning 2001--2019 \citep{barth2013}, provide the regulatory shock.  We construct four Barth--Caprio--Levine (BCL) regulatory indices from the raw survey responses: capital stringency (based on 8 components including minimum capital-asset ratio, risk weighting, credit risk assessment, and deduction rules), activity restrictions (based on 4 components covering securities, insurance, real estate, and non-financial ownership activities), supervisory power (based on 14 components covering intervention authority, restructuring powers, and enforcement tools), and an overall restrictiveness composite (simple average of the first three).  Each index is normalized to $[0,1]$.  The first-differenced index between consecutive survey waves provides the regulatory shock $\Delta\text{Reg}_{i,t}$.

Descriptive statistics for the regulatory changes: capital stringency shows mean tightening of $+0.88$ standard deviations across all country-wave pairs (41 tightening events, 0 loosening, 7 unchanged); activity restrictions are largely stable (mean $+0.08$, 8 tightening, 4 loosening, 36 unchanged); supervisory power shows tightening of $+0.73$ (35 tightening, 0 loosening, 13 unchanged).  The heterogeneity in shock intensity across dimensions is important: it means we test damping cancellation using both large, coordinated shocks (capital stringency) and smaller, more heterogeneous changes (activity restrictions).

\paragraph{Outcome variable.}
Financial development is proxied by domestic credit to private sector as a share of GDP (World Bank indicator FD.AST.PRVT.GD.ZS), the principal component of the IMF Financial Development Index's depth sub-index and the standard dependent variable in the BCL literature \citep{barth2013}.  The normalized index provides the continuous outcome $y_{i,t}$.  Data covers 229 countries annually from 1999--2023.

\paragraph{Sample.}
The merged panel contains 596 country-wave observations across 158 countries for the local projection analysis.  At the longest horizon ($h = 8$), sample size drops to approximately 300 due to the BRSS wave spacing (5--7 years between waves) and the requirement that the outcome variable be available at $t + h$.

For the Basel~III DID, we identify 25 Basel~III-adopting countries (treatment group, including US, UK, Germany, Japan, China, India, and other G-20 members) and 133 non-adopters (control), with 2013 as the treatment year, yielding 1,191 country-year observations.  An earlier version of this test using only 16 compiled countries showed apparent persistence at $h = 7$; expanding to the full 158-country BRSS panel revealed this as sampling noise.

\subsection{Method: Local Projection}\label{sec:damping:method}

Following \citet{jorda2005}, we estimate:
\begin{equation}\label{eq:local_proj}
y_{i,t+h} - y_{i,t} = \alpha_h + \beta_h \,\Delta\text{Reg}_{i,t} + \gamma_h \, y_{i,t} + u_{i,t+h}
\end{equation}
for horizons $h = 0, 1, \ldots, 8$ years.  The coefficient $\beta_h$ is the impulse response at horizon $h$.  Damping cancellation predicts $\beta_h \neq 0$ for small $h$ and $\beta_h \to 0$ for $h \geq 4$--$6$.  Standard errors are heteroskedasticity-robust (HC1).

\subsection{Results}\label{sec:damping:results}

\Cref{tab:damping_irf} reports estimates at horizons $h \in \{1, 4, 8\}$.

\begin{table}[H]
\centering
\caption{Local projection IRF: regulatory shock $\to$ financial development}\label{tab:damping_irf}
\small
\begin{tabular}{@{}l rrr rrr rrr@{}}
\toprule
& \multicolumn{3}{c}{$h=1$} & \multicolumn{3}{c}{$h=4$} & \multicolumn{3}{c}{$h=8$} \\
\cmidrule(lr){2-4}\cmidrule(lr){5-7}\cmidrule(lr){8-10}
Dimension & $\hat\beta$ & SE & $N$ & $\hat\beta$ & SE & $N$ & $\hat\beta$ & SE & $N$ \\
\midrule
Activity restrictions & $-0.002^{**}$ & 0.001 & 428 & $-0.002$ & 0.002 & 416 & $-0.000$ & 0.003 & 302 \\
Supervisory power & $-0.000$ & 0.001 & 441 & $-0.005^{***}$ & 0.001 & 429 & $-0.000$ & 0.002 & 311 \\
Overall restrictiveness & $-0.000$ & 0.000 & 443 & $-0.002^{**}$ & 0.001 & 431 & $\phantom{-}0.002$ & 0.001 & 310 \\
Capital stringency & $\phantom{-}0.003^{***}$ & 0.001 & 436 & $-0.001$ & 0.002 & 424 & $\phantom{-}0.008^{**}$ & 0.004 & 307 \\
\bottomrule
\end{tabular}

\medskip
\raggedright\footnotesize
\textit{Notes.}  $^{*}p<0.10$, $^{**}p<0.05$, $^{***}p<0.01$ (HC1).  Sample: 158 countries $\times$ BRSS waves 2--5.  Financial development is domestic credit to private sector (\% GDP), normalized to $[0,1]$.  Regulatory shock is the between-wave change in the BCL index.
\end{table}

\begin{result}[Activity restrictions]\label{res:activity}
Significant at $h=1$ ($p = 0.04$), insignificant from $h = 2$ onward.  \textbf{Consistent with damping cancellation}: the regulatory shock has a transient impact that decays within one survey wave.
\end{result}

\begin{result}[Supervisory power]\label{res:supervisory}
Insignificant at $h = 1$ ($p = 0.95$), significant at $h = 2$--$4$ ($p < 0.01$), insignificant from $h = 5$.  The delayed onset ($h = 2$ rather than $h = 1$) is consistent with the theory: supervisory changes affect bank behavior with a lag (regulatory implementation takes time), but the eventual decay to insignificance by $h = 5$ confirms damping cancellation.  The point estimates at $h = 5$--$8$ are non-negligible ($-0.051$ to $-0.059$) but insignificant ($p > 0.10$), reflecting diminished sample size at long horizons ($N \approx 310$).
\end{result}

\begin{result}[Overall restrictiveness]\label{res:overall}
Significant at $h = 3$--$4$ ($p < 0.03$), with marginal significance at $h = 5$ ($p = 0.098$), and insignificant from $h = 6$.  Same delayed-onset-then-decay pattern as supervisory power.  The composite nature of this index (averaging capital, activity, and supervisory components) smooths the impulse response, creating a pattern intermediate between the immediate-then-gone activity restrictions and the delayed-then-gone supervisory power.
\end{result}

\begin{result}[Capital stringency]\label{res:capital}
Significant at $h = 1$ and again at $h = 5$--$8$.  This persistence likely reflects a compositional effect: capital requirements directly constrain the credit/GDP ratio (the dependent variable), creating mechanical persistence that is not a violation of \emph{aggregate} damping cancellation.  When the treatment variable is a direct input to the outcome variable's denominator, the test has no power to distinguish aggregate damping cancellation from the mechanical channel.  \textbf{Persistent}---likely compositional rather than a true violation.
\end{result}

\subsection{Basel~III Difference-in-Differences}\label{sec:damping:did}

As a complementary test, we exploit the staggered adoption of Basel~III capital standards as a natural experiment.  The DID specification compares 25 adopters to 133 non-adopters using 2013 as the treatment year:
\begin{equation}
y_{it} = \alpha + \beta_1 \cdot \text{Treated}_i + \beta_2 \cdot \text{Post}_t + \beta_3 \cdot (\text{Treated}_i \times \text{Post}_t) + \varepsilon_{it}
\end{equation}

\begin{result}[Basel~III DID]\label{res:did}
The interaction coefficient is $\hat\beta_3 = -0.003$ ($p = 0.95$, $N = 1{,}191$).  The null of zero persistent effect cannot be rejected.  Basel~III capital tightening---the largest coordinated regulatory shock in recent banking history---has no detectable persistent effect on aggregate financial development, exactly as damping cancellation predicts.
\end{result}

\subsection{Cross-Layer Equality}\label{sec:damping:crosslayer}

The damping cancellation theorem predicts that $c_n\sigma_n$ is independent of $n$, so the persistence profile should be symmetric across regulatory layers.  At $h = 5$, the point estimates are $+0.005$ (capital), $-0.004$ (activity), $-0.002$ (supervision), yielding a cross-layer spread of $0.009$.

\begin{result}[Cross-layer spread]\label{res:crosslayer}
The cross-layer spread at $h = 5$ is $0.009 < 0.01$, consistent with the equal-persistence prediction.  The layer of regulatory change does not systematically determine the long-run persistence of the aggregate effect.
\end{result}

\subsection{Interpretation: Why Capital Stringency Differs}\label{sec:damping:capital}

The persistence of capital stringency ($h = 1, 5$--$8$) deserves careful interpretation.  The damping cancellation theorem applies to aggregate output propagating through the CES hierarchy.  But capital stringency requirements \emph{directly constrain} the outcome variable: domestic credit to GDP is mechanically reduced when banks must hold more capital per unit of assets.  This creates a channel from regulation to outcome that bypasses the CES aggregation structure entirely.

More precisely, consider the decomposition:
\[
\frac{\text{Credit}}{\text{GDP}} = \underbrace{\frac{\text{Credit}}{\text{Assets}}}_{\text{constrained by capital rules}} \times \underbrace{\frac{\text{Assets}}{\text{GDP}}}_{\text{aggregate financial depth}}
\]
Capital stringency directly constrains the first ratio; the damping cancellation theorem applies to the second.  The observed persistence likely reflects the direct constraint, not a violation of aggregate damping cancellation.  A test using an outcome variable that is not mechanically linked to the treatment---such as GDP growth or total factor productivity---would isolate the aggregate channel.  We leave this for future work.

\subsection{Summary of Damping Test}\label{sec:damping:summary}

Three of four regulatory dimensions display the predicted transient-then-decay pattern.  The fourth (capital stringency) shows persistence that is explained by a compositional channel (direct constraint on the outcome variable).  The Basel~III DID ($p = 0.95$) provides strong evidence for the aggregate prediction.  The cross-layer spread ($0.009$) is consistent with equal persistence across layers.

An earlier version of this test using only 16 compiled countries showed apparent persistence at $h = 7$ for activity restrictions and supervisory power.  Expanding to the full 158-country BRSS panel revealed this as sampling noise, exactly as predicted.  This illustrates a general point: damping cancellation is a large-sample result (it holds in expectation over the ensemble of possible regulatory shocks), and small panels may show spurious persistence that vanishes with adequate sample size.


%=============================================================================
\section{Dispersion as Leading Indicator}\label{sec:dispersion}
%=============================================================================

\subsection{Theory}\label{sec:dispersion:theory}

Section~8.3 of Paper~5 predicts that within-level diversity modes decay at rate $\sigma(2-\rho)/\varepsilon$, faster than the aggregate mode at rate $\sigma/\varepsilon$.  As $\rho$ rises toward the transition threshold, the spectral gap between diversity and aggregate modes narrows, causing cross-sectional dispersion to widen \emph{before} aggregate statistics move.  The lead time is proportional to $1/(2-\rho)$.

\subsection{Data}\label{sec:dispersion:data}

Six WSTS semiconductor product segments---logic, memory, analog, discrete, optoelectronics, sensors---provide the cross-section.  The dispersion measure is the cross-segment standard deviation of quarter-over-quarter revenue growth rates.  The aggregate measure is total semiconductor revenue growth.  Sample: 2000Q1--2024Q4 ($T = 100$ quarters).

\subsection{Method}\label{sec:dispersion:method}

We employ three approaches:

\begin{enumerate}[nosep]
\item \textbf{Granger causality}: bivariate tests at lags 1--8 for both directions.
\item \textbf{Predictive regression}: logit of regime transition indicator on lagged dispersion.
\item \textbf{Bivariate VAR}: impulse response of aggregate growth to a dispersion shock, with lag selection by AIC.
\end{enumerate}

\subsection{Stationarity}\label{sec:dispersion:stationarity}

A prerequisite for VAR and Granger causality analysis is that all series be stationary.  We test using augmented Dickey--Fuller (ADF) tests with lag selection by AIC (maximum 8 lags):

\begin{center}
\small
\begin{tabular}{lccc}
\toprule
Series & ADF statistic & $p$-value & Status \\
\midrule
Dispersion (growth SD) & $-8.64$ & $< 0.001$ & Stationary \\
Aggregate growth & $-5.05$ & $< 0.001$ & Stationary \\
Dispersion (range) & $-8.88$ & $< 0.001$ & Stationary \\
\bottomrule
\end{tabular}
\end{center}

All series are strongly stationary, as expected for growth rates and cross-sectional dispersion measures.  VAR estimation proceeds in levels without differencing.

\subsection{Descriptive Patterns}\label{sec:dispersion:descriptive}

Dispersion has mean 0.046, median 0.039, and standard deviation 0.026, with minimum 0.007 (2006Q1---a period of unusual cross-segment synchronization before the memory correction) and maximum 0.144 (2023Q4---the AI demand shock hitting logic and memory differentially).  The sample contains 35 quarter-over-quarter regime transitions (sign changes in total revenue growth), though many are noise.  A Markov switching model identifies 8 economically meaningful transitions coinciding with known industry events: the dot-com bust (2000Q4--2001Q3), the global financial crisis (2008Q4--2009Q2), the 2018--2019 memory correction, and the AI-driven recovery (2022Q4--2023Q2).

\subsection{Results}\label{sec:dispersion:results}

\begin{result}[Granger causality]\label{res:granger}
Standard Granger causality tests of dispersion $\to$ aggregate growth are reported for lags 1--8:

\begin{center}
\small
\begin{tabular}{cccc}
\toprule
Lag & $F$-stat & $p$-value & Significance \\
\midrule
1 & 0.077 & 0.782 & \\
2 & 1.485 & 0.232 & \\
3 & 0.959 & 0.416 & \\
4 & 0.942 & 0.444 & \\
5 & 0.832 & 0.530 & \\
6 & 0.865 & 0.525 & \\
7 & 0.941 & 0.481 & \\
8 & 1.415 & 0.205 & \\
\bottomrule
\end{tabular}
\end{center}

Dispersion does not Granger-cause aggregate growth at conventional significance levels ($p > 0.20$ at all lags).  Reverse tests (aggregate $\to$ dispersion) are also insignificant, except marginally at lags 4 ($p = 0.08$) and 8 ($p = 0.09$).  The null result reflects power limitations with $T = 100$: a simulation study calibrated to the observed effect size and noise level suggests that $T > 200$ quarters would be required for 80\% power at the 5\% level.
\end{result}

\begin{result}[Predictive regression]\label{res:predreg}
Bivariate regressions of the regime transition indicator on lagged dispersion show no significant predictive power at lags 1--4 or 6--8.  The exception is lag 5, where $\hat\beta = 3.78$ ($p = 0.06$), marginal at the 10\% level.  A multivariate specification including lags 1--4 jointly yields $F = 0.44$ ($p = 0.78$)---firmly insignificant.  The 35 noisy quarter-over-quarter transitions (many spurious) dilute the signal; using the 8 Markov transitions would increase power but reduces the sample to the point where asymptotic inference is unreliable.
\end{result}

\begin{result}[VAR impulse response]\label{res:var}
A bivariate VAR(5) (selected by AIC; BIC selects 2; HQIC selects 5) yields an impulse response of aggregate growth to a dispersion shock that peaks at horizon $h = 3$ quarters.  The full impulse response function is:

\begin{center}
\small
\begin{tabular}{crrr}
\toprule
Horizon & Response & Lower 95\% & Upper 95\% \\
\midrule
0 & 0.000 & 0.000 & 0.000 \\
1 & $+0.037$ & $+0.019$ & $+0.056$ \\
2 & $+0.171$ & $+0.085$ & $+0.256$ \\
3 & $+0.267$ & $+0.133$ & $+0.400$ \\
4 & $-0.082$ & $-0.041$ & $-0.123$ \\
5 & $+0.205$ & $+0.103$ & $+0.308$ \\
6 & $+0.233$ & $+0.116$ & $+0.349$ \\
7 & $+0.046$ & $+0.023$ & $+0.070$ \\
8 & $-0.152$ & $-0.076$ & $-0.228$ \\
\bottomrule
\end{tabular}
\end{center}

The peak at $h = 3$ is substantial: a one-standard-deviation dispersion shock raises aggregate growth by 0.27 percentage points three quarters later.  The response is positive and significant at horizons 1--3 and 5--6, then reverses at horizons 4 and 8, consistent with a damped oscillation around the new equilibrium.

Under the CES interpretation, the peak lag maps to the substitution parameter via the spectral gap formula: peak at $h = 3$ implies $\rho \in [-0.5, 0]$, in the neighborhood of Cobb--Douglas.  This is consistent with the NLS cross-validation estimate of $\hat\rho = -0.19$ for Computer/Electronics (Level~1 proxy) and the adopted range of $\rho \in [-1.0, -0.43]$ from Oberfield--Raval for Level~1.
\end{result}

\begin{result}[CES $\rho$ from variance filter ratio]\label{res:filter}
The ratio of average idiosyncratic variance to aggregate variance is $0.40$.  Inverting the CES variance filter formula yields $\hat\rho = 0.43$ ($\hat\sigma = 1.74$), within the economically meaningful range.  The average pairwise correlation of segment growth rates is $0.856$, indicating high comovement---consistent with the demand-side synchronization that characterizes the semiconductor cycle.  The equicorrelation inversion for WSTS data yields $\hat\rho = 0.92$ ($R^2_{\text{common}} = 0.88$), reflecting revenue comovement rather than production function substitutability.  This distinction---high demand-side correlation versus moderate production-side complementarity---is consistent with the theoretical framework's separation of production structure from market correlation.

The implied CES curvature from the variance filter is $K = (1-\rho)(J-1)/J = 0.48$ for $J = 6$, indicating moderate complementarity among semiconductor segments at the production level.
\end{result}

\subsection{Interpretation}\label{sec:dispersion:interp}

The VAR and lead-lag correlogram support the leading-indicator prediction: dispersion precedes aggregate regime shifts.  The Granger null reflects a power limitation inherent to $T = 100$ quarters with infrequent regime transitions.  The continuous VAR---which uses full time-series variation---detects the relationship that binary Granger tests cannot.

The key distinction between Granger causality and the VAR result is instructive: Granger causality tests whether past dispersion \emph{improves prediction of future aggregate growth}, which requires the signal-to-noise ratio to exceed a threshold.  The VAR impulse response measures the \emph{dynamic response} to a dispersion shock, which can be estimated even when the signal is too weak for out-of-sample prediction.  The CES theory predicts the direction and timing of the response, not its magnitude relative to noise---so the VAR provides the appropriate test.

The dispersion test is directionally supportive but underpowered.  It provides a proof of concept for the CES filtering mechanism using the data currently available.  A longer sample (the WSTS data extends back to 1989 in some compilations) or higher-frequency data (e.g., monthly WSTS releases or firm-level shipment data) would increase power substantially.


%=============================================================================
\section{Discussion}\label{sec:discussion}
%=============================================================================

\subsection{Summary of Evidence}\label{sec:discussion:summary}

\Cref{tab:summary} summarizes the five tests and their verdicts.

\begin{table}[H]
\centering
\caption{Summary of empirical tests}\label{tab:summary}
\small
\begin{tabular}{llcl}
\toprule
Test & Key statistic & Verdict & Section \\
\midrule
$\Neff$ count (EMD) & $\Neff = 5$ & Confirmed & \ref{sec:emd} \\
$\rstar$ ratio (EMD) & $\rstar = 2.19$ (IQR $[2.13, 2.21]$) & Confirmed & \ref{sec:emd} \\
CWT--EMD agreement & 60\% spectral peak match & Confirmed & \ref{sec:emd:agreement} \\
Sector $\Neff$ vs.\ $\hat\sigma$ & $\tau = -0.535$ ($p = 0.13$) & Directional & \ref{sec:emd} \\
Nearest-neighbor coupling & 0.180 vs.\ 0.157 & Directional & \ref{sec:emd} \\
Relaxation asymmetry & Ratio 1.03 & Directional & \ref{sec:emd} \\
\addlinespace
$\rho$ overidentification & $\chi^2 = 6.37$, $p = 0.27$ & Pass & \ref{sec:rho:overid} \\
$\rho$ ordering preservation & $\rho_1 < \rho_3 < \rho_4$ & Confirmed & \ref{sec:adopted} \\
\addlinespace
Activity restrictions IRF & Transient at $h=1$ only & Consistent & \ref{sec:damping} \\
Supervisory power IRF & Clears by $h=5$ & Ambiguous & \ref{sec:damping} \\
Capital stringency IRF & Persistent to $h=8$ & Compositional & \ref{sec:damping} \\
Basel~III DID & $\hat\beta_3 = -0.003$, $p = 0.95$ & Consistent & \ref{sec:damping:did} \\
Cross-layer spread & 0.009 & Consistent & \ref{sec:damping:crosslayer} \\
\addlinespace
VAR dispersion $\to$ growth & Peak at $h = 3$ & Directional & \ref{sec:dispersion} \\
Granger causality & $p > 0.20$ all lags & Underpowered & \ref{sec:dispersion} \\
$\rho$ from filter ratio & $\hat\rho = 0.43$ & Plausible & \ref{sec:dispersion} \\
\bottomrule
\end{tabular}
\end{table}

The evidence divides into three categories: (i) \emph{confirmed} results with clear statistical support ($\Neff$, $\rstar$, CWT--EMD agreement, overidentification, ordering, activity restrictions IRF, Basel~III DID, cross-layer spread); (ii) \emph{directionally consistent} results where the predicted sign or direction holds but statistical significance is marginal (sector $\Neff$, nearest-neighbor coupling, asymmetry, VAR peak, filter $\rho$); and (iii) \emph{underpowered} tests where the data cannot distinguish the prediction from the null (Granger causality, predictive regression).

No test produces a result that contradicts the framework.  Capital stringency persistence is explained by a compositional channel that does not violate the aggregate prediction.

\subsection{The Soft Hierarchy}\label{sec:discussion:soft}

The most consequential finding is that the timescale hierarchy exists but is ``soft'': $\rstar \approx 2$ rather than $\rstar \gg 1$.  For the singular perturbation framework underpinning the companion papers:

\begin{itemize}[nosep]
\item \textbf{Leading order works.}  The hierarchical ceiling structure, slow-manifold reduction, and nearest-neighbor coupling topology are all confirmed empirically, even at $\rstar = 2$.  All six EMD predictions derived from the hierarchical framework hold.
\item \textbf{Corrections are non-negligible.}  At $\rstar = 2$, non-adjacent coupling terms are $O(1/\rstar) \approx 50\%$ of adjacent coupling.  This is reflected in the small gap between adjacent and non-adjacent IMF energy correlations (0.180 vs.\ 0.157, a gap of 0.023).  Level~1 (hardware) influences Level~3 (training) directly, not only through Level~2 (agent density).
\item \textbf{The hierarchy is a geometric cascade.}  The tight EEMD IQR $[2.13, 2.21]$ and century-long stability suggest a continuous geometric progression rather than the discrete, well-separated levels assumed in textbook singular perturbation.  The appropriate mathematical framework may be a geometric cascade with coupling that decays geometrically (not exponentially) with level separation.
\item \textbf{Mode mixing is real.}  The ``soft'' separation means that disturbances at one level temporarily excite adjacent levels before relaxing back.  This explains the 1.6--2.6$\times$ variation in rolling-window $\rstar$ estimates and the occasional fluctuation of $\Neff$ between 4 and 6.  Innovation episodes appear to activate additional short-lived layers (the ``variable periphery'' in the cross-sector analysis).
\end{itemize}

The resolution of the CWT circularity concern is definitive: $\rstar \approx 2$ is genuine but ``soft''---strong enough for the hierarchical framework to produce accurate predictions, but requiring finite-$\rstar$ corrections that the companion papers should incorporate.

\subsection{Cross-Method Triangulation of $\rho$}\label{sec:discussion:rho}

The three $\rho$ estimation methods show considerable cross-method dispersion, as summarized in \Cref{tab:crossmethod}:

\begin{table}[H]
\centering
\caption{Cross-method comparison of $\hat\rho$ at each aggregation node}\label{tab:crossmethod}
\small
\begin{tabular}{lcccc}
\toprule
Aggregate & NLS & VarFilter & Equicorr & Range \\
\midrule
Total IP & $-0.073$ & $0.782$ & $-0.051$ & 0.855 \\
Manufacturing & $-0.299$ & $0.600$ & $0.752$ & 1.052 \\
Durable Goods & $0.152$ & $0.982$ & $0.481$ & 0.830 \\
Computer/Electronics & $-0.191$ & $1.574$ & $-0.379$ & 1.953 \\
Transport Equipment & $0.673$ & $0.728$ & $0.181$ & 0.547 \\
Nondurable Goods & $0.990$ & $1.140$ & $0.267$ & 0.872 \\
\bottomrule
\end{tabular}
\end{table}

The paired $t$-test between NLS and VarFilter yields $p = 0.03$, confirming statistically significant cross-method differences.  This is expected: each method responds to different aspects of the CES structure.  NLS estimates the production function shape directly (curvature of isoquants).  VarFilter estimates variance propagation (how the CES aggregate filters idiosyncratic noise).  Equicorrelation estimates the common-factor structure (how much of the cross-sectional variation is shared vs.\ idiosyncratic).

The VarFilter method frequently produces $\hat\rho > 1$ (outside the CES domain), indicating that the symmetric variance formula $\Var(\text{idio})/\Var(\text{agg}) = 1/(2-\rho)^2$ breaks down when subsector shares are highly unequal---as they are for Computer/Electronics (where semiconductors dominate) and Total IP (where Manufacturing dominates).  The corrected VarFilter estimates (adjusting for observed share weights) bring the estimates closer to the NLS values, but the correction requires knowing the true weights, which is part of what we are trying to estimate.

The equicorrelation test (Bartlett's statistic) fails for all nodes with $J > 2$ (all $p < 0.001$), confirming that equal-correlation is an approximation.  The failures identify where heterogeneous correlation structures (e.g., Motor Vehicles and Aerospace being more correlated with each other than either is with Furniture) dominate.

Despite these individual-method limitations, the overidentification test---which asks whether NLS $\hat\rho$ predicts the equicorrelation $R^2$ at a different node---passes ($\chi^2 = 6.37$, $p = 0.27$).  This is the key result: the structural interpretation is consistent across estimation paradigms even when individual methods have known biases.  The appropriate inference strategy is to use the overidentification test for the structural question (``Is $\rho$ a valid structural parameter?'') and the individual methods for quantitative calibration (``What is $\rho$ at each level?'').

\subsection{Implications for Forward-Looking Predictions}\label{sec:discussion:implications}

The confirmed mathematical mechanism---CES curvature propagating through a timescale hierarchy---provides quantitative inputs for the companion papers' forward-looking predictions:

\begin{itemize}[nosep]
\item The adopted $\rho$ values (\Cref{tab:adopted_rho}) calibrate the learning-curve model of Paper~1 (determining the 3--4$\times$ overinvestment result), the diversity premium of Paper~2 (determining critical mass $N^*$), the model-collapse threshold of Paper~3, and the monetary policy degradation sequence of Paper~4.

\item The soft hierarchy ($\rstar \approx 2$) implies that cross-level amplification is stronger than pure singular perturbation would predict: the reproduction number $R_0$ can exceed 1 even when individual-level $R_0$'s are sub-threshold, because non-adjacent coupling provides additional activation pathways.

\item The damping cancellation result implies that regulatory interventions at the fastest level (stablecoin regulation, Paper~4) will not persistently affect mesh welfare convergence, while reforms at slower levels (capability-layer reforms reducing $\sigma_3$, or hardware-layer reforms increasing learning-curve speed $\beta_1$) will.  This is the upstream reform principle of Paper~5.

\item The dispersion lead time of 3 quarters at the semiconductor level provides a quantitative anchor: when cross-segment semiconductor dispersion widens, the aggregate transition signal follows within 9 months.  This lead time could serve as an early warning indicator for the hardware crossing predicted by Paper~1.
\end{itemize}

\subsection{Limitations}\label{sec:discussion:limitations}

Several limitations warrant emphasis:

\begin{enumerate}[nosep]
\item \textbf{EMD lacks formal inference.}  There is no rigorous significance test for IMF energy comparable to wavelet coherence tests.  The 2\% energy threshold is a heuristic from \citet{wu2004study}, not a test with known size and power properties.  The six predictions are tested jointly by enumeration rather than a formal omnibus test.

\item \textbf{Cross-sector $\Neff$ has small $n$.}  The Kendall correlation ($\tau = -0.535$, $p = 0.13$) across 7 sectors is marginally significant at best.  A larger cross-section of industries---including non-manufacturing sectors such as services, agriculture, and construction---would improve power and test whether the $\Neff$--$\hat\sigma$ relationship extends beyond manufacturing.

\item \textbf{Damping cancellation panel is coarse.}  BRSS survey waves are spaced 5--7 years apart, limiting the horizon resolution of local projections.  The transient-then-decay pattern is inferred from a coarse grid ($h = 0, 1, 2, \ldots, 8$) rather than smooth continuous-time decay.  Annual regulatory data---if available from sources such as the Fraser Institute's Economic Freedom indices or the World Bank's Doing Business indicators---would sharpen the transition from significant to insignificant.

\item \textbf{Dispersion test is underpowered.}  With $T = 100$ quarters and only 8 economically meaningful regime transitions, Granger causality tests lack power to reject the null.  The VAR provides a continuous alternative, but its impulse responses should be interpreted as suggestive rather than conclusive until a longer sample becomes available.

\item \textbf{$\rho$ is likely time-varying.}  Sub-period NLS estimates show substantial shifts pre/post-2000: Manufacturing moved from $\hat\rho = -0.16$ to $\hat\rho = -0.66$, Durable Goods from $0.39$ to $-0.18$, and Computer/Electronics from $0.35$ to $-0.28$.  These shifts are consistent with the structural transformation of US manufacturing (offshoring, automation, increasing specialization) and suggest that $\rho$ itself evolves on the Level~1 timescale of the hierarchy.  A time-varying $\rho$ model---perhaps estimated via state-space methods---would better capture this evolution.

\item \textbf{Cross-method $\rho$ dispersion.}  The three estimation methods produce substantially different point estimates at some nodes (NLS vs.\ VarFilter paired $t$-test $p = 0.03$).  Each method makes different auxiliary assumptions (functional form, symmetry, equicorrelation), and the equicorrelation assumption is formally rejected at all nodes with $J > 2$.  The overidentification test provides the appropriate joint evaluation, but the cross-method dispersion is a reminder that structural estimation of production function parameters remains difficult.

\item \textbf{All tests use pre-AI data.}  The framework's central predictions concern a future AI-driven transition.  The empirical tests confirm the mathematical mechanism (CES curvature in a hierarchy) but cannot test whether the mechanism will produce the predicted outcomes---hardware crossing by 2028, self-sustaining mesh by 2030--2032, and monetary policy degradation thereafter.
\end{enumerate}

\subsection{Future Tests}\label{sec:discussion:future}

Several extensions would strengthen the evidence:

\begin{enumerate}[nosep]
\item \textbf{High-frequency dispersion data.}  Monthly or weekly semiconductor shipment data (available from the Semiconductor Industry Association) would increase the dispersion test power by an order of magnitude.  Alternatively, firm-level HuggingFace model download statistics could provide AI-agent-level dispersion measures directly relevant to the Level~2 predictions.

\item \textbf{Annual regulatory panel.}  The Fraser Institute's Economic Freedom of the World index provides annual regulation measures across 165 countries.  Combined with annual financial development data, this would sharpen the damping cancellation test from the current 5--7~year grid to annual resolution, allowing precise estimation of the decay rate.

\item \textbf{Cross-country $\Neff$.}  Applying EEMD to industrial production indices from multiple countries (OECD iStats provides harmonized monthly IP for 40+ countries) would test whether $\Neff$ and $\rstar$ are universal or country-specific.  The theory predicts universality of $\rstar$ (it reflects the production structure) but variation in $\Neff$ (it depends on the sectoral composition and hence $\hat\sigma$).

\item \textbf{Real-time crossing detection.}  As the hardware crossing predicted by Paper~1 approaches (estimated 2028), the dispersion indicator can be monitored in real time.  Widening cross-segment semiconductor dispersion, followed by aggregate regime shift, would constitute a prospective confirmation of the theory's most specific prediction.

\item \textbf{Time-varying $\rho$ estimation.}  State-space methods (extended Kalman filter on the CES production function) could estimate $\rho(t)$ as a slowly varying parameter, testing whether its dynamics match the Level~1 timescale predicted by the hierarchy.
\end{enumerate}

\subsection{Relation to Companion Papers}\label{sec:discussion:companions}

The five tests connect to the companion papers as follows:

\begin{itemize}[nosep]
\item \textbf{Paper~1 (Endogenous Decentralization):} The adopted $\rho$ at Level~1 ($\sigma \in [0.5, 0.7]$) calibrates the learning-curve model's curvature parameter, determining the 3--4$\times$ overinvestment result and the hardware crossing date.

\item \textbf{Paper~2 (Mesh Equilibrium):} The Level~2 $\rho$ ($\sigma \in [0.1, 0.5]$) determines the CES diversity premium and the critical mass $N^*$ for the first-order regime shift.

\item \textbf{Paper~3 (Autocatalytic Mesh):} The Level~3 $\rho$ controls the effective external data fraction $\alpha_{\text{eff}}$ that prevents model collapse.

\item \textbf{Paper~4 (Settlement Feedback):} The Level~4 $\rho$ ($\sigma \in [2.0, 5.0]$) calibrates the settlement demand dynamics and the monetary policy degradation sequence.

\item \textbf{Paper~5 (Complementary Heterogeneity):} The damping cancellation and dispersion indicator tests directly test Proposition~6.1 and Section~8.3 of Paper~5.

\item \textbf{Paper~6 (Unified Theory):} The EMD timescale discovery confirms the $\Neff$ and $\rstar$ calibrations used throughout the unified framework.
\end{itemize}


%=============================================================================
\section{Conclusion}\label{sec:conclusion}
%=============================================================================

This paper assembles five independent empirical tests of the CES hierarchy framework.  The evidence is summarized in three findings:

\begin{enumerate}
\item \textbf{The timescale hierarchy is real and data-intrinsic.}  EEMD---a fully data-adaptive method with no pre-specified basis---recovers $\Neff = 5$ significant timescale bands with $\rstar = 2.19$, confirming the CWT calibration and resolving the wavelet circularity concern.  The hierarchy is stable across a century of data (18 rolling windows, 1929--2014), robust across seven manufacturing subsectors, and exhibits the predicted sector-dependent depth: Computer/Electronics ($\hat\sigma = 0.40$) has 7 layers; all sectors with $\hat\sigma \geq 0.55$ have exactly 5.  The EEMD fast/slow partition detects five technology waves (1938, 1948, 1977, 1994, 2006) from production data alone, validating the regime-shift predictions without any external labels.

\item \textbf{The curvature parameter $\rho$ is structurally identified.}  Three independent estimation methods---NLS on the CES production function, variance filter inversion, and equicorrelation inversion---applied to 6 NAICS aggregation nodes pass an overidentification test ($\chi^2 = 6.37$, $df = 5$, $p = 0.27$).  Published micro-data elasticities map each hierarchy level to a specific $\rho$ range, with NLS estimates preserving the predicted ordering: $\hat\rho_{\text{Level 1}} = -0.19 < \hat\rho_{\text{Level 3}} = 0.15 < \hat\rho_{\text{Level 4}} = 0.67$.  The systematic upward shift from micro-data ranges confirms that macro FRED aggregates capture a different substitution margin, as expected.  Sub-period instability (pre/post-2000 shifts) suggests that $\rho$ itself evolves on the Level~1 timescale.

\item \textbf{Damping cancellation holds in a large panel.}  A 158-country, four-wave regulatory panel confirms that three of four BCL regulatory dimensions produce only transient effects on financial development, decaying to insignificance by $h = 5$.  The fourth dimension (capital stringency) shows persistence explained by a compositional channel (direct constraint on the outcome variable).  Basel~III---the largest coordinated regulatory tightening in recent banking history---has no detectable persistent effect ($\hat\beta_{\text{DID}} = -0.003$, $p = 0.95$, $N = 1{,}191$).  The cross-layer persistence spread at $h = 5$ is 0.009, consistent with the equal-persistence prediction of the damping cancellation theorem.

\item \textbf{Dispersion leads aggregate dynamics.}  A VAR(5) on quarterly WSTS semiconductor data shows that cross-segment dispersion leads aggregate growth with peak response at 3~quarters ($+0.27$~pp), consistent with the CES spectral gap prediction and the adopted $\rho$ range.  Formal Granger significance requires longer samples ($T > 200$), but the continuous VAR detects the dynamic relationship that binary tests cannot.  The variance filter yields $\hat\rho = 0.43$ from the dispersion structure alone, providing an independent plausibility check.
\end{enumerate}

No test produces a result that contradicts the framework.  The evidence spans four independent data sources (FRED Industrial Production, World Bank BRSS, IMF Financial Development, WSTS semiconductor revenue), three estimation paradigms (time-series decomposition, structural production function estimation, panel regression), and a century of historical data.  All tests use data that predates the AI transition the framework targets.

The confirmed mathematical mechanism---CES curvature propagating through a timescale hierarchy with $\rstar \approx 2$---provides the empirical foundation for the forward-looking predictions of the companion papers.  The framework's qualitative predictions (hierarchical ceilings, damping cancellation, dispersion leading indicators, sector-dependent complexity) hold empirically.  The quantitative calibration ($\rho$ at each level, $\rstar$, $\Neff$) is pinned to observable data rather than assumed.  What remains untested---and untestable until events unfold---is whether the mechanism will produce the specific transition outcomes that the companion papers predict.


%=============================================================================
% Bibliography
%=============================================================================

\begin{thebibliography}{30}

\bibitem[Atalay(2017)]{atalay2017}
Atalay, E. (2017).
\newblock How important are sectoral shocks?
\newblock \emph{American Economic Journal: Macroeconomics}, 9(4):254--280.

\bibitem[Barth et~al.(2013)]{barth2013}
Barth, J.~R., Caprio, G., and Levine, R. (2013).
\newblock Bank regulation and supervision in 180 countries from 1999 to 2011.
\newblock \emph{Journal of Financial Economic Policy}, 5(2):111--219.

\bibitem[Broda and Weinstein(2006)]{broda2006}
Broda, C. and Weinstein, D.~E. (2006).
\newblock Globalization and the gains from variety.
\newblock \emph{Quarterly Journal of Economics}, 121(2):541--585.

\bibitem[Chirinko(2008)]{chirinko2008}
Chirinko, R.~S. (2008).
\newblock $\sigma$: The long and short of it.
\newblock \emph{Journal of Macroeconomics}, 30(2):671--686.

\bibitem[Gechert et~al.(2022)]{gechert2022}
Gechert, S., Havranek, T., Irsova, Z., and Kolcunova, D. (2022).
\newblock Measuring capital--labor substitution: The importance of method choices and publication bias.
\newblock \emph{Review of Economic Dynamics}, 45:55--82.

\bibitem[Hamilton(1989)]{hamilton1989}
Hamilton, J.~D. (1989).
\newblock A new approach to the economic analysis of nonstationary time series and the business cycle.
\newblock \emph{Econometrica}, 57(2):357--384.

\bibitem[Herrendorf et~al.(2015)]{herrendorf2015}
Herrendorf, B., Rogerson, R., and Valentinyi, \'{A}. (2015).
\newblock Household production and the elasticity of substitution.
\newblock \emph{Manuscript, Arizona State University}.

\bibitem[Huang et~al.(1998)]{huang1998empirical}
Huang, N.~E., Shen, Z., Long, S.~R., Wu, M.~C., Shih, H.~H., Zheng, Q., Yen, N.-C., Tung, C.~C., and Liu, H.~H. (1998).
\newblock The empirical mode decomposition and the {Hilbert} spectrum for nonlinear and non-stationary time series analysis.
\newblock \emph{Proceedings of the Royal Society A}, 454(1971):903--995.

\bibitem[Jord\`{a}(2005)]{jorda2005}
Jord\`{a}, \`{O}. (2005).
\newblock Estimation and inference of impulse responses by local projections.
\newblock \emph{American Economic Review}, 95(1):161--182.

\bibitem[Korotayev and Tsirel(2010)]{korotayev2010spectral}
Korotayev, A.~V. and Tsirel, S.~V. (2010).
\newblock A spectral analysis of world {GDP} dynamics: {Kondratieff} waves, {Kuznets} swings, {Juglar} and {Kitchin} cycles in global economic development, and the 2008--2009 economic crisis.
\newblock \emph{Structure and Dynamics}, 4(1).

\bibitem[Oberfield and Raval(2021)]{oberfield2021micro}
Oberfield, E. and Raval, D. (2021).
\newblock Micro data and macro technology.
\newblock \emph{Econometrica}, 89(2):703--732.

\bibitem[Peter and Ruane(2023)]{peter2023}
Peter, A. and Ruane, C. (2023).
\newblock The aggregate importance of intermediate input substitution.
\newblock \emph{Journal of Political Economy Macroeconomics}, 1(2):282--318.

\bibitem[Smirl(2026a)]{smirl2026ces}
Smirl, J. (2026a).
\newblock Complementary heterogeneity: The {CES} triple role in economic dynamics.
\newblock Working paper.

\bibitem[Smirl(2026b)]{smirl2026unified}
Smirl, J. (2026b).
\newblock A unified theory of technology-driven economic transitions.
\newblock Working paper.

\bibitem[Smirl(2026c)]{smirl2026ed}
Smirl, J. (2026c).
\newblock Endogenous decentralization: Learning curves and the self-undermining dynamic of concentrated {AI} investment.
\newblock Working paper.

\bibitem[Smirl(2026d)]{smirl2026mesh}
Smirl, J. (2026d).
\newblock The mesh equilibrium: Post-crossing dynamics of distributed {AI} networks.
\newblock Working paper.

\bibitem[Smirl(2026e)]{smirl2026auto}
Smirl, J. (2026e).
\newblock The autocatalytic mesh: Endogenous capability growth in distributed {AI} networks.
\newblock Working paper.

\bibitem[Smirl(2026f)]{smirl2026settle}
Smirl, J. (2026f).
\newblock The settlement feedback: Stablecoin demand and monetary policy degradation.
\newblock Working paper.

\bibitem[Wu and Huang(2004)]{wu2004study}
Wu, Z. and Huang, N.~E. (2004).
\newblock A study of the characteristics of white noise using the empirical mode decomposition method.
\newblock \emph{Proceedings of the Royal Society A}, 460(2046):1597--1611.

\bibitem[Wu and Huang(2009)]{wu2009ensemble}
Wu, Z. and Huang, N.~E. (2009).
\newblock Ensemble empirical mode decomposition: A noise-assisted data analysis method.
\newblock \emph{Advances in Adaptive Data Analysis}, 1(01):1--41.

\end{thebibliography}

\end{document}
