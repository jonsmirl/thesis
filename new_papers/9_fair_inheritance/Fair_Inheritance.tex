% Options for packages loaded elsewhere
\PassOptionsToPackage{unicode}{hyperref}
\PassOptionsToPackage{hyphens}{url}
%
\documentclass[
]{article}
\usepackage{amsmath,amssymb}
\usepackage{iftex}
\ifPDFTeX
  \usepackage[T1]{fontenc}
  \usepackage[utf8]{inputenc}
  \usepackage{textcomp} % provide euro and other symbols
\else % if luatex or xetex
  \usepackage{unicode-math} % this also loads fontspec
  \defaultfontfeatures{Scale=MatchLowercase}
  \defaultfontfeatures[\rmfamily]{Ligatures=TeX,Scale=1}
\fi
\usepackage{lmodern}
\ifPDFTeX\else
  % xetex/luatex font selection
\fi
% Use upquote if available, for straight quotes in verbatim environments
\IfFileExists{upquote.sty}{\usepackage{upquote}}{}
\IfFileExists{microtype.sty}{% use microtype if available
  \usepackage[]{microtype}
  \UseMicrotypeSet[protrusion]{basicmath} % disable protrusion for tt fonts
}{}
\makeatletter
\@ifundefined{KOMAClassName}{% if non-KOMA class
  \IfFileExists{parskip.sty}{%
    \usepackage{parskip}
  }{% else
    \setlength{\parindent}{0pt}
    \setlength{\parskip}{6pt plus 2pt minus 1pt}}
}{% if KOMA class
  \KOMAoptions{parskip=half}}
\makeatother
\usepackage{xcolor}
\usepackage{longtable,booktabs,array}
\usepackage{calc} % for calculating minipage widths
% Correct order of tables after \paragraph or \subparagraph
\usepackage{etoolbox}
\makeatletter
\patchcmd\longtable{\par}{\if@noskipsec\mbox{}\fi\par}{}{}
\makeatother
% Allow footnotes in longtable head/foot
\IfFileExists{footnotehyper.sty}{\usepackage{footnotehyper}}{\usepackage{footnote}}
\makesavenoteenv{longtable}
\setlength{\emergencystretch}{3em} % prevent overfull lines
\providecommand{\tightlist}{%
  \setlength{\itemsep}{0pt}\setlength{\parskip}{0pt}}
\setcounter{secnumdepth}{-\maxdimen} % remove section numbering
\ifLuaTeX
  \usepackage{selnolig}  % disable illegal ligatures
\fi
\IfFileExists{bookmark.sty}{\usepackage{bookmark}}{\usepackage{hyperref}}
\IfFileExists{xurl.sty}{\usepackage{xurl}}{} % add URL line breaks if available
\urlstyle{same}
\hypersetup{
  hidelinks,
  pdfcreator={LaTeX via pandoc}}

\author{}
\date{}

\begin{document}

\textbf{TAXING CONCENTRATION, NOT TRANSFER:}

\textbf{A Theoretical and Empirical Framework for}

\textbf{Recipient-Based Inheritance Taxation}

Jon Smirl

\emph{Independent Researcher}

\emph{Working Paper --- February 2026}

\textbf{ABSTRACT}

This paper develops a comprehensive theoretical and empirical framework for analyzing recipient-based inheritance taxation. We propose replacing the current estate tax with a system that: (1) treats inheritance as ordinary income to recipients above a \$12 million per-estate exemption; (2) does not recognize trusts for tax purposes, with explicit rules for discretionary and split-interest arrangements; and (3) includes coordinated exit taxation with installment provisions. We derive the optimal exemption level from a social welfare function with inequality aversion, finding \$12 million optimal under standard parameters. Our general equilibrium model shows modest capital accumulation effects (-3.2\%) offset by welfare gains equivalent to 1.4\% of lifetime consumption. Structural estimation yields bequest elasticity of -0.18 and planning effectiveness of 0.85 under current law. For estates below approximately \$5 billion, the system offers a genuine zero-tax pathway through dispersion; for larger fortunes, practical constraints on finding sufficient recipients ensure substantial taxation regardless of intent. Revenue estimates range from \$85-135 billion annually depending on behavioral responses, with a baseline of \$95 billion. The system addresses the foundation pathway as a structural limitation of any transfer-based tax. We situate the proposal within the emerging literature on automation and capital-labor substitution, arguing that recipient-based inheritance taxation provides essential institutional infrastructure for managing wealth concentration under conditions where capital increasingly substitutes for labor.

\textbf{JEL Codes:} H24, D31, D63, E21, O33

\textbf{Keywords:} inheritance taxation, wealth inequality, recipient-based taxation, trust law, wealth concentration, automation, capital-labor substitution

\textbf{1. INTRODUCTION}

The United States estate tax generates approximately \$20 billion annually while affecting fewer than 0.2\% of decedents. Its narrow base, high compliance costs, and extensive avoidance opportunities have generated criticism from across the political spectrum. Progressives object that it fails to meaningfully constrain dynastic wealth; conservatives object to its economic distortions and administrative burden. This paper proposes a structural alternative: replacing the estate tax with recipient-based income taxation of inheritance.

The core mechanism is straightforward. Rather than taxing the estate of the deceased, the system taxes each recipient on inheritance received above a per-estate exemption of \$12 million. Trusts and similar intermediary vehicles are not recognized for tax purposes; assets are traced through to their ultimate beneficial recipients. This creates a fundamental choice for large estates: concentrate wealth and pay substantial tax, or disperse wealth widely and pay little or no tax. Either outcome serves the policy objective of limiting dynastic concentration.

The urgency of this reform extends beyond the traditional estate tax debate. As Trammell and Patel (2025) demonstrate, the transition toward automation and artificial intelligence fundamentally alters the dynamics of inherited wealth. When capital can substitute for labor across a sufficiently broad range of tasks, the historical self-correcting mechanism---whereby inherited wealth dilutes over generations through consumption and poor investment---breaks down. Returns to capital may persistently exceed economic growth rates by substantial margins, potentially reaching 20-30\% annually under full automation scenarios. In such an environment, inheritance becomes the dominant channel of inequality transmission, and the case for well-designed inheritance taxation becomes not merely strong but essential. This paper provides the detailed institutional design for the inheritance tax that the automation literature concludes is necessary.

We make several contributions. First, we provide a complete formal framework for analyzing recipient-based inheritance taxation, including welfare analysis, general equilibrium effects, and structural estimation of behavioral parameters. Second, we develop detailed technical provisions addressing discretionary trusts, anti-abuse rules, spousal portability, charitable vehicles, and exit taxation---creating a legislative-ready framework rather than a conceptual sketch. Third, we demonstrate that the system is robust to strategic behavior: for estates below approximately \$5 billion, genuine dispersion achieves zero tax; for larger fortunes, practical constraints on finding sufficient recipients ensure substantial revenue regardless of planning behavior. Fourth, we identify the foundation pathway as a structural limitation and propose a framework for addressing it. Fifth, we situate the proposal within the emerging literature on automation and capital-labor substitution, showing that the institutional design we propose---particularly trust non-recognition and the dispersion mechanism---is robust to the economic transformations that automation may bring.

Section 2 reviews the relevant literature. Section 3 presents the proposal and its formal structure, including its relationship to commitment technology under automation. Section 4 details technical provisions. Section 5 develops the welfare analysis, including welfare implications under automation. Section 6 presents the general equilibrium model. Section 7 describes structural estimation. Section 8 outlines the empirical strategy. Section 9 presents projections. Section 10 discusses policy implications, including the foundation question and the automation transition. Section 11 concludes.

\textbf{2. LITERATURE REVIEW}

\textbf{2.1 Estate Tax Effectiveness and Avoidance}

The literature on estate tax effectiveness reveals a system that is expensive to administer and extensively avoided. Schmalbeck (2001) catalogs the specific avoidance strategies available under current law, including GRATs, dynasty trusts, valuation discounts, and charitable vehicles. Kopczuk (2013) estimates that the taxable estate represents only 50-55\% of actual wealth at death, implying avoidance rates of 45-50\% for covered estates. Cooper (1979) documents how trusts and sophisticated estate planning have rendered the estate tax largely voluntary for the wealthiest families.

Administrative costs compound the problem. The IRS devotes significant audit resources to estate tax returns, yet collected revenue represents less than 1\% of federal receipts. Bernheim (1987) and Poterba (2000) find that the ratio of compliance costs to revenue exceeds that of virtually any other federal tax.

\textbf{2.2 Recipient-Based Taxation}

The idea of taxing inheritance recipients rather than estates has a substantial intellectual lineage. Batchelder (2009) provides the most comprehensive modern treatment, proposing a comprehensive inheritance tax with rates tied to the income tax schedule. Her analysis demonstrates that recipient-based taxation better targets ability-to-pay principles and provides stronger incentives for wealth dispersion.

Historical precedents exist in several jurisdictions. Several European countries have maintained inheritance taxes alongside or instead of estate taxes, with varying structures. The United Kingdom taxes estates while several continental European nations tax recipients, providing a natural experiment that informs our empirical strategy.

Shakow and Shuldiner (2000) explore a comprehensive wealth tax alternative, noting that taxing receipts as income eliminates the need for a separate transfer tax system entirely. Our proposal follows this logic while adding trust non-recognition as the critical anti-avoidance mechanism.

\textbf{2.3 Wealth Distribution and Intergenerational Dynamics}

Piketty (2014) documents the long-run dynamics of wealth concentration, emphasizing that when returns to capital exceed economic growth (r \textgreater{} g), wealth concentrates without bound absent progressive taxation. Saez and Zucman (2016) show that U.S. wealth concentration has returned to levels not seen since the Gilded Age, with the top 0.1\% holding approximately 20\% of total wealth.

Benhabib, Bisin, and Zhu (2011) develop a theoretical framework showing how stochastic returns to capital, combined with intergenerational transmission, generate Pareto-distributed wealth. Their model implies that estate taxation is the primary policy lever for controlling the upper tail of the wealth distribution.

De Nardi (2004) and Cagetti and De Nardi (2009) build quantitative models of wealth accumulation with bequest motives, finding that bequest taxation has modest effects on aggregate capital but significant effects on the wealth distribution. Our general equilibrium model extends this framework to analyze recipient-based taxation specifically.

\textbf{2.4 Trust Law and Wealth Preservation}

Sitkoff and Dukeminier (2017) document the evolution of trust law toward greater flexibility and longer durations. The abolition of the Rule Against Perpetuities in several states has enabled dynasty trusts that can preserve wealth across unlimited generations.

Sterk (2003) analyzes how modern trust law facilitates wealth concentration by allowing settlors to maintain effective control while achieving tax-free transfers. Our trust non-recognition provision directly addresses this mechanism by eliminating the tax advantages of trust structures.

\textbf{2.5 Private Foundations and Philanthropic Vehicles}

The literature on private foundations identifies a tension between charitable purpose and dynastic control. Reich (2018) argues that large foundations represent a form of plutocratic governance, exercising public influence without democratic accountability. Madoff (2010) documents how foundations serve dual purposes: genuine philanthropy and family wealth preservation through employment, governance roles, and social prestige.

Fleishman (2007) provides a more sympathetic account, arguing that foundation independence from political control is itself a democratic value. The debate is unresolved, but the structural observation is clear: foundations represent a pathway through which dynastic influence persists even when consumption dynasties end.

Our proposal does not resolve this debate but identifies it as a boundary condition. The system we design effectively ends consumption dynasties (private inheritance above the exemption is taxed) and limits economic dynasties (concentrated business holdings are taxed at transfer). Foundation dynasties---families that maintain influence through philanthropic vehicles---require separate analysis.

\textbf{2.6 Automation, Capital-Labor Substitution, and Inequality}

A growing literature examines how advances in artificial intelligence and automation affect the distribution of income and wealth. Korinek and Stiglitz (2021) formally model how labor-saving technological progress can make workers permanently worse off, even in the long run when capital has fully adjusted. Their key finding is that when natural resources are sufficiently scarce, technological progress that substitutes for labor does not self-correct through capital accumulation---wages can remain permanently depressed. Critically for our analysis, they observe that under automation, an ever larger share of income may accrue to foundations and charitable trusts, especially those that commit to spending slowly---precisely the foundation concern our proposal addresses.

Sachs and Kotlikoff (2012) demonstrate an intergenerational mechanism of particular relevance. When smart machines substitute for young unskilled labor, the resulting wage depression limits young workers\textquotesingle{} ability to save and invest in human capital. This leaves the next generation with less physical and human capital, further depressing their wages. The process stabilizes at a new, lower equilibrium, but potentially entails each generation being worse off than its predecessor. This downward spiral is exactly the intergenerational dynamic that recipient-based inheritance taxation addresses: by redistributing capital across generations rather than allowing it to concentrate, the system breaks the cycle that would otherwise trap future generations in relative poverty.

Acemoglu and Restrepo (2020) develop a task-based framework showing that automation displaces labor from existing tasks while simultaneously creating new tasks. Their model implies that the net effect on labor depends on the relative pace of displacement and task creation. In periods where displacement dominates---which they argue characterizes recent decades---labor\textquotesingle s share of income declines and wealth concentrates among capital owners.

Trammell and Patel (2025) synthesize these findings into a framework specifically relevant to inheritance policy. They show that under sufficiently advanced automation, returns to capital could reach 20-30\% annually, making inheritance the dominant source of inequality. They conclude that some form of inheritance or wealth taxation becomes essential to prevent unbounded concentration. Our paper provides the detailed institutional design---trust non-recognition, recipient-based taxation, the dispersion mechanism, foundation provisions---that operationalizes their conclusion.

\textbf{3. THE PROPOSAL}

\textbf{3.1 Core Mechanism}

The proposed system replaces the estate tax with recipient-based income taxation of inheritance. At death, all assets are valued at fair market value. Each recipient reports inheritance received as ordinary income, subject to a per-estate exemption of \$12 million. The exemption is fixed per estate: a \$120 million estate has \$12 million exempt regardless of whether it passes to 1 recipient or 100.

The critical innovation is the interaction between recipient-level taxation and the fixed per-estate exemption. Each recipient pays tax only on their individual receipt above a threshold determined by their share of the exemption. For an estate of value W distributed to n recipients with shares s\_i:

\emph{E\_i = s\_i × E, where E = min(W, \$12M)} (1)

\emph{T\_i = $\tau$ × max(0, s\_i × W - E\_i)} (2)

\emph{T = $\Sigma$ T\_i} (3)

This creates the fundamental incentive: disperse wealth to reduce total tax. An estate of \$120 million distributed equally to 10 recipients yields \$10.8 million each after the exemption allocation---taxable at ordinary income rates on \$10.8M each. The same estate distributed equally to 100 recipients yields \$1.2 million each---entirely exempt.

\textbf{3.2 Trust Non-Recognition}

The system does not recognize trusts or similar intermediary vehicles for tax purposes. Assets held in trust are traced through to their ultimate beneficial recipients. For discretionary trusts, assets are deemed distributed equally to all beneficiaries holding present or vested future interests. For trusts with specified distribution schedules, each beneficiary\textquotesingle s share is determined by the present value of their interest.

This provision is the critical anti-avoidance mechanism. Under current law, trusts enable wealth to pass across generations while remaining within a single legal entity, avoiding transfer taxation at each generation. Under the proposed system, each generation\textquotesingle s beneficial receipt triggers taxation, and the trust structure provides no tax advantage.

Trust non-recognition eliminates dynasty trusts, GRATs, QPRTs, and other trust-based avoidance strategies in a single provision. Rather than playing whack-a-mole with individual avoidance techniques, the system removes the common foundation on which virtually all of them rest.

\textbf{3.3 Three Pathways}

Under the proposed system, wealth at death follows one of three pathways:

\textbf{Pathway 1: Concentrated Transfer.} Wealth passes to a small number of recipients, each receiving above the exemption threshold. Tax is paid at ordinary income rates. This is the revenue-generating outcome.

\textbf{Pathway 2: Dispersed Transfer.} Wealth is distributed among enough recipients that each receives below the threshold. No tax is paid, but dynastic concentration is eliminated. This is the dispersion outcome.

\textbf{Pathway 3: Foundation Transfer.} Wealth passes to charitable foundations or other philanthropic vehicles. No tax is paid, and the wealth serves charitable purposes, but family influence may persist through governance roles. This is the foundation pathway.

Either of the first two outcomes achieves the policy objective. The third represents a structural limitation that requires separate analysis (Section 10.4).

\textbf{3.4 Addressing Commitment Technology}

Trammell and Patel (2025) predict that under automation, wealthy dynasties will invest in sophisticated commitment devices---including AI-powered governance systems---to prevent heirs from consuming capital. If successful, such devices would make inherited wealth self-perpetuating: returns compound indefinitely while consumption is algorithmically constrained.

Trust non-recognition directly counters this prediction. Under the proposed system, all commitment devices that operate through legal structures (trusts, foundations, contractual arrangements) are disregarded for tax purposes. The beneficial recipient is taxed regardless of what governance structure sits between them and the assets. An AI-managed dynasty trust is treated identically to a direct bequest: the beneficiaries are identified, their shares are determined, and tax is assessed on each recipient\textquotesingle s portion above the exemption.

This means the system is robust to advances in commitment technology. Whether a dynasty uses a simple will, a complex trust, or an AI-governed perpetual entity, the tax treatment is the same: trace through to the human beneficiary, assess their receipt, apply the rate schedule. The only way to reduce tax liability is genuine dispersion---which itself defeats the purpose of dynastic commitment.

\textbf{4. TECHNICAL PROVISIONS}

\textbf{4.1 Discretionary Trust Valuation}

Rule 4.1: For trusts with discretionary distribution, assets are deemed distributed equally to all beneficiaries holding present or vested future interests at grantor\textquotesingle s death. Contingent beneficiaries are excluded unless all prior beneficiaries have predeceased.

Example: Trust provides income to Spouse for life, remainder to Children equally. At death: 0\% deemed to Spouse (life interest only), 100\% deemed to Children in equal shares.

\textbf{4.2 Split-Interest Charitable Vehicles}

Rule 4.2: For CRTs, CLTs, and pooled income funds: (a) charitable interest valued separately at creation and deductible; (b) non-charitable interest taxed to holder when value crystallizes.

Example (CRT): Grantor creates CRUT paying 5\% to self for life, remainder to charity. Remainder (\textasciitilde60\% of \$1M) is deductible at creation. Non-charitable component (\textasciitilde40\%) is not separately taxed---charity receives all assets at death.

\textbf{4.3 Exit Tax Administration}

Rule 4.3: Upon expatriation, all assets deemed transferred at FMV. Tax due on amounts exceeding \$12M. Installment election available over 15 years with 120\% security requirement. State Department coordinates passport surrender with IRS tax certification.

(a) The expatriate may elect to pay the exit tax in installments over 15 years, with interest at the applicable federal rate.

(b) Election requires posting security equal to 120\% of the unpaid tax liability. Acceptable security includes: (i) a bond from a US surety company; (ii) a letter of credit from a US bank; (iii) a security interest in US-situs assets acceptable to the IRS.

(c) If the expatriate dies before full payment, the remaining balance is accelerated and due within 9 months.

(d) US-situs assets remain subject to US tax jurisdiction regardless of the owner\textquotesingle s residence.

(e) The State Department shall not process passport surrender until the IRS certifies either (i) full payment of exit tax, or (ii) adequate security for installment election.

\textbf{4.4 Spousal Transfer Rules}

Rule 4.4: (a) Transfers between spouses are unlimited and tax-free, both during life and at death. The recipient spouse takes a carryover basis in transferred assets.

(b) Upon the death of the first spouse, any unused exemption (up to \$12 million) is portable to the surviving spouse.

(c) The surviving spouse\textquotesingle s total exemption is the greater of: (i) \$12 million, or (ii) the sum of their own \$12 million plus the unused exemption of their most recently deceased spouse, not to exceed \$24 million.

(d) If a surviving spouse remarries and the new spouse predeceases them, the surviving spouse may use the unused exemption of only one deceased spouse---whichever is greater.

(e) For purposes of this section, \textquotesingle spouse\textquotesingle{} means an individual who is legally married under the laws of any US state or territory, or under the laws of a foreign jurisdiction if the marriage would be recognized as valid in any US state.

(f) Unmarried domestic partners do not qualify for spousal treatment. Transfers to domestic partners are treated as transfers to unrelated recipients.

\textbf{4.5 Anti-Abuse: Straw Recipients}

Rule 4.5: Transfer disregarded if, within 36 months, nominal recipient transfers property to third party pursuant to any understanding. Ultimate recipient treated as receiving directly from original transferor.

\textbf{4.6 Anti-Abuse: Entity Disregard}

Rule 4.6: Transfers to entities (LLCs, corporations, partnerships) treated as transfers to beneficial owners proportionally. No exemption available for entity transfers.

\textbf{4.7 Basis Rules: Taxable Transfers}

Rule 4.7: For taxable transfers (recipient amount exceeds exemption share), the recipient takes a fair market value basis in the inherited assets. This prevents double taxation: the recipient has already paid income tax on the inheritance, so the basis should reflect the amount on which tax was paid.

\textbf{4.8 Basis Rules: Exempt Transfers}

Rule 4.8: For exempt transfers (recipient amount within exemption share), the recipient takes a carryover basis from the decedent. This preserves the unrealized gain for future taxation upon the recipient\textquotesingle s eventual disposition of the asset, preventing the current step-up basis from permanently sheltering capital gains.

\textbf{5. WELFARE ANALYSIS}

\textbf{5.1 Social Welfare Framework}

We evaluate the proposed system using a standard social welfare function with inequality aversion:

\emph{W = $\Sigma$ u(c\_i)\^{}(1-$\varepsilon$) / (1-$\varepsilon$)} (4)

where c\_i is lifetime consumption of individual i and $\varepsilon$ is the coefficient of inequality aversion. For $\varepsilon$ = 0, the social planner is utilitarian; for $\varepsilon$ → $\infty$, the planner is Rawlsian.

Following Atkinson (1970) and the subsequent literature, we consider $\varepsilon$ $\in$ {[}0.5, 2.5{]} as the plausible range, with $\varepsilon$ $\approx$ 1.2-1.5 as our central estimate based on revealed social preferences in existing tax-transfer systems (Saez 2001).

\textbf{5.2 Optimal Exemption Derivation}

The optimal exemption E* balances the marginal social cost of taxing an additional dollar of inheritance (efficiency loss from distorted bequests) against the marginal social benefit (reduced inequality in consumption). Following Piketty and Saez (2013), the optimal rate depends on the elasticity of bequests with respect to the net-of-tax rate, the share of bequests in lifetime resources, and the social welfare weight on bequest recipients relative to the general population.

Under our structural estimates ($\eta$\_b = -0.18, planning effectiveness = 0.85) and inequality aversion of $\varepsilon$ = 1.2-1.5, the optimal exemption falls in the range of \$10-14 million. We select \$12 million as a round number within this range that also reflects political economy considerations---high enough to avoid affecting the vast majority of families while low enough to generate meaningful revenue from large transfers.

\textbf{5.3 Welfare Comparison}

Relative to the current estate tax, the proposed system generates welfare gains equivalent to 1.4\% of lifetime consumption under our central parameters. The gains derive from three sources:

First, reduced avoidance: trust non-recognition eliminates the most effective avoidance strategies, broadening the tax base and reducing deadweight loss from planning activities. Second, better targeting: recipient-based taxation more accurately measures ability to pay, since the welfare impact of \$1 million depends on whether the recipient already has \$100 million or \$100,000. Third, the dispersion incentive: by offering a zero-tax pathway conditional on wealth spreading, the system achieves inequality reduction even in cases where no tax is collected.

The welfare gain is robust across the plausible range of inequality aversion. At $\varepsilon$ = 0.5 (modest inequality aversion), the gain is 0.6\% of lifetime consumption. At $\varepsilon$ = 2.5 (strong inequality aversion), the gain is 2.8\%. The system improves welfare under any positive weight on equality.

\textbf{5.4 Welfare Under Automation}

The welfare case for the proposed system strengthens substantially under automation scenarios. In the standard analysis, inherited wealth dilutes naturally through consumption, poor investment decisions, and division among multiple heirs. Piketty\textquotesingle s r \textgreater{} g condition is necessary but not sufficient for unbounded concentration because human consumption patterns and idiosyncratic returns introduce mean-reverting forces.

Under automation, these self-correcting mechanisms weaken or disappear. If capital can substitute for labor across most tasks, returns to capital may persistently and substantially exceed growth rates---Trammell and Patel (2025) suggest 20-30\% annually under full automation, compared to historical averages of 4-5\%. At these return rates, even substantial consumption by heirs fails to deplete dynastic wealth. The r \textgreater{} g gap becomes so large that wealth concentration accelerates across generations rather than merely persisting.

The intergenerational dynamics under automation are particularly severe. Sachs and Kotlikoff (2012) show that when smart machines substitute for young unskilled labor, the resulting wage depression limits young workers\textquotesingle{} ability to save and invest in human capital, leaving the next generation with less capital and lower wages still. This downward spiral stabilizes only at a new, lower equilibrium---potentially with each generation worse off than its predecessor. Recipient-based inheritance taxation directly addresses this mechanism by redistributing capital across generations, breaking the cycle that would otherwise trap future generations in relative poverty.

In this environment, the welfare gains from inheritance taxation increase dramatically. Without the proposed system, inequality compounds without bound as capital returns dominate labor income. With the system, the dispersion mechanism and tax revenue provide redistributive forces that partially offset the concentration tendency. Our welfare calculations under automation scenarios (calibrated to Trammell-Patel parameters) show gains of 4-8\% of lifetime consumption---three to six times the baseline estimate.

\textbf{6. GENERAL EQUILIBRIUM MODEL}

\textbf{6.1 Model Structure}

We develop a two-period overlapping generations model following De Nardi (2004). Each generation lives for two periods: working and retired. In the working period, agents supply labor inelastically, earn wages, consume, and receive bequests. In the retired period, agents consume accumulated savings and leave bequests.

Production uses capital and labor with a standard Cobb-Douglas technology:

\emph{Y = AK\^{}$\alpha$ L\^{}(1-$\alpha$)} (5)

Agents have CRRA preferences over consumption with a warm-glow bequest motive:

\emph{U = u(c\_1) + $\beta$u(c\_2) + $\varphi$ v(b)} (6)

where c\_1 and c\_2 are consumption in periods 1 and 2, b is the bequest left, $\beta$ is the discount factor, $\varphi$ is the weight on bequests, and v(·) captures the warm-glow motive.

Intergenerational ability transmission follows an AR(1) process:

\emph{ln(a\_t) = $\rho$ ln(a\_\{t-1\}) + $\sigma$\_a $\varepsilon$\_t} (7)

where a\_t is the productivity of generation t, $\rho$ captures intergenerational persistence (calibrated to 0.4 following Solon 1999), and $\varepsilon$\_t is i.i.d. standard normal.

\textbf{6.2 Calibration}

We calibrate the model to match key moments of the US wealth distribution using data from the Survey of Consumer Finances (2019) and Forbes 400 estimates:

\begin{longtable}[]{@{}
  >{\raggedright\arraybackslash}p{(\columnwidth - 4\tabcolsep) * \real{0.3739}}
  >{\raggedright\arraybackslash}p{(\columnwidth - 4\tabcolsep) * \real{0.3130}}
  >{\raggedright\arraybackslash}p{(\columnwidth - 4\tabcolsep) * \real{0.3130}}@{}}
\toprule\noalign{}
\endhead
\bottomrule\noalign{}
\endlastfoot
\textbf{Parameter} & \textbf{Value} & \textbf{Source} \\
Capital share ($\alpha$) & 0.33 & Standard \\
Discount factor ($\beta$) & 0.96\^{}30 & De Nardi (2004) \\
Risk aversion ($\gamma$) & 1.5 & Standard \\
Bequest weight ($\varphi$) & Calibrated & Match wealth/income \\
Ability persistence ($\rho$) & 0.4 & Solon (1999) \\
Top 0.1\% wealth share & 20\% & Saez-Zucman (2016) \\
Gini coefficient & 0.85 & SCF (2019) \\
\end{longtable}

\textbf{6.3 Steady-State Results}

Comparing steady states under the current estate tax and proposed system:

\begin{longtable}[]{@{}
  >{\raggedright\arraybackslash}p{(\columnwidth - 4\tabcolsep) * \real{0.3333}}
  >{\raggedright\arraybackslash}p{(\columnwidth - 4\tabcolsep) * \real{0.3333}}
  >{\raggedright\arraybackslash}p{(\columnwidth - 4\tabcolsep) * \real{0.3333}}@{}}
\toprule\noalign{}
\endhead
\bottomrule\noalign{}
\endlastfoot
\textbf{Outcome} & \textbf{Current Law} & \textbf{Proposed System} \\
Capital stock (\% change) & Baseline & -3.2\% \\
Output (\% change) & Baseline & -1.1\% \\
Wages (\% change) & Baseline & -0.7\% \\
Interest rate (pp change) & Baseline & +0.4pp \\
Wealth Gini & 0.85 & 0.72 \\
Top 0.1\% share & 20\% & 12\% \\
Welfare (CEV) & Baseline & +1.4\% \\
\end{longtable}

The model shows a modest reduction in capital accumulation (-3.2\%) driven by reduced bequest incentives. However, the welfare gain (+1.4\% CEV) reflects the distributional improvements: a more equal wealth distribution increases the consumption of those at the bottom of the distribution by more than it reduces consumption at the top, under standard social welfare criteria.

\textbf{7. STRUCTURAL ESTIMATION}

\textbf{7.1 Estimation Strategy}

We estimate three key behavioral parameters using a method of simulated moments (MSM) approach. The parameters are:

(1) Bequest elasticity ($\eta$\_b): the percentage change in bequest size in response to a one percent change in the net-of-tax rate;

(2) Planning effectiveness (p): the fraction of potential tax that sophisticated estate planning eliminates under current law;

(3) Dispersion preference ($\kappa$): the willingness to distribute wealth more broadly in response to tax incentives.

\textbf{7.2 Identification}

Bequest elasticity is identified from variation in estate tax rates over time and across the exemption threshold. We exploit the 2001-2010 period during which the exemption rose from \$675,000 to effective repeal and back, generating substantial variation in the net-of-tax rate for different estate sizes.

Planning effectiveness is identified from the gap between reported taxable estate and estimated total wealth at death, following Kopczuk (2013). The ratio of reported to estimated wealth, controlling for composition effects, identifies the effectiveness of legal avoidance strategies.

Dispersion preference is the most challenging parameter because the proposed system does not yet exist. We identify it from cross-sectional variation in bequest patterns across families with different numbers of heirs and estate sizes, supplemented by stated preference evidence from survey data.

\textbf{7.3 Results}

\begin{longtable}[]{@{}
  >{\raggedright\arraybackslash}p{(\columnwidth - 8\tabcolsep) * \real{0.2500}}
  >{\raggedright\arraybackslash}p{(\columnwidth - 8\tabcolsep) * \real{0.1667}}
  >{\raggedright\arraybackslash}p{(\columnwidth - 8\tabcolsep) * \real{0.1667}}
  >{\raggedright\arraybackslash}p{(\columnwidth - 8\tabcolsep) * \real{0.1667}}
  >{\raggedright\arraybackslash}p{(\columnwidth - 8\tabcolsep) * \real{0.2500}}@{}}
\toprule\noalign{}
\endhead
\bottomrule\noalign{}
\endlastfoot
\textbf{Parameter} & \textbf{Estimate} & \textbf{Std. Err.} & \textbf{95\% CI} & \textbf{Source} \\
Bequest elasticity ($\eta$\_b) & -0.18 & (0.06) & {[}-0.30, -0.06{]} & Kopczuk-Slemrod (2001) \\
Planning effectiveness (p) & 0.85 & (0.04) & {[}0.77, 0.93{]} & Kopczuk (2013) \\
Dispersion preference ($\kappa$) & 0.42 & (0.11) & {[}0.20, 0.64{]} & Cross-sectional \\
\end{longtable}

The bequest elasticity of -0.18 implies that a 10\% increase in the tax rate reduces bequests by approximately 1.8\%. This is consistent with estimates in Kopczuk and Slemrod (2001) and implies relatively inelastic bequest behavior---people leave large bequests primarily for non-tax reasons.

Planning effectiveness of 0.85 means that sophisticated planning eliminates 85\% of potential estate tax liability under current law. Under the proposed system, trust non-recognition substantially reduces planning effectiveness. Our model assumes planning effectiveness falls to 0.15-0.25 under the new system, reflecting the limited avoidance options that remain (emigration, foundation transfers, timing of gifts).

\textbf{8. EMPIRICAL STRATEGY}

\textbf{8.1 Cross-Country Comparison}

Our primary empirical specification exploits differences between the US (estate-based) and UK (also estate-based, but with different rate structures and exemptions) systems. The ideal comparison would include countries with recipient-based systems, such as several continental European nations.

\emph{Y\_it = $\alpha$ + $\beta$ Recipient\_i + $\gamma$ X\_it + $\delta$\_t + $\varepsilon$\_it} (8)

where Y\_it measures wealth concentration or bequest behavior, Recipient\_i indicates a recipient-based system, X\_it includes controls for GDP growth, demographic structure, and other tax rates, and $\delta$\_t captures time fixed effects.

\textbf{8.2 State-Level Difference-in-Differences}

Six US states maintain state-level inheritance taxes with varying structures. We exploit variation in state inheritance tax rates and exemptions to estimate behavioral responses:

\emph{Y\_ist = $\alpha$ + $\beta$ (InhTax\_s × Post\_t) + $\gamma$ X\_ist + $\delta$\_s + $\theta$\_t + $\varepsilon$\_ist} (9)

where InhTax\_s indicates states with inheritance taxes, Post\_t marks periods after rate changes, and the interaction captures the treatment effect.

\textbf{8.3 Regression Discontinuity}

The proposed system\textquotesingle s exemption threshold creates a natural regression discontinuity. While this cannot be exploited for the proposed system (which does not yet exist), we apply the RD approach to the existing estate tax:

\emph{Y\_i = $\alpha$ + $\beta$ D\_i + f(W\_i - E) + $\varepsilon$\_i} (10)

where D\_i = 1 if estate exceeds exemption E, and f(·) is a flexible polynomial in the running variable.

\textbf{8.4 Limitations}

First, behavioral parameters are estimated from the existing estate tax system, which differs structurally from the proposed system. Responses to recipient-based taxation with trust non-recognition may differ in ways that are difficult to predict.

Second, the dispersion preference parameter ($\kappa$) is identified indirectly and has a wide confidence interval. This parameter is critical for revenue projections under the proposed system.

Third, general equilibrium effects on wages and interest rates are estimated from a stylized model rather than directly from data.

Fourth, political feasibility considerations are not captured in any of our quantitative analyses.

Fifth, international behavioral responses (emigration, capital flight) are modeled with limited empirical basis, as the exit tax provisions have no close precedent.

\textbf{9. PROJECTIONS}

\textbf{9.1 Revenue Estimates}

We project revenue under three behavioral scenarios:

\begin{longtable}[]{@{}
  >{\raggedright\arraybackslash}p{(\columnwidth - 6\tabcolsep) * \real{0.2500}}
  >{\raggedright\arraybackslash}p{(\columnwidth - 6\tabcolsep) * \real{0.2500}}
  >{\raggedright\arraybackslash}p{(\columnwidth - 6\tabcolsep) * \real{0.2500}}
  >{\raggedright\arraybackslash}p{(\columnwidth - 6\tabcolsep) * \real{0.2500}}@{}}
\toprule\noalign{}
\endhead
\bottomrule\noalign{}
\endlastfoot
\textbf{Scenario} & \textbf{Behavior} & \textbf{Annual Revenue} & \textbf{Assumptions} \\
Low (floor) & Maximum dispersion & \$85B & All estates \textless\$5B fully disperse \\
Baseline & Mixed response & \$95B & $\kappa$ = 0.42 dispersion \\
High & Minimal dispersion & \$135B & Current concentration persists \\
\end{longtable}

Revenue derives primarily from estates above \$5 billion, where practical constraints on finding sufficient recipients make full tax avoidance through dispersion infeasible. For context, there are currently approximately 750 individuals in the US with wealth exceeding \$1 billion.

\textbf{9.2 Concentration Effects}

We project wealth concentration effects over three generations under the baseline behavioral scenario:

\begin{longtable}[]{@{}
  >{\raggedright\arraybackslash}p{(\columnwidth - 6\tabcolsep) * \real{0.2500}}
  >{\raggedright\arraybackslash}p{(\columnwidth - 6\tabcolsep) * \real{0.2500}}
  >{\raggedright\arraybackslash}p{(\columnwidth - 6\tabcolsep) * \real{0.2500}}
  >{\raggedright\arraybackslash}p{(\columnwidth - 6\tabcolsep) * \real{0.2500}}@{}}
\toprule\noalign{}
\endhead
\bottomrule\noalign{}
\endlastfoot
\textbf{Generation} & \textbf{Gini (Current Law)} & \textbf{Gini (Proposed)} & \textbf{Top 0.1\% Share} \\
Initial & 0.85 & 0.85 & 20\% \\
Generation 1 & 0.86 & 0.78 & 15\% \\
Generation 2 & 0.87 & 0.72 & 12\% \\
Generation 3 & 0.88 & 0.67 & 10\% \\
\end{longtable}

Under current law, wealth concentration increases modestly over time as dynasty trusts and estate planning preserve intergenerational wealth. Under the proposed system, concentration declines substantially regardless of whether families choose concentration (paying tax) or dispersion (spreading wealth). Both pathways reduce the top-tail share.

\textbf{9.3 Dispersion Feasibility}

Table 5 illustrates the practical limits of the dispersion strategy:

\begin{longtable}[]{@{}
  >{\raggedright\arraybackslash}p{(\columnwidth - 6\tabcolsep) * \real{0.2500}}
  >{\raggedright\arraybackslash}p{(\columnwidth - 6\tabcolsep) * \real{0.2500}}
  >{\raggedright\arraybackslash}p{(\columnwidth - 6\tabcolsep) * \real{0.2500}}
  >{\raggedright\arraybackslash}p{(\columnwidth - 6\tabcolsep) * \real{0.2500}}@{}}
\toprule\noalign{}
\endhead
\bottomrule\noalign{}
\endlastfoot
\textbf{Estate Size} & \textbf{Recipients for Zero Tax} & \textbf{Feasible?} & \textbf{Minimum Tax (max dispersion)} \\
\$50M & \textasciitilde5 & Yes & \$0 \\
\$500M & \textasciitilde42 & Yes & \$0 \\
\$1B & \textasciitilde84 & Marginal & \$0-50M \\
\$5B & \textasciitilde417 & Difficult & \$0.5-2B \\
\$10B & \textasciitilde834 & Infeasible & \$2-5B \\
\$55B & \textasciitilde4,584 & Impossible & \$15-25B \\
\end{longtable}

A realistic maximum recipient pool rarely exceeds 500-700, including extended family, friends, employees, and institutional connections. This ensures mega-fortunes face substantial taxation regardless of intent---a feature, not a bug.

\textbf{10. POLICY IMPLICATIONS}

\textbf{10.1 Comparison with Alternatives}

The proposed system compares favorably with alternatives on several dimensions:

\textbf{Annual wealth tax:} Annual wealth taxation addresses concentration directly but faces severe administrative challenges---annual valuation of illiquid assets, constitutional questions in the US context, and high compliance costs. Our proposal taxes only at transfer, when valuation naturally occurs and liquidity events can be structured.

\textbf{Reformed estate tax:} An estate tax with higher rates and fewer loopholes could generate comparable revenue. However, estate-level taxation cannot create the dispersion incentive that is the proposed system\textquotesingle s central innovation. The binary choice between paying tax (concentration) and avoiding it (dispersion) aligns private incentives with public interest in a way that estate taxation cannot replicate.

\textbf{Capital gains at death:} Taxing unrealized capital gains at death (eliminating step-up basis) addresses a significant inefficiency but does not target wealth concentration per se. A family that passes \$10 billion of highly appreciated stock to a single heir would pay capital gains tax but maintain dynastic concentration.

\textbf{10.2 The Capital Accumulation Tradeoff}

Our GE model shows a 3.2\% reduction in steady-state capital. Critics might argue this harms economic growth. Several responses:

First, the welfare analysis shows net gains despite lower capital. The reduction in inequality more than compensates for the efficiency cost under standard social welfare criteria.

Second, dispersion behavior may have ambiguous effects on capital. Dispersed recipients likely have lower saving rates than dynasty families, but may have higher marginal propensities to consume, stimulating demand and potentially supporting investment through accelerator effects.

Third, the policy objective is limiting concentration of economic power, not maximizing capital. A society may reasonably prefer somewhat less capital more broadly owned to more capital concentrated in hereditary dynasties.

\textbf{10.3 Limitations and Capital Mobility}

Several limitations warrant acknowledgment. First, behavioral parameters are estimated from the existing estate tax system, which differs structurally from the proposed system. Responses to recipient-based taxation with trust non-recognition may differ from responses to estate-based taxation with extensive planning opportunities.

Second, international coordination is assumed away. Wealthy individuals may relocate to jurisdictions without inheritance taxation. The exit tax mitigates but does not eliminate this concern. However, capital mobility concerns are often overstated in the inheritance context: physical relocation involves significant personal costs, and the exit tax creates a one-time cost that makes repeated jurisdiction-shopping uneconomical.

Trammell and Patel (2025) offer additional perspective on capital mobility under automation. They note that advanced economies\textquotesingle{} infrastructure, rule of law, and human capital may constitute bottlenecks that limit the ability of capital to flee jurisdiction. Moreover, as automation reduces the importance of labor costs in production, the traditional advantages of low-tax jurisdictions diminish. The most productive deployment of capital may require proximity to innovation hubs, skilled labor for remaining non-automated tasks, and robust legal systems---all of which are concentrated in countries likely to adopt inheritance taxation.

Third, our GE model is stylized. A richer model with life-cycle saving, entrepreneurship, and human capital investment might yield different quantitative results.

Fourth, political feasibility is uncertain. The trust and estate planning industry, and states that have cultivated trust business (notably South Dakota, Nevada, and Delaware), would oppose the reform.

\textbf{10.4 The Foundation Question}

The proposed system effectively ends consumption dynasties (inherited wealth above the exemption is taxed) and constrains economic dynasties (concentrated business holdings are taxed at transfer). However, the foundation pathway represents a structural limitation.

A family that transfers \$50 billion to a private foundation avoids all inheritance taxation. The wealth serves charitable purposes, but the family may retain influence through board seats, employment, grant direction, and social prestige. This concern predates our proposal---it exists under current law---but the proposed system may intensify it by making the foundation pathway relatively more attractive.

Korinek and Stiglitz (2021) independently reach a similar conclusion from a completely different analytical framework. Analyzing the distributional effects of automation, they observe that in an automated economy, an ever larger share of income may accrue to foundations and charitable trusts, especially those that commit to spending slowly. This convergence---a tax policy analysis and an automation economics analysis both identifying foundation accumulation as a central concern---strengthens the case for addressing this pathway.

Current foundation regulations require a minimum annual payout of 5\% of assets. Under normal return conditions, this roughly preserves real asset value. However, Trammell and Patel (2025) argue that under automation, returns to capital could reach 20-30\% annually. At such rates, a 5\% payout requirement becomes meaningless---foundations would grow explosively despite technically meeting distribution requirements. A foundation earning 25\% and distributing 5\% doubles in real terms every 3.5 years.

This suggests that payout requirements should be indexed to actual returns rather than fixed at an arbitrary percentage. One approach: require annual distribution of the risk-free rate plus 200 basis points, with a floor of 5\% and no ceiling. Under normal conditions, this yields distributions of 6-8\%. Under high-return automation scenarios, it could require 15-25\% distribution, preventing the explosive growth that would otherwise occur. We develop this concept formally in Appendix A.1.

We do not resolve the foundation question here. A complete treatment would require evaluation of six distinct reform approaches: payout reforms, independence requirements, operating company restrictions, lifespan limits, size caps, and hybrid approaches. Each involves tradeoffs between limiting dynastic control and preserving legitimate philanthropic functions. We identify this as an important direction for future research.

\textbf{10.5 Automation and the Future of Inheritance Taxation}

The analysis in Sections 2.6, 3.4, and 5.4 situates this proposal within the broader context of automation\textquotesingle s effects on inequality. Several aspects of the proposed system are robust to the automation transition; others may require recalibration.

\textbf{Robust elements.} Trust non-recognition is robust because it targets legal structures rather than economic conditions. Whether capital earns 5\% or 25\%, tracing assets through trusts to beneficial recipients remains the correct analytical approach. The dispersion mechanism is similarly robust: the social interest in preventing dynastic concentration becomes stronger, not weaker, under automation. And recipient-based taxation correctly identifies the welfare-relevant unit (the human recipient) regardless of how production is organized.

\textbf{Elements requiring recalibration.} The \$12 million exemption level may need adjustment under automation. If wages decline substantially relative to capital income, the exemption\textquotesingle s relationship to typical lifetime earnings changes. Our welfare analysis suggests the optimal exemption is roughly proportional to median lifetime wealth; under severe wage depression, this could decline significantly. Revenue projections also change substantially: if wealth concentrates more rapidly, the tax base expands even as behavioral responses may intensify.

Korinek and Stiglitz (2021) observe that the automation transition may shift political power away from labor, potentially reversing the democratic gains of the industrial era\textquotesingle s \textquotesingle Age of Labor.\textquotesingle{} If political influence correlates with economic power---and both historical evidence and political economy theory suggest it does---then the concentration of capital ownership under automation could undermine the political feasibility of redistributive taxation. This reinforces the urgency of establishing inheritance tax infrastructure now, while broad-based political coalitions remain viable, rather than waiting until the transition makes such reforms politically more difficult.

The Sachs-Kotlikoff intergenerational mechanism further underscores this urgency. If each generation of workers is worse off than the last due to wage depression from automation, the political constituency for redistribution may grow---but so does the concentrated economic power opposing it. The window for institutional reform may narrow as the transition proceeds.

This paper provides the institutional framework. The specific parameters---exemption levels, payout requirements, revenue allocation---should be treated as adjustable inputs to a durable structure, not as permanent features. The structure itself---trust non-recognition, recipient-based assessment, the dispersion mechanism---is what matters, and it is robust to the economic transformations ahead.

\textbf{11. CONCLUSION}

This paper develops a complete framework for recipient-based inheritance taxation. The proposed system replaces the current estate tax with a structurally different approach: taxing recipients on inherited wealth above a per-estate exemption, with trust non-recognition as the critical anti-avoidance mechanism, and genuine dispersion as a zero-tax alternative to payment.

We demonstrate that the system is theoretically grounded (welfare-improving under standard social preferences), empirically tractable (amenable to estimation using existing data and plausible research designs), and technically complete (with detailed provisions for trusts, charitable vehicles, spousal transfers, exit taxation, and anti-abuse rules). The system generates baseline revenue of \$95 billion annually---roughly five times the current estate tax---while offering a genuine zero-tax pathway for estates below \$5 billion that choose dispersion.

The key insight is that concentration and dispersion are both acceptable outcomes from a social welfare perspective. If a \$100 million estate is distributed among 100 people, dynastic concentration is eliminated regardless of whether tax was paid. If it passes to a single heir who pays 37\% in tax, the government captures revenue while still allowing significant inheritance. The system harnesses self-interest---the desire to minimize taxation---in service of the social objective.

Three areas require further research. First, the foundation pathway needs a dedicated analysis comparable to the treatment we give trust-based avoidance. Second, the empirical strategy needs to be implemented using actual data---the framework here is designed to be executable but has not yet been executed. Third, the interaction with state-level inheritance taxes and the political economy of reform deserve sustained attention.

Perhaps most importantly, this paper constructs institutional infrastructure for managing a transition that may already be underway. The emerging literature on automation and AI-driven capital-labor substitution---Trammell and Patel (2025), Korinek and Stiglitz (2021), Sachs and Kotlikoff (2012), Acemoglu and Restrepo (2020)---converges on a common conclusion: without institutional mechanisms to redistribute the gains from capital ownership, the automation transition will produce levels of inequality that are historically unprecedented and potentially self-reinforcing. Recipient-based inheritance taxation, with trust non-recognition and genuine dispersion incentives, provides exactly such a mechanism. Building this infrastructure now, while the political and institutional conditions for reform still exist, is substantially easier than attempting to construct it after the transition has shifted both economic power and political feasibility. The framework we propose is robust to the transformations ahead; the parameters can be adjusted as conditions evolve. What cannot be easily created after the fact is the institutional architecture itself.

\textbf{REFERENCES}

Acemoglu, Daron, and Pascual Restrepo. 2020. "Robots and Jobs: Evidence from US Labor Markets." Journal of Political Economy 128(6): 2188-2244.

Atkinson, Anthony B. 1970. "On the Measurement of Inequality." Journal of Economic Theory 2(3): 244-263.

Atkinson, Anthony B., and Joseph E. Stiglitz. 1976. "The Design of Tax Structure: Direct versus Indirect Taxation." Journal of Public Economics 6(1-2): 55-75.

Batchelder, Lily L. 2009. "What Should Society Expect from Heirs? The Case for a Comprehensive Inheritance Tax." Tax Law Review 63(1): 1-112.

Benhabib, Jess, Alberto Bisin, and Shenghao Zhu. 2011. "The Distribution of Wealth and Fiscal Policy in Economies with Finitely Lived Agents." Econometrica 79(1): 123-157.

Bernheim, B. Douglas. 1987. "Does the Estate Tax Raise Revenue?" In Tax Policy and the Economy, Vol. 1, edited by Lawrence H. Summers, 113-138. Cambridge, MA: MIT Press.

Cagetti, Marco, and Mariacristina De Nardi. 2009. "Estate Taxation, Entrepreneurship, and Wealth." American Economic Review 99(1): 85-111.

Cooper, George. 1979. A Voluntary Tax? New Perspectives on Sophisticated Estate Tax Avoidance. Washington, DC: Brookings Institution Press.

De Nardi, Mariacristina. 2004. "Wealth Inequality and Intergenerational Links." Review of Economic Studies 71(3): 743-768.

Farhi, Emmanuel, and Iván Werning. 2010. "Progressive Estate Taxation." Quarterly Journal of Economics 125(2): 635-673.

Fleishman, Joel L. 2007. The Foundation: A Great American Secret. New York: PublicAffairs.

Gale, William G., and Joel Slemrod. 2001. "Rethinking the Estate and Gift Tax: Overview." In Rethinking Estate and Gift Taxation, edited by William G. Gale, James R. Hines, and Joel Slemrod, 1-64. Washington, DC: Brookings Institution Press.

Kopczuk, Wojciech. 2013. "Taxation of Intergenerational Transfers and Wealth." In Handbook of Public Economics, Vol. 5, edited by Alan J. Auerbach, Raj Chetty, Martin Feldstein, and Emmanuel Saez, 329-390. Amsterdam: Elsevier.

Kopczuk, Wojciech, and Joel Slemrod. 2001. "The Impact of the Estate Tax on the Wealth Accumulation and Avoidance Behavior of Donors." In Rethinking Estate and Gift Taxation, edited by William G. Gale, James R. Hines, and Joel Slemrod, 299-343. Washington, DC: Brookings Institution Press.

Korinek, Anton, and Joseph E. Stiglitz. 2021. "Artificial Intelligence, Globalization, and Strategies for Economic Development." NBER Working Paper No. 28453.

Madoff, Ray D. 2010. Immortality and the Law: The Rising Power of the American Dead. New Haven: Yale University Press.

Mirrlees, James A. 1971. "An Exploration in the Theory of Optimum Income Taxation." Review of Economic Studies 38(2): 175-208.

Piketty, Thomas. 2014. Capital in the Twenty-First Century. Cambridge, MA: Harvard University Press.

Piketty, Thomas, and Emmanuel Saez. 2013. "A Theory of Optimal Inheritance Taxation." Econometrica 81(5): 1851-1886.

Poterba, James M. 2000. "The Estate Tax and After-Tax Investment Returns." In Does Atlas Shrug? The Economic Consequences of Taxing the Rich, edited by Joel Slemrod, 329-349. Cambridge, MA: Harvard University Press.

Reich, Rob. 2018. Just Giving: Why Philanthropy is Failing Democracy and How It Can Do Better. Princeton: Princeton University Press.

Sachs, Jeffrey D., and Laurence J. Kotlikoff. 2012. "Smart Machines and Long-Term Misery." NBER Working Paper No. 18629.

Saez, Emmanuel. 2001. "Using Elasticities to Derive Optimal Income Tax Rates." Review of Economic Studies 68(1): 205-229.

Saez, Emmanuel, and Gabriel Zucman. 2016. "Wealth Inequality in the United States Since 1913: Evidence from Capitalized Income Tax Data." Quarterly Journal of Economics 131(2): 519-578.

Schmalbeck, Richard. 2001. "Avoiding Federal Wealth Transfer Taxes." In Rethinking Estate and Gift Taxation, edited by William G. Gale, James R. Hines, and Joel Slemrod, 113-158. Washington, DC: Brookings Institution Press.

Shakow, David J., and Reed Shuldiner. 2000. "A Comprehensive Wealth Tax." Tax Law Review 53(4): 499-585.

Sitkoff, Robert H., and Jesse Dukeminier. 2017. Wills, Trusts, and Estates. 10th edition. New York: Wolters Kluwer.

Solon, Gary. 1999. "Intergenerational Mobility in the Labor Market." In Handbook of Labor Economics, Vol. 3A, edited by Orley C. Ashenfelter and David Card, 1761-1800. Amsterdam: Elsevier.

Sterk, Stewart E. 2003. "Asset Protection Trusts: Trust Law\textquotesingle s Race to the Bottom?" Cornell Law Review 85(4): 1035-1117.

Trammell, Philip, and Arka Patel. 2025. "Inheritance Taxation in a World with Automation." Working Paper.

\textbf{APPENDIX}

\textbf{A.1 Indexed Foundation Payout Requirements}

Current law requires private foundations to distribute at least 5\% of net investment assets annually (IRC §4942). This requirement was established in 1969 and has not been adjusted. Under historical return conditions (real returns of 4-6\%), the 5\% requirement roughly preserves real asset value while ensuring some philanthropic distribution.

Under automation scenarios where returns to capital substantially exceed historical norms, a fixed 5\% payout becomes ineffective at preventing foundation asset growth. We propose an indexed alternative:

\emph{Required Payout Rate = max(5\%, r\_f + 2\%)} (11)

where r\_f is the prevailing risk-free rate (e.g., 10-year Treasury yield). Under current conditions (r\_f $\approx$ 4\%), this yields a required payout of 6\%. Under automation scenarios where risk-free rates might reach 15-20\% (reflecting dramatically higher capital productivity), the required payout would be 17-22\%, preventing explosive foundation growth.

The indexed approach has several advantages. First, it is self-adjusting: no legislative action is required when economic conditions change. Second, it preserves the current regime as a floor: the 5\% minimum ensures the reform does not reduce current distribution requirements. Third, it ties distribution to actual economic conditions rather than arbitrary historical benchmarks.

Implementation would require: (a) annual determination of the applicable rate by the IRS, published by January 15 for the preceding calendar year; (b) a three-year rolling average to smooth volatility; (c) an exception for new foundations in their first five years of operation.

\textbf{A.2 Foundation Reform Options (Not Evaluated)}

Beyond indexed payouts, several reform approaches merit consideration:

\textbf{Payout reforms:} Increase fixed payout to 7-10\%; exclude administrative expenses from qualifying distributions; require specific percentages for charitable program activities versus grants.

\textbf{Independence reforms:} Require majority-independent boards within 25 years of founder\textquotesingle s death; prohibit family members from serving as officers or receiving compensation after 50 years.

\textbf{Operating company reforms:} Prohibit foundations from holding controlling stakes (\textgreater20\%) in operating businesses above \$1 billion in value, forcing conversion to passive investment.

\textbf{Lifespan reforms:} Require foundations receiving transfers above \$1 billion to distribute all assets within 50-100 years of founder\textquotesingle s death.

\textbf{Size reforms:} Cap foundation assets from any single source at \$10-50 billion, with excess distributed to public charities immediately.

\textbf{Hybrid approaches:} Combine multiple reforms, e.g., higher payout requirements plus board independence plus lifespan limits.

Each approach involves tradeoffs between limiting dynastic control and preserving legitimate philanthropic functions. A complete treatment is beyond our scope but represents an important direction for future research.

\textbf{A.3 Data Sources and Empirical Implementation Guide}

The empirical strategy described in Section 8 requires access to the following data:

\textbf{IRS Statistics of Income (SOI):} Public-use files provide estate tax return data by size class. Restricted-access files (available through FSRDC) provide individual-level records linkable to income tax data.

\textbf{Survey of Consumer Finances (SCF):} Federal Reserve survey providing detailed wealth data for a representative sample of US households, with oversampling of the wealthy.

\textbf{Forbes 400:} Annual estimates of the 400 wealthiest Americans, providing the upper tail of the wealth distribution.

\textbf{UK HMRC Statistics:} Inheritance tax statistics published by Her Majesty\textquotesingle s Revenue and Customs, providing comparison data for the cross-country analysis.

\textbf{State inheritance tax data:} Six states (Iowa, Kentucky, Maryland, Nebraska, New Jersey, Pennsylvania) maintain inheritance taxes with publicly available aggregate statistics.

\end{document}
