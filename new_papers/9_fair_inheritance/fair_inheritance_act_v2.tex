% Options for packages loaded elsewhere
\PassOptionsToPackage{unicode}{hyperref}
\PassOptionsToPackage{hyphens}{url}
%
\documentclass[
]{article}
\usepackage{amsmath,amssymb}
\usepackage{iftex}
\ifPDFTeX
  \usepackage[T1]{fontenc}
  \usepackage[utf8]{inputenc}
  \usepackage{textcomp} % provide euro and other symbols
\else % if luatex or xetex
  \usepackage{unicode-math} % this also loads fontspec
  \defaultfontfeatures{Scale=MatchLowercase}
  \defaultfontfeatures[\rmfamily]{Ligatures=TeX,Scale=1}
\fi
\usepackage{lmodern}
\ifPDFTeX\else
  % xetex/luatex font selection
\fi
% Use upquote if available, for straight quotes in verbatim environments
\IfFileExists{upquote.sty}{\usepackage{upquote}}{}
\IfFileExists{microtype.sty}{% use microtype if available
  \usepackage[]{microtype}
  \UseMicrotypeSet[protrusion]{basicmath} % disable protrusion for tt fonts
}{}
\makeatletter
\@ifundefined{KOMAClassName}{% if non-KOMA class
  \IfFileExists{parskip.sty}{%
    \usepackage{parskip}
  }{% else
    \setlength{\parindent}{0pt}
    \setlength{\parskip}{6pt plus 2pt minus 1pt}}
}{% if KOMA class
  \KOMAoptions{parskip=half}}
\makeatother
\usepackage{xcolor}
\usepackage{longtable,booktabs,array}
\usepackage{calc} % for calculating minipage widths
% Correct order of tables after \paragraph or \subparagraph
\usepackage{etoolbox}
\makeatletter
\patchcmd\longtable{\par}{\if@noskipsec\mbox{}\fi\par}{}{}
\makeatother
% Allow footnotes in longtable head/foot
\IfFileExists{footnotehyper.sty}{\usepackage{footnotehyper}}{\usepackage{footnote}}
\makesavenoteenv{longtable}
\setlength{\emergencystretch}{3em} % prevent overfull lines
\providecommand{\tightlist}{%
  \setlength{\itemsep}{0pt}\setlength{\parskip}{0pt}}
\setcounter{secnumdepth}{-\maxdimen} % remove section numbering
\ifLuaTeX
  \usepackage{selnolig}  % disable illegal ligatures
\fi
\IfFileExists{bookmark.sty}{\usepackage{bookmark}}{\usepackage{hyperref}}
\IfFileExists{xurl.sty}{\usepackage{xurl}}{} % add URL line breaks if available
\urlstyle{same}
\hypersetup{
  hidelinks,
  pdfcreator={LaTeX via pandoc}}

\author{}
\date{}

\begin{document}

\textbf{THE FAIR INHERITANCE ACT}

\emph{Spread or Be Taxed: A Proposal to End Dynastic Wealth Concentration}

POSITION PAPER

Executive Summary

The Core Principle

This proposal does not tax wealth transfer. It taxes wealth concentration.

Under this system, a billionaire can pay zero tax on transferring their entire fortune. The only requirement is spreading the wealth broadly enough that no single recipient receives more than \$12 million. The billionaire chooses: concentrate wealth in a few heirs and pay tax, or disperse it widely and pay nothing.

Either outcome achieves the policy goal. Dynasties end, whether through taxation or through dispersion. The system is indifferent to which path the wealthy choose.

The Mechanism

The proposal replaces the federal estate and gift tax with a simple rule: inheritances and large gifts are taxable income to the recipient, but only above \$12 million per estate (\$24 million for married couples). Trusts and similar structures are not recognized; assets are deemed transferred directly to ultimate beneficiaries. An identical exit tax applies to expatriation.

This creates a straightforward choice for every wealthy individual:

\begin{longtable}[]{@{}
  >{\raggedright\arraybackslash}p{(\columnwidth - 2\tabcolsep) * \real{0.5000}}
  >{\raggedright\arraybackslash}p{(\columnwidth - 2\tabcolsep) * \real{0.5000}}@{}}
\toprule\noalign{}
\endhead
\bottomrule\noalign{}
\endlastfoot
\textbf{Option A: Concentrate} & \textbf{Option B: Disperse} \\
Leave \$10B to 3 children & Leave \$12M each to 834 people \\
Pay \textasciitilde\$3.6B in tax & Pay zero tax \\
Dynasty continues (diminished) & No dynasty; 834 families transformed \\
\end{longtable}

The goal is not revenue. The goal is preventing permanent dynasties. If billionaires achieve zero tax by dispersing wealth so broadly that no dynasty forms, the system has succeeded completely. Revenue is merely the penalty for choosing concentration. Dispersion is the tax-free alternative.

Part I: The Problem of Dynastic Wealth

What Is a Dynasty?

A dynasty exists when a family maintains concentrated economic and political power across multiple generations based solely on inherited wealth rather than ongoing contribution. The defining feature is not wealth itself but the perpetuation of control without merit.

The Cargill-MacMillan family illustrates this phenomenon. With 14 billionaires and combined wealth exceeding \$55 billion, they own 88\% of America\textquotesingle s largest private company. This fortune traces to a single grain storage facility opened in 1865. For 160 years, descendants have maintained control over a substantial portion of America\textquotesingle s food supply based on an ancestor\textquotesingle s decision six generations ago. No current family member created this enterprise. They inherited control, and under current law, they will pass it to their descendants indefinitely.

Why Current Law Fails

The federal estate tax nominally applies a 40\% rate to estates above \$13.61 million. In practice, wealthy families pay far less. Dynasty trusts, grantor retained annuity trusts, family limited partnerships, and valuation discounts allow sophisticated planners to reduce effective rates to single digits or zero. The step-up in basis at death permanently eliminates all unrealized capital gains, representing approximately \$58 billion annually in foregone revenue.

The result is a two-tier system. Moderately wealthy families, those with \$15-30 million in assets, may pay estate taxes because they lack access to sophisticated planning. Ultra-wealthy families, those with the resources to employ the wealth defense industry, largely avoid them. The Cargill-MacMillans will pass \$55 billion to the next generation with minimal taxation. The system designed to prevent dynasties instead perpetuates them.

The Step-Up Loophole

The most significant gap in current law is the stepped-up basis at death. When assets pass to heirs, all unrealized capital gains are permanently eliminated. This enables the strategy known as "buy, borrow, die": wealthy individuals hold appreciated assets, borrow against them for living expenses, and pass them to heirs with no capital gains ever paid.

The step-up in basis costs more than the estate tax collects. It benefits primarily the wealthiest families. And it ensures that much of America\textquotesingle s greatest fortunes will never be taxed at all.

Part II: The Proposed System

Five Simple Rules

The proposal replaces the estate and gift tax system with five clear rules:

\textbf{Rule 1: Inheritance as Income}

All inheritances above a \$12 million per-estate exemption are taxable as ordinary income to recipients. For married couples, the combined exemption is \$24 million, with full portability to the surviving spouse.

\textbf{Rule 2: Gifts as Income}

All lifetime gifts above a \$12 million cumulative exemption per donor are taxable as ordinary income to recipients.

\textbf{Rule 3: Unlimited Spousal Transfers}

Transfers between spouses remain unlimited and tax-free, with full portability of unused exemptions at death.

\textbf{Rule 4: No Trust Recognition}

Trusts and similar structures are not recognized for transfer tax purposes. Assets are deemed transferred directly from the original owner to the ultimate beneficial recipients.

\textbf{Rule 5: Exit Tax}

Expatriation triggers deemed transfer of all worldwide assets, taxable above the \$12 million exemption at ordinary income rates. This prevents escape through renunciation of citizenship.

Why Per-Estate Exemption

The exemption is per-estate, not per-recipient. This is the critical design choice that prevents gaming.

Under a per-recipient exemption, a billionaire could multiply exemptions by adding beneficiaries. With a \$12 million per-recipient exemption, naming 100 beneficiaries would shelter \$1.2 billion. Under the per-estate approach, the exemption is fixed regardless of how many people inherit. A \$1 billion estate pays tax on \$988 million whether it goes to 2 heirs or 200.

This design has a crucial implication: the only way to reduce tax liability is to reduce the size of individual inheritances below \$12 million. This requires spreading wealth broadly, which is exactly the outcome the system seeks.

Part III: The Billionaire\textquotesingle s Choice

Zero Tax Is Available

A billionaire can pay zero tax under this system. The path is simple: ensure no single recipient receives more than \$12 million.

\begin{longtable}[]{@{}
  >{\raggedright\arraybackslash}p{(\columnwidth - 2\tabcolsep) * \real{0.5000}}
  >{\raggedright\arraybackslash}p{(\columnwidth - 2\tabcolsep) * \real{0.5000}}@{}}
\toprule\noalign{}
\endhead
\bottomrule\noalign{}
\endlastfoot
\textbf{Estate Size} & \textbf{Recipients Needed for Zero Tax} \\
\$12 million & 1 \\
\$100 million & 9 \\
\$500 million & 42 \\
\$1 billion & 84 \\
\$10 billion & 834 \\
\$55 billion (Cargill family) & 4,584 \\
\end{longtable}

If the Cargill-MacMillans want to pay zero tax, they can. They simply have to give their fortune to 4,584 different people, each receiving under \$12 million. That is not a dynasty. That is wealth dispersion on a transformative scale.

The Nature of the Choice

Every billionaire faces a genuine choice under this system:

\textbf{Concentrate and be taxed:}

Leave the fortune to a small number of heirs. Pay approximately 35\% tax on amounts above exemption. The heirs remain extremely wealthy. A diminished dynasty may continue for another generation or two before dissipating through repeated taxation.

\textbf{Disperse and pay nothing:}

Spread the fortune across hundreds or thousands of recipients. Pay zero tax. No dynasty forms. Hundreds or thousands of families receive life-changing wealth.

The billionaire retains complete control over this choice. They can fund thousands of scholarships. They can give \$12 million to every long-tenured employee. They can endow community foundations across the country. They can transform their extended family for generations. Any combination they choose.

The only thing they cannot do is concentrate billions in a handful of heirs without tax consequences. If they choose concentration over dispersion, they have chosen taxation. That choice is theirs.

Part IV: Effects by Wealth Level

No Change for Most Americans

The median inheritance in the United States is under \$100,000. The average is approximately \$72,000. At a \$12 million per-estate exemption, the vast majority of Americans experience no tax on inheritances whatsoever. Approximately 99.9\% of estates remain completely exempt.

A family with a \$5 million estate, a \$10 million estate, or even an \$11 million estate pays nothing. The system affects only the very largest concentrations of wealth.

Moderate Wealth: \$12-50 Million

Estates in this range experience modest taxation if they concentrate inheritance in few heirs. A \$30 million estate to two children results in approximately \$6.7 million in total tax (\$18 million taxable above the exemption), leaving each child with approximately \$11.7 million.

However, these families also have the dispersion option. A \$30 million estate distributed to 3 children and 4 grandchildren at approximately \$4.3 million each pays substantially less: only \$2.6 million each is taxable above each recipient's share of the exemption, yielding lower marginal rates. The choice between concentration (higher tax) and broader family distribution (lower tax) is available.

Significant Wealth: \$50-500 Million

These estates face a real choice. A \$200 million estate concentrated in 2 heirs generates approximately \$63 million in tax. The same estate dispersed among 14 or more recipients pays nothing.

At this level, the tax creates meaningful incentive toward broader distribution. Families may choose to include more descendants, fund charitable foundations, or establish trusts for community benefit (which, being charitable, do not count as taxable transfers).

Dynastic Wealth: \$500 Million and Above

The ultra-wealthy face the starkest choice. Concentration becomes extremely expensive. A \$1 billion estate to 3 heirs generates approximately \$360 million in tax. A \$10 billion estate generates approximately \$3.6 billion.

The dispersion alternative requires genuine commitment to spreading wealth. A \$10 billion estate requires 834 recipients to achieve zero tax. This is not trivial to organize, but it is achievable for anyone who genuinely wishes to avoid taxation.

The question becomes: Is maintaining dynastic concentration worth hundreds of millions or billions in taxes? Some will say yes and pay. Others will choose dispersion. Either outcome serves the policy goal.

Part V: Why Dynasties End

The Mathematics of Concentration

Under this system, wealth concentration becomes unsustainable across generations. Consider a family that chooses to concentrate wealth and pay taxes:

\begin{longtable}[]{@{}
  >{\raggedright\arraybackslash}p{(\columnwidth - 6\tabcolsep) * \real{0.2500}}
  >{\raggedright\arraybackslash}p{(\columnwidth - 6\tabcolsep) * \real{0.2500}}
  >{\raggedright\arraybackslash}p{(\columnwidth - 6\tabcolsep) * \real{0.2500}}
  >{\raggedright\arraybackslash}p{(\columnwidth - 6\tabcolsep) * \real{0.2500}}@{}}
\toprule\noalign{}
\endhead
\bottomrule\noalign{}
\endlastfoot
\textbf{Generation} & \textbf{Starting Wealth} & \textbf{After Tax} & \textbf{With 5\% Growth} \\
Founder & \$1 billion & n/a & \$1 billion \\
Generation 2 & \$1.6 billion & \$1.05 billion & \$1.05 billion \\
Generation 3 & \$1.7 billion & \$1.1 billion & \$680 million \\
Generation 4 & \$1.1 billion & \$715 million & \$440 million \\
Generation 5 & \$700 million & \$460 million & \$285 million \\
\end{longtable}

Even with strong investment returns, concentrated wealth erodes substantially across generations. A \$1 billion fortune becomes \$285 million by the fifth generation, split among numerous descendants. The descendants remain comfortable but no longer exercise dynastic control.

The Dispersion Alternative

Families that choose dispersion avoid this erosion entirely but achieve the same policy outcome through a different path. A \$1 billion fortune dispersed to 67 recipients in the first generation transforms 67 families. No tax is paid. No dynasty forms. Wealth spreads through the economy.

In either case, the founding generation\textquotesingle s achievement does not create permanent control over economic or political systems. This is the purpose of the proposal.

What the System Preserves

The system does not prevent individuals from building great wealth through their own efforts. It does not punish entrepreneurship or innovation. It does not eliminate inheritance. It does not impose confiscatory rates on moderate estates.

It preserves the ability to become a billionaire. It preserves the ability to pass meaningful inheritance to children. It only prevents that inheritance from creating permanent dynastic control across unlimited generations.

The principle is simple: You can get rich. Your children can inherit enough to have every advantage in life. Your grandchildren will need to do something with their lives beyond managing inherited wealth.

Part VI: Eliminating the Double Taxation Argument

The Problem with the Estate Tax Frame

The most effective attack on inheritance taxation has always been the claim of double taxation: the decedent already paid taxes on their earnings, and the government taxes the same money again at death. This argument resonates emotionally. The "death tax" rebranding was devastatingly effective politics.

The estate tax frames the deceased as the victim, a hardworking person being taxed one final time simply for dying. This framing persists even though much estate wealth consists of unrealized gains that were never taxed.

Why the Argument Fails Here

The proposed system fundamentally changes the framing. It is not a tax on the dead person\textquotesingle s estate. It is a tax on the living recipient\textquotesingle s windfall income.

When two different people are taxed on two different events, it is not double taxation. It is simply taxation. A corporation pays tax on its profits; when the corporation pays wages, the employee pays income tax. No one calls this double taxation.

Similarly, when a parent earns income and pays tax, and later the child receives an inheritance and pays tax, two different people have experienced two separate taxable events. The parent paid tax on their earnings. The child pays tax on their receipt. These are distinct.

The Windfall Principle

The tax code already treats windfalls as ordinary income to the recipient. Lottery winnings are taxed. Gambling winnings are taxed. Prizes and awards are taxed. No one argues that lottery winnings should be tax-free because the ticket purchasers already paid income tax on the money they used to buy tickets.

Inheritance is economically identical to these other windfalls. The recipient did nothing to earn the money. It arrived as a windfall. Taxing the recipient on their windfall is consistent with tax treatment of every other windfall.

The Political Reframing

The shift from taxing the estate to taxing the recipient transforms political dynamics:

\emph{Old framing: Government punishing success and taxing grieving families at death.}

\emph{New framing: Heirs pay tax on their windfall like everyone else, or they spread wealth broadly and pay nothing.}

The victim is no longer the hardworking deceased parent. The taxpayer is a person who just received millions of dollars for doing nothing, and who had the option to avoid all tax by sharing more broadly.

The message becomes: Americans who work for their income pay taxes. Americans who inherit millions can either pay taxes or spread the wealth. The choice is theirs.

Part VII: Closing All Escape Routes

The Trust Elimination

The proposal does not recognize trusts for transfer tax purposes. This single rule eliminates virtually the entire estate planning toolkit:

Dynasty trusts, which currently allow wealth to pass for generations without estate tax, are disregarded. Assets are deemed transferred directly to beneficiaries, who pay income tax on amounts above exemption.

Grantor Retained Annuity Trusts (GRATs), which currently transfer appreciation tax-free, are disregarded. The appreciation passes to beneficiaries as taxable income.

Family Limited Partnerships, which currently generate valuation discounts of 20-40\%, are disregarded. Assets are valued at fair market value and deemed transferred to beneficial owners.

Life insurance trusts, which currently exclude insurance proceeds from estates, are disregarded. Proceeds pass to beneficiaries as taxable income above exemption.

The wealth defense industry\textquotesingle s sophisticated structures become irrelevant. The only questions that matter are: Who receives the assets? How much does each person receive?

Attempted Escapes and Why They Fail

\begin{longtable}[]{@{}
  >{\raggedright\arraybackslash}p{(\columnwidth - 2\tabcolsep) * \real{0.5000}}
  >{\raggedright\arraybackslash}p{(\columnwidth - 2\tabcolsep) * \real{0.5000}}@{}}
\toprule\noalign{}
\endhead
\bottomrule\noalign{}
\endlastfoot
\textbf{Strategy} & \textbf{Why It Fails} \\
More beneficiaries to multiply exemptions & Exemption is per estate, not per recipient \\
Dynasty trusts & Trusts not recognized \\
GRATs and IDGTs & Trusts not recognized \\
Family LLCs for valuation discounts & Entities disregarded; FMV applies \\
Lifetime gifts to avoid estate & Same rules apply to gifts \\
Renounce citizenship & Exit tax at same rate \\
\end{longtable}

What Actually Works

Three strategies reduce tax liability. First, charitable giving: wealth given to charity escapes taxation entirely, but it goes to charity, not heirs. Second, spending: wealth consumed during life is not transferred. Third, dispersion: spreading wealth broadly enough that each recipient's share of the taxable amount is minimized.

All three outcomes serve the policy goal. Charity benefits society. Spending returns wealth to the economy. Dispersion prevents dynasty formation. The system is indifferent to which path the wealthy choose.

Part VIII: The Cargill Test Case

The Current Situation

The Cargill-MacMillan family represents the paradigm case for this proposal. With 14 billionaires, combined wealth exceeding \$55 billion, and 88\% ownership of America\textquotesingle s largest private company, they exercise economic power based entirely on inheritance from a founder who died over a century ago.

Under current law, this fortune will continue indefinitely. Dynasty trusts in favorable jurisdictions, sophisticated valuation techniques, and the step-up in basis ensure minimal taxation at each generational transfer. The family\textquotesingle s wealth defense advisors are among the most skilled in the world.

Under the Proposed System

The family faces a clear choice:

\textbf{Option A: Maintain concentration}

The approximately 90 family members continue to hold concentrated ownership. At generational transfers, estates pay tax on amounts above \$12 million per estate. With average individual holdings of roughly \$600 million, each estate generates approximately \$218 million in tax.

Over one generation, the family pays approximately \$19 billion in aggregate tax. The remaining wealth, approximately \$36 billion, remains substantial but represents a significant reduction. Over three generations, the fortune erodes to approximately \$15 billion, fragmented among hundreds of descendants. The dynasty ends.

\textbf{Option B: Disperse the wealth}

The family decides to spread wealth broadly. At \$12 million per recipient, \$55 billion requires 4,584 recipients. The family could fund transformative gifts for thousands of employees, community foundations in every state where Cargill operates, scholarships at agricultural colleges across the country, or any combination of recipients they choose.

Tax paid: zero. Dynasty status: ended. Societal impact: thousands of families and institutions transformed.

Either Outcome Succeeds

The system does not prefer one outcome over the other. If the Cargills choose concentration and pay \$19 billion in tax, that revenue funds public purposes. If they choose dispersion and pay nothing, wealth spreads through thousands of recipients. Both outcomes end the dynasty. Both serve the public interest.

The question is simply whether concentrated control of the American food supply should pass indefinitely based on an 1865 business decision. The answer under this system is no, but the family chooses how that answer is implemented.

Part IX: Implementation

Administrative Simplicity

The proposed system is dramatically simpler than current law. Estates issue 1099-INHERITANCE forms to beneficiaries showing gross inheritance, exempt portion (pro-rata share of \$12 million estate exemption), and taxable portion. Recipients include taxable amounts on individual income tax returns.

For gifts, donors issue 1099-GIFT forms for transfers exceeding annual reporting thresholds. Donors track cumulative gifts against their \$12 million lifetime exemption.

The IRS already has robust infrastructure for 1099 reporting and matching. This system leverages existing infrastructure rather than requiring specialized estate tax enforcement.

Liquidity Provisions

Heirs of illiquid assets, particularly family businesses and real estate, may face cash tax obligations on non-liquid inheritance. The proposal includes installment payment provisions allowing payment over 10-15 years for qualifying illiquid assets, with interest at market rates.

This addresses legitimate concerns about forced liquidation while ensuring taxes are ultimately paid.

Transition Rules

Existing irrevocable trusts would be grandfathered as to corpus at enactment. Future distributions to beneficiaries would be taxable as income above exemption amounts.

Assets held at enactment would receive a one-time step-up in basis to fair market value, creating a clean break.

An effective date 12 months after enactment would provide transition time for estate planning adjustments.

Constitutional Basis

The proposal treats inheritance as income under the Sixteenth Amendment. The Supreme Court\textquotesingle s Glenshaw Glass decision defined income as any "accession to wealth, clearly realized, over which the taxpayer has complete dominion." Inheritance meets this test: the recipient gains wealth, the receipt is a realization event, and the recipient has complete control.

Alternatively, the system could be structured as an excise tax on the privilege of receiving property by inheritance, similar to how current estate and gift taxes are structured.

Part X: Conclusion

The Complete System

\begin{longtable}[]{@{}
  >{\raggedright\arraybackslash}p{(\columnwidth - 2\tabcolsep) * \real{0.5000}}
  >{\raggedright\arraybackslash}p{(\columnwidth - 2\tabcolsep) * \real{0.5000}}@{}}
\toprule\noalign{}
\endhead
\bottomrule\noalign{}
\endlastfoot
\textbf{Element} & \textbf{Treatment} \\
Inheritance & Income above \$12M per estate (\$24M married) \\
Lifetime gifts & Income above \$12M cumulative per donor \\
Spousal transfers & Unlimited, tax-free, full portability \\
Trusts & Not recognized; deemed direct transfer \\
Charity & Fully deductible \\
Expatriation & Exit tax at same rate \\
\end{longtable}

The Choice, Not the Tax

This proposal is not fundamentally about taxation. It is about choice.

Every wealthy individual can choose to pay zero tax. The only requirement is spreading wealth broadly enough that no dynasty forms. If they prefer to concentrate wealth in a few heirs, they pay tax on that concentration. The choice is entirely theirs.

The goal is not revenue, though revenue will be collected from those who choose concentration. The goal is ending permanent dynasties. If every billionaire in America chose to disperse their wealth across thousands of recipients and paid zero tax, the system would have succeeded completely.

The Fundamental Principle

America has never been comfortable with hereditary aristocracy. The Founders rejected titles of nobility. Each generation has valued the ideal, if not always the reality, of advancement through merit rather than birth.

Dynastic wealth concentration contradicts this principle. When fourteen members of a single family are billionaires based on a business decision made in 1865, when a family controls a substantial portion of the food supply based on inheritance rather than contribution, the principle of merit has yielded to hereditary privilege.

This proposal offers a simple remedy: spread or be taxed. Billionaires who wish to create lasting impact can do so by transforming thousands of lives through broad dispersion of their wealth. Those who prefer dynastic concentration will pay meaningful tax on that choice. Either way, permanent hereditary control of economic and political power ends.

The principle can be stated simply: You can get rich. Your children can inherit enough to have every advantage. Your grandchildren will need to do something with their lives.

That is not punishment. That is America.

\end{document}
