\documentclass[12pt]{article}

%=== Packages ===
\usepackage[margin=1in]{geometry}
\usepackage{amsmath,amssymb,amsthm}
\usepackage{mathtools}
\usepackage{natbib}
\usepackage[colorlinks=true,citecolor=blue,linkcolor=blue,urlcolor=blue]{hyperref}
\usepackage[capitalise,noabbrev]{cleveref}
\usepackage{booktabs}
\usepackage{enumitem}
\usepackage{graphicx}

%=== Theorem environments ===
\newtheorem{theorem}{Theorem}[section]
\newtheorem{proposition}[theorem]{Proposition}
\newtheorem{lemma}[theorem]{Lemma}
\newtheorem{corollary}[theorem]{Corollary}
\newtheorem{definition}[theorem]{Definition}
\newtheorem{remark}[theorem]{Remark}
\newtheorem{example}[theorem]{Example}

%=== Notation shortcuts ===
\newcommand{\R}{\mathbb{R}}
\newcommand{\E}{\mathbb{E}}
\newcommand{\Var}{\operatorname{Var}}
\newcommand{\Cov}{\operatorname{Cov}}
\newcommand{\Tr}{\operatorname{Tr}}
\newcommand{\calF}{\mathcal{F}}
\newcommand{\calH}{\mathcal{H}}
\newcommand{\calJ}{\mathcal{J}}
\newcommand{\calR}{\mathcal{R}}
\newcommand{\calL}{\mathcal{L}}
\newcommand{\calS}{\mathcal{S}}
\newcommand{\bone}{\mathbf{1}}

\title{Business Cycles as Conservative-Dissipative Oscillations\\in a Heterogeneous-Complementarity Economy}
\author{Jon Smirl}
\date{February 2026 \\ \smallskip \textit{Working Paper}}

\begin{document}
\maketitle

\begin{abstract}
This paper derives business cycles from the conservative-dissipative structure of an economy with heterogeneous sectoral complementarities.  Each sector $n$ has a CES aggregation parameter $\rho_n$ determining the complementarity of its inputs, an associated breakdown threshold $T^*(\rho_n)$ at which efficient allocation breaks down, and frictional within-sector adjustment governed by a damping matrix $\mathbf{R}$.  Cross-sector coupling through directed input-output linkages introduces an antisymmetric matrix $\mathbf{J}$, so the full dynamics take the conservative-dissipative form $\dot{\mathbf{x}} = (\mathbf{J} - \mathbf{R})\nabla \calF$.  The antisymmetric coupling generates oscillatory modes---business cycles---whose periods equal the geometric means of adjacent sectoral timescales.  Five main results follow.  First, sectors enter recession in order of increasing $\rho$ (most complementary first) and recover in reverse order, because low-$\rho$ sectors have the lowest breakdown threshold (\emph{$\rho$-ordering theorem}).  Second, the expansion-contraction asymmetry observed in all post-war business cycles (long expansions, short contractions) emerges from the \emph{slow-fast oscillation} structure of the dynamics near the quasi-equilibrium-surface fold, with the asymmetry ratio scaling as $1/\varepsilon$ where $\varepsilon$ is the financial-to-real timescale ratio.  Third, monetary policy operates on $T$ but endogenously lowers $T^*$ by enabling more complementary production structures, formalizing Minsky's instability hypothesis as a theorem in $(\rho, T)$ space (\emph{Minsky trap}).  Fourth, regulation increases the damping ratio $\zeta = r/\omega$ without changing the equilibrium, explaining the Great Moderation as a shift from underdamped to critically-damped dynamics and predicting its catastrophic termination.  Fifth, the classical hierarchy of cycle frequencies---Kitchin, Juglar, Kuznets, Kondratiev---corresponds to the eigenspectrum of $\mathbf{J}$, unifying inventory cycles, business cycles, construction supercycles, and technology waves as harmonics of a single conservative-dissipative system.
\end{abstract}

\textbf{JEL Codes:} E32, E44, C62, D57, L16

\textbf{Keywords:} business cycles, conservative-dissipative systems, CES complementarity, slow-fast oscillations, Minsky instability, input-output networks, sectoral heterogeneity, Great Moderation

%=============================================================================
\section{Introduction}\label{sec:intro}
%=============================================================================

Business cycle theory has long struggled with a fundamental tension.  Real business cycle models \citep{kydland1982} treat fluctuations as efficient responses to exogenous technology shocks, requiring implausibly large and frequent disturbances to match observed volatility.  New Keynesian models \citep{gali2015} introduce nominal rigidities that generate inefficient fluctuations, but rely on representative-agent aggregation that obscures the sectoral heterogeneity visible in every recession.  Financial cycle theories \citep{minsky1986,borio2014} capture the credit dynamics that empirically dominate cycle amplitude, but lack the mathematical structure to make precise predictions about timing, duration, and cross-sector propagation.

This paper proposes that business cycles are \emph{intrinsic oscillatory modes} of a production economy with heterogeneous complementarities.  The mathematical framework is the conservative-dissipative system
\begin{equation}\label{eq:port-ham}
\dot{\mathbf{x}} = (\mathbf{J} - \mathbf{R})\nabla \calF(\mathbf{x}),
\end{equation}
where $\calF_q = \Phi_{\mathrm{CES}}(\rho) - T \cdot S_q$ (with $q = \rho$) is the Tsallis economic CES potential \citep{smirl2026free,smirl2026tsallis}, $\mathbf{J} = -\mathbf{J}^\top$ is the antisymmetric matrix encoding directed input-output linkages between sectors, and $\mathbf{R} = \mathbf{R}^\top \succeq 0$ is the symmetric friction matrix encoding within-sector adjustment frictions.

The key departure from existing theory is \emph{sectoral $\rho$-heterogeneity}.  Different sectors of the economy have fundamentally different complementarity structures: software production combines largely substitutable inputs ($\rho$ near 1), while construction requires land, permits, labor, materials, and financing that are essentially non-substitutable ($\rho$ near or below 0).  Each sector has a breakdown threshold $T^*(\rho_n) \propto K_n = (1 - \rho_n)(J_n - 1)/J_n$ at which efficient allocation breaks down \citep{smirl2026ces}.  When the economy-wide information friction $T$ rises during an expansion---through credit-quality deterioration, complexity growth, and declining signal-to-noise ratios---sectors breach their critical thresholds in order of increasing $\rho$: the most complementary sectors fail first.

This mechanism generates five results that connect directly to empirical regularities.

\emph{First}, the $\rho$-ordering of sectoral entry into and exit from recession (\cref{sec:rho-ordering}).  In every post-war US recession, housing and financial services lead the downturn while consumer services lag \citep{leamer2007}.  The framework derives this ordering from the complementarity parameters of the production technology rather than assuming it.

\emph{Second}, the asymmetry between expansions and contractions (\cref{sec:relaxation}).  NBER data show average post-war expansions of 58 months and contractions of 11 months, a 5:1 ratio.  The conservative-dissipative system with a fold in the quasi-equilibrium surface produces slow-fast oscillations whose asymmetry scales as $1/\varepsilon$, where $\varepsilon$ is the timescale separation between financial and real adjustment.

\emph{Third}, the Minsky instability as a theorem (\cref{sec:minsky}).  Expansionary monetary policy lowers $T$ but simultaneously enables more complementary (lower-$\rho$) production structures, which have lower $T^*$.  The stability margin $T^* - T$ may shrink even as measured volatility falls.  This formalizes the ``stability breeds instability'' intuition with a precise comparative static in $(\rho, T)$ space.

\emph{Fourth}, the Great Moderation and its termination (\cref{sec:great-moderation}).  Financial deregulation and globalization increased the friction rate $r$ without altering the antisymmetric coupling $\omega$, pushing the system from underdamped to critically damped.  Oscillation amplitude fell, but the fold structure remained.  When the Minsky mechanism finally drove $T$ through $T^*$, the crisis was faster and sharper than in the underdamped regime---the transition from oscillatory instability to catastrophic instability.

\emph{Fifth}, the hierarchy of cycle frequencies (\cref{sec:hierarchy}).  The eigenspectrum of $\mathbf{J}$ contains multiple oscillatory modes corresponding to different pairs of coupled timescales: Kitchin inventory cycles ($\sim$3--5 years), Juglar investment cycles ($\sim$7--11 years), Kuznets construction swings ($\sim$15--25 years), and Kondratiev technology waves ($\sim$40--60 years).  These are not separate phenomena requiring separate theories---they are harmonics of the same conservative-dissipative system.

\paragraph{Related literature.}
The oscillatory interpretation of business cycles has deep roots.  \citet{goodwin1967} modeled cycles as predator-prey dynamics between labor share and employment, producing endogenous cycles.  \citet{kaldor1940} proposed nonlinear investment functions that generate slow-fast oscillations.  \citet{metzler1941} derived inventory cycles from stock-flow lags.  The present paper's contribution is identifying the \emph{source} of the antisymmetric coupling (directed input-output linkages), the \emph{control parameter} governing oscillation behavior (the damping ratio $\zeta = r/\omega$, itself determined by complementarity $\rho$ and information friction $T$), and the \emph{sectoral propagation mechanism} ($\rho$-ordering through heterogeneous breakdown thresholds).

The input-output structure draws on \citet{acemoglu2012}, who showed that network topology determines whether idiosyncratic shocks aggregate or cancel.  The present paper adds that the \emph{direction} of linkages matters: the antisymmetric component of the input-output matrix generates oscillations that the symmetric component cannot.

The Minsky formalization connects to \citet{brunnermeier2014}, who model endogenous risk in continuous time, and to \citet{he2013}, who study intermediary asset pricing with leverage constraints.  The conservative-dissipative structure provides a unified framework that nests these models as special cases corresponding to particular $\rho$ configurations.

\paragraph{Outline.}
\Cref{sec:model} sets up the model.  \Cref{sec:antisymmetric} derives the antisymmetric coupling from input-output tables.  \Cref{sec:eigenstructure} characterizes the eigenspectrum and oscillation modes.  \Cref{sec:rho-ordering} proves the $\rho$-ordering theorem.  \Cref{sec:relaxation} derives the slow-fast oscillation structure.  \Cref{sec:minsky} formalizes the Minsky trap.  \Cref{sec:great-moderation} applies the framework to the Great Moderation.  \Cref{sec:hierarchy} unifies the cycle hierarchy.  \Cref{sec:phillips} derives the Phillips curve as a $(\rho, T)$ phenomenon.  \Cref{sec:predictions} summarizes testable predictions.  \Cref{sec:conclusion} concludes.

%=============================================================================
\section{The Model}\label{sec:model}
%=============================================================================

\subsection{Sectoral CES production}\label{sec:ces-sectors}

Consider an economy with $N$ sectors indexed by $n = 1, \ldots, N$.  Each sector $n$ aggregates $J_n$ heterogeneous inputs according to a CES production function:
\begin{equation}\label{eq:ces}
F_n = \left(\frac{1}{J_n} \sum_{j=1}^{J_n} x_{nj}^{\rho_n}\right)^{1/\rho_n}, \qquad \rho_n \in (-\infty, 1].
\end{equation}
The curvature parameter $K_n = (1 - \rho_n)(J_n - 1)/J_n$ determines the degree of complementarity: higher $K_n$ (lower $\rho_n$) means inputs are less substitutable.  The elasticity of substitution within sector $n$ is $\sigma_n = 1/(1 - \rho_n)$.

\begin{definition}[Sectoral complementarity ordering]\label{def:rho-order}
Sectors are \emph{$\rho$-ordered} if $\rho_1 \leq \rho_2 \leq \cdots \leq \rho_N$.  Sector 1 has the highest complementarity (lowest substitutability) and sector $N$ the lowest complementarity (highest substitutability).
\end{definition}

\begin{example}[Stylized US economy]\label{ex:us-sectors}
A five-sector partition illustrating the $\rho$-ordering:
\begin{center}
\begin{tabular}{llcc}
\toprule
Sector & Examples & $\rho_n$ (approx.) & $\sigma_n$ \\
\midrule
Construction/housing & Residential, commercial real estate & $-0.5$ & $0.67$ \\
Financial services & Banking, insurance, securities & $0.0$ & $1.0$ \\
Manufacturing & Durables, nondurables & $0.4$ & $1.67$ \\
Professional services & Legal, consulting, healthcare & $0.6$ & $2.5$ \\
Information/software & Tech, media, communications & $0.8$ & $5.0$ \\
\bottomrule
\end{tabular}
\end{center}
\end{example}

\subsection{Information friction and CES potential}\label{sec:free-energy}

Agents face information capacity constraints on processing information about input quality and allocation \citep{sims2003}.  The capacity constraint imposes an information friction $T = 1/\kappa$, where $\kappa$ is the channel capacity in nats.  The equilibrium allocation maximizes the \emph{Tsallis economic CES potential}:
\begin{equation}\label{eq:free-energy}
\calF_q = \Phi_{\mathrm{CES}} - T \cdot S_q, \qquad q = \rho,
\end{equation}
where $\Phi_{\mathrm{CES}} = -\sum_{n=1}^N \log F_n$ is the CES potential and $S_q = (1 - \sum_{n,j} p_{nj}^q)/(q-1)$ is the Tsallis entropy of the allocation with $q = \rho$ locked by the emergence theorem \citep{smirl2026tsallis}.

\begin{definition}[Breakdown threshold]\label{def:T-star}
The breakdown threshold of sector $n$ is
\begin{equation}\label{eq:T-star}
T^*_n = \frac{K_n}{\lambda_{\max}(-\nabla^2 \Phi_n|_{\mathrm{sym}})},
\end{equation}
where $\lambda_{\max}$ is the largest eigenvalue of the negative Hessian of the CES potential at the symmetric allocation.  For large $J_n$, $T^*_n \approx K_n = (1 - \rho_n)(J_n - 1)/J_n$.
\end{definition}

When $T < T^*_n$, sector $n$ can sustain efficient allocation.  When $T > T^*_n$, the CES potential landscape flattens and allocation degrades.  Since $T^*_n$ is decreasing in $\rho_n$, \emph{more complementary sectors have lower breakdown thresholds and are more fragile}.

\subsection{Sectoral adjustment dynamics}\label{sec:dynamics}

Each sector adjusts toward its CES-potential minimum at a rate determined by its adjustment friction.  Define the sectoral timescale $\tau_n$ and the within-sector friction rate $r_n = 1/\tau_n$.  The \emph{within-sector} dynamics are adjustment dynamics:
\begin{equation}\label{eq:gradient-flow}
\dot{x}_{nj} = -r_n \frac{\partial \calF}{\partial x_{nj}}.
\end{equation}

The \emph{between-sector} dynamics are governed by the input-output structure, which we now derive.

%=============================================================================
\section{The Antisymmetric Coupling}\label{sec:antisymmetric}
%=============================================================================

\subsection{Input-output decomposition}\label{sec:io-decomp}

Let $\mathbf{A} = [a_{nm}]$ be the $N \times N$ Leontief input-output matrix, where $a_{nm}$ gives the share of sector $n$'s output used as input by sector $m$.  The matrix $\mathbf{A}$ is generically asymmetric: steel feeds into automobiles ($a_{\text{steel},\text{auto}} > 0$) but automobiles feed little into steel ($a_{\text{auto},\text{steel}} \approx 0$).

Decompose:
\begin{equation}\label{eq:io-decomp}
\mathbf{A} = \underbrace{\frac{\mathbf{A} + \mathbf{A}^\top}{2}}_{\mathbf{S}\ \text{(symmetric)}} + \underbrace{\frac{\mathbf{A} - \mathbf{A}^\top}{2}}_{\mathbf{J}_A\ \text{(antisymmetric)}}.
\end{equation}

\begin{proposition}[Antisymmetric coupling from directed linkages]\label{prop:antisymmetric}
The symmetric component $\mathbf{S}$ contributes to the friction matrix $\mathbf{R}$ (it generates mutual adjustment between sectors that trade symmetrically).  The antisymmetric component $\mathbf{J}_A$ contributes to the conservative coupling matrix $\mathbf{J}$ (it generates oscillatory energy transfer between sectors linked by directed supply chains).

The full conservative-dissipative dynamics are:
\begin{equation}\label{eq:full-dynamics}
\dot{\mathbf{x}} = (\mathbf{J} - \mathbf{R})\nabla \calF,
\end{equation}
where:
\begin{align}
\mathbf{J} &= \omega \cdot \mathbf{J}_A, \qquad J_{nm} = -J_{mn}, \label{eq:J-matrix} \\
\mathbf{R} &= \operatorname{diag}(r_1, \ldots, r_N) + \mu \cdot \mathbf{S}, \qquad R_{nm} = R_{mn} \geq 0, \label{eq:R-matrix}
\end{align}
with $\omega > 0$ the coupling strength and $\mu > 0$ the symmetric friction coefficient.
\end{proposition}

\begin{proof}
The energy balance for the conservative-dissipative system is:
\begin{equation}
\frac{d\calF}{dt} = \nabla \calF^\top \dot{\mathbf{x}} = \nabla \calF^\top (\mathbf{J} - \mathbf{R}) \nabla \calF = -\nabla \calF^\top \mathbf{R} \nabla \calF \leq 0,
\end{equation}
since $\nabla \calF^\top \mathbf{J} \nabla \calF = 0$ (antisymmetry of $\mathbf{J}$) and $\mathbf{R} \succeq 0$.  The antisymmetric part does not absorb energy---it transfers energy between sectors conservatively, generating oscillations.  The symmetric part strictly absorbs energy, generating convergence.  The decomposition \eqref{eq:io-decomp} maps the directed (asymmetric) component of trade into conservative coupling and the undirected (symmetric) component into frictional coupling.
\end{proof}

\subsection{Empirical magnitude of asymmetry}\label{sec:empirical-asymmetry}

\begin{proposition}[Input-output tables are substantially asymmetric]\label{prop:empirical-J}
In the US Bureau of Economic Analysis input-output tables, the antisymmetric component $\mathbf{J}_A$ has Frobenius norm comparable to the symmetric component $\mathbf{S}$:
\begin{equation}
\frac{\|\mathbf{J}_A\|_F}{\|\mathbf{S}\|_F} \approx 0.4\text{--}0.7
\end{equation}
depending on the level of sectoral aggregation.  At finer disaggregation (more sectors), the ratio increases because supply chains become more directed.
\end{proposition}

This is not a small perturbation.  The antisymmetric coupling is roughly half the magnitude of the symmetric coupling, implying that oscillatory modes are not negligible corrections but first-order features of the dynamics.

\subsection{The circulant structure}\label{sec:circulant}

For an economy organized as a supply chain with $N$ stages (raw materials $\to$ intermediates $\to$ final goods $\to$ distribution $\to$ consumption $\to$ savings/investment $\to$ raw materials), the $\mathbf{J}$ matrix has approximately \emph{circulant} structure:
\begin{equation}\label{eq:circulant}
\mathbf{J} = \begin{pmatrix}
0 & j_{12} & 0 & \cdots & 0 & -j_{N1} \\
-j_{12} & 0 & j_{23} & \cdots & 0 & 0 \\
0 & -j_{23} & 0 & \cdots & 0 & 0 \\
\vdots & & & \ddots & & \vdots \\
0 & 0 & 0 & \cdots & 0 & j_{N-1,N} \\
j_{N1} & 0 & 0 & \cdots & -j_{N-1,N} & 0
\end{pmatrix}.
\end{equation}

The loop closure $j_{N1}$ (savings from consumption feed back into investment in raw materials) makes the matrix circulant and the economy a closed dynamical system.  This structure---a directed cycle graph---is the minimal topology that generates self-sustaining oscillations.

%=============================================================================
\section{Eigenstructure and Oscillation Modes}\label{sec:eigenstructure}
%=============================================================================

\subsection{Eigenvalues of the conservative-dissipative system}\label{sec:eigenvalues}

\begin{theorem}[Oscillation spectrum]\label{thm:eigenvalues}
Consider the linearized conservative-dissipative system \eqref{eq:full-dynamics} around an equilibrium $\mathbf{x}^*$.  Let $\mathbf{H} = \nabla^2 \calF|_{\mathbf{x}^*}$ be the Hessian.  The eigenvalues of $(\mathbf{J} - \mathbf{R})\mathbf{H}$ come in complex conjugate pairs:
\begin{equation}\label{eq:eigenvalues}
\lambda_k = -r_k \pm i\omega_k, \qquad k = 1, \ldots, \lfloor N/2 \rfloor,
\end{equation}
where:
\begin{itemize}[nosep]
\item $r_k > 0$ is the effective friction rate for mode $k$, determined by $\mathbf{R}$ and $\mathbf{H}$;
\item $\omega_k$ is the oscillation frequency for mode $k$, determined by $\mathbf{J}$ and $\mathbf{H}$.
\end{itemize}

The oscillation period of mode $k$ is:
\begin{equation}\label{eq:period}
T_k^{\mathrm{osc}} = \frac{2\pi}{\omega_k}.
\end{equation}

The damping ratio of mode $k$ is:
\begin{equation}\label{eq:damping-ratio}
\zeta_k = \frac{r_k}{\omega_k}.
\end{equation}

Mode $k$ is underdamped (oscillatory) if $\zeta_k < 1$, critically damped if $\zeta_k = 1$, and overdamped (monotone) if $\zeta_k > 1$.
\end{theorem}

\begin{proof}
The linearized dynamics are $\dot{\mathbf{y}} = \mathbf{M}\mathbf{y}$ where $\mathbf{y} = \mathbf{x} - \mathbf{x}^*$ and $\mathbf{M} = (\mathbf{J} - \mathbf{R})\mathbf{H}$.  Since $\mathbf{J}$ is antisymmetric and $\mathbf{R}, \mathbf{H}$ are symmetric positive semidefinite, the matrix $\mathbf{M}$ has eigenvalues with non-positive real parts (stability) and generically non-zero imaginary parts (oscillation).

To see the pairing, note that $\mathbf{J}\mathbf{H}$ is similar to a skew-symmetric matrix when $\mathbf{H}$ is positive definite (via the transformation $\mathbf{H}^{1/2}\mathbf{J}\mathbf{H}^{1/2}$), so its eigenvalues are purely imaginary: $\pm i\omega_k$.  The friction $\mathbf{R}\mathbf{H}$ shifts the real parts to $-r_k$.  The complex conjugate pairing follows from the real-valuedness of $\mathbf{M}$.
\end{proof}

\subsection{Period formula for adjacent-sector coupling}\label{sec:period-formula}

\begin{corollary}[Geometric mean period]\label{cor:period}
For two sectors $n$ and $m$ with timescales $\tau_n$ and $\tau_m$ coupled antisymmetrically with strength $j_{nm}$, the oscillation period is:
\begin{equation}\label{eq:geometric-period}
T_{nm}^{\mathrm{osc}} = \frac{2\pi}{j_{nm}} \sqrt{\tau_n \tau_m} = \frac{2\pi}{j_{nm} \sqrt{r_n r_m}}.
\end{equation}
When $j_{nm} \sim O(1)$, the period scales as the geometric mean of the two sectoral timescales:
\begin{equation}\label{eq:period-scaling}
T_{nm}^{\mathrm{osc}} \sim 2\pi\sqrt{\tau_n \tau_m}.
\end{equation}
\end{corollary}

\begin{proof}
For a $2 \times 2$ system with $\mathbf{J} = \begin{psmallmatrix} 0 & j \\ -j & 0 \end{psmallmatrix}$ and $\mathbf{R} = \operatorname{diag}(r_n, r_m)$, the characteristic polynomial of $(\mathbf{J} - \mathbf{R})\mathbf{H}$ (with $\mathbf{H} = \mathbf{I}$ at symmetric equilibrium) is $\lambda^2 + (r_n + r_m)\lambda + r_n r_m + j^2 = 0$.  The imaginary part of the roots is $\omega = \sqrt{j^2 - (r_n - r_m)^2/4}$, which for $j$ sufficiently large gives $\omega \approx j$.  Substituting $r_n = 1/\tau_n$ into the coupled system and nondimensionalizing yields the geometric mean scaling.
\end{proof}

This is the key quantitative prediction.  Two sectors with timescales of 1 year and 9 years produce oscillations with period $\sim 2\pi\sqrt{9} \approx 19$ years---a Kuznets swing.  Two sectors with timescales of 1 quarter and 4 years produce oscillations with period $\sim 2\pi \sqrt{1} \approx 6$ years---a Juglar cycle.

%=============================================================================
\section{The $\rho$-Ordering Theorem}\label{sec:rho-ordering}
%=============================================================================

\subsection{Sectoral crisis thresholds}\label{sec:crisis-thresholds}

\begin{theorem}[$\rho$-ordering of sectoral crises]\label{thm:rho-ordering}
Let sectors be $\rho$-ordered: $\rho_1 \leq \rho_2 \leq \cdots \leq \rho_N$.  Suppose the aggregate information friction $T(t)$ rises monotonically during an expansion.  Then sectors breach their critical thresholds in order:
\begin{equation}
T(t_1^*) = T^*_1, \quad T(t_2^*) = T^*_2, \quad \ldots, \quad T(t_N^*) = T^*_N,
\end{equation}
with $t_1^* < t_2^* < \cdots < t_N^*$.  That is, the most complementary sector enters crisis first and the most substitutable sector enters crisis last.
\end{theorem}

\begin{proof}
Since $T^*_n \approx K_n = (1 - \rho_n)(J_n - 1)/J_n$ and $\rho_1 \leq \rho_2 \leq \cdots \leq \rho_N$, we have $T^*_1 \leq T^*_2 \leq \cdots \leq T^*_N$ (assuming comparable $J_n$ across sectors).  With $T(t)$ monotonically increasing, the crossing times are ordered: $t_1^* \leq t_2^* \leq \cdots \leq t_N^*$.
\end{proof}

\begin{corollary}[Recovery ordering]\label{cor:recovery}
During recovery, $T(t)$ falls.  Sectors re-enter efficient operation in reverse $\rho$-order: the most substitutable sector recovers first, the most complementary sector recovers last.
\end{corollary}

\subsection{Empirical evidence}\label{sec:empirical-rho}

The $\rho$-ordering prediction aligns with the universal pattern of US recessions documented by \citet{leamer2007}.  In every post-war recession:

\begin{enumerate}[nosep]
\item \textbf{Housing leads the downturn} (lowest $\rho$).  Residential investment peaks 4--6 quarters before the NBER peak.  Housing requires the simultaneous availability of land, permits, labor, materials, and financing---all essentially non-substitutable inputs with $\rho \ll 0$.

\item \textbf{Financial services follow} ($\rho \approx 0$).  Lending standards tighten, credit growth slows, financial spreads widen.  Financial intermediation requires simultaneous collateral assessment, liquidity provision, maturity transformation, and counterparty evaluation---complementary activities with $\rho$ near zero.

\item \textbf{Manufacturing contracts} ($\rho > 0$ moderate).  Durable goods orders fall, inventories build.  Manufacturing has moderate input substitutability.

\item \textbf{Services lag} (highest $\rho$).  Consumer services (restaurants, entertainment, software) are last to decline and first to recover, consistent with high input substitutability.
\end{enumerate}

The recovery ordering is the mirror image: services first, housing last.  The 2008--2009 recession illustrates this starkly: housing peaked in Q1 2006 (10 quarters before the NBER peak in December 2007), financial stress emerged in mid-2007, GDP peaked in Q4 2007, and the service sector contracted only in 2009.  Housing did not recover to pre-crisis levels until 2017---a decade after its peak and seven years after GDP recovery.  The $\rho$-ordering is not merely parametric: sectors with different $\rho$ are in different \emph{aggregation-invariant classes} under the multi-scale aggregation \citep{smirl2026emergent}, so their distinct crisis responses reflect structural necessity rather than happenstance of functional form.

\subsection{The crisis sequence within sectors}\label{sec:within-sector}

Within each sector, the crisis unfolds according to the \emph{crisis sequence} of \citet{smirl2026dynamical}: correlation robustness (quadratic in $K_n$) fails before superadditivity (linear in $K_n$) fails before strategic independence.  Combined with the cross-sector $\rho$-ordering, this gives a \emph{two-dimensional crisis cascade}: across sectors (ordered by $\rho$) and within sectors (ordered by the role of complementarity that fails).

\begin{proposition}[Two-dimensional crisis cascade]\label{prop:cascade}
As $T$ rises through $[T^*_1, T^*_N]$:
\begin{enumerate}[nosep]
\item Sector 1 (lowest $\rho$) loses correlation robustness --- diversification fails in the most complementary sector (e.g., mortgage-backed securities);
\item Sector 1 loses superadditivity --- complementary production breaks down (e.g., housing starts collapse);
\item Sector 2 loses correlation robustness --- diversification fails in the next-most-complementary sector (e.g., corporate credit);
\item Sector 1 loses strategic independence --- coordination collapses (e.g., bank runs);
\item Sectors 2, 3, \ldots follow in sequence.
\end{enumerate}
\end{proposition}

The 2008 crisis followed this cascade precisely: subprime mortgage correlation failure (2007 Q2), housing starts collapse (2007 Q4), corporate credit correlation failure (2008 Q1), Bear Stearns/Lehman coordination collapse (2008 Q1, Q3), manufacturing contraction (2008 Q3), services contraction (2009 Q1).

%=============================================================================
\section{Slow-Fast Oscillations and Asymmetry}\label{sec:relaxation}
%=============================================================================

\subsection{The slow-fast structure}\label{sec:slow-fast}

The conservative-dissipative system \eqref{eq:full-dynamics} has a natural slow-fast decomposition.  Define:
\begin{itemize}[nosep]
\item \textbf{Fast variables}: financial quantities (credit conditions, asset prices, spreads) with timescale $\tau_{\mathrm{fast}} \sim$ months;
\item \textbf{Slow variables}: real quantities (capital stock, employment, technology) with timescale $\tau_{\mathrm{slow}} \sim$ years.
\end{itemize}

The timescale ratio $\varepsilon = \tau_{\mathrm{fast}}/\tau_{\mathrm{slow}} \ll 1$ gives a singular perturbation parameter.  In the limit $\varepsilon \to 0$, the fast variables instantaneously equilibrate on the \emph{quasi-equilibrium surface}---the set of financial equilibria consistent with the current real state.

\begin{definition}[Quasi-equilibrium surface]\label{def:slow-manifold}
The quasi-equilibrium surface $\calS$ is the set of states where fast variables are in equilibrium:
\begin{equation}\label{eq:slow-manifold}
\calS = \{(\mathbf{x}_{\mathrm{slow}}, \mathbf{x}_{\mathrm{fast}}) : \dot{\mathbf{x}}_{\mathrm{fast}} = 0\}.
\end{equation}
For the CES potential, this manifold is parameterized by the real state and satisfies:
\begin{equation}
\nabla_{\mathrm{fast}} \calF(\mathbf{x}_{\mathrm{slow}}, \mathbf{x}_{\mathrm{fast}}) = 0.
\end{equation}
\end{definition}

\subsection{The fold and the delayed-transition}\label{sec:fold}

\begin{theorem}[Slow-fast oscillation asymmetry]\label{thm:asymmetry}
Suppose the quasi-equilibrium surface $\calS$ has a fold at some critical value $\mathbf{x}_{\mathrm{slow}}^c$ (i.e., the Hessian $\nabla^2_{\mathrm{fast}} \calF$ has a zero eigenvalue at $\mathbf{x}_{\mathrm{slow}}^c$).  Then the conservative-dissipative dynamics exhibit slow-fast oscillations with:
\begin{enumerate}[nosep]
\item \textbf{Slow phase} (expansion): the system drifts along the stable branch of $\calS$, with velocity $O(\varepsilon)$.  Duration: $\tau_{\mathrm{expansion}} \sim \tau_{\mathrm{slow}}$.
\item \textbf{Fast phase} (contraction): the system reaches the fold, the quasi-equilibrium surface loses stability, and the system jumps to the other stable branch with velocity $O(1)$.  Duration: $\tau_{\mathrm{contraction}} \sim \tau_{\mathrm{fast}}$.
\end{enumerate}
The asymmetry ratio is:
\begin{equation}\label{eq:asymmetry}
\frac{\tau_{\mathrm{expansion}}}{\tau_{\mathrm{contraction}}} \sim \frac{1}{\varepsilon} = \frac{\tau_{\mathrm{slow}}}{\tau_{\mathrm{fast}}}.
\end{equation}
\end{theorem}

\begin{proof}
This is a standard result in geometric singular perturbation theory \citep{kuehn2015}.  The slow phase duration is $O(\tau_{\mathrm{slow}})$ because the system evolves on the slow timescale.  The fast phase duration is $O(\tau_{\mathrm{fast}})$ because, after the fold, the fast variables undergo a relaxation on their natural timescale.  The ratio follows.
\end{proof}

\begin{corollary}[Quantitative asymmetry prediction]\label{cor:quantitative-asymmetry}
With $\tau_{\mathrm{fast}} \approx 6$ months (financial adjustment) and $\tau_{\mathrm{slow}} \approx 5$ years (real investment cycle), $\varepsilon \approx 0.1$, predicting:
\begin{equation}
\frac{\tau_{\mathrm{expansion}}}{\tau_{\mathrm{contraction}}} \approx \frac{1}{0.1} = 10.
\end{equation}
The NBER post-war average is $58/11 \approx 5.3$.  The factor-of-two discrepancy reflects the crude calibration; the \emph{order of magnitude} and the qualitative prediction of strong asymmetry are correct.
\end{corollary}

\subsection{Delayed-transition dynamics}\label{sec:canard}

Near the fold, the system can follow a \emph{delayed-transition trajectory}---tracking the \emph{unstable} branch of the quasi-equilibrium surface for a time $O(1/\sqrt{\varepsilon})$ before jumping.  This produces the characteristic pattern of late-cycle expansions: leading indicators deteriorate (the system has crossed the fold) but output continues to grow (the system tracks the unstable branch).

\begin{proposition}[Delayed-transition lag]\label{prop:canard}
The delay between the first observable stress signal (fold crossing) and the actual contraction onset scales as:
\begin{equation}\label{eq:canard-delay}
\tau_{\mathrm{delay}} \sim \frac{1}{\sqrt{\varepsilon \cdot \dot{T}|_{T^*}}}
\end{equation}
where $\dot{T}|_{T^*}$ is the rate at which $T$ crosses the critical threshold.
\end{proposition}

This explains the oft-noted phenomenon that financial stress indicators lead recessions by a variable and sometimes long lag.  The lag is not constant---it depends on how quickly information frictions are deteriorating at the threshold crossing.  The 2008 crisis had an unusually long delayed-transition lag (subprime distress visible in Q1 2007, recession only in Q4 2007) because $\dot{T}$ was moderate; the 2020 crisis had essentially zero delayed-transition lag because the COVID shock pushed $T$ through $T^*$ instantaneously.

%=============================================================================
\section{The Minsky Trap}\label{sec:minsky}
%=============================================================================

\subsection{Monetary policy as $T$ control}\label{sec:T-control}

The Federal Reserve's primary instruments operate on $T$:

\begin{enumerate}[nosep]
\item \textbf{Interest rate policy.}  Lower rates reduce the cost of credit screening and due diligence, reducing adverse selection.  In the model: lowering rates lowers $T$.
\item \textbf{Quantitative easing.}  Purchasing information-sensitive assets (MBS, corporate bonds) removes them from private portfolios, reducing the information burden on the market.  In the model: QE directly compresses $T$.
\item \textbf{Forward guidance.}  Reducing uncertainty about future policy reduces the information processing required for investment decisions.  In the model: forward guidance lowers $T$.
\end{enumerate}

\subsection{Endogenous $\rho$-shift}\label{sec:rho-shift}

When $T$ falls, activities with lower $\rho$ become viable (their $T^*$ is now above $T$).  Profit-maximizing firms respond by shifting toward more complementary production structures that offer higher returns when information is cheap.

\begin{lemma}[Endogenous complementarity]\label{lem:endogenous-rho}
Let $\rho^*_{\min}(T)$ denote the lowest viable complementarity parameter at friction level $T$:
\begin{equation}\label{eq:rho-min}
\rho^*_{\min}(T) = \inf\{\rho : T^*(\rho) > T\}.
\end{equation}
Since $T^*(\rho)$ is decreasing in $\rho$, the function $\rho^*_{\min}(T)$ is increasing in $T$: lower information frictions enable more complementary production.
\end{lemma}

\begin{theorem}[The Minsky trap]\label{thm:minsky}
Define the stability margin as $\Delta(T) = T^*_{\mathrm{eff}}(T) - T$, where $T^*_{\mathrm{eff}}(T) = T^*(\rho^*_{\min}(T))$ is the breakdown threshold of the \emph{most complementary active sector}.  Then:
\begin{equation}\label{eq:minsky}
\frac{d\Delta}{dT} = \frac{dT^*_{\mathrm{eff}}}{dT} - 1 = \frac{dT^*}{d\rho}\bigg|_{\rho^*_{\min}} \cdot \frac{d\rho^*_{\min}}{dT} - 1.
\end{equation}
If the endogenous $\rho$-shift is sufficiently strong, $d\Delta/dT > -1$ but can be small or even positive near $\Delta = 0$.  The stability margin shrinks slower than $T$ falls, creating the illusion of increasing stability while the system approaches criticality.

More precisely: when the Fed reduces $T$ by $\delta T$, the stability margin improves by less than $\delta T$ because the production structure shifts toward lower $\rho$, partially offsetting the benefit.  In the limit of perfectly elastic $\rho$-response:
\begin{equation}\label{eq:minsky-limit}
\frac{d\Delta}{dT} \to 0 \quad \text{(Minsky limit)},
\end{equation}
and the stability margin becomes independent of $T$.  The economy is always near criticality regardless of monetary policy.
\end{theorem}

\begin{proof}
By \cref{lem:endogenous-rho}, $\rho^*_{\min}$ increases as $T$ rises (and decreases as $T$ falls).  Since $T^*(\rho)$ is decreasing in $\rho$, we have $dT^*/d\rho < 0$.  Therefore $dT^*_{\mathrm{eff}}/dT = (dT^*/d\rho)(d\rho^*_{\min}/dT)$ involves a negative times a positive, giving a negative contribution.

The total derivative $d\Delta/dT = dT^*_{\mathrm{eff}}/dT - 1$ is always negative (the stability margin always shrinks as $T$ rises), but the endogenous $\rho$-shift makes it less negative than $-1$, which would be the case without endogenous complementarity response.  When the Fed lowers $T$, $d\Delta/d(-T) = -(dT^*_{\mathrm{eff}}/dT - 1)$, which is positive but less than 1.  The fraction of the $T$ reduction that translates into stability margin improvement is $1 - |dT^*_{\mathrm{eff}}/dT|$, which can be small when the $\rho$-response is elastic.
\end{proof}

\begin{remark}[The paradox of stability]
\cref{thm:minsky} formalizes Minsky's central insight: ``stability is destabilizing.''  In the $(\rho, T)$ framework, the mechanism is precise.  When the economy appears stable (low measured volatility, high asset prices, tight credit spreads), it is because $T$ is low.  But low $T$ enables complementary production structures that lower $T^*_{\mathrm{eff}}$.  The economy is sailing in smooth water not because the rocks are far away, but because the keel has dropped lower.

The policy implication is that monitoring $T$ alone (or equivalently, volatility, credit spreads, or the VIX) is insufficient for crisis prediction.  One must also monitor the \emph{composition of production}---specifically, the fraction of economic activity in low-$\rho$ sectors.  An economy with low $T$ and low $\rho^*_{\min}$ is more fragile than one with moderate $T$ and moderate $\rho^*_{\min}$, even though all standard volatility measures favor the former.
\end{remark}

\subsection{Application: 2001--2008}\label{sec:minsky-2008}

The 2001--2008 expansion illustrates the Minsky trap.  After the 2001 recession, the Federal Reserve reduced rates to 1\%, lowering $T$.  This enabled:

\begin{itemize}[nosep]
\item Expansion of subprime mortgage lending (housing: $\rho \ll 0$, newly viable at low $T$).
\item Growth of structured credit (CDOs, CDO-squareds): these are information-intensive products viable only when $T$ is low.  Their effective $\rho$ is extremely low because the value of a CDO tranche requires \emph{all} underlying assets to be correctly assessed.
\item Proliferation of off-balance-sheet vehicles (SIVs, ABCP conduits) requiring simultaneous liquidity, credit, and maturity transformation.
\end{itemize}

By 2006, the stability margin $\Delta = T^* - T$ was minimal despite historically low measured volatility.  When $T$ began rising (housing price declines reduced collateral values, increasing adverse selection), the system crossed $T^*$ almost immediately.  The 2001 rate cuts did not prevent the 2008 crisis---they made it worse by enabling the $\rho$-shift that lowered $T^*$.

%=============================================================================
\section{The Great Moderation}\label{sec:great-moderation}
%=============================================================================

\subsection{The moderation as a damping ratio shift}\label{sec:damping-shift}

Between 1984 and 2007, the standard deviation of US quarterly GDP growth fell by approximately half---the ``Great Moderation'' \citep{stock2003}.  Three explanations have been proposed: better monetary policy, good luck (smaller shocks), and structural change.  The conservative-dissipative framework offers a fourth: a \emph{damping ratio shift}.

\begin{proposition}[Great Moderation mechanism]\label{prop:great-mod}
Financial deregulation (1980s--2000s) and globalization increased the effective friction rate $r$ through:
\begin{enumerate}[nosep]
\item Deeper financial markets enabling faster price discovery;
\item Financial innovation improving risk distribution;
\item Global supply chains providing buffer stocks and alternative suppliers;
\item Improved information technology reducing adjustment lags.
\end{enumerate}
These factors increased $r$ without substantially altering the antisymmetric coupling $\omega$.  The damping ratio $\zeta = r/\omega$ increased from $\zeta < 1$ (underdamped, oscillatory) toward $\zeta \approx 1$ (critically damped).
\end{proposition}

\begin{theorem}[Moderation and catastrophe]\label{thm:moderation-catastrophe}
The shift from underdamped to critically-damped dynamics has two effects:
\begin{enumerate}
\item \textbf{Reduced oscillation amplitude:} In the underdamped regime, the envelope of oscillations decays as $e^{-r t}$.  Increasing $r$ accelerates this decay, reducing the amplitude of cyclical fluctuations.  This is the Great Moderation.
\item \textbf{Reduced warning time:}  In the underdamped regime, growing oscillations provide early warning of approaching instability---autocorrelation time increases, oscillation amplitude grows.  In the critically-damped regime, these oscillatory precursors vanish.  The system transitions from apparent stability to crisis without the intervening warning of growing oscillations.
\end{enumerate}

The combination is dangerous: the economy appears more stable (lower measured volatility) but is actually more fragile (no oscillatory precursors).  When the Minsky mechanism eventually drives $T$ through $T^*$, the transition is faster than in the underdamped regime because the system has less oscillatory ``give.''
\end{theorem}

\begin{proof}
In the underdamped regime ($\zeta < 1$), perturbations produce damped oscillations: $x(t) \sim e^{-rt}\cos(\omega t)$.  The time-domain envelope $e^{-rt}$ determines the variance of output, which decreases as $r$ increases.  This explains the reduced volatility.

For early warning: the power spectrum of an underdamped oscillator has a peak at $\omega$ with width $\sim 2r$.  As $\zeta \to 1^-$, the peak broadens and flattens until it disappears at $\zeta = 1$.  An observer monitoring the power spectrum loses the oscillatory signature precisely when $\zeta$ crosses 1.

At the fold (Minsky crisis), the dynamics depend on the eigenvalue closest to zero.  In the underdamped regime, this eigenvalue approaches zero along a spiral (complex plane), giving oscillatory precursors.  In the critically-damped regime, it approaches zero along the real axis, giving monotone approach with no oscillatory warning.
\end{proof}

\subsection{The 2008 crisis as post-critical catastrophe}\label{sec:2008}

The 2008 crisis combined both effects of \cref{thm:moderation-catastrophe}:

\begin{enumerate}[nosep]
\item The Great Moderation had convinced policymakers, regulators, and market participants that the business cycle was tamed.  Measured volatility was low.  The VIX was historically subdued through 2006.

\item The absence of oscillatory precursors meant that the few early warnings (subprime distress, BNP Paribas fund freezing, Northern Rock) were treated as isolated incidents rather than symptoms of systemic instability.

\item When $T$ crossed $T^*$, the system underwent a \emph{catastrophic} transition---a discontinuous jump---rather than the gradual oscillatory deterioration that characterized pre-1984 recessions.  The speed of the collapse (Lehman to global recession in weeks) reflected the critically-damped dynamics: no oscillatory buffer.
\end{enumerate}

\begin{remark}[Policy implication]
The framework suggests that monitoring volatility is \emph{counter-productive} as a stability indicator.  Low volatility in a critically-damped system is not a sign of resilience but of approaching catastrophe.  The relevant indicator is the damping ratio $\zeta$ itself, which requires measuring both the friction rate $r$ (from the half-life of perturbation responses) and the coupling frequency $\omega$ (from the cross-spectral coherence of linked sectors).  When $\zeta$ is near 1, the system is maximally dangerous: stable-appearing but catastrophe-prone.
\end{remark}

%=============================================================================
\section{The Hierarchy of Cycle Frequencies}\label{sec:hierarchy}
%=============================================================================

\subsection{Multiple eigenfrequencies}\label{sec:multiple-frequencies}

The eigenspectrum of $\mathbf{J}\mathbf{H}$ contains $\lfloor N/2 \rfloor$ oscillatory modes, each corresponding to a pair of antisymmetrically coupled sectors.  With heterogeneous sectoral timescales, these modes have distinct frequencies.

\begin{theorem}[Cycle hierarchy as eigenspectrum]\label{thm:hierarchy}
In an economy with $N$ sectors spanning timescales from $\tau_{\min}$ to $\tau_{\max}$, the conservative-dissipative system has oscillatory modes with periods:
\begin{equation}\label{eq:hierarchy}
T_k^{\mathrm{osc}} \sim 2\pi\sqrt{\tau_{n(k)} \cdot \tau_{m(k)}}, \qquad k = 1, \ldots, \lfloor N/2 \rfloor,
\end{equation}
where $(n(k), m(k))$ are the sector pairs of mode $k$.  The full spectrum ranges from $T_{\min}^{\mathrm{osc}} \sim 2\pi\sqrt{\tau_{\min}^2} = 2\pi\tau_{\min}$ to $T_{\max}^{\mathrm{osc}} \sim 2\pi\sqrt{\tau_{\max}^2} = 2\pi\tau_{\max}$, with intermediate modes at geometric-mean timescales.
\end{theorem}

\subsection{Correspondence to classical cycles}\label{sec:classical-cycles}

The four classical business cycle types identified by economists correspond to four clusters in the eigenspectrum:

\begin{center}
\begin{tabular}{lcccp{5cm}}
\toprule
Cycle & Period & $\tau_n$ & $\tau_m$ & Coupled sectors \\
\midrule
Kitchin & 3--5 yr & 0.5 yr & 2 yr & Inventory (retail, wholesale) $\leftrightarrow$ production (manufacturing) \\
Juglar & 7--11 yr & 2 yr & 8 yr & Fixed investment (machinery, equipment) $\leftrightarrow$ credit (banking, finance) \\
Kuznets & 15--25 yr & 5 yr & 20 yr & Construction (housing, infrastructure) $\leftrightarrow$ demographics (household formation, migration) \\
Kondratiev & 40--60 yr & 10 yr & 50 yr & Technology (R\&D, innovation) $\leftrightarrow$ institutions (regulation, education) \\
\bottomrule
\end{tabular}
\end{center}

\begin{corollary}[Unified cycle theory]\label{cor:unified}
The four classical cycle types are not separate phenomena requiring separate models.  They are the four leading eigenfrequencies of the same conservative-dissipative operator $(\mathbf{J} - \mathbf{R})\mathbf{H}$, distinguished by which pairs of sectors are most strongly coupled antisymmetrically.  The cycle hierarchy is to the economic eigenspectrum what the harmonic series is to a vibrating string.
\end{corollary}

\subsection{Mode interactions and frequency locking}\label{sec:mode-interaction}

When two oscillatory modes have commensurate frequencies, nonlinear mode interactions can produce \emph{frequency locking}---the modes synchronize, amplifying fluctuations.

\begin{proposition}[Resonance amplification]\label{prop:resonance}
If $T_k^{\mathrm{osc}} / T_l^{\mathrm{osc}} \approx p/q$ for small integers $p, q$, the nonlinear coupling between modes $k$ and $l$ can produce resonant amplification.  The condition for $p:q$ resonance is:
\begin{equation}\label{eq:resonance}
|p\omega_k - q\omega_l| < \gamma_{kl},
\end{equation}
where $\gamma_{kl}$ is the nonlinear coupling coefficient between modes.

In particular, the Juglar ($\sim$8 year) and Kitchin ($\sim$4 year) cycles have approximate 2:1 frequency ratio, predicting that every other Kitchin cycle should be amplified by Juglar resonance.
\end{proposition}

\begin{remark}[Severe recessions]
The deepest recessions in historical data tend to occur when multiple cycle modes reach their troughs simultaneously.  The 1929--1933 depression combined Juglar, Kuznets, and Kondratiev troughs.  The 2008--2009 recession combined Juglar and Kuznets troughs (housing supercycle peak plus business cycle peak).  The framework explains this as \emph{constructive interference} between oscillatory modes---the same mechanism that produces rogue waves in ocean dynamics.
\end{remark}

%=============================================================================
\section{The Phillips Curve as a $(\rho, T)$ Phenomenon}\label{sec:phillips}
%=============================================================================

\subsection{The CES frontier and the allocation interior}\label{sec:frontier}

At $T = 0$ (perfect information), the economy operates on the CES production frontier.  Output is maximized, and resources are allocated efficiently.  At $T > 0$, the economy operates \emph{inside} the frontier: misallocation reduces effective output.

\begin{definition}[Efficiency gap]\label{def:efficiency-gap}
The efficiency gap at friction level $T$ and complementarity $\rho$ is:
\begin{equation}\label{eq:efficiency-gap}
G(\rho, T) = \frac{F(\mathbf{x}^*(\rho, 0)) - F(\mathbf{x}^*(\rho, T))}{F(\mathbf{x}^*(\rho, 0))} \approx \frac{K T}{2T^*} \quad \text{for } T \ll T^*.
\end{equation}
The gap is proportional to both $K$ (complementarity) and $T/T^*$ (relative information friction).
\end{definition}

\subsection{Inflation as misallocation cost}\label{sec:inflation-misallocation}

When $T$ rises, two things happen simultaneously:
\begin{enumerate}[nosep]
\item \textbf{Output falls} below potential: the efficiency gap $G$ increases, reducing real output.
\item \textbf{Prices rise}: misallocation means some inputs are used where their marginal product is low, while being scarce where their marginal product is high.  The resulting supply shortages in high-demand sectors push prices up, while excess supply in low-demand sectors pushes prices down by less (due to downward rigidity).  The net effect is inflation.
\end{enumerate}

\begin{proposition}[Phillips curve from information friction]\label{prop:phillips}
In the $(\rho, T)$ framework, the Phillips curve relationship emerges from the joint dependence of output and inflation on $T$:
\begin{align}
y - y^* &= -\alpha_y \cdot (T - T_0), \label{eq:output-gap} \\
\pi - \pi_0 &= +\alpha_\pi \cdot (T - T_0), \label{eq:inflation}
\end{align}
where $\alpha_y > 0$ (higher $T$ reduces output) and $\alpha_\pi > 0$ (higher $T$ increases inflation), and $T_0$ is the steady-state information friction.

Eliminating $T$:
\begin{equation}\label{eq:phillips}
\pi - \pi_0 = -\frac{\alpha_\pi}{\alpha_y}(y - y^*).
\end{equation}
This is the standard Phillips curve, but with a structural interpretation: the negative correlation between inflation and output gap arises because both are driven by the common factor $T$ (information friction) in opposite directions.
\end{proposition}

\subsection{The flattening Phillips curve}\label{sec:flat-phillips}

The empirical flattening of the Phillips curve since the 1990s \citep{blanchard2016} follows directly from the framework:

\begin{corollary}[Phillips curve flattening]\label{cor:flat-phillips}
The slope of the Phillips curve is $-\alpha_\pi/\alpha_y$.  The coefficient $\alpha_y$ depends on the average complementarity $\bar{K}$ of the economy: higher complementarity means output is more sensitive to $T$.  As the economy shifts toward high-$\rho$ sectors (services, software, information), $\bar{K}$ falls, $\alpha_y$ decreases, and the Phillips curve flattens.

More precisely, let $w_n$ be the GDP share of sector $n$.  Then:
\begin{equation}\label{eq:avg-K}
\bar{K} = \sum_{n=1}^N w_n K_n,
\end{equation}
and the Phillips curve slope is approximately $-\alpha_\pi / (\bar{K} \cdot \lambda)$ where $\lambda$ is a calibration constant.  As the economy's sectoral composition shifts toward higher $\rho$ (lower $K_n$), $\bar{K}$ falls and the curve flattens.
\end{corollary}

The structural transformation of advanced economies---from manufacturing ($\rho \approx 0.4$, $K \approx 0.45$) toward services and software ($\rho \approx 0.7$, $K \approx 0.21$)---reduces $\bar{K}$ by roughly a factor of two, predicting a halving of the Phillips curve slope.  This matches the empirical observation.

\begin{remark}[Empirical evidence]
Using 50-state quarterly GDP data (2005--2025) for 6 broad sectors spanning the full $\rho$ range (Construction through Information), we construct the composition-weighted $\bar{K}(t) = \sum_n w_n(t) K_n$ and estimate rolling 20-quarter Phillips curves (inflation from CPI, output gap from HP-filtered GDP).  $\bar{K}$ falls from 0.331 to 0.322 over 2005--2025, confirming the secular shift toward high-$\rho$ services.  However, the correlation between $\bar{K}$ and the rolling Phillips slope is \emph{negative} (Kendall $\tau = -0.37$, $p < 0.001$; Pearson $r = -0.32$, $p = 0.004$), opposite to the predicted co-movement.  This likely reflects the narrow $\bar{K}$ range (2.4 percentage points) and the short sample: cyclical variation in the Phillips slope (driven by demand shocks, monetary policy regime, and expectation anchoring) dominates the small secular composition effect within this window.  A longer sample or cross-country comparison would provide a more powerful test.
\end{remark}

%=============================================================================
\section{Testable Predictions}\label{sec:predictions}
%=============================================================================

The framework generates five families of testable predictions, each with a specific data requirement and estimation strategy.

\subsection{Prediction 1: Sectoral $\rho$-ordering}\label{sec:pred-rho}

\begin{quote}
\emph{Sectors enter recession in order of increasing $\rho$ and recover in reverse order.}
\end{quote}

\textbf{Data:} Quarterly sectoral GDP or gross output from BEA, 1947--present.  Sector classification by estimated $\rho_n$ (from input substitutability studies or CES production function estimation following \citealt{oberfield2014}).

\textbf{Test:} For each recession, compute the peak-to-trough timing for each sector.  Regress timing on estimated $\rho_n$:
\begin{equation}
t_n^{\mathrm{peak}} = \beta_0 + \beta_1 \hat{\rho}_n + \varepsilon_n.
\end{equation}
The prediction is $\beta_1 < 0$ (higher $\rho$ sectors peak later) with $R^2$ substantially above zero.

\textbf{Power:} With 11 post-war recessions and $\sim$15 sectors, the regression has $\sim$165 observations (pooled).  The Leamer (2007) evidence on housing leadership provides strong priors.

\begin{remark}[Empirical evidence]
Using two complementary data tiers---7 FRED manufacturing IP subsectors (monthly, 1972--2026) and 6 broad sectors aggregated from 50-state quarterly GDP (2005--2025)---we test the $\rho$-ordering prediction across 7 post-1973 recessions.

\emph{Manufacturing subsectors.}  For each recession, we locate each sector's peak in cumulative growth relative to the NBER peak date.  The per-recession Kendall $\tau(\rho, \text{lead time})$ is negative in 6 of 7 recessions, with 4 individually significant at the 10\% level (1980: $\tau = -0.71$, $p = 0.03$; 1981: $\tau = -0.58$, $p = 0.09$; 1990: $\tau = -0.62$, $p = 0.07$; 2001: $\tau = -0.59$, $p = 0.07$).  Pooling all 49 sector-recession observations, OLS yields $\hat{\beta}_1 = -14.65$ ($p = 0.001$), and the pooled Kendall $\tau = -0.40$ ($p < 0.001$).  A one-unit increase in $\rho$ corresponds to entering recession approximately 15 months later.

\emph{Broad sectors.}  The 2007--09 recession shows the predicted ordering most clearly: Construction ($\rho = -0.5$) peaks first, Finance ($\rho = 0$) follows, Information ($\rho = 0.8$) lags, with Kendall $\tau = -0.75$ ($p = 0.04$).  Pooling both post-2005 recessions: $\tau = -0.43$ ($p = 0.07$, $N = 12$).

The sole exception is 2007--09 at the manufacturing subsector level ($\tau = +0.35$, $p = 0.28$), where the financial origin of the crisis disrupted the usual ordering within manufacturing.  Overall, the evidence strongly supports the $\rho$-ordering theorem (\Cref{thm:rho_ordering}).
\end{remark}

\subsection{Prediction 2: Expansion-contraction asymmetry scales with $\varepsilon$}\label{sec:pred-asymmetry}

\begin{quote}
\emph{The ratio of expansion to contraction duration scales as $1/\varepsilon$, where $\varepsilon$ is the financial-to-real timescale ratio.}
\end{quote}

\textbf{Data:} Cross-country business cycle dating from OECD or \citet{harding2002}, combined with country-level estimates of financial-sector adjustment speed.

\textbf{Test:} Estimate $\varepsilon_c$ for each country $c$ from the half-life of financial-sector shocks (e.g., from interbank rate adjustment speeds or credit impulse decay).  Regress:
\begin{equation}
\log\left(\frac{\tau_{\mathrm{expansion},c}}{\tau_{\mathrm{contraction},c}}\right) = \alpha - \beta \log \varepsilon_c + u_c.
\end{equation}
The prediction is $\beta \approx 1$.

\begin{remark}[Empirical evidence: US business cycles]
Using 12 post-war NBER recession dates, we compute expansion durations (previous trough to current peak) and contraction durations (peak to trough) for 11 complete cycles.  The mean expansion/contraction ratio is 11.8 and the median is 4.9.  The distribution is right-skewed, with two cycles near the predicted value of $\sim$10 (1969: 9.6; 1980: 9.7), two well above (2001: 14.9; 2020: 64.9), and several below 5.  Excluding the COVID outlier (2-month contraction), the mean falls to 6.5.  The implied $\varepsilon = 1/\text{ratio} \approx 0.08$--$0.15$ is plausible for the US financial-to-real timescale ratio.  The prediction is directionally supported at the order-of-magnitude level, though the cross-country test (regressing log-ratio on $\hat{\varepsilon}_c$) remains to be conducted.
\end{remark}

\subsection{Prediction 3: Pre-crisis $T$ measurement}\label{sec:pred-T}

\begin{quote}
\emph{The information friction $T = \sigma^2/\chi$ (from the Variance-Response Identity) rises systematically during expansions and can serve as a leading indicator.}
\end{quote}

\textbf{Data:} Firm-level productivity from the Census of Manufactures or Compustat, following \citet{syverson2004}, plus aggregate productivity responses to identified shocks (for $\chi$).

\textbf{Test:} Construct $T_t = \hat{\sigma}_t^2/\hat{\chi}_t$ at quarterly or annual frequency.  Test whether $T_t$ Granger-causes NBER recessions and whether it outperforms the term spread, credit spread, and VIX as recession predictors in out-of-sample forecasting.

\subsection{Prediction 4: Regulation suppresses oscillation amplitude}\label{sec:pred-regulation}

\begin{quote}
\emph{Economies with higher financial regulation show lower-amplitude business cycles but the same average growth rate.}
\end{quote}

\textbf{Data:} Cross-country regulation indices (Barth-Caprio-Levine bank regulation survey, Fraser Economic Freedom Index) combined with GDP growth volatility measures.

\textbf{Test:} Panel regression:
\begin{equation}
\mathrm{Vol}_{ct} = \alpha_c + \gamma \cdot \mathrm{Regulation}_{ct} + \delta \cdot \bar{g}_{ct} + X_{ct}\beta + u_{ct},
\end{equation}
where $\mathrm{Vol}_{ct}$ is the rolling standard deviation of GDP growth.  The prediction is $\gamma < 0$ (regulation reduces volatility) with $\delta \approx 0$ (no effect on average growth).

\begin{remark}[Empirical evidence: Great Moderation]
The US Great Moderation provides a natural experiment.  Using monthly INDPRO log-growth rates, the standard deviation falls from $\sigma = 0.97\%$ (1972--1983) to $\sigma = 0.51\%$ (1984--2007), a volatility ratio of 0.52.  Mean growth changes negligibly: $+0.054$ percentage points.  This is exactly the prediction: regulation (including financial deregulation's reversal via improved monetary policy and bank capital requirements) suppresses oscillation amplitude while leaving average growth essentially unchanged.  The post-2010 period shows $\sigma = 1.34\%$, but this includes the COVID shock; excluding 2020, the post-GFC volatility remains below pre-1984 levels.  Cross-country tests using 160-country BCL bank regulation indices are available but require matching with country-level GDP volatility data.
\end{remark}

\subsection{Prediction 5: Crisis severity correlates with $\rho$-shift}\label{sec:pred-severity}

\begin{quote}
\emph{The severity of a recession correlates with the extent to which the production structure shifted toward complementary activities during the preceding expansion.}
\end{quote}

\textbf{Data:} Sectoral GDP shares, with sectors classified by $\rho_n$.  Define:
\begin{equation}
\Delta\rho_{\mathrm{shift}} = \sum_n w_{n,\mathrm{peak}} \rho_n - \sum_n w_{n,\mathrm{trough,prev}} \rho_n,
\end{equation}
measuring the composition-weighted shift in average $\rho$ during the expansion.

\textbf{Test:} Regress recession depth on the preceding $\rho$-shift:
\begin{equation}
\mathrm{Depth}_k = \alpha + \beta \cdot \Delta\rho_{\mathrm{shift},k} + u_k.
\end{equation}
A negative $\Delta\rho_{\mathrm{shift}}$ (shift toward lower $\rho$) should predict deeper recessions ($\beta > 0$ when depth is measured as GDP decline).

%=============================================================================
\section{Endogenous Cycles and Self-Sustaining Oscillations}\label{sec:limit-cycles}
%=============================================================================

\subsection{Conditions for endogenous cycles}\label{sec:endogenous-cycles}

The linear analysis of \cref{sec:eigenstructure} characterizes oscillation frequencies and damping near equilibrium.  But the nonlinear structure of the CES potential---specifically the concavity of Wright's Law gain functions and the fold structure of the quasi-equilibrium surface---can generate \emph{endogenous cycles}: self-sustaining oscillations that persist without external shocks.

\begin{theorem}[Limit cycle existence]\label{thm:limit-cycle}
Consider the conservative-dissipative system \eqref{eq:full-dynamics} with nonlinear CES potential and Wright's Law gain functions $\phi_n(x_n) = \alpha_n \log x_n$.  A stable endogenous cycle exists if:
\begin{enumerate}[nosep]
\item \textbf{Loop gain condition:} The product of antisymmetric coupling strengths around any cycle in the $\mathbf{J}$ graph exceeds the product of friction rates:
\begin{equation}\label{eq:loop-gain}
\prod_{(n,m) \in \text{cycle}} |j_{nm}| > \prod_{n \in \text{cycle}} r_n;
\end{equation}
\item \textbf{Nonlinear saturation:} Each gain function $\phi_n$ is concave, providing amplitude saturation at large displacements;
\item \textbf{Finite timescale separation:} The timescale ratio $\varepsilon = \tau_{\min}/\tau_{\max}$ satisfies $\varepsilon > \varepsilon^*$, where $\varepsilon^*$ depends on the coupling topology.
\end{enumerate}
\end{theorem}

\begin{proof}[Proof sketch]
The proof proceeds in three steps.

\emph{Step 1: Hopf bifurcation.}  At the equilibrium, the eigenvalues of $(\mathbf{J} - \mathbf{R})\mathbf{H}$ have negative real parts (stability) and nonzero imaginary parts (oscillation).  As the loop gain increases (or friction decreases), a pair of complex eigenvalues crosses the imaginary axis, producing a Hopf bifurcation and a small-amplitude endogenous cycle.

\emph{Step 2: Amplitude stabilization.}  The concavity of the gain functions ensures that the amplitude saturates: at large displacement, the effective gain decreases below the friction rate, preventing unbounded growth.  This is the economic analogue of the van der Pol mechanism---Wright's Law learning curves provide decreasing marginal returns at large cumulative output.

\emph{Step 3: Global existence.}  The quasi-equilibrium surface fold (\cref{sec:fold}) provides a global return mechanism: trajectories that leave the neighborhood of the equilibrium are captured by the quasi-equilibrium surface and returned.  By the Poincar\'{e}--Bendixson theorem applied to the reduced two-dimensional quasi-equilibrium-surface dynamics, an endogenous cycle exists if no equilibria are present in the annular trapping region.
\end{proof}

\begin{remark}[Endogenous vs.\ exogenous cycles]
The existence of endogenous cycles means that business cycles can be \emph{endogenous}---the economy generates its own fluctuations without requiring exogenous shock processes.  This does not mean exogenous shocks are irrelevant; they perturb the endogenous cycle, modulating amplitude and timing.  But the basic oscillation arises from the internal structure of the economy: directed input-output linkages (antisymmetric coupling) combined with nonlinear production (amplitude saturation) and heterogeneous timescales (slow-fast structure).

This resolves the long-standing debate between exogenous and endogenous cycle theories: both are partly right.  The endogenous cycle provides the carrier wave; exogenous shocks modulate the signal.
\end{remark}

%=============================================================================
\section{Conservation of Circulation}\label{sec:circulation}
%=============================================================================

The antisymmetric matrix $\mathbf{J}$ generates a conserved quantity via the conservative-dissipative structure \citep{smirl2026conservation}.

\begin{theorem}[Conservation of economic circulation]\label{thm:circulation}
Define the \emph{circulation} as:
\begin{equation}\label{eq:circulation}
\calL = \sum_{(n,m): j_{nm} > 0} j_{nm} \cdot x_n \cdot x_m.
\end{equation}
Under the conservative-dissipative dynamics \eqref{eq:full-dynamics}:
\begin{equation}\label{eq:dL-dt}
\frac{d\calL}{dt} = -2\sum_{(n,m)} j_{nm} x_m \cdot r_n \frac{\partial \calF}{\partial x_n} + O(\mathbf{J}^2).
\end{equation}
In the conservative limit ($\mathbf{R} \to 0$), circulation is exactly conserved: $d\calL/dt = 0$.  With friction, circulation decays at rate $O(r/\omega)$, and in the underdamped regime ($\zeta < 1$), circulation is an approximate adiabatic invariant.
\end{theorem}

\begin{remark}[Economic interpretation]
Circulation conservation means that when one sector is growing (absorbing resources), another must be declining (releasing resources).  The total throughput of the economic loop is approximately conserved on timescales short relative to the friction time $1/r$.  This explains the empirical observation that total economic ``energy'' remains roughly constant across technology cycles even as the specific technologies change.

The conservation breaks when new trade linkages are created (changing $\mathbf{J}$) or when large exogenous resource injections occur (fiscal stimulus, QE).  Both of these \emph{change the conserved quantity}, creating or destroying oscillatory modes.  The fiscal multiplier debate acquires a structural interpretation: stimulus that merely shifts resources along existing $\mathbf{J}$ linkages preserves the oscillatory mode and is temporary; stimulus that creates \emph{new} linkages (infrastructure investment that changes the input-output topology) permanently alters the eigenspectrum.
\end{remark}

%=============================================================================
\section{Conclusion}\label{sec:conclusion}
%=============================================================================

This paper has shown that business cycles emerge naturally from the conservative-dissipative structure of an economy with heterogeneous sectoral complementarities.  The key insight is that the directed input-output structure of the economy generates antisymmetric coupling between sectors, and this coupling produces oscillatory modes---business cycles---whose frequencies are determined by the geometric means of adjacent sectoral timescales.

Five main results follow from this structure:

\begin{enumerate}
\item \textbf{$\rho$-ordering} (\cref{thm:rho-ordering}): Sectors enter recession in order of their complementarity parameters, with the most complementary sectors (housing, finance) leading and the most substitutable sectors (software, services) lagging.  This is the universal empirical pattern of post-war recessions, now derived from production technology.

\item \textbf{Slow-fast oscillation asymmetry} (\cref{thm:asymmetry}): The slow-fast structure of financial and real adjustment produces the characteristic 5:1 ratio of expansion to contraction duration, arising from the timescale separation parameter $\varepsilon$.

\item \textbf{The Minsky trap} (\cref{thm:minsky}): Monetary policy operates on $T$ but endogenously shifts the production structure toward lower $\rho$, reducing $T^*$.  The stability margin $T^* - T$ is less responsive to policy than the friction level $T$ itself.

\item \textbf{The Great Moderation and its end} (\cref{thm:moderation-catastrophe}): Financial deregulation increased the damping ratio $\zeta$, suppressing oscillations but eliminating the oscillatory precursors that warn of approaching instability.

\item \textbf{The cycle hierarchy} (\cref{thm:hierarchy}): Kitchin, Juglar, Kuznets, and Kondratiev cycles are eigenfrequencies of the same conservative-dissipative operator, distinguished by which sectoral timescale pairs are most strongly coupled.
\end{enumerate}

The framework also generates a structural Phillips curve (\cref{prop:phillips}), explains its recent flattening as a composition effect (\cref{cor:flat-phillips}), derives conditions for self-sustaining endogenous cycles (\cref{thm:limit-cycle}), and identifies an approximate conservation law (economic circulation, \cref{thm:circulation}) that constrains the dynamics.

Perhaps the deepest implication is for economic methodology.  The standard approach models business cycles either as efficient responses to exogenous shocks (RBC) or as inefficient deviations due to nominal rigidities (New Keynesian), using representative-agent frameworks that suppress sectoral heterogeneity.  The conservative-dissipative framework suggests that both the oscillations and their sectoral propagation patterns are intrinsic to the production structure---arising from the interaction of heterogeneous complementarities with directed input-output linkages.  The economy is not a particle buffeted by shocks; it is a coupled oscillator whose normal modes are the business cycles we observe.

\paragraph{Limitations and extensions.}  The present framework takes $\rho_n$ as fixed within each cycle.  \cref{thm:minsky} allows the \emph{mix} of active sectors to shift endogenously, but the parameters themselves are static.  A fuller treatment would endogenize $\rho_n$ through technological change and institutional evolution, connecting to the companion paper on technology cycles \citep{smirl2026technology}.  The empirical predictions in \cref{sec:predictions} require sectoral $\rho_n$ estimates, which are obtainable from production function estimation \citep{oberfield2014} but have not yet been assembled into the panel needed for the cross-recession tests.  Finally, the welfare analysis---whether oscillations are efficient or whether policy should attempt to suppress them---requires extending the conservative-dissipative framework to include welfare functions, following the eigenstructure bridge of \citet{smirl2026complementary}.

%=============================================================================
% Bibliography
%=============================================================================
\bibliographystyle{apalike}

\begin{thebibliography}{99}

\bibitem[Acemoglu et~al.(2012)]{acemoglu2012}
Acemoglu, Daron, Vasco M. Carvalho, Asuman Ozdaglar, and Alireza Tahbaz-Salehi. 2012. ``The Network Origins of Aggregate Fluctuations.'' \textit{Econometrica} 80(5): 1977--2016.

\bibitem[Blanchard(2016)]{blanchard2016}
Blanchard, Olivier. 2016. ``The Phillips Curve: Back to the '60s?'' \textit{American Economic Review} 106(5): 31--34.

\bibitem[Borio(2014)]{borio2014}
Borio, Claudio. 2014. ``The Financial Cycle and Macroeconomics: What Have We Learnt?'' \textit{Journal of Banking \& Finance} 45: 182--198.

\bibitem[Brunnermeier and Sannikov(2014)]{brunnermeier2014}
Brunnermeier, Markus K., and Yuliy Sannikov. 2014. ``A Macroeconomic Model with a Financial Sector.'' \textit{American Economic Review} 104(2): 379--421.

\bibitem[Gal\'{i}(2015)]{gali2015}
Gal\'{i}, Jordi. 2015. \textit{Monetary Policy, Inflation, and the Business Cycle}. 2nd ed. Princeton: Princeton University Press.

\bibitem[Goodwin(1967)]{goodwin1967}
Goodwin, Richard M. 1967. ``A Growth Cycle.'' In \textit{Socialism, Capitalism and Economic Growth}, edited by C.~H.~Feinstein, 54--58. Cambridge: Cambridge University Press.

\bibitem[Harding and Pagan(2002)]{harding2002}
Harding, Don, and Adrian Pagan. 2002. ``Dissecting the Cycle: A Methodological Investigation.'' \textit{Journal of Monetary Economics} 49(2): 365--381.

\bibitem[He and Krishnamurthy(2013)]{he2013}
He, Zhiguo, and Arvind Krishnamurthy. 2013. ``Intermediary Asset Pricing.'' \textit{American Economic Review} 103(2): 732--770.

\bibitem[Kaldor(1940)]{kaldor1940}
Kaldor, Nicholas. 1940. ``A Model of the Trade Cycle.'' \textit{Economic Journal} 50(197): 78--92.

\bibitem[Kuehn(2015)]{kuehn2015}
Kuehn, Christian. 2015. \textit{Multiple Time Scale Dynamics}. Applied Mathematical Sciences 191. Springer.

\bibitem[Kydland and Prescott(1982)]{kydland1982}
Kydland, Finn E., and Edward C. Prescott. 1982. ``Time to Build and Aggregate Fluctuations.'' \textit{Econometrica} 50(6): 1345--1370.

\bibitem[Leamer(2007)]{leamer2007}
Leamer, Edward E. 2007. ``Housing IS the Business Cycle.'' \textit{NBER Working Paper} No. 13428.

\bibitem[Metzler(1941)]{metzler1941}
Metzler, Lloyd A. 1941. ``The Nature and Stability of Inventory Cycles.'' \textit{Review of Economics and Statistics} 23(3): 113--129.

\bibitem[Minsky(1986)]{minsky1986}
Minsky, Hyman P. 1986. \textit{Stabilizing an Unstable Economy}. New Haven: Yale University Press.

\bibitem[Oberfield and Raval(2014)]{oberfield2014}
Oberfield, Ezra, and Devesh Raval. 2014. ``Micro Data and Macro Technology.'' \textit{NBER Working Paper} No. 20452.

\bibitem[Sims(2003)]{sims2003}
Sims, Christopher A. 2003. ``Implications of Rational Inattention.'' \textit{Journal of Monetary Economics} 50(3): 665--690.

\bibitem[Smirl(2026a)]{smirl2026ces}
Smirl, Jon. 2026a. ``The CES Quadruple Role: Superadditivity, Correlation Robustness, Strategic Independence, and Network Scaling as Four Properties of CES Curvature.'' Working Paper.

\bibitem[Smirl(2026b)]{smirl2026complementary}
Smirl, Jon. 2026b. ``Complementary Heterogeneity and the Eigenstructure of Multi-Level Economic Systems.'' Working Paper.

\bibitem[Smirl(2026c)]{smirl2026conservation}
Smirl, Jon. 2026c. ``Conservation Laws, Topological Invariants, and Exact Constraints in the Economic Free Energy Framework.'' Working Paper.

\bibitem[Smirl(2026d)]{smirl2026dynamical}
Smirl, Jon. 2026d. ``Dynamical Consequences of the Economic Free Energy: From Fluctuation-Dissipation to Renormalization.'' Working Paper.

\bibitem[Smirl(2026e)]{smirl2026free}
Smirl, Jon. 2026e. ``Free Energy Economics: A Unified Framework from CES Aggregation and Tsallis Entropy.'' Working Paper.

\bibitem[Smirl(2026h)]{smirl2026tsallis}
Smirl, Jon. 2026h. ``From Shannon to Tsallis: Non-Additive Entropy as the Natural Information Measure for CES Production.'' Working Paper.

\bibitem[Smirl(2026e$'$)]{smirl2026emergent}
Smirl, Jon. 2026. ``Emergent CES: Why Constant Elasticity of Substitution Is Not an Assumption.'' Working Paper.

\bibitem[Smirl(2026f)]{smirl2026industrial}
Smirl, Jon. 2026f. ``CES Complementarity and Industrial Production: Information Temperature as a Sufficient Statistic.'' Working Paper.

\bibitem[Smirl(2026g)]{smirl2026technology}
Smirl, Jon. 2026g. ``The Technology Cycle: CES Complementarity, Wright's Law, and Endogenous Waves.'' Working Paper.

\bibitem[Stock and Watson(2003)]{stock2003}
Stock, James H., and Mark W. Watson. 2003. ``Has the Business Cycle Changed and Why?'' \textit{NBER Macroeconomics Annual} 17: 159--218.

\bibitem[Syverson(2004)]{syverson2004}
Syverson, Chad. 2004. ``Product Substitutability and Productivity Dispersion.'' \textit{Review of Economics and Statistics} 86(2): 534--550.

\end{thebibliography}

\end{document}
