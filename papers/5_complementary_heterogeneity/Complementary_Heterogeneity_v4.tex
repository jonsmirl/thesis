\documentclass[12pt,letterpaper]{article}

% Page layout
\usepackage[margin=1in]{geometry}
\usepackage{setspace}
\onehalfspacing

% Math
\usepackage{amsmath,amssymb,amsthm,mathtools}

% Tables and figures
\usepackage{booktabs}
\usepackage{array}
\usepackage{tabularx}
\usepackage{multirow}
\usepackage{graphicx}
\usepackage{float}

% Typography
\usepackage[T1]{fontenc}
\usepackage[expansion=false]{microtype}
\usepackage{enumitem}

% References
\usepackage{xcolor}
\usepackage[colorlinks=true,linkcolor=blue,citecolor=blue,urlcolor=blue]{hyperref}

% Theorem environments
\newtheorem{theorem}{Theorem}[section]
\newtheorem{proposition}[theorem]{Proposition}
\newtheorem{lemma}[theorem]{Lemma}
\newtheorem{corollary}[theorem]{Corollary}
\newtheorem{definition}[theorem]{Definition}
\theoremstyle{remark}
\newtheorem{remark}[theorem]{Remark}

% Custom commands
\newcommand{\Pcycle}{P_{\text{cycle}}}
\newcommand{\FCES}{F_{\text{CES}}}
\newcommand{\phieff}{\varphi_{\text{eff}}}
\DeclareMathOperator*{\argmax}{arg\,max}
\DeclareMathOperator{\tr}{tr}
\DeclareMathOperator{\sgn}{sgn}
\DeclareMathOperator{\diag}{diag}
\DeclareMathOperator{\Var}{Var}

% Section formatting
\usepackage{titlesec}
\titleformat{\section}{\large\bfseries}{\thesection.}{0.5em}{}
\titleformat{\subsection}{\normalsize\bfseries}{\thesubsection}{0.5em}{}
\titleformat{\subsubsection}{\normalsize\itshape}{\thesubsubsection}{0.5em}{}

\begin{document}

% -------------------------------------------------------------------
% TITLE PAGE
% -------------------------------------------------------------------
\begin{titlepage}
\centering
\vspace*{1.5cm}
{\LARGE\bfseries Complementary Heterogeneity\\[8pt]
in Hierarchical Economies\par}
\vspace{0.6cm}
{\large\itshape CES Aggregation, Derived Architecture, and\\
Cross-Sector Activation in Multi-Timescale Systems\par}
\vspace{1.5cm}
{\large Jon Smirl\par}
\vspace{0.3cm}
{Independent Researcher\par}
\vspace{0.3cm}
{February 2026\par}
\vspace{0.5cm}
{\scshape Working Paper\par}
\vspace{0.8cm}
\begin{abstract}
\noindent What happens to a multi-sector economy when AI agents enter as market participants at machine speed?
This paper develops a theory of hierarchical economies with CES production at each level, operating across a variable number of hierarchical timescales (empirically 4--5 in the 2020s): hardware (decades), network formation (years), capability accumulation (months), and financial settlement (days).

Taking the CES Quadruple Role (Smirl~2026a) as given---within each sector, the curvature parameter $K = (1-\rho)(J-1)/J$ simultaneously controls diversity gains, informational robustness, and resistance to manipulation---the paper asks what happens \emph{between} sectors.
Five results follow from the geometry.

First, the \emph{architecture} of cross-sector interaction is derived, not assumed: CES geometry forces aggregate coupling (only sector-level output matters to other sectors), directed feed-forward structure (asymmetric flows are necessary for structural transitions), and nearest-neighbor topology (sectors cannot skip levels in the hierarchy).
Second, a \emph{spectral activation threshold} governs whether the hierarchy sustains non-trivial activity: individual sectors can each be too weak to sustain themselves, yet the system as a whole sustains activity through cross-sector feedbacks.
Third, a \emph{hierarchical ceiling cascade} bounds each sector's output by the sector below it in the timescale hierarchy, with the long-run growth rate set by the slowest sector.
Fourth, a \emph{welfare decomposition} attributes inefficiency to each sector through a computable distance function; the binding welfare constraint is the most institutionally rigid sector, not the most visible disequilibrium.
Fifth, a \emph{damping cancellation} theorem shows that tightening regulation at any sector has zero net welfare effect---reform must target the upstream sector.

The transition from the low-activity to high-activity equilibrium takes $O(1/\sqrt{\varepsilon_{\text{drift}}})$ time, where $\varepsilon_{\text{drift}}$ is the rate of secular improvement.
At Wright's Law semiconductor improvement rates, this yields an approximately 8-year transition window.
Three policy principles follow from theorems: reform upstream not locally, increase cross-sector responsiveness at any level, and invest in the weakest cross-sector link.
\end{abstract}

\vspace{0.5cm}
\noindent\textbf{Keywords:} CES aggregation, hierarchical economies, cross-sector activation, welfare decomposition, multi-timescale dynamics, autonomous agents, institutional rigidity

\vspace{0.3cm}
\noindent\textbf{JEL:} C62, D24, D85, E44, L14, O33, O41
\end{titlepage}

% -------------------------------------------------------------------
% 1. INTRODUCTION
% -------------------------------------------------------------------
\section{Introduction}\label{sec:intro}

Autonomous AI agents are entering capital markets.
They process information at machine speed, optimize portfolios continuously, arbitrage mispricings in milliseconds, and settle transactions through programmable stablecoin infrastructure.
No formal theory describes the dynamics of an economy where the marginal market participant is a machine.
This paper provides one.

The paper studies a hierarchical economy with four sectors operating at different timescales:
\begin{enumerate}[leftmargin=2em]
\item \textbf{Hardware} (decades): semiconductor learning curves, capital investment, the pace set by Wright's Law.
\item \textbf{Network} (years): formation of AI agent ecosystems, adoption dynamics, platform competition.
\item \textbf{Capability} (months): training efficiency, model improvement, autocatalytic skill accumulation.
\item \textbf{Settlement} (days): financial settlement, stablecoin flows, monetary policy transmission.
\end{enumerate}

Each sector aggregates heterogeneous inputs using CES production technology---the unique aggregator compatible with constant returns to scale and scale-consistent nesting \citep{smirl2026emergent}.
The CES Quadruple Role (Smirl~2026a, henceforth ``Paper~A'') guarantees that within each sector, diversity is productive (superadditivity gap $\propto K$), informationally robust (effective dimensionality bonus $\propto K^2$), and resistant to strategic manipulation (penalty $\propto K$).
These are static, within-sector properties assumed henceforth.

The paper asks: what happens \emph{between} sectors?
The answer is surprising---the architecture of cross-sector interaction is not a free modeling choice but is \emph{derived} from the CES geometry.
Five results follow.

\medskip
\noindent\textbf{Result 1: Derived architecture} (Section~\ref{sec:topology}).
Each sector communicates with others only through its aggregate output.
The coupling must be directed (non-reciprocal, feed-forward).
Long-range cross-sector links have no effect on dynamics.
The architecture is a nearest-neighbor chain.
These are consequences of the CES isoquant geometry, not assumptions of the model.

\medskip
\noindent\textbf{Result 2: Activation threshold} (Section~\ref{sec:threshold}).
The hierarchy activates when the spectral radius of a cross-sector amplification matrix exceeds~1.
Individual sectors can each be too weak to sustain themselves, yet the system as a whole can sustain activity through cross-sector feedback.
The system's activation is bottlenecked by its weakest cross-sector link.

\medskip
\noindent\textbf{Result 3: Hierarchical ceiling} (Section~\ref{sec:transition}).
Each sector's output is bounded by the sector below it in the timescale hierarchy.
The long-run growth rate equals the growth rate of the slowest sector (hardware).
The Baumol cost disease and the Triffin dilemma are the same mathematical object---a quasi-equilibrium-surface constraint---at adjacent layers in the hierarchy.

\medskip
\noindent\textbf{Result 4: Welfare decomposition} (Section~\ref{sec:welfare}).
A computable welfare-distance function attributes inefficiency to each sector.
The binding welfare constraint is the most institutionally \emph{rigid} sector, not the most \emph{visible} disequilibrium.

\medskip
\noindent\textbf{Result 5: Damping cancellation} (Section~\ref{sec:damping}).
Tightening regulation at sector~$n$ has zero net welfare effect---faster convergence exactly offsets lower equilibrium output.
To improve welfare at sector~$n$, reform sector~$n-1$ (upstream) or increase the responsiveness of sector~$n$'s cross-sector coupling.

\medskip
The transition from the low-activity to the high-activity equilibrium takes $O(1/\sqrt{\varepsilon_{\text{drift}}})$ time, where $\varepsilon_{\text{drift}}$ is the rate of secular improvement (e.g., Wright's Law cost declines).
At current semiconductor improvement rates ($\approx 15\%$ annual cost decline), this yields an approximately 8-year transition window (Section~\ref{sec:transition}).

The paper proceeds as follows.
Section~\ref{sec:setup} establishes the hierarchical economy and the CES potential.
Section~\ref{sec:topology} proves the Port Topology Theorem, the Moduli Space characterization, and the endogenous hierarchy depth result (the number of effective levels $N_{\mathrm{eff}}$ is determined by the technology's timescale structure, not assumed).
Section~\ref{sec:threshold} derives the next-generation matrix and the spectral activation threshold.
Section~\ref{sec:levels} applies the framework to the four economic levels of the current AI/crypto transition.
Section~\ref{sec:bridge} proves the Eigenstructure Bridge and welfare decomposition.
Section~\ref{sec:damping} proves the damping cancellation and derives the upstream reform principle.
Section~\ref{sec:transition} derives the hierarchical ceiling cascade and the transition duration.
Section~\ref{sec:predictions} presents eleven empirical predictions.
Section~\ref{sec:limitations} states limitations.
Section~\ref{sec:conclusion} concludes.


% -------------------------------------------------------------------
% 2. SETUP
% -------------------------------------------------------------------
\section{Setup}\label{sec:setup}

\subsection{The CES Aggregate}\label{sec:ces}

For $J \geq 2$ components at level $n$ of an $N$-level hierarchy, the \textbf{CES aggregate} with equal weights is
\begin{equation}\label{eq:CES}
F_n(\mathbf{x}_n) = \left(\frac{1}{J}\sum_{j=1}^{J} x_{nj}^{\,\rho}\right)^{1/\rho}, \qquad \mathbf{x}_n = (x_{n1}, \ldots, x_{nJ}) \in \mathbb{R}_{+}^J
\end{equation}
where $\rho < 1$, $\rho \neq 0$, is the substitution parameter and $\sigma_{\text{sub}} = 1/(1-\rho)$ is the elasticity of substitution.
The \textbf{curvature parameter} is $K = (1-\rho)(J-1)/J$.

By Paper~A, $K$ simultaneously controls three within-sector properties:
\begin{itemize}[leftmargin=2em]
\item \emph{Superadditivity:} combining diverse inputs produces more than the sum, with gap $\geq \Omega(K) \cdot \text{diversity}$.
\item \emph{Correlation robustness:} the effective dimensionality bonus from the CES nonlinearity is $\Omega(K^2) \cdot \text{idiosyncratic variation}$.
\item \emph{Strategic independence:} any coalition redistribution reduces aggregate output, with penalty $\leq -\Omega(K) \cdot \text{deviation}^2$.
\end{itemize}
These are static properties of the CES aggregate.
We do not reprove them here; see Paper~A for the complete development.

\subsection{The Hierarchical Economy}\label{sec:hierarchy}

The dynamics of the hierarchical economy are
\begin{equation}\label{eq:dynamics}
\varepsilon_n \dot{x}_{nj} = T_n(\mathbf{x}_{n-1}) \cdot \frac{\partial F_n}{\partial x_{nj}} - \sigma_n x_{nj}, \qquad n = 1, \ldots, N, \quad j = 1, \ldots, J
\end{equation}
where each term has an economic interpretation:
\begin{itemize}[leftmargin=2em]
\item $\varepsilon_n > 0$: characteristic adjustment speed of sector~$n$. Hardware adjusts over decades ($\varepsilon_1 = 1$); financial settlement adjusts over days ($\varepsilon_4 / \varepsilon_1 \approx 10^{-3}$--$10^{-4}$).
\item $T_n(\mathbf{x}_{n-1}) = \phi_n(F_{n-1}(\mathbf{x}_{n-1}))$: demand from the downstream sector---the resources flowing into sector~$n$, determined by the aggregate performance of sector~$n-1$.
\item $\partial F_n / \partial x_{nj}$: the marginal product of component~$j$ within sector~$n$---how much adding one more unit of component~$j$ increases the sector's output.
\item $\sigma_n > 0$: depreciation, friction, or institutional drag at sector~$n$---how fast gains erode without continued input.
\end{itemize}

\begin{table}[htbp]
\centering
\caption{Derived objects and notation.}\label{tab:notation}
\begin{tabular}{lll}
\toprule
\textbf{Symbol} & \textbf{Definition} & \textbf{Name} \\
\midrule
$K$ & $(1-\rho)(J-1)/J$ & Curvature parameter \\
$\mathbf{K}$ & Next-generation matrix (bold) & Cross-sector amplification matrix \\
$\rho(\mathbf{K})$ & Spectral radius of $\mathbf{K}$ & Activation threshold parameter \\
$\Phi$ & $-\sum_n \log F_n$ & Production difficulty (CES potential) \\
$V$ & $\sum_n c_n\,D_{KL}(\mathbf{x}_n \| \mathbf{x}_n^*)$ & Welfare distance (welfare loss function) \\
$W$ & $\diag(W_{11}, \ldots, W_{NN})$ & Institutional quality matrix \\
$\Pcycle$ & $\prod_n k_{n+1,n}$ & Cycle product \\
\bottomrule
\end{tabular}
\end{table}

\medskip
\noindent\textbf{Standing Assumptions.}
Throughout: (1)~$\rho < 1$, $\rho \neq 0$: inputs within each sector are not perfect substitutes;
(2)~$J \geq 2$ components per level, $N \geq 2$ levels: nontrivial diversity and nontrivial hierarchy (Section~\ref{sec:endogenousN} shows $N$ is endogenously determined by the technology's timescale structure; $N = 4$ is the current-era application);
(3)~timescale separation $\varepsilon_n / \varepsilon_{n+1} > r^*$ for each adjacent pair, where $r^* \geq 2$ is the minimum ratio for the singular perturbation reduction to hold: hardware adjusts much slower than finance (empirically uncontroversial; wavelet analysis of US industrial production 1919--2025 gives median adjacent-band ratio $\approx 2.1$, IQR $[1.84, 2.63]$). The nearest-neighbor topology of Theorem~\ref{thm:topology}(iv) is an $O(1/r^*)$ approximation; non-adjacent coupling terms are suppressed by this factor but not negligible when $r^* \approx 2$;
(4)~positive depreciation $\sigma_n > 0$: without continued investment, capability erodes;
(5)~monotone coupling $\phi_n : \mathbb{R}_+ \to \mathbb{R}_+$, $C^1$, $\phi_n(0) = 0$, $\phi_n' > 0$: more upstream output means more downstream demand; no output means no demand.

\subsection{The CES Potential}\label{sec:freeenergy}

Define the \textbf{production difficulty} (CES potential) as
\begin{equation}\label{eq:Phi}
\Phi = -\sum_{n=1}^{N} \log F_n.
\end{equation}
This is the log of the reciprocal of aggregate output at each level---a natural internal energy function of the production technology.
Its convexity (all eigenvalues of $\nabla^2\Phi$ positive at the symmetric allocation) means the economy is always locally ``trying'' to increase output.
But institutional friction ($\sigma_n$) and the directed structure (feed-forward coupling) prevent it from reaching a global optimum.

At the symmetric allocation $x_{nj} = \bar{x}$ for all~$j$, the Hessian of the CES aggregate is (Paper~A, Curvature Lemma):
\begin{equation}\label{eq:hessian}
\nabla^2 F_n = \frac{(1-\rho)}{J^2 \bar{x}}\bigl[\mathbf{1}\mathbf{1}^T - J\,I\bigr]
\end{equation}
with eigenvalue~$0$ on~$\mathbf{1}$ (by Euler's theorem for degree-1 homogeneous functions) and eigenvalue $-(1-\rho)/(J\bar{x})$ on~$\mathbf{1}^{\perp}$ (the ``diversity'' directions).

The within-level Jacobian at equilibrium is
\begin{equation}\label{eq:jacobian}
Df_n = \frac{\sigma_n}{\varepsilon_n}\left[\frac{(1-\rho)}{J}\,\mathbf{1}\mathbf{1}^T - (2-\rho)\,I\right]
\end{equation}
with eigenvalue $-\sigma_n/\varepsilon_n$ on~$\mathbf{1}$ (aggregate mode) and $-\sigma_n(2-\rho)/\varepsilon_n$ on~$\mathbf{1}^{\perp}$ (diversity modes).
The diversity modes decay $(2-\rho)$ times faster than the aggregate mode.

\medskip
\noindent\textbf{Economic interpretation.}
The $(2-\rho)$ factor quantifies a ``macro is blind to micro'' result: compositional changes within a sector---which agent is doing what---decay $(2-\rho)$ times faster than changes in the sector's aggregate output.
By the time a downstream sector responds, the upstream sector's composition has already equilibrated.
Only the aggregate matters.


% -------------------------------------------------------------------
% 3. THE PORT TOPOLOGY THEOREM
% -------------------------------------------------------------------
\section{The Port Topology Theorem}\label{sec:topology}

This section contains the most surprising result: the network architecture of the hierarchical economy is \emph{derived} from the CES geometry, not assumed.

\begin{theorem}[CES-Forced Topology]\label{thm:topology}
Under the standing assumptions of Section~\ref{sec:hierarchy}, the hierarchical CES economy has the following topological properties:
\begin{enumerate}[label=(\roman*)]
\item \emph{(Aggregate coupling)} Each sector communicates with other sectors only through its aggregate output $F_n$. Individual component states within a sector are invisible to other sectors.
\item \emph{(Directed coupling)} The between-sector coupling is necessarily feed-forward (non-reciprocal). Any bidirectional coupling yields an unconditionally stable system incapable of structural transition.
\item \emph{(Port alignment)} The coupling enters through the gradient of the CES aggregate (proportional to $\mathbf{1}$ at the symmetric allocation). The coupling \emph{functions} $\phi_n$---how much downstream demand responds to upstream output---are free parameters not determined by $\rho$.
\item \emph{(Nearest-neighbor topology)} Under timescale separation, long-range coupling (e.g., from hardware directly to settlement, skipping network and capability) has no dynamical effect. The effective topology is a nearest-neighbor chain.
\end{enumerate}
\end{theorem}

\noindent\textit{Proof sketches follow for each claim. Full proofs are in Appendix~\ref{app:topology}.}

\medskip
\noindent\textbf{Claim (i): Aggregate coupling.}
Three steps.
First, at any equilibrium, the CES first-order conditions force all components to equalize: $x_{nj} = \bar{x}_n$ for all~$j$ (equilibrium uniqueness from $\rho < 1$).
Second, the diversity modes decay $(2-\rho)$ times faster than the aggregate mode, creating a spectral gap (Equation~\ref{eq:jacobian}).
Third, this spectral gap ensures that the fast-decaying diversity modes create an invariant manifold parameterized by $F_n$ alone, which persists under perturbation.

\medskip
\noindent\textbf{Economic interpretation.}
The aggregate output of a sector is a sufficient statistic for cross-sector planning.
A policymaker or downstream sector does not need to know the composition of upstream output---only its level.
This is a consequence of CES complementarity, not an assumption of the model.
The $(2-\rho)$ factor provides the mechanism: compositional changes within a sector are invisible to other sectors because they decay $(2-\rho)$ times faster than aggregate changes.
By the time the downstream sector responds, the composition has already equilibrated.

\medskip
\noindent\textbf{Claim (ii): Directed coupling.}
Consider a two-sector system on the quasi-equilibrium surface (scalar dynamics per sector).
With bidirectional coupling, the Jacobian is
$\mathcal{J}_{\text{bidir}} = \begin{psmallmatrix} -\sigma_1 & -c/J \\ c/J & -\sigma_2 \end{psmallmatrix}$
with both eigenvalues having strictly negative real parts for all $c, \sigma_1, \sigma_2 > 0$.
Any passive bidirectional coupling satisfies $\dot{V}_{\text{coupling}} \leq 0$, which only strengthens stability.
The structural transition at $\rho(\mathbf{K}) = 1$ requires net energy injection through the hierarchy, which requires directed coupling.

\medskip
\noindent\textbf{Economic interpretation.}
Feedback between sectors must be asymmetric---semiconductor improvements enable AI training, but AI training does not directly improve semiconductor fabrication (at least not on the same timescale).
If the coupling were symmetric, the economy would be unconditionally stable---which means it could never undergo a structural transition.
The possibility of structural change (the ``crossing point'' where the AI economy activates) requires directed, asymmetric flows.

\medskip
\noindent\textbf{Claim (iii): Port alignment.}
At a symmetric equilibrium, the equilibrium condition requires the coupling direction $\mathbf{b}_n \propto \mathbf{1}$.
Since $\nabla F_n = (1/J)\,\mathbf{1}$ at the symmetric point (Paper~A, Proposition~2.1), the natural CES-compatible coupling direction is the gradient.
The gain functions $\phi_n$ (how much downstream demand responds to upstream output) are free parameters: for power-law gains $\phi_n(z) = a_n z^{\beta_n}$, the exponents $\beta_n$ are not determined by $\rho$.

\medskip
\noindent\textbf{Economic interpretation.}
The geometry determines \emph{which direction} cross-sector coupling takes (through the aggregate, not through individual components).
The \emph{strength} of coupling---the gain functions $\phi_n$---is where the discipline-specific economics lives.
Semiconductor economics determines $\phi_1$ (learning curves).
Platform economics determines $\phi_2$ (network effects).
AI research determines $\phi_3$ (training efficiency).
Monetary economics determines $\phi_4$ (settlement demand).

\medskip
\noindent\textbf{Claim (iv): Nearest-neighbor topology.}
Consider a three-level system with long-range coupling from level~1 to level~3.
Level~1, being fastest ($\varepsilon_1 \ll \varepsilon_2$), equilibrates to $F_1^* = \beta_1/(\sigma_1 J)$, a constant on the quasi-equilibrium surface.
The long-range coupling becomes a constant, absorbed into a modified input to level~3.
The reduced Jacobian is lower-triangular---independent of the long-range coupling strength.
Induction extends to $N$ levels.

\medskip
\noindent\textbf{Economic interpretation.}
You cannot skip levels in the hierarchy.
A semiconductor subsidy does not directly improve financial settlement---it first improves hardware, which enables more AI agents, which develop capabilities, which create settlement demand.
Each link in the chain must be traversed.
Interventions that try to ``jump'' levels (e.g., mandating stablecoin adoption without the underlying AI ecosystem) have the same equilibrium effect as doing nothing.


\subsection{The Moduli Space Theorem}\label{sec:moduli}

\begin{theorem}[Structural Determination]\label{thm:moduli}
Fix $\rho < 1$ and the structural integers $(J, N)$.
Then:

\emph{Qualitative invariants} (determined by $\rho$):
within-sector eigenstructure, coupling topology (aggregate, directed, nearest-neighbor), existence of an activation threshold, the Quadruple Role (from Paper~A), and the Eigenstructure Bridge (Section~\ref{sec:bridge}).

\emph{Free parameters} (quantitative degrees of freedom):
timescales $(\varepsilon_1, \ldots, \varepsilon_N)$, damping rates $(\sigma_1, \ldots, \sigma_N)$, and gain functions $(\phi_1, \ldots, \phi_N)$.
\end{theorem}

\noindent\textbf{Economic interpretation.}
The CES geometry determines the \emph{architecture} of the economy---which variables couple to which, through what channels, in what direction.
What the CES geometry leaves free is where the economics lives: how fast each sector adjusts, how much friction each faces, and how responsive cross-sector coupling is.
This is a model selection result: the space of possible models collapses from an arbitrary $N$-sector interaction graph to a specific nearest-neighbor chain with scalar coupling.
The analogy to electrical circuits is apt: Kirchhoff's laws constrain the circuit topology; component values (resistors, capacitors) are free.

The gain functions are not a gap in the theory---they \emph{are} the economic content.
They encode: $\phi_1$ = learning curve shape (semiconductor economics), $\phi_2$ = network recruitment (platform economics), $\phi_3$ = training efficiency (AI capability research), $\phi_4$ = settlement demand (monetary economics).
The CES geometry provides the architecture; the gain functions are where the discipline-specific economics lives.

\subsection{Endogenous Hierarchy Depth}\label{sec:endogenousN}

Theorem~\ref{thm:moduli} treats $N$ as a structural integer.
We now show it is determined endogenously by the technology's timescale structure.

\begin{definition}[Spectral gap sequence]\label{def:spectralgap}
Given $M$ economic processes with characteristic adjustment timescales $\tau_1 \leq \tau_2 \leq \cdots \leq \tau_M$, the \textbf{spectral gap sequence} is
\[
g_i = \log(\tau_{i+1}/\tau_i), \qquad i = 1, \ldots, M-1.
\]
The \textbf{effective hierarchy depth} is
\begin{equation}\label{eq:Neff}
N_{\mathrm{eff}} = 1 + \#\{i : g_i > \log r^*\}
\end{equation}
where $r^* \geq 1$ is the minimum timescale ratio for the singular perturbation reduction to hold (the quasi-equilibrium surface exists and the $MJ \to N_{\mathrm{eff}}J$ dimensional reduction is valid).

\medskip
\noindent\emph{Empirical calibration.}
Continuous wavelet transform (Morlet) analysis of US industrial production (FRED INDPRO, 1919--2025) identifies spectral peaks at decade resolution with median adjacent-peak separation ratio $2.1$ (IQR~$[1.84, 2.63]$, maximum $5.5$).
Setting $r^* = 2$ (one wavelet octave) yields $N_{\mathrm{eff}} = 4$--$5$ across all decades, with mean $4.5$ and standard deviation $1.0$.
This calibration implies that the singular perturbation reduction underlying the nearest-neighbor topology is an $O(1/r^*) \approx O(0.5)$ approximation: the quasi-equilibrium surface exists but corrections from non-adjacent coupling are of order $50\%$, not negligible.
The hierarchy is better understood as a \emph{wavelet multiresolution}---a continuous spectrum organized into approximately octave-spaced bands---rather than a cleanly separated discrete ladder.
\end{definition}

Processes whose timescales are separated by less than $r^*$ collapse into a single effective level: their dynamics are entangled and cannot be reduced via the quasi-equilibrium surface.
Processes separated by more than $r^*$ define distinct levels with the nearest-neighbor topology of Theorem~\ref{thm:topology}(iv).

\begin{theorem}[Endogenous hierarchy depth]\label{thm:endogenousN}
Under the standing assumptions with $N$ replaced by $N_{\mathrm{eff}}$ from~\eqref{eq:Neff}:
\begin{enumerate}[label=\emph{(\roman*)}]
\item \emph{(Structural determination.)} All qualitative results of this paper---port topology, activation threshold, eigenstructure bridge, damping cancellation, upstream reform principle---hold for arbitrary $N_{\mathrm{eff}} \geq 2$.
The four-level specification $N=4$ is a quantitative application to the current AI/crypto transition, not a structural assumption.

\item \emph{(Lifecycle dynamics.)} $N_{\mathrm{eff}}$ follows a technology lifecycle.
During innovation phases, novel processes create new timescale gaps, increasing $N_{\mathrm{eff}}$.
During maturation, adjacent timescales converge as processes standardize, closing gaps and decreasing $N_{\mathrm{eff}}$.

\item \emph{(Threshold hardening.)} The spectral radius $\rho(\mathbf{K})$ of the next-generation matrix is weakly decreasing in $N_{\mathrm{eff}}$ for fixed gain elasticities.
More hierarchy levels means more places the activation cascade can break: the system requires stronger complementarity (lower $\rho$) or larger gain elasticities at each level to sustain $\rho(\mathbf{K}) > 1$.

\item \emph{(Level crystallization.)} A new level ``crystallizes'' when a timescale gap $g_i$ crosses $\log r^*$ from below.
This is a structural regime shift: the conservative-dissipative system gains a node, the next-generation matrix gains a row and column, and the activation threshold changes discontinuously.

\item \emph{(Sequential waves.)} A single sector can undergo multiple lifecycle episodes.
If a new technological paradigm opens timescale gaps that the previous maturation closed, $N_{\mathrm{eff}}$ rises again.
The signature is observable only when the new paradigm changes the \emph{production process}, not merely the discovery frontier: genomics revolutionized pharmaceutical R\&D without altering drug manufacturing (no change in production-measured $N_{\mathrm{eff}}$), while electrification of motor vehicles replaces the entire powertrain supply chain (large change in $N_{\mathrm{eff}}$).

\item \emph{(Approximation quality.)} The nearest-neighbor coupling topology (Theorem~\ref{thm:topology}(iv)) is exact in the limit $r^* \to \infty$ and an $O(1/r^*)$ approximation for finite $r^*$.  At the empirically calibrated $r^* \approx 2$, non-adjacent coupling terms are suppressed by a factor of $\sim 1/2$ relative to adjacent terms.  All qualitative results (activation threshold, ceiling cascade, damping cancellation) are robust to this correction; quantitative predictions (transition duration, spectral radius) carry $O(1/r^*)$ uncertainty.
\end{enumerate}
\end{theorem}

\begin{proof}[Proof sketch]
(i) Inspection of the proofs of Theorems~\ref{thm:topology}--\ref{thm:lyapunov} and Propositions~\ref{prop:damping}--\ref{thm:upstream}: $N$ enters only as the dimension of the quasi-equilibrium-surface system.
All arguments go through for $N_{\mathrm{eff}} \geq 2$.

(ii) Innovation creates new processes (e.g., model training separating from hardware manufacturing circa 2017), opening spectral gaps.
Maturation routinizes processes onto common timescales (e.g., steel production in 1880 had distinct blast-furnace, rolling, and finishing timescales; by 1970 these operate as a single continuous flow).

(iii) The characteristic polynomial of the $N \times N$ next-generation matrix has degree~$N$.
For the lower-triangular structure of Theorem~\ref{thm:topology}(ii), $\rho(\mathbf{K}) = \max_n k_{n+1,n}/\sigma_n$.
Adding a level $N+1$ contributes a factor $(z - k_{N+2,N+1}/\sigma_{N+1})$ to the characteristic polynomial.
Unless $k_{N+2,N+1}/\sigma_{N+1} \geq \rho(\mathbf{K})$, the spectral radius does not increase.

(iv) At $g_i = \log r^*$, the quasi-equilibrium surface for the pair $(\tau_i, \tau_{i+1})$ bifurcates: below, the two processes share a manifold; above, they separate.
The dimension of the reduced system jumps from $N_{\mathrm{eff}}$ to $N_{\mathrm{eff}}+1$.

(v) A new paradigm (e.g., electric powertrains replacing internal combustion) introduces processes on timescales separated from the matured ones by more than $r^*$, re-opening spectral gaps via the same mechanism as~(ii).
Whether this is observable in production data depends on whether the new paradigm changes the physical production process: if the new timescale gaps appear only in R\&D workflows (e.g., genomics in drug discovery), they are not visible in the production time series that defines $N_{\mathrm{eff}}$.
\end{proof}

\begin{remark}[Maturity, sequential waves, and the cross-section]\label{rem:maturity}
The lifecycle dynamics of clauses~(ii) and~(v) explain several empirical patterns.

\medskip
\noindent\emph{Cross-sectional R\&D is a poor predictor.}
A high-R\&D mature industry (pharmaceuticals, R\&D/sales $\approx 10\%$) may have \emph{fewer} effective levels than a lower-R\&D industry in its innovation phase.
The correct predictor is the \emph{rate of change in the production process}---lifecycle position---not the \emph{level} of technical sophistication.

\medskip
\noindent\emph{Maturation.}
Steel manufacturing circa 1890 likely had $N_{\mathrm{eff}} = 3$--$4$ (ore processing, blast furnace chemistry, rolling/finishing, distribution---each on distinct timescales).
By 1970, continuous casting collapsed these into $N_{\mathrm{eff}} = 2$.
Historical FRED data confirms this: Primary Metals production (1972--2025) shows \emph{declining} effective dimensionality (Kendall $\tau = -0.28$), consistent with late-stage maturation after the Bessemer/continuous-casting consolidation.
Computer/Electronics in 2025, however, shows a non-monotone trajectory: rolling-window $N_{\mathrm{eff}}$ peaks at $4$ around 1987 and 2007, consistent with mid-innovation bursts, but returns to $N_{\mathrm{eff}} = 2$ in recent windows.
The cross-sectional contrast is clearest in spectral correlation: mature sectors (Primary Metals, Food, Transport, Chemicals) correlate at $r > 0.91$, while Computer/Electronics correlates at only $r = 0.30$--$0.50$ with this cluster---confirming that the spectral fingerprint of a sector depends on its lifecycle position.

\medskip
\noindent\emph{Sequential waves.}
Motor vehicles (assembly line $\sim$1913; electric powertrain $\sim$2012) and basic chemicals (synthetic chemistry $\sim$1920; specialty/green chemistry $\sim$2000) show rising $N_{\mathrm{eff}}$ in 1972--2025 (both $\tau \approx 0.31$, $p < 0.005$), indicating active second waves.
Pharmaceuticals, despite genomics and mRNA, shows no production-dimensionality trend ($\tau = -0.05$, $p = 0.67$): the revolution is in the R\&D pipeline, not the manufacturing process.
This confirms clause~(v): only paradigm shifts that restructure the \emph{production} timescale distribution are visible in $N_{\mathrm{eff}}$.

\medskip
\noindent\emph{General-purpose technology synchronization.}
At known GPT adoption dates (postwar electrification, internet, deep learning), aggregate industrial production dimensionality \emph{decreases} (all shifts $-4$ to $-6$ modes, $p < 0.001$).
GPTs create a dominant cross-sector mode that synchronizes previously independent industries, reducing aggregate $N_{\mathrm{eff}}$ while potentially increasing within-sector complexity.
\end{remark}

\noindent\textbf{Updated Moduli Space characterization.}
Theorem~\ref{thm:moduli} is now sharper: the ``structural integers'' $(J, N)$ include $N = N_{\mathrm{eff}}$, which is itself determined by the technology's timescale distribution.
The free parameters remain: timescales $(\varepsilon_1, \ldots, \varepsilon_{N_{\mathrm{eff}}})$ (subject to the gap constraint $\varepsilon_n / \varepsilon_{n+1} > r^*$), damping rates, and gain functions.
The number of free parameters is now endogenous.


\subsection{The Secular Equation and the Value of Diversity}\label{sec:secular}

The equal-weight case ($a_j = 1/J$) used throughout the paper is analytically clean but empirically restrictive.
Paper~A (Section~8) develops the general-weight theory in full.
The key objects are summarized here because they control the quantitative strength of every result in this paper.

\begin{proposition}[Secular equation, Paper~A]\label{prop:secular_summary}
For a CES aggregate with weights $a_1, \ldots, a_J > 0$ (summing to~1), define the \textbf{effective shares} $p_j = a_j^{1/(1-\rho)}$, the \textbf{inverse effective shares} $w_j = 1/p_j$, and order them $w_{(1)} \leq \cdots \leq w_{(J)}$.
The $J-1$ principal curvatures of the CES isoquant at the cost-minimizing point are determined by the roots $\mu_1 < \cdots < \mu_{J-1}$ of the \textbf{secular equation}
\begin{equation}\label{eq:secular}
\sum_{j=1}^{J} \frac{1}{w_j - \mu} = 0.
\end{equation}
The roots strictly interlace the poles:
\[
w_{(1)} < \mu_1 < w_{(2)} < \mu_2 < \cdots < w_{(J-1)} < \mu_{J-1} < w_{(J)}.
\]
\end{proposition}

The interlacing property is the mathematical engine.
Because exactly one root sits in each interval $(w_{(k)}, w_{(k+1)})$, adding or removing a component shifts the roots in a predictable, monotone way---this gives the \emph{exact marginal value of diversity}.
The smallest root $R_{\min} = \mu_1$ controls the generalized curvature parameter:
\begin{equation}\label{eq:genK}
K(\rho, \mathbf{a}) = (1-\rho)\,\frac{J-1}{J}\,\Phi^{1/\rho}\,R_{\min}
\end{equation}
where $\Phi = \sum_j p_j$.
At equal weights, $R_{\min} = J^{\sigma}$, $\Phi^{1/\rho} = J^{-\sigma}$, and $K$ reduces to $(1-\rho)(J-1)/J$.

\noindent\textbf{Why this matters for Paper~B.}
Every quantitative result in this paper---the superadditivity gap (Section~\ref{sec:ces}), the spectral threshold (Section~\ref{sec:threshold}), the welfare weights (Section~\ref{sec:bridge}), the manipulation penalty---depends on $K$.
Under general weights, $K$ is replaced by $K(\rho, \mathbf{a})$ from Equation~\eqref{eq:genK}, and $R_{\min}$ enters as a known function of the weight vector through the secular equation.

For a coalition $S \subset \{1, \ldots, J\}$ with $|S| = k$, the \textbf{coalition curvature} $K_S = (1-\rho)(k-1)\Phi_S^{1/\rho}R_{\min,S}/k$ uses the smallest root of the secular equation restricted to $S$.
The interlacing property guarantees $K_S > 0$ for all coalitions of size $k \geq 2$, all $\rho < 1$, all weight vectors---there is no configuration of weights that eliminates the manipulation penalty.

\noindent\textbf{Applicability.}
The secular equation applies to \emph{any} CES aggregate in economics with heterogeneous shares:
Dixit-Stiglitz monopolistic competition (firm-level productivity weights), Armington trade aggregation (country-level trade shares), capital-labor-energy production functions (factor weights), CES demand systems (expenditure shares).
In each case, compute $w_j = a_j^{-\sigma}$, find the smallest root $\mu_1$ of $\sum 1/(w_j - \mu) = 0$ numerically (an $O(J)$ computation), and evaluate $K(\rho, \mathbf{a})$.
This single number then enters all four parts of the Quadruple Role: superadditivity ($\propto K$), correlation robustness ($\propto K^2$), strategic independence ($\propto K$), and network scaling ($\propto J^{1/\rho}$).


% -------------------------------------------------------------------
% 4. THE REDUCED SYSTEM AND ACTIVATION THRESHOLD
% -------------------------------------------------------------------
\section{The Reduced System and Activation Threshold}\label{sec:threshold}

\subsection{Reduction to $N$ Dimensions}\label{sec:reduction}

On the quasi-equilibrium surface---where the fast sector equilibrates before the slow sector moves appreciably (Theorem~\ref{thm:topology}(i))---the $NJ$-dimensional system reduces to $N$ scalar equations.

\begin{proposition}[Reduced dynamics]\label{prop:reduced}
On the quasi-equilibrium surface, the aggregate dynamics are
\begin{equation}\label{eq:reduced}
\varepsilon_n \dot{F}_n = \phi_n(F_{n-1})/J - \sigma_n F_n, \qquad n = 1, \ldots, N.
\end{equation}
\end{proposition}

\begin{proof}
By Theorem~\ref{thm:topology}(i), the quasi-equilibrium surface at each level is parameterized by $F_n$.
Project the dynamics onto the aggregate: $\dot{F}_n = \nabla F_n \cdot \dot{\mathbf{x}}_n = (1/J)\mathbf{1} \cdot \dot{\mathbf{x}}_n$.
Summing~\eqref{eq:dynamics} over $j$: $\varepsilon_n J \dot{F}_n = T_n - \sigma_n J F_n$, using $\sum_j \partial F_n/\partial x_{nj} = 1$ (Euler's theorem for degree-1 homogeneous functions) and $\sum_j x_{nj} = J F_n$ (symmetric allocation).
Thus $\varepsilon_n \dot{F}_n = T_n/J - \sigma_n F_n = \phi_n(F_{n-1})/J - \sigma_n F_n$.
\end{proof}

\noindent\textbf{Economic interpretation.}
Each sector's aggregate output evolves as a balance between incoming demand from the upstream sector ($\phi_n(F_{n-1})/J$) and depreciation ($\sigma_n F_n$).
The $NJ$-dimensional system of individual components collapses to $N$ scalar equations in the sector aggregates.
This reduction is not an approximation---it is exact on the quasi-equilibrium surface, with $O(\varepsilon)$ corrections from the fast dynamics.

\subsection{The Next-Generation Matrix}\label{sec:NGM}

At the nontrivial equilibrium, decompose the Jacobian of the reduced system as $\mathcal{J}_{\text{agg}} = T + \Sigma$, where $\Sigma = \diag(-\sigma_1/\varepsilon_1, \ldots, -\sigma_N/\varepsilon_N)$ encodes depreciation and $T$ encodes cross-sector amplification.
The \textbf{next-generation matrix} is
\begin{equation}\label{eq:NGM}
\mathbf{K} = -T\Sigma^{-1}.
\end{equation}

Entry $K_{n,n-1} = k_{n,n-1} = \phi_n'(\bar{F}_{n-1})\bar{F}_{n-1}/|\sigma_{n-1}|$ measures: if sector~$n-1$ generates one additional unit of output, how many ``next-generation'' units of sector~$n$ activity does this produce, accounting for depreciation?
This is the cross-sector multiplier.

\subsection{Characteristic Polynomial}\label{sec:charpoly}

\begin{theorem}[NGM characteristic polynomial]\label{thm:charpoly}
For a cyclic $N$-level system with diagonal entries $d_1, \ldots, d_N$ and nearest-neighbor coupling $k_{n+1,n}$, the characteristic polynomial of the next-generation matrix is
\begin{equation}\label{eq:charpoly}
p(\lambda) = \prod_{i=1}^{N}(d_i - \lambda) - \Pcycle
\end{equation}
where $\Pcycle = \prod_n k_{n+1,n}$ is the cycle product.
\end{theorem}

\begin{proof}
In the Leibniz expansion $\det(\mathbf{K} - \lambda I) = \sum_{\pi \in S_N} \sgn(\pi) \prod_i (\mathbf{K} - \lambda I)_{i,\pi(i)}$, a permutation $\pi$ contributes a nonzero product only if every factor is nonzero.
The matrix $\mathbf{K} - \lambda I$ has diagonal entries $(d_i - \lambda)$, subdiagonal entries $k_{n+1,n}$, corner entry $k_{1N}$, and all others zero.
Only two permutations have all nonzero factors:
(a)~the identity, contributing $\prod_i (d_i - \lambda)$;
(b)~the full $N$-cycle, contributing $(-1)^{N-1} \Pcycle$.
The sign gives $p(\lambda) = \prod(d_i - \lambda) - \Pcycle$.
\end{proof}

\noindent\textbf{Economic interpretation.}
$\Pcycle$ is the geometric mean of all cross-sector amplification rates around the loop.
The sensitivity $\partial\rho(\mathbf{K})/\partial k_{ij} \propto \Pcycle/k_{ij}$ is largest for the smallest $k_{ij}$.
The system's activation is bottlenecked by its weakest cross-sector link.
\emph{Investment thesis: invest in the weakest link, not the strongest.}

\subsection{The Spectral Threshold}\label{sec:spectral}

\begin{theorem}[Activation threshold]\label{thm:activation}
The nontrivial equilibrium (positive activity at all levels) exists and is stable if and only if $\rho(\mathbf{K}) > 1$.
The transition at $\rho(\mathbf{K}) = 1$ is a structural transition: below threshold, only the trivial equilibrium (no activity) is stable; above threshold, a non-trivial equilibrium (positive activity at all levels) becomes stable.

The system can be globally activated ($\rho(\mathbf{K}) > 1$) from individually sub-threshold sectors ($d_n < 1$ for all $n$) when $\Pcycle^{1/N} > 1 - \max_i d_i$: the cross-sector amplification compensates for sub-threshold individual sectors.
\end{theorem}

\noindent\textbf{Economic interpretation.}
No single sector---not AI hardware, not the agent ecosystem, not training capability, not financial settlement---is individually strong enough to sustain itself.
But the cross-sector feedbacks (cheaper hardware $\to$ more agents $\to$ better training $\to$ more settlement demand $\to$ more investment $\to$ cheaper hardware) create a self-sustaining cycle.
The threshold $\rho(\mathbf{K}) = 1$ is the ``crossing point'' where this cycle becomes self-sustaining.

Each sector alone is too weak to sustain itself, yet the cross-sector feedbacks are strong enough that the system as a whole sustains activity.
This is the fundamental mechanism: the hierarchical economy is activated by its interconnections, not by any single sector's strength.


% -------------------------------------------------------------------
% 5. THE FOUR ECONOMIC LEVELS
% -------------------------------------------------------------------
\section{The Four Economic Levels}\label{sec:levels}

This section applies the general $N_{\mathrm{eff}}$-level framework (Theorem~\ref{thm:endogenousN}) to the four levels of the current AI/crypto transition ($N_{\mathrm{eff}} = 4$).
For each level, we present the economic content, the gain function $\phi_n$, the relevant CES property from the Quadruple Role (Paper~A), and the ceiling from the quasi-equilibrium surface cascade.
The four-level structure reflects the 2020s-era timescale distribution; historical and future transitions may have different $N_{\mathrm{eff}}$ (Remark~\ref{rem:maturity}).

\begin{table}[htbp]
\centering
\caption{Timescale hierarchy.}\label{tab:timescales}
\begin{tabular}{lllll}
\toprule
\textbf{Level} & \textbf{Domain} & \textbf{Timescale} & \textbf{Ordering} & \textbf{Approx.\ period} \\
\midrule
$n=1$ (slowest) & Hardware (learning curves) & $\varepsilon_1 = 1$ & Reference & $\sim$15--30\,yr \\
$n=2$ & Network (mesh formation) & $\varepsilon_2$ & $\varepsilon_1/\varepsilon_2 > r^* \approx 2$ & $\sim$5--12\,yr \\
$n=3$ & Capability (training) & $\varepsilon_3$ & $\varepsilon_2/\varepsilon_3 > r^*$ & $\sim$1--5\,yr \\
$n=4$ (fastest) & Settlement (finance) & $\varepsilon_4$ & $\varepsilon_3/\varepsilon_4 > r^*$ & $< 1$\,yr \\
\bottomrule
\end{tabular}
\end{table}

\noindent Wavelet energy band analysis shows that the business-cycle (3--8\,yr) and technology-wave (8--20\,yr) bands are active in $\geq 82\%$ of decades since 1919, forming a persistent two-level core.  The sub-annual (settlement) and multi-decade (Kondratieff/hardware) bands are intermittent, activating during innovation phases and structural transitions---consistent with the $R_0$ activation threshold: a level sustains activity only when $\rho(\mathbf{K}) > 1$.

\subsection{Hardware (Level 1, Slowest---Decades)}\label{sec:hardware}

Wright's Law learning curves provide the gain function: as cumulative production doubles, unit cost falls by a constant fraction.
The gain function $\phi_1$ is a power law with exponent determined by the learning rate (empirically $\alpha \approx 0.23$ for semiconductors).
This is the pace car---it sets the long-run growth rate of the entire economy.

At this level, the CES aggregate captures the complementarity between different semiconductor technologies: DRAM, HBM, logic chips, and specialized accelerators.
No single chip type substitutes perfectly for the others ($\rho < 1$), so diversity in hardware capability is productive.

\textbf{Ceiling.}
Hardware is the slowest level.
Its growth rate is determined exogenously by the Wright's Law drift rate $\varepsilon_{\text{drift}}$ and by the feedback from the settlement layer (in the cyclic specification).
All other levels are ultimately bounded by this pace.

\subsection{Network (Level 2---Years)}\label{sec:network}

After the crossing point ($\rho(\mathbf{K}) > 1$), heterogeneous AI agents with diverse capabilities self-organize into a mesh network.
The adoption dynamics are logistic:
\begin{equation}\label{eq:adoption}
\dot{F}_2 = \beta(F_1) \cdot F_2 \cdot (1 - F_2/N^*(F_1)) - \mu F_2
\end{equation}
where $\beta(F_1)$ is the adoption rate (increasing in hardware capability) and $N^*(F_1)$ is the carrying capacity (also increasing in $F_1$).

The CES superadditivity (Paper~A, Theorem~3.1) quantifies the diversity premium: combining agents with different capability profiles produces more aggregate capability than the sum of their individual contributions.
The premium is proportional to $K$ and to the squared geodesic distance between agents' capability profiles on the unit isoquant.

By Theorem~\ref{thm:topology}(i), $F_n$ is a sufficient statistic for the level's state.
Individual agent capabilities are invisible to other levels.

\textbf{Ceiling.}
Network size is bounded by hardware: $F_2 \leq N^*(F_1)$.
More hardware capability enables a larger carrying capacity for the agent network.

\subsection{Capability (Level 3---Months)}\label{sec:capability}

When capability becomes a dynamical variable, three mechanisms make growth endogenous: training agents improve other agents (autocatalytic capability growth), operation generates training data (self-referential learning), and the mesh modifies its own composition (endogenous variety expansion).
The effective production multiplier including autocatalytic feedback is $\phieff = \phi_0/(1 - \beta_{\text{auto}}\cdot\phi_0)$.

Three regimes emerge with sharp boundaries:
\begin{enumerate}[leftmargin=2em]
\item Convergence to a ceiling $C_{\max}$ when $\phieff < 1$ and variety is bounded (the most likely near-term regime).
\item Exponential growth when autocatalytic coupling pushes $\phieff$ to unity and variety expands endogenously.
\item Finite-time singularity when $\phieff > 1$ with no saturation (conditions unlikely to hold simultaneously).
\end{enumerate}

The CES correlation robustness (Paper~A, Theorem~3.2) provides collapse protection: the diversity of the training data prevents model collapse \cite{shumailov2024}.
The curvature parameter $K$ controls both the diversity premium for capability aggregation and the diversity protection against model collapse.

\textbf{The Baumol bottleneck.}
As the mesh automates progressively more inference tasks, the remaining non-automated task---frontier model training---becomes the binding constraint.
Mesh growth converges to the frontier training rate.
This is the first instance of the hierarchical ceiling: capability is bounded by the network, which is bounded by hardware.

\subsection{Settlement (Level 4, Fastest---Days)}\label{sec:settlement}

The mesh requires a programmable settlement layer for routing compensation.
Dollar stablecoins, backed by US Treasuries, provide a cost advantage over fiat payment rails.
As mesh operations scale, settlement demand grows faster than inference demand.

The CES strategic independence (Paper~A, Theorem~3.3) makes manipulation unprofitable: diversity modes decay $(2-\rho)$ times faster than aggregate modes, suppressing manipulation signals faster than legitimate price signals.

\textbf{Monetary policy degradation.}
As the fraction~$\phi$ of capital managed by autonomous agents increases, monetary policy tools degrade in sequence.
Forward guidance degrades first (it depends on information processing delay, which mesh agents eliminate).
Quantitative easing degrades second (it depends on arbitrage speed, which mesh agents improve).
Financial repression degrades last but most sharply (it depends on captive savings, which collapse discontinuously when stablecoin access crosses a critical threshold).
The surviving channels---interest rate and lender-of-last-resort---operate through real economy dynamics rather than market frictions.

\textbf{The Triffin squeeze.}
Stablecoin demand pushes Treasury supply~$b$ upward while mesh participation makes the safety boundary~$\bar{b}(\phi)$ lower.
The squeeze is self-reinforcing when $\dot{b} > 0$ and $\dot{\bar{b}} < 0$ simultaneously.
This is the same mathematical object as the Baumol bottleneck at Level~3---a quasi-equilibrium surface constraint at an adjacent layer.

\textbf{Ceiling.}
Settlement is bounded by capability: $F_4 \leq \bar{S}(F_3)$.
Settlement infrastructure cannot grow faster than the capability layer that generates demand for it.


% -------------------------------------------------------------------
% 6. THE EIGENSTRUCTURE BRIDGE AND WELFARE DECOMPOSITION
% -------------------------------------------------------------------
\section{The Eigenstructure Bridge and Welfare Decomposition}\label{sec:bridge}

\subsection{The Non-Gradient Obstruction}\label{sec:nongradient}

\begin{proposition}[Non-gradient structure]\label{prop:nongradient}
The hierarchical CES economy does \emph{not} follow potential-based adjustment dynamics.
The lower-triangular Jacobian (from directed coupling) is a topological obstruction---no coordinate transformation can symmetrize it.
\end{proposition}

\noindent\textbf{Economic interpretation.}
There is no social planner's problem whose first-order conditions generate these dynamics.
The economy's directed, hierarchical structure is fundamentally incompatible with welfare optimization.
Standard welfare theorems do not apply.
This is not a market failure---it is a structural feature of hierarchical economies with directed cross-sector flows.
The absence of a potential function is \emph{the reason} the Bridge equation and the damping cancellation are nontrivial results.

\subsection{The Storage Function}\label{sec:storage}

\begin{theorem}[Welfare distance function]\label{thm:lyapunov}
Define
\begin{equation}\label{eq:lyapunov}
V(\mathbf{x}) = \sum_{n=1}^{N} c_n \sum_{j=1}^{J} \left(\frac{x_{nj}}{x_{nj}^*} - 1 - \log\frac{x_{nj}}{x_{nj}^*}\right) = \sum_{n=1}^{N} c_n\, D_{KL}(\mathbf{x}_n \| \mathbf{x}_n^*)
\end{equation}
with tree coefficients $c_n = \Pcycle / k_{n,n-1}$.
Then $V$ is a welfare distance function for the nontrivial equilibrium: $V \geq 0$ with $V = 0$ if and only if $\mathbf{x} = \mathbf{x}^*$, and $V$ always decreases along the economy's trajectory ($\dot{V} \leq 0$).
\end{theorem}

\begin{proof}[Proof sketch]
Nonnegativity follows from $g(z) = z - 1 - \log z \geq 0$ with equality iff $z = 1$.
Along trajectories, the within-level contributions $\dot{V}_{\text{within}} = -\sum_n c_n \sigma_n \sum_j (x_{nj} - x_{nj}^*)^2/x_{nj} \leq 0$.
The cross-level contributions cancel by the tree condition on $c_n$---this is the Li-Shuai-van den Driessche~\cite{li2010} construction applied to the cycle-graph topology.
The specific coefficients $c_n = \Pcycle/k_{n,n-1}$ are those required for cancellation on the cycle graph.
See Appendix~\ref{app:lyapunov} for the full proof.
\end{proof}

\noindent\textbf{Economic interpretation.}
$V$ measures the total welfare loss from being out of equilibrium, decomposed by sector.
Each sector's contribution is $c_n \cdot D_{KL}(\mathbf{x}_n \| \mathbf{x}_n^*)$, where $c_n$ is determined by the system's structure.
$V$ always decreases along the economy's trajectory---the economy always moves toward equilibrium.
The decomposition identifies which sector is contributing most to total welfare loss.

\subsection{The Bridge Equation}\label{sec:bridgeeq}

\begin{theorem}[Eigenstructure Bridge]\label{thm:bridge}
On the quasi-equilibrium surface:
\begin{equation}\label{eq:bridge}
\nabla^2\Phi\big|_{\text{slow}} = W^{-1} \cdot \nabla^2 V
\end{equation}
where $W = \diag(W_{11}, \ldots, W_{NN})$ is the \textbf{institutional quality matrix} with entries
\begin{equation}\label{eq:Wnn}
W_{nn} = \frac{\Pcycle}{|\sigma_n|\,\varepsilon_{T_n}}
\end{equation}
and $\varepsilon_{T_n} = T_n'(\bar{F}_{n-1})\bar{F}_{n-1}/T_n(\bar{F}_{n-1})$ is the elasticity of the coupling at level~$n$.
\end{theorem}

\begin{proof}[Proof sketch]
On the quasi-equilibrium surface, $\Phi|_{\text{slow}} = -\sum_n \log F_n$ and $V = \sum_n c_n \bar{F}_n\,g(F_n/\bar{F}_n)$.
Their Hessians at equilibrium are diagonal:
$(\nabla^2\Phi|_{\text{slow}})_{nn} = 1/\bar{F}_n^2$ and $(\nabla^2 V)_{nn} = c_n/\bar{F}_n$.
The ratio is $W_{nn}^{-1} = 1/(c_n \bar{F}_n)$.
Expressing $c_n = \Pcycle/k_{n,n-1}$ and using the equilibrium relation yields the stated $W_{nn}$.
See Appendix~\ref{app:bridge} for the full derivation.
\end{proof}

\noindent\textbf{Economic interpretation.}
Three objects, three meanings:
\begin{itemize}[leftmargin=2em]
\item $\Phi$ (production difficulty): what the economy \emph{can} do---the curvature of the technology landscape.
\item $V$ (welfare distance): how far the economy \emph{is} from efficiency---the curvature of the welfare landscape.
\item $W$ (institutional quality): how efficiently the economy \emph{adjusts}---the conversion factor between technological possibility and welfare realization.
\end{itemize}

The Bridge says: the curvature of the technology landscape determines the curvature of the welfare landscape, up to a level-specific scaling factor $W_{nn}$ that depends on institutional quality.
Countries with better institutions (lower $W_{nn}$, meaning lower friction $\sigma_n$ or higher coupling elasticity $\varepsilon_{T_n}$) have tighter correspondence between technological possibility and welfare realization.

The production technology ($\rho$) determines \emph{which} adjustments are fast and which are slow (eigenvectors).
The institutional parameters ($\sigma_n$, $\phi_n$) determine \emph{how fast} (eigenvalues).
Different countries have different $\sigma_n$ and $\phi_n$, so they converge at different rates, but along the same directions.
This is a Lucas-critique-compatible statement: the structure is policy-invariant; the dynamics are not.

\subsection{Welfare Loss Decomposition}\label{sec:welfare}

\begin{proposition}[Closed-form welfare loss]\label{prop:welfare}
With power-law gain functions $\phi_n(z) = a_n z^{\beta_n}$, the tree coefficients are
$c_n = \Pcycle\,\sigma_{n-1}/(\beta_n\,\sigma_n\,J\,\bar{F}_n)$
and the welfare distance function simplifies to
\begin{equation}\label{eq:welfare}
V = \frac{\Pcycle}{J}\sum_{n=1}^{N}\frac{\sigma_{n-1}}{\beta_n\,\sigma_n}\;g\!\left(\frac{F_n}{\bar{F}_n}\right).
\end{equation}
Under uniform depreciation $\sigma_n = \sigma$:
\begin{equation}\label{eq:welfare_uniform}
V = \frac{\Pcycle}{\sigma J}\sum_{n=1}^{N}\frac{1}{\beta_n}\;g\!\left(\frac{F_n}{\bar{F}_n}\right)
\end{equation}
where $g(z) = z - 1 - \log z \geq 0$.
The contribution of level~$n$ to welfare loss is proportional to $g(F_n/\bar{F}_n)/\beta_n$.
\end{proposition}

\noindent\textbf{Economic interpretation.}
Sectors with \emph{inelastic} gain functions (small $\beta_n$---cross-sector coupling responds weakly to upstream improvements) contribute more welfare loss per unit of disequilibrium.
The binding welfare constraint is the most institutionally rigid sector, not the most visibly disrupted one.

\textbf{Current implication:}
The welfare-relevant bottleneck is more likely at the capability layer (slow-moving training pipelines, regulatory barriers to AI deployment) than at the settlement layer (fast-moving fintech).
A policymaker focused on stablecoin regulation is optimizing the wrong margin.

\subsection{The Logistic Fragility Condition}\label{sec:logistic}

The power-law gain functions above have constant elasticity $\varepsilon_{T_n} = \beta_n$.
The logistic case reveals a sharper phenomenon.

\begin{proposition}[Logistic fragility]\label{prop:logistic}
For logistic gain $\phi_n(z) = r_n z(1 - z/K_n)$, the elasticity at equilibrium depends on the utilization ratio $u_n = \bar{F}_{n-1}/K_n$:
\begin{equation}\label{eq:logistic_elast}
\varepsilon_{T_n} = \frac{1 - 2u_n}{1 - u_n}.
\end{equation}
The tree coefficient has a pole at $u_n = 1/2$:
\begin{equation}\label{eq:logistic_cn}
c_n = \frac{\Pcycle\,\sigma_{n-1}(1 - u_n)}{\sigma_n\,J\,\bar{F}_n(1 - 2u_n)}.
\end{equation}
Stability of the welfare distance function requires $u_n < 1/2$ (operating below the logistic inflection point).
At $u_n > 1/2$, the elasticity goes negative, the tree coefficient changes sign, and $V$ ceases to be a welfare distance function.
\end{proposition}

\noindent\textbf{Economic interpretation.}
As the upstream level approaches half its carrying capacity, the welfare weight at level~$n$ diverges---perturbations at that level dominate the welfare loss.
This is the approach to the logistic peak: the system is maximally sensitive to perturbations near the inflection point of the S-curve.
Operating above the logistic inflection is destabilizing, not merely inefficient.

\textbf{Prediction:} variance of mesh-related indicators spikes when agent density reaches approximately 50\% of infrastructure capacity---at the inflection point, not at saturation.
\textbf{Design criterion:} engineer carrying capacity so equilibrium utilization stays well below 50\%.


% -------------------------------------------------------------------
% 7. THE DAMPING CANCELLATION AND POLICY
% -------------------------------------------------------------------
\section{The Damping Cancellation and Policy}\label{sec:damping}

\subsection{The Damping-Speed Tradeoff}\label{sec:tradeoff}

\begin{proposition}[Damping cancellation]\label{prop:damping}
For the reduced system on the quasi-equilibrium surface:
\begin{enumerate}[label=(\roman*)]
\item The convergence speed at level~$n$ is $\sigma_n/\varepsilon_n$, strictly increasing in $\sigma_n$---more friction means faster convergence.
\item The equilibrium output is $\bar{F}_n = \phi_n(\bar{F}_{n-1})/(\sigma_n J)$, strictly decreasing in $\sigma_n$---more friction means lower output.
\item The welfare dissipation rate at level~$n$ near equilibrium is
\begin{equation}\label{eq:dissipation}
-\dot{V}_n \approx \frac{\Pcycle\,\sigma_{n-1}}{\beta_n\,J\,\bar{F}_n}\cdot\frac{(\delta F_n)^2}{\bar{F}_n}
\end{equation}
(under power-law gains), which is \textbf{independent of $\sigma_n$ itself}.
\end{enumerate}
\end{proposition}

\begin{proof}
(i)~The eigenvalue of the reduced Jacobian is $-\sigma_n/\varepsilon_n$ (Equation~\ref{eq:jacobian} restricted to the aggregate mode).
(ii)~Direct from the equilibrium condition.
(iii)~$\dot{V}_n = -c_n\sigma_n(\delta F_n)^2/\bar{F}_n$.
Substituting $c_n$ from Proposition~\ref{prop:welfare}: $c_n\sigma_n = \Pcycle\sigma_{n-1}/(\beta_n J\bar{F}_n)$, which is independent of $\sigma_n$.
\end{proof}

\noindent\textbf{Economic interpretation.}
Tightening regulation at sector~$n$ speeds up convergence to equilibrium but lowers the equilibrium itself.
These two effects \emph{exactly cancel} in the welfare dissipation.
The net welfare effect of local regulation is zero.
The welfare dissipation at sector~$n$ depends on $\sigma_{n-1}$ (upstream friction) and $\beta_n$ (the sector's own responsiveness to upstream improvements), not on $\sigma_n$.

\subsection{The Upstream Reform Principle}\label{sec:upstream}

\begin{theorem}[Upstream reform principle]\label{thm:upstream}
To accelerate welfare-relevant adjustment at sector~$n$:
\begin{enumerate}[leftmargin=2em]
\item Increase $\beta_n$---make the sector more responsive to upstream improvements, OR
\item Reduce $\sigma_{n-1}$---reduce friction at the upstream sector.
\item Do NOT increase $\sigma_n$---tightening local regulation has zero net welfare effect.
\end{enumerate}
\end{theorem}

The policy chain:
\begin{itemize}[leftmargin=2em]
\item Fix settlement ($n=4$): reform capability aggregation ($\sigma_3$) or increase settlement elasticity ($\beta_4$).
\item Fix capability ($n=3$): reform network recruitment ($\sigma_2$) or increase training elasticity ($\beta_3$).
\item Fix network ($n=2$): reform hardware investment ($\sigma_1$) or increase recruitment elasticity ($\beta_2$).
\item Fix hardware ($n=1$): reduce $\gamma_c$ directly (CHIPS Act, semiconductor subsidies).
\end{itemize}

\begin{corollary}[Zero welfare effect of stablecoin regulation]\label{cor:stablecoin}
Stablecoin regulation ($\sigma_4$) has zero marginal welfare effect.
Capability-layer reform ($\sigma_3$ or $\beta_4$) has positive marginal welfare effect.
This is a theorem, not a heuristic.
\end{corollary}

\subsection{The Global Welfare Ordering}\label{sec:ordering}

\begin{corollary}[Welfare ordering]\label{cor:ordering}
Under the partial order $\boldsymbol{\beta} \succeq \boldsymbol{\beta}'$ (all gain elasticities weakly higher):
\begin{enumerate}[label=(\roman*)]
\item $W_{nn}(\boldsymbol{\beta}) \leq W_{nn}(\boldsymbol{\beta}')$ for all $n$ (the Bridge tightens---institutional quality improves).
\item $V(\boldsymbol{\beta}) \leq V(\boldsymbol{\beta}')$ at every non-equilibrium state (welfare loss decreases).
\end{enumerate}
\end{corollary}

\noindent\textbf{Economic interpretation.}
Increasing the responsiveness of cross-sector coupling at any level is unambiguously welfare-improving, regardless of the current state of the economy.
Policies that increase cross-sector responsiveness are always welfare-improving.
Policies that flatten response curves are always welfare-reducing.

\subsection{The Rigidity Ordering}\label{sec:rigidity}

From the institutional quality matrix: $W_{11} > W_{22} > W_{33} > W_{44}$ (hardware stiffest, settlement loosest) when the timescale and depreciation orderings align.
Policy interventions at stiff layers (semiconductor subsidies, export controls) have persistent effects.
Interventions at loose layers (stablecoin regulation) have transient effects the system routes around.


% -------------------------------------------------------------------
% 8. TRANSITION DYNAMICS
% -------------------------------------------------------------------
\section{Transition Dynamics}\label{sec:transition}

\subsection{The Hierarchical Ceiling Cascade}\label{sec:ceiling}

\begin{proposition}[Ceiling functions]\label{prop:ceiling}
Under the timescale ordering, successive equilibration yields:
\begin{itemize}[leftmargin=2em]
\item \textbf{Level 4} (fastest): $F_4 \leq \bar{S}(F_3)$. Settlement is bounded by capability.
\item \textbf{Level 3}: $F_3 \leq (\phieff/\delta_C) \cdot \FCES(N^*(F_1))$. Capability is bounded by network and hardware.
\item \textbf{Level 2}: $F_2 \leq N^*(F_1)$. Network is bounded by hardware.
\end{itemize}
\end{proposition}

The cascade of ceilings $F_1 \to F_2 \leq N^*(F_1) \to F_3 \leq (\phieff/\delta_C)\FCES(N^*) \to F_4 \leq \bar{S}(F_3)$ bounds each level by a function of the level below in the timescale hierarchy.
The long-run growth rate equals the hardware improvement rate---the slowest-adapting sector.

\noindent\textbf{Economic interpretation.}
The Baumol cost disease and the Triffin squeeze are the same mathematical object---a quasi-equilibrium surface constraint---at adjacent layers.
The Baumol bottleneck says: as the mesh automates inference, the remaining non-automated task (frontier training) becomes the binding constraint.
The Triffin squeeze says: as stablecoin demand grows, Treasury supply must grow faster than the safety boundary shrinks.
Both are instances of a faster sector being bounded by its slower parent.

\subsection{The Transition Duration}\label{sec:canard}

When $\rho(\mathbf{K})$ crosses~1, the economy undergoes a structural transition---the low-activity equilibrium loses stability and the high-activity equilibrium becomes stable.
But the transition takes time.

\begin{theorem}[Transition duration]\label{thm:canard}
If the bifurcation parameter drifts at rate $\varepsilon_{\text{drift}}$, the transition duration is
\begin{equation}\label{eq:canard}
\Delta t_{\text{crisis}} = \frac{\pi}{\sqrt{|a|\,\varepsilon_{\text{drift}}}} + O\!\left(\frac{\log(1/\delta)}{\sqrt{|a|\,\varepsilon_{\text{drift}}}}\right)
\end{equation}
where $a = \partial^2 g/\partial F_1 \partial\mu|_{\text{bif}}$ is the sensitivity of the growth rate to the bifurcation parameter.

If the drift is in institutional friction ($\gamma_c$ improving): $a = -1$, and the duration is $\pi/\sqrt{\varepsilon_{\text{drift}}}$, independent of all other parameters.

At Wright's Law semiconductor improvement rates ($\approx 15\%$ annual cost decline): $\Delta t \approx \pi/\sqrt{0.15} \approx 8$ years.
\end{theorem}

\begin{proof}[Proof sketch]
At the structural transition, the dynamics admit the local normal form $\dot{y} = a\epsilon y + b y^2 + O(|y|^3 + |\epsilon|^2)$.
If the bifurcation parameter drifts linearly, the rescaled system becomes the Weber equation plus a quadratic perturbation.
The passage through zero eigenvalue creates a delay of $\pi$ time units in the rescaled variable.
Converting back gives the stated duration.
This is the standard delayed loss of stability result \cite{neishtadt1987,neishtadt1988,berglund2006}.
See Appendix~\ref{app:canard} for the full development.
\end{proof}

\noindent\textbf{Economic interpretation.}
After conditions become favorable for the high-activity equilibrium, the economy lingers near the old equilibrium for $O(1/\sqrt{\varepsilon_{\text{drift}}})$ time before snapping to the new one.
This is the ``crisis duration''---the period of structural transition.
It is computable from observable drift rates.

The mixed partial $a$ has two natural cases depending on which parameter drifts:
\begin{itemize}[leftmargin=2em]
\item \textbf{Case 1:} $\mu = \gamma_c$ (institutional friction at the slowest level improves). Then $a = -1$ and the duration is $\pi/\sqrt{\varepsilon_{\text{drift}}}$, independent of all system parameters except the drift rate. This is the simplest case: if what's improving is the institutional friction, the transition time depends only on how fast it improves.
\item \textbf{Case 2:} $\mu = \delta_c$ (investment efficiency improves). Then $|a|$ depends on the product of cascade elasticities through the entire hierarchy:
\begin{equation}\label{eq:cascade_elast}
\frac{\Psi'(\bar{F}_1)\bar{F}_1}{\Psi(\bar{F}_1)} = \varepsilon_I \cdot \varepsilon_{\bar{S}} \cdot \varepsilon_{\FCES} \cdot \varepsilon_{h_2}
\end{equation}
where $\varepsilon_I$, $\varepsilon_{\bar{S}}$, $\varepsilon_{\FCES}$, and $\varepsilon_{h_2}$ are the elasticities of the settlement investment function, the settlement ceiling, the CES capability aggregate, and the network ceiling respectively. The CES elasticity at the symmetric allocation is $\varepsilon_{\FCES} = 1/J$.
\end{itemize}

\begin{proposition}[Curvature dependence of the transition]\label{prop:canard_K}
The second-order coefficient $b = \frac{1}{2}\partial^2 g/\partial F_1^2|_{\text{bif}}$ contains the CES second derivative
\begin{equation}\label{eq:FCES_second}
\frac{\partial^2 \FCES}{\partial F_2^2}\bigg|_{\text{sym}} = -\frac{K}{J\bar{F}_2}
\end{equation}
(from the CES Hessian restricted to the aggregate direction, Paper~A).
Therefore $|b|$ increases with $K$: higher curvature (stronger complementarity) increases the sharpness of the transition.
However, $b$ does not appear in the leading-order transition duration $\pi/\sqrt{|a|\varepsilon_{\text{drift}}}$---it enters only in the correction terms and in the amplitude of the post-transition trajectory.
\end{proposition}

\noindent\textbf{Economic interpretation.}
$K$ controls the \emph{sharpness} of the transition (how quickly the economy accelerates once it begins transitioning), not the \emph{duration}.
Higher complementarity means a faster snap to the new equilibrium, with less overshooting.
The economically important timing question ``how long does the transition take?'' has the answer: $O(1/\sqrt{\varepsilon_{\text{drift}}})$, with the constant controlled by the chain of gain elasticities (Equation~\ref{eq:cascade_elast}), not by the CES substitution parameter.

\subsection{Dispersion as Leading Indicator}\label{sec:dispersion}

At the structural transition, the spectral gap between diversity and aggregate modes closes.
Within-sector heterogeneity stops being slaved to the aggregate.

\textbf{Prediction:} cross-sectional variance of agent performance widens \emph{before} aggregate statistics move.
The within-mesh Gini coefficient rises before the crossing and collapses after (as diversity modes re-equilibrate on the new quasi-equilibrium surface).


% -------------------------------------------------------------------
% 9. EMPIRICAL PREDICTIONS
% -------------------------------------------------------------------
\section{Empirical Predictions}\label{sec:predictions}

\subsection{Calibration Inputs}\label{sec:calibration}

Semiconductor learning curves provide the drift rate $\varepsilon_{\text{drift}}$: at Wright's Law rates of approximately 15\% annual cost decline, the transition duration formula yields a transition window of order 8 years.
The monetary productivity gap (6.4 percentage points) anchors the settlement cost advantage.

\subsection{Predictions}\label{sec:predlist}

\noindent\textbf{P1--P3: Testing the Quadruple Role (Paper~A, applied at each level).}

\noindent(P1)~Cross-agent capability profiles on the unit isoquant diverge as the mesh matures, with superadditivity gap proportional to $K \cdot d_{\mathcal{I}}^2$.
\emph{Falsification:} agent capability profiles converge rather than diverge.

\noindent(P2)~Model collapse incidence remains below threshold for mesh-trained agents, with effective quality bounded below by a function of $K$.
\emph{Falsification:} mesh-trained agents exhibit systematic model collapse.

\noindent(P3)~Coalition manipulation gain in mesh-mediated markets satisfies $\Delta(S) \leq 0$ with penalty proportional to $K_S$.
\emph{Falsification:} sustained profitable manipulation by coalitions in mesh-mediated markets.

\medskip
\noindent\textbf{P4: Testing the spectral threshold.}

\noindent Cross-layer acceleration occurs with delay $\approx \pi/\sqrt{|a|\varepsilon_{\text{drift}}}$ after the drift parameter crosses the threshold.
At Wright's Law rates: 6--10 year transition window.
\emph{Falsification:} no acceleration by 2035.

\medskip
\noindent\textbf{P5--P6: Testing monetary policy degradation.}

\noindent(P5)~Forward guidance effectiveness declines before QE effectiveness, which declines before financial repression collapses.
\emph{Falsification:} policy tools degrade in a different order.

\noindent(P6)~The duration of market impact from FOMC statements declines as autonomous agent market share grows.
\emph{Falsification:} impact duration increases or remains constant.

\medskip
\noindent\textbf{P7--P8: Testing settlement feedback.}

\noindent(P7)~Stablecoin Treasury holdings exceed 5\% of short-duration Treasury supply by 2028.
\emph{Falsification:} stablecoin Treasury share below 3\% by 2029.

\noindent(P8)~At least one country group experiences stablecoin-mediated dollarization by 2030.
\emph{Falsification:} no countries show stablecoin-driven dollarization patterns by 2031.

\medskip
\noindent\textbf{P9: Testing the hierarchical ceiling.}

\noindent The ratio of mesh capability growth to frontier training rate converges: $\dot{C}_{\text{mesh}}/\dot{C}_{\text{frontier}} \to 1$.
\emph{Falsification:} mesh capability growth consistently exceeds frontier training rate.

\medskip
\noindent\textbf{P10: Testing damping cancellation.}

\noindent Tightening stablecoin regulation ($\sigma_4$) has no persistent effect on mesh welfare convergence.
Capability-layer reforms (reducing $\sigma_3$ or increasing $\beta_3$) do.
\emph{Falsification:} stablecoin regulation measurably slows or accelerates mesh formation over a 3+ year horizon.

\medskip
\noindent\textbf{P11: Testing the dispersion indicator.}

\noindent Cross-sectional variance of AI agent performance metrics widens before aggregate mesh statistics shift.
\emph{Falsification:} aggregate leads dispersion.

\medskip
\noindent\textbf{P12: Testing endogenous hierarchy depth.}

\noindent Three testable implications of Theorem~\ref{thm:endogenousN}:

\emph{(a) Innovation vs.\ maturation.}
Sectors mid-innovation show increasing $N_{\mathrm{eff}}(t)$; mature sectors are flat or declining.
Preliminary evidence: Computer/Electronics (NAICS~334) shows $\tau = 0.26$ ($p = 0.018$); Food (NAICS~311) shows $\tau = 0.15$ ($p = 0.16$); historical steel ingots (1899--1939, maturation phase) show $\tau = -0.27$ ($p = 0.002$).

\emph{(b) Sequential waves.}
Sectors with production-visible second paradigms---motor vehicles (electric powertrain, $\tau = 0.31$, $p = 0.004$) and basic chemicals (green/specialty chemistry, $\tau = 0.31$, $p = 0.004$)---show rising $N_{\mathrm{eff}}$ despite being ``mature'' by first-wave standards.
Pharmaceuticals, whose genomics revolution changes R\&D but not manufacturing, shows no production-dimensionality trend ($\tau = -0.05$, $p = 0.67$).

\emph{(c) GPT synchronization.}
General-purpose technology adoption reduces \emph{aggregate} $N_{\mathrm{eff}}$ by creating a dominant cross-sector mode: all three tested GPT dates (postwar electrification, internet, deep learning) show significant negative shifts of 4--6 modes ($p < 0.001$ each).

\emph{Falsification:} (a)~no sector shows lifecycle-position-dependent dimensionality trends by 2035; (b)~a sector with purely R\&D-side innovation shows production-$N_{\mathrm{eff}}$ increase; (c)~a GPT adoption \emph{increases} aggregate dimensionality.

\subsection{Preliminary Evidence: Damping Cancellation}\label{sec:evidence_damping}

Predictions P1--P12 target 2027--2040.  But the mathematical structure of damping cancellation (Proposition~\ref{prop:damping}) applies to \emph{any} regulatory shock propagating through a hierarchical system with CES coupling---not only to future AI dynamics.  We test the mechanism using an existing natural experiment: five waves of the World Bank's Bank Regulation and Supervision Survey (BRSS) across 162 countries, spanning 2001--2019~\cite{barth2013}.

\medskip
\noindent\textbf{Data.}
The treatment variable is the change in Barth--Caprio--Levine (BCL) regulatory indices between BRSS survey waves.  Five indices are constructed from the raw survey responses: capital stringency, activity restrictions, supervisory power, entry barriers, and private monitoring.  The outcome variable is financial development, proxied by domestic credit to private sector as a share of GDP (World Bank indicator FD.AST.PRVT.GD.ZS), the principal component of the IMF Financial Development Index's depth sub-index and the standard dependent variable in the BCL literature~\cite{barth2013}.  The merged panel contains 596 country-wave observations across 158 countries.

\medskip
\noindent\textbf{Method.}
We estimate local projection impulse response functions following Jord\`a~\cite{jorda2005}:
\begin{equation}\label{eq:local_proj}
y_{i,t+h} - y_{i,t} = \alpha_h + \beta_h \,\Delta\text{Reg}_{i,t} + \gamma_h \, y_{i,t} + u_{i,t+h}
\end{equation}
where $y_{i,t}$ is normalized financial development, $\Delta\text{Reg}_{i,t}$ is the change in a BCL regulatory index between consecutive survey waves, and $h = 0, 1, \ldots, 8$ years.
Under damping cancellation, $\beta_h$ should be significant at short horizons ($h \leq 2$) and insignificant by $h = 4$--$6$.
Standard errors are heteroskedasticity-robust (HC1).

\medskip
\noindent\textbf{Results.}
Table~\ref{tab:damping_irf} reports the estimates.  Three of four dimensions display the predicted transient-then-decay pattern:

\begin{table}[H]
\centering
\caption{Local projection IRF: regulatory shock $\to$ financial development}\label{tab:damping_irf}
\small
\begin{tabular}{@{}l rrr rrr rrr@{}}
\toprule
& \multicolumn{3}{c}{$h=1$} & \multicolumn{3}{c}{$h=4$} & \multicolumn{3}{c}{$h=8$} \\
\cmidrule(lr){2-4}\cmidrule(lr){5-7}\cmidrule(lr){8-10}
Dimension & $\hat\beta$ & SE & $N$ & $\hat\beta$ & SE & $N$ & $\hat\beta$ & SE & $N$ \\
\midrule
Activity restrictions & $-0.002^{**}$ & 0.001 & 428 & $-0.002$ & 0.002 & 416 & $-0.000$ & 0.003 & 302 \\
Supervisory power & $-0.000$ & 0.001 & 441 & $-0.005^{***}$ & 0.001 & 429 & $-0.000$ & 0.002 & 311 \\
Overall restrictiveness & $-0.000$ & 0.000 & 443 & $-0.002^{**}$ & 0.001 & 431 & $\phantom{-}0.002$ & 0.001 & 310 \\
Capital stringency & $\phantom{-}0.003^{***}$ & 0.001 & 436 & $-0.001$ & 0.002 & 424 & $\phantom{-}0.008^{**}$ & 0.004 & 307 \\
\bottomrule
\end{tabular}

\medskip
\raggedright\footnotesize
\textit{Notes.}  $^{*}p<0.10$, $^{**}p<0.05$, $^{***}p<0.01$ (HC1).  Sample: 158 countries $\times$ BRSS waves 2--5.
Financial development is domestic credit to private sector (\% GDP), normalized to $[0,1]$.  Regulatory shock is the between-wave change in the BCL index.  Full horizon-by-horizon estimates available upon request.
\end{table}

\begin{itemize}[leftmargin=2em]
\item \emph{Activity restrictions}: significant at $h=1$ ($p=0.04$), insignificant from $h=2$ onward.  Consistent with damping cancellation.
\item \emph{Supervisory power}: significant at $h=2$--$4$ ($p<0.01$), insignificant from $h=5$.  The theory predicts decay by $h=4$--$6$; the data clear significance at exactly $h=5$.
\item \emph{Overall restrictiveness}: significant at $h=3$--$4$, insignificant from $h=5$.  Same pattern.
\item \emph{Capital stringency}: significant at $h=1$ and again at $h=5$--$8$.  This persistence likely reflects a compositional effect: capital requirements directly constrain the credit/GDP ratio (the dependent variable), creating mechanical persistence that is not a violation of \emph{aggregate} damping cancellation.
\end{itemize}

\noindent\textbf{Basel~III difference-in-differences.}
As a complementary test, we estimate a DID specification comparing 25 Basel~III-adopting countries to 133 non-adopters, using 2013 as the treatment year.
The interaction term is $\hat{\beta}_{\text{DID}} = -0.003$ ($p = 0.95$, $N = 1{,}191$).  The null of zero persistent effect cannot be rejected, consistent with Corollary~\ref{cor:stablecoin}.

\medskip
\noindent\textbf{Cross-layer equality.}
Proposition~\ref{prop:damping}(iii) predicts that $c_n\sigma_n$ is independent of $n$, so the persistence profile should be symmetric across regulatory layers.  At $h=5$, the point estimates are $+0.005$ (capital), $-0.004$ (activity), $-0.002$ (supervision), with a cross-layer spread of $0.009$---consistent with the equal-persistence prediction.

\medskip
\noindent\textbf{Interpretation.}
With $N \approx 300$ at long horizons, these are not underpowered tests.  An earlier version using only 16 compiled countries showed apparent persistence at $h=7$ for activity restrictions and supervisory power; expanding to the full 158-country BRSS panel revealed this as sampling noise, exactly as predicted.


\subsection{Preliminary Evidence: Dispersion as Leading Indicator}\label{sec:evidence_dispersion}

Section~\ref{sec:dispersion} predicts that cross-sectional dispersion widens \emph{before} aggregate regime shifts, because the spectral gap between diversity and aggregate modes closes at the transition.  We test this using quarterly semiconductor revenue data from the World Semiconductor Trade Statistics (WSTS) organization, 2000Q1--2024Q4 ($T = 100$).

\medskip
\noindent\textbf{Data.}
Six product segments---logic, memory, analog, discrete, optoelectronics, sensors---provide the cross-section.  The dispersion measure is the cross-segment standard deviation of quarter-over-quarter revenue growth rates.  The aggregate measure is total semiconductor revenue growth.

\medskip
\noindent\textbf{VAR impulse response.}
A bivariate VAR(5) (selected by AIC) in aggregate growth and cross-segment dispersion yields an impulse response of aggregate growth to a dispersion shock that peaks at horizon $h=3$ quarters.
Under the CES interpretation, peak lag maps to the substitution parameter: $h=3$ implies $\rho \in [-0.5, 0]$, in the neighborhood of Cobb--Douglas.  This is consistent with moderately complementary semiconductor segments.

\medskip
\noindent\textbf{CES $\rho$ from the variance filter ratio.}
The ratio of average idiosyncratic variance to aggregate variance is $0.40$.  Inverting the CES variance filter formula yields $\hat\rho = 0.43$, within the economically meaningful range and consistent with the VAR-implied estimate.

\medskip
\noindent\textbf{Regime identification.}
A two-state Markov switching model~\cite{hamilton1989} on aggregate growth identifies 8 regime transitions, coinciding with known industry events: the dot-com bust (2000Q4--2001Q3), the global financial crisis (2008Q4--2009Q2), the 2018--2019 memory correction, and the AI-driven recovery (2022Q4--2023Q2).  This collapses the 35 noisy quarter-over-quarter sign changes to economically meaningful regime shifts.

\medskip
\noindent\textbf{Granger causality.}
Dispersion does not Granger-cause aggregate growth at conventional significance levels ($p > 0.20$ at all lags 1--8), nor does it Granger-cause Markov regime transitions ($p > 0.30$).  The lead-lag correlogram, however, shows positive correlation at leads $+2$ to $+3$ quarters, consistent with the VAR peak.

\medskip
\noindent\textbf{Interpretation.}
The VAR and correlogram support the leading-indicator prediction; the Granger null reflects a power limitation.  With only $T = 100$ quarters and 8 Markov transitions, the binary regime variable has very low variation.  The continuous VAR---which uses the full time-series variation---detects the relationship that the binary Granger test cannot.  A longer sample or higher-frequency data would resolve this.

\medskip
The two tests bracket the empirical evidence: the damping cancellation result is strong (large $N$, clear transient-decay pattern, powerful DID null), while the dispersion indicator result is directionally supportive but underpowered.  Both are tests of the \emph{mathematical mechanism}---CES curvature propagating through a hierarchy---using historical data that predates the AI transition the predictions target.


% -------------------------------------------------------------------
% 10. LIMITATIONS
% -------------------------------------------------------------------
\section{Limitations}\label{sec:limitations}

Two categories of limitations, stated without apology.

\subsection{Mathematical}\label{sec:mathlimits}

\begin{enumerate}[leftmargin=2em]
\item \emph{Gain functions genuinely free.}
Theorem~\ref{thm:topology}(iii) establishes this as a proved impossibility: the exponents and coefficients of $\phi_n$ are not determined by $\rho$.
Since $\phi_n$ determines the equilibrium cascade $\{\bar{F}_n\}$, the welfare weights $\{c_n\}$, and the institutional quality matrix $W$, this freedom propagates through the quantitative predictions.

\item \emph{Timescale separation cannot be eliminated.}
The fast sector equilibrates before the slow sector moves appreciably (Standing Assumption~3) is required for the quasi-equilibrium surface to exist and for the $NJ \to N$ dimensional reduction.
Without it, the quasi-equilibrium surface need not exist, and the reduction is unjustified.
The nearest-neighbor topology (Theorem~\ref{thm:topology}(iv)) also requires timescale separation.
Theorem~\ref{thm:endogenousN} makes the number of levels endogenous but does not weaken this requirement---it formalizes which process pairs have sufficient separation to constitute distinct levels.

\item \emph{The system does not follow potential-based adjustment dynamics.}
The lower-triangular Jacobian is a topological obstruction (Proposition~\ref{prop:nongradient}).
No coordinate transformation can make the system follow potential-based adjustment dynamics while preserving directed coupling.
Standard welfare theorems do not apply.

\item \emph{Local stability only.}
Theorem~\ref{thm:lyapunov} proves $\dot{V} \leq 0$, establishing local asymptotic stability.
Global asymptotic stability requires boundary analysis depending on the specific gain functions.

\item \emph{Symmetric weights for quantitative bounds.}
General weights yield results via the secular equation (Section~\ref{sec:secular}; Paper~A, Section~8), with $R_{\min}$ replacing the equal-weight curvature.
The qualitative results are unchanged; bounds are less clean but remain computable.

\item \emph{$O(\varepsilon)$ approximation error.}
The sufficient statistic property holds up to $O(\varepsilon)$ corrections.
During crisis episodes, within-sector composition may matter.

\item \emph{Delayed-transition duration is leading order.}
Correction terms involve $K$ through the quadratic coefficient~$b$; amplitude behavior is less precisely bounded.
\end{enumerate}

\subsection{Empirical}\label{sec:emplimits}

\begin{enumerate}[leftmargin=2em]
\item Gain elasticities $(\beta_1, \ldots, \beta_4)$ uncalibrated.
\item Damping rates $(\sigma_1, \ldots, \sigma_4)$ uncalibrated.
\item Predictions span 2027--2040 (long horizon).
\item Which specific layer binds first depends on uncalibrated $\beta_n$.
\item Crisis duration estimate requires $\varepsilon_{\text{drift}}$, which is itself uncertain.
\end{enumerate}

\subsection{Frameworks Considered and Rejected}\label{sec:rejected}

\emph{Mean field games} (Lasry-Lions): agents are not exchangeable.
The CES structure ($\rho < 1$) ensures non-exchangeability; MFG would average over the heterogeneity that drives both efficiency results and collusion resistance.

\emph{Minsky financial instability hypothesis}: insufficiently formalized for the results needed here.
The Brunnermeier-Sannikov~\cite{brunnermeier2014} framework captures the same insight rigorously.

\emph{Full continuous-time general equilibrium}: intractable.
The four-ODE deterministic skeleton captures qualitative dynamics; a stochastic extension is deferred.


% -------------------------------------------------------------------
% 11. CONCLUSION
% -------------------------------------------------------------------
\section{Conclusion}\label{sec:conclusion}

One production difficulty function $\Phi = -\sum_n \log F_n$, one derived architecture, five results, three policy principles.

The CES geometry forces the architecture: aggregate coupling, directed feed-forward, nearest-neighbor chain.
The architecture is derived, not assumed.

The spectral threshold $\rho(\mathbf{K}) = 1$ activates the hierarchy---individual sectors can each be too weak to sustain themselves, yet the system as a whole sustains activity through cross-sector feedback.
The hierarchical ceiling cascade bounds each layer by its parent.
The welfare distance function connects the technology ($\Phi$) to the welfare loss ($V$) through the institutional quality ($W$).

Three policy principles follow from theorems:
\begin{enumerate}[leftmargin=2em]
\item \emph{Reform upstream, not locally} (damping cancellation, Proposition~\ref{prop:damping}).
Tightening regulation at any sector has zero net welfare effect.
To improve welfare at sector~$n$, reform sector~$n-1$.
\item \emph{Increase cross-sector responsiveness at any layer} (global welfare ordering, Corollary~\ref{cor:ordering}).
More responsive gain functions are unambiguously welfare-improving.
\item \emph{Invest in the weakest cross-sector link} (cycle product sensitivity, Theorem~\ref{thm:charpoly}).
The system's activation is bottlenecked by its weakest coupling.
The transition takes $O(1/\sqrt{\varepsilon_{\text{drift}}})$---at current semiconductor improvement rates, approximately 8 years.
\end{enumerate}

Eleven predictions, spanning 2027--2040, test the theory.


% -------------------------------------------------------------------
% REFERENCES
% -------------------------------------------------------------------
\newpage
\begin{thebibliography}{99}

% Companion papers
\bibitem{smirl2026a} Smirl, J. (2026a). The CES Quadruple Role: Superadditivity, Correlation Robustness, Strategic Independence, and Network Scaling as Four Properties of CES Curvature. Working Paper.

\bibitem{smirl2026emergent} Smirl, J. (2026). Emergent CES: Renormalization, Functional Equations, and Maximum Entropy Derivations of the CES Aggregate. Working Paper.

% Core mathematical framework
\bibitem{li2010} Li, M.~Y., Shuai, Z., and van den Driessche, P. (2010). Global-stability problem for coupled systems of differential equations on networks. \emph{J.\ Differential Equations} 248, 1--20.

\bibitem{shuai2013} Shuai, Z., and van den Driessche, P. (2013). Global stability of infectious disease models using Lyapunov functions. \emph{SIAM J.\ Appl.\ Math.} 73, 1513--1532.

\bibitem{diekmann1990} Diekmann, O., Heesterbeek, J.~A.~P., and Metz, J.~A.~J. (1990). On the definition and the computation of the basic reproduction ratio $R_0$ in models for infectious diseases in heterogeneous populations. \emph{J.\ Math.\ Biol.} 28, 365--382.

\bibitem{vandendriessche2002} Van den Driessche, P., and Watmough, J. (2002). Reproduction numbers and sub-threshold endemic equilibria for compartmental models of disease transmission. \emph{Math.\ Biosci.} 180, 29--48.

\bibitem{fenichel1979} Fenichel, N. (1979). Geometric singular perturbation theory for ordinary differential equations. \emph{J.\ Differential Equations} 31, 53--98.

\bibitem{vanderschaft2014} van der Schaft, A., and Jeltsema, D. (2014). Port-Hamiltonian systems theory: An introductory overview. \emph{Found.\ Trends Syst.\ Control} 1(2--3), 173--378.

\bibitem{docarmo1992} do Carmo, M.~P. (1992). \emph{Riemannian Geometry}. Birkh\"auser.

% Delayed-transition / delayed bifurcation
\bibitem{neishtadt1987} Neishtadt, A.~I. (1987). Persistence of stability loss for dynamical bifurcations, I. \emph{Differential Equations} 23, 1385--1391.

\bibitem{neishtadt1988} Neishtadt, A.~I. (1988). Persistence of stability loss for dynamical bifurcations, II. \emph{Differential Equations} 24, 171--176.

\bibitem{berglund2006} Berglund, N., and Gentz, B. (2006). \emph{Noise-Induced Phenomena in Slow-Fast Dynamical Systems}. Springer.

% CES and production theory
\bibitem{arrow1961} Arrow, K.~J., Chenery, H.~B., Minhas, B.~S., and Solow, R.~M. (1961). Capital-labor substitution and economic efficiency. \emph{Rev.\ Econ.\ Stat.} 43, 225--250.

\bibitem{dixit1977} Dixit, A.~K., and Stiglitz, J.~E. (1977). Monopolistic competition and optimum product diversity. \emph{Amer.\ Econ.\ Rev.} 67, 297--308.

\bibitem{jones2005} Jones, C.~I. (2005). The shape of production functions and the direction of technical change. \emph{Quart.\ J.\ Econ.} 120, 517--549.

\bibitem{shapley1971} Shapley, L.~S. (1971). Cores of convex games. \emph{Int.\ J.\ Game Theory} 1, 11--26.

% Network science
\bibitem{barabasi1999} Barab\'asi, A.-L., and Albert, R. (1999). Emergence of scaling in random networks. \emph{Science} 286, 509--512.

\bibitem{pastor2001} Pastor-Satorras, R., and Vespignani, A. (2001). Epidemic spreading in scale-free networks. \emph{Phys.\ Rev.\ Lett.} 86, 3200--3203.

% Growth theory
\bibitem{jones1995} Jones, C.~I. (1995). R\&D-based models of economic growth. \emph{J.\ Polit.\ Econ.} 103, 759--784.

\bibitem{romer1990} Romer, P.~M. (1990). Endogenous technological change. \emph{J.\ Polit.\ Econ.} 98, S71--S102.

\bibitem{aghion2018} Aghion, P., Jones, B.~F., and Jones, C.~I. (2018). Artificial intelligence and economic growth. In \emph{The Economics of Artificial Intelligence}, University of Chicago Press.

\bibitem{bloom2020} Bloom, N., Jones, C.~I., Van Reenen, J., and Webb, M. (2020). Are ideas getting harder to find? \emph{Amer.\ Econ.\ Rev.} 110, 1104--1144.

\bibitem{baumol1967} Baumol, W.~J. (1967). Macroeconomics of unbalanced growth. \emph{Amer.\ Econ.\ Rev.} 57, 415--426.

% Monetary economics
\bibitem{brunnermeier2014} Brunnermeier, M.~K., and Sannikov, Y. (2014). A macroeconomic model with a financial sector. \emph{Amer.\ Econ.\ Rev.} 104, 379--421.

\bibitem{woodford2003} Woodford, M. (2003). \emph{Interest and Prices}. Princeton University Press.

\bibitem{lucas1976} Lucas, R.~E. (1976). Econometric policy evaluation: A critique. \emph{Carnegie-Rochester Conf.\ Ser.\ Public Policy} 1, 19--46.

% Dollarization
\bibitem{uribe1997} Uribe, M. (1997). Hysteresis in a simple model of currency substitution. \emph{J.\ Monet.\ Econ.} 40, 185--202.

\bibitem{farhi2018} Farhi, E., and Maggiori, M. (2018). A model of the international monetary system. \emph{Quart.\ J.\ Econ.} 133, 295--355.

\bibitem{triffin1960} Triffin, R. (1960). \emph{Gold and the Dollar Crisis}. Yale University Press.

% Market microstructure
\bibitem{kyle1985} Kyle, A.~S. (1985). Continuous auctions and insider trading. \emph{Econometrica} 53, 1315--1335.

% Autocatalysis
\bibitem{hordijk2004} Hordijk, W., and Steel, M. (2004). Detecting autocatalytic, self-sustaining sets in chemical reaction systems. \emph{J.\ Theor.\ Biol.} 227, 451--461.

% Model collapse
\bibitem{shumailov2024} Shumailov, I., et al. (2024). The curse of recursion: Training on generated data makes models forget. \emph{Nature} (forthcoming).

% Empirical test references
\bibitem{barth2013} Barth, J.~R., Caprio, G., and Levine, R. (2013). Bank regulation and supervision in 180 countries from 1999 to 2011. \emph{J.\ Finan.\ Econ.\ Policy} 5, 111--219.

\bibitem{jorda2005} Jord\`a, \`O. (2005). Estimation and inference of impulse responses by local projections. \emph{Amer.\ Econ.\ Rev.} 95, 161--182.

\bibitem{hamilton1989} Hamilton, J.~D. (1989). A new approach to the economic analysis of nonstationary time series and the business cycle. \emph{Econometrica} 57, 357--384.

\end{thebibliography}


% -------------------------------------------------------------------
% APPENDICES
% -------------------------------------------------------------------
\newpage
\appendix

\section{Proof of the Port Topology Theorem}\label{app:topology}

\subsection{Proof of Claim (i): Aggregate Coupling}

\begin{proof}
\emph{Step 1: Equilibrium uniqueness.}
At equilibrium, $(T_n/J)\,x_{nj}^{\rho-1}\,F_n^{1-\rho} = \sigma_n x_{nj}$, so $x_{nj}^{\rho-2} = T_n F_n^{1-\rho}/(J\sigma_n)$.
The right side is independent of $j$.
Since $\rho < 1$ implies $\rho - 2 < 0$, the map $x \mapsto x^{\rho-2}$ is injective on $\mathbb{R}_+$, so $x_{nj} = \bar{x}_n$ for all $j$.

\emph{Step 2: Normal hyperbolicity.}
Define the equilibrium manifold $\mathcal{M}_n = \{x_{nj} = F_n / J^{1/\rho} \text{ for all } j\}$.
By Equation~\eqref{eq:jacobian}, the transverse eigenvalues (on $\mathbf{1}^{\perp}$) are $-\sigma_n(2-\rho)/\varepsilon_n$ and the tangential eigenvalue (on $\mathbf{1}$) is $-\sigma_n/\varepsilon_n$.
The spectral gap is
$\sigma_n(2-\rho)/\varepsilon_n - \sigma_n/\varepsilon_n = \sigma_n(1-\rho)/\varepsilon_n > 0$ for all $\rho < 1$.

\emph{Step 3: Fenichel persistence.}
By Fenichel's geometric singular perturbation theorem~\cite{fenichel1979}, the normally hyperbolic equilibrium manifold $\mathcal{M}_n$ persists as a locally invariant quasi-equilibrium surface $\mathcal{M}_n^{\varepsilon}$ within $O(\varepsilon)$ of $\mathcal{M}_n$, smoothly parameterized by $F_n$.
On $\mathcal{M}_n^{\varepsilon}$, the within-level state is determined by $F_n$ up to $O(\varepsilon)$ corrections.
Consequently, $F_n$ is a sufficient statistic for level~$n$'s state on the quasi-equilibrium surface.
\end{proof}

\subsection{Proof of Claim (ii): Directed Coupling}

\begin{proof}
\emph{Step 1: Power-preserving bidirectional coupling.}
Suppose levels~1 and~2 are coupled bidirectionally with port powers summing to zero.
The power-preservation constraint forces linear coupling $\phi(F_1) = cF_1$ and $\psi(F_2) = cF_2$ for some constant $c > 0$.
The Jacobian is then
$\mathcal{J}_{\text{bidir}} = \begin{psmallmatrix} -\sigma_1 & -c/J \\ c/J & -\sigma_2 \end{psmallmatrix}$
with eigenvalues $-(\sigma_1+\sigma_2)/2 \pm \sqrt{(\sigma_1-\sigma_2)^2/4 - c^2/J^2}$.
Whether the discriminant is positive (two real negative eigenvalues) or negative (complex pair with real part $-(\sigma_1+\sigma_2)/2 < 0$), both eigenvalues have strictly negative real part.
The system is unconditionally stable for all $c, \sigma_1, \sigma_2 > 0$.

\emph{Step 2: Extension to passive bidirectional coupling.}
Any passive bidirectional coupling satisfies $\dot{V}_{\text{coupling}} \leq 0$ by definition.
This contributes a negative-semidefinite term to the effective Jacobian, which can only strengthen stability.
Therefore any bidirectional coupling---whether power-preserving or merely passive---yields an unconditionally stable system.

\emph{Step 3: Necessity of directed coupling.}
The structural transition at $\rho(\mathbf{K}) = 1$ requires that the spectral radius of the next-generation matrix reach~1, requiring net energy injection through the hierarchy.
Power-preserving bidirectional coupling contributes zero net energy; passive coupling contributes negative net energy.
Neither can produce $\rho(\mathbf{K}) = 1$.
Therefore the CES geometry, combined with the requirement for nontrivial dynamics, forces the between-level coupling to be non-reciprocal with an external energy source.
\end{proof}

\subsection{Proof of Claim (iii): Port Alignment}

\begin{proof}
\emph{Step 1: Port direction is forced.}
At a symmetric equilibrium $\mathbf{x}_n^* = \bar{x}\,\mathbf{1}$, the equilibrium condition requires the coupling direction $\mathbf{b}_n \propto \mathbf{1}$.
Furthermore, $\nabla F_n = (1/J)\,\mathbf{1} \propto \mathbf{1}$ at the symmetric point (Paper~A), so $\mathbf{b}_n = \nabla F_n$ is the natural CES-compatible port direction.

\emph{Step 2: Asymmetric ports are penalized.}
For $\mathbf{b}_n \not\propto \mathbf{1}$, the equilibrium $\mathbf{x}_n^* \propto \mathbf{b}_n$ is asymmetric.
By Jensen's inequality applied to the concave CES function ($\rho < 1$), $F(\mathbf{x}) \leq F(\bar{x}\,\mathbf{1})$ for any $\mathbf{x}$ with $\sum x_j = J\bar{x}$.
Asymmetric ports produce less aggregate output per unit input.

\emph{Step 3: Gain function is free.}
At equilibrium, $\phi_n(\bar{F}_{n-1}) = \sigma_n J \bar{F}_n$.
For power-law gains $\phi_n(z) = a_n z^{\beta_n}$, the exponents $\beta_n$ are free parameters not determined by $\rho$.
The coefficients $a_n$ adjust to satisfy the equilibrium condition but depend on $\sigma_n$, $J$, and the cascade---not on $\rho$ alone.
\end{proof}

\subsection{Proof of Claim (iv): Nearest-Neighbor Topology}

\begin{proof}
Consider a three-level system with long-range coupling from level~1 to level~3.
Level~1, being fastest ($\varepsilon_1 \ll \varepsilon_2$), equilibrates to $F_1^* = \beta_1/(\sigma_1 J)$, an algebraic constant on the quasi-equilibrium surface.
The long-range coupling $\phi_{31}(F_1^*) = \phi_{31}(\beta_1/(\sigma_1 J)) \equiv \tilde{\beta}_3$ becomes a constant, absorbed into the exogenous input to level~3.
The effective dynamics become identical to a nearest-neighbor system with modified exogenous input.

The Jacobian of the reduced system is lower-triangular---independent of the long-range coupling strength.
Long-range coupling affects the equilibrium location but not the dynamics or stability near that equilibrium.

The argument generalizes by induction to $N$ levels: after all levels faster than level~$m$ equilibrate, any coupling $\phi_{nm}(F_m)$ with $m$ fast becomes $\phi_{nm}(F_m^*) = \text{const}$, absorbed into a redefined exogenous input.
The effective topology is nearest-neighbor.
This result requires the timescale separation of Standing Assumption~(3).
\end{proof}


\section{Proof of the Welfare Distance Function}\label{app:lyapunov}

\begin{proof}
Nonnegativity follows from $g(z) = z - 1 - \log z \geq 0$ with equality iff $z = 1$.
Along trajectories:
\[
\dot{V} = \sum_n c_n \sum_j \left(1 - \frac{x_{nj}^*}{x_{nj}}\right) f_{nj}(\mathbf{x}).
\]
The within-level contributions are
\[
\dot{V}_{\text{within}} = -\sum_n c_n \sigma_n \sum_j \frac{(x_{nj} - x_{nj}^*)^2}{x_{nj}} \leq 0.
\]
The cross-level contributions involve terms of the form
\[
c_n \sum_j \left(1 - \frac{x_{nj}^*}{x_{nj}}\right)\frac{T_n}{J}x_{nj}^{\rho-1}F_n^{1-\rho}.
\]
At the symmetric equilibrium $x_{nj}^* = \bar{x}_n$, these contributions cancel by the tree condition on $c_n = \Pcycle/k_{n,n-1}$.
This is the Li-Shuai-van den Driessche~\cite{li2010} construction applied to the cycle-graph topology, using the Volterra-Lyapunov identity
$a - b\log(a/b) \leq a - b + b\log(b/a)$
for each coupling term.

For the 4-level cycle, there is exactly one spanning in-tree per root, and the specific coefficients $c_n = \Pcycle/k_{n,n-1}$ are exactly those required for cancellation.
\end{proof}


\section{Proof of the Eigenstructure Bridge}\label{app:bridge}

\begin{proof}
On the quasi-equilibrium surface, $\Phi|_{\text{slow}} = -\sum_n \log F_n$ and $V = \sum_n c_n \bar{F}_n\,g(F_n/\bar{F}_n)$.
Their Hessians at equilibrium are diagonal:
\[
(\nabla^2\Phi|_{\text{slow}})_{nn} = \frac{1}{\bar{F}_n^2}, \qquad (\nabla^2 V)_{nn} = \frac{c_n}{\bar{F}_n}.
\]
The ratio is $(\nabla^2\Phi)_{nn}/(\nabla^2 V)_{nn} = 1/(c_n \bar{F}_n) = W_{nn}^{-1}$.

Expressing $c_n = \Pcycle/k_{n,n-1}$ and $k_{n,n-1} = T_n'(\bar{F}_{n-1})\bar{F}_{n-1}/|\sigma_{n-1}|$, together with the equilibrium relation $T_n(\bar{F}_{n-1}) = |\sigma_n|\bar{F}_n$:
\begin{align*}
c_n\bar{F}_n &= \frac{\Pcycle}{k_{n,n-1}}\cdot\bar{F}_n = \Pcycle\cdot\frac{|\sigma_{n-1}|}{T_n'\bar{F}_{n-1}}\cdot\bar{F}_n \\
&= \frac{\Pcycle}{|\sigma_n|}\cdot\frac{T_n}{T_n'\bar{F}_{n-1}} = \frac{\Pcycle}{|\sigma_n|\,\varepsilon_{T_n}}
\end{align*}
where $\varepsilon_{T_n} = T_n'(\bar{F}_{n-1})\bar{F}_{n-1}/T_n(\bar{F}_{n-1})$ is the elasticity of the coupling at level~$n$.

Special cases:
\emph{Power-law coupling} ($\phi_n(z) = a_n z^{\beta_n}$): the elasticity is constant, $\varepsilon_{T_n} = \beta_n$, giving $W_{nn} = \Pcycle/(\beta_n|\sigma_n|)$.

\emph{Linear coupling} ($\beta_n = 1$) with uniform damping ($\sigma_n = \sigma$): $W = (\Pcycle/\sigma)\,I$, so $\nabla^2\Phi|_{\text{slow}} = (\sigma/\Pcycle)\,\nabla^2 V$.
This is the only case where the system ``almost'' follows potential-based adjustment dynamics.
\end{proof}


\section{Transition Dynamics: The Normal Form}\label{app:canard}

\subsection{The Transcritical Normal Form}

\begin{proof}
At the bifurcation point $(\bar{F}_1, \mu^*)$ where $g = 0$ and $\partial g/\partial F_1 = 0$, the dynamics admit the local normal form
\[
\dot{y} = a\,\epsilon\,y + b\,y^2 + O(|y|^3 + |\epsilon|^2)
\]
where $y = F_1 - \bar{F}_1$, $\epsilon = \mu - \mu^*$, and
\[
a = \frac{\partial^2 g}{\partial F_1\,\partial\mu}\bigg|_{\text{bif}}, \qquad b = \frac{1}{2}\,\frac{\partial^2 g}{\partial F_1^2}\bigg|_{\text{bif}}.
\]
The conditions $g = 0$ and $\partial_F g = 0$ eliminate the constant and linear terms.
The nondegeneracy conditions $a \neq 0$ and $b \neq 0$ are the transversality requirements for a structural transition.
\end{proof}

\subsection{Computing the Mixed Partial}

\noindent\textbf{Case 1: $\mu = \gamma_c$ (institutional friction improves).}
$g = \delta_c\Psi(F_1)^\alpha F_1^{\phi_c} - \gamma_c F_1$, so $\partial g/\partial\gamma_c = -F_1$ and $\partial^2 g/\partial F_1\partial\gamma_c = -1$.
Thus $a = -1$.
The transition duration is $\pi/\sqrt{\varepsilon_{\text{drift}}}$, independent of all system parameters except the drift rate.

\noindent\textbf{Case 2: $\mu = \delta_c$ (investment efficiency improves).}
$\partial^2 g/\partial F_1\partial\delta_c = [\alpha\Psi'/\Psi\cdot F_1 + \phi_c]\Psi^\alpha F_1^{\phi_c - 1}$.
The term $\Psi'/\Psi$ propagates through the entire cascade of ceiling functions.
We now derive this cascade explicitly.

\emph{Step 1: Decompose the composite ceiling.}
The full ceiling function $\Psi(F_1) = I(\bar{S}(\FCES(h_2(F_1))))$ chains four maps:
\begin{itemize}[leftmargin=2em]
\item $h_2$: the network ceiling $F_2 \leq N^*(F_1)$, with elasticity $\varepsilon_{h_2} = h_2'(\bar{F}_1)\bar{F}_1/h_2(\bar{F}_1)$;
\item $\FCES$: the CES aggregate at the capability level, with elasticity $\varepsilon_{\FCES}$;
\item $\bar{S}$: the settlement ceiling, with elasticity $\varepsilon_{\bar{S}}$;
\item $I$: the settlement investment function, with elasticity $\varepsilon_I$.
\end{itemize}

\emph{Step 2: Apply the chain rule.}
By the chain rule for elasticities:
\[
\frac{\Psi'(\bar{F}_1)\bar{F}_1}{\Psi(\bar{F}_1)} = \varepsilon_I \cdot \varepsilon_{\bar{S}} \cdot \varepsilon_{\FCES} \cdot \varepsilon_{h_2}.
\]
This is Equation~\eqref{eq:cascade_elast} in the main text.

\emph{Step 3: Evaluate the CES elasticity.}
At the symmetric allocation $x_{nj} = \bar{x}_n$ for all $j$:
\[
\varepsilon_{\FCES} = \frac{\partial \FCES}{\partial F_2}\bigg|_{\text{sym}} \cdot \frac{\bar{F}_2}{\bar{F}_{\text{CES}}} = \frac{1}{J}
\]
since each of $J$ symmetric inputs contributes equally to the aggregate.
Hence the cascade elasticity simplifies to $\varepsilon_I \cdot \varepsilon_{\bar{S}} \cdot \varepsilon_{h_2}/J$.

\subsection{Where $K$ Enters}\label{app:K_enters}

\begin{proof}[Proof of Proposition~\ref{prop:canard_K}]
\emph{Step 1: CES second derivative at the symmetric point.}
From Paper~A, the CES function $\FCES = \left(\frac{1}{J}\sum_{j=1}^J x_j^\rho\right)^{1/\rho}$ has second derivative along the aggregate direction $\mathbf{1}$:
\[
\frac{\partial^2 \FCES}{\partial F_2^2}\bigg|_{\text{sym}} = \frac{\rho - 1}{J\bar{F}_2} = -\frac{K}{J\bar{F}_2}
\]
where $K = (1-\rho)(J-1)/J$ and the restriction to the aggregate direction absorbs the $(J-1)/J$ factor as $(1-\rho)/J$.

\emph{Step 2: Propagation to $b$.}
The coefficient $b = \frac{1}{2}\partial^2 g/\partial F_1^2|_{\text{bif}}$ inherits a contribution from $\partial^2 \FCES/\partial F_2^2$ through the chain of ceiling functions.
The chain rule for the second derivative of the composite $\Psi(F_1)$ yields terms involving $\Psi''$, which in turn contains $\partial^2\FCES/\partial F_2^2$.
Therefore $|b|$ is increasing in $K$: higher curvature (stronger complementarity) amplifies the quadratic nonlinearity.

\emph{Step 3: Separation of roles.}
$K$ enters $b$ (the \emph{sharpness} coefficient) but not $a$ (the \emph{duration} coefficient).
The leading-order transition time $\pi/\sqrt{|a|\varepsilon_{\text{drift}}}$ is independent of $K$.
$K$ controls only the correction terms and the post-transition trajectory amplitude.
\end{proof}

\subsection{Passage Time}

In the rescaled variable $\tau = \sqrt{|a|\varepsilon_{\text{drift}}}\,t$, the normal form becomes the Weber equation plus a quadratic perturbation.
The passage through zero eigenvalue creates a delay of $\pi$ time units in the rescaled variable.
Converting back: $\Delta t = \pi/\sqrt{|a|\varepsilon_{\text{drift}}}$.
This is the standard delayed loss of stability result for passage through a structural transition \cite{neishtadt1987,neishtadt1988,berglund2006}.

The logarithmic correction accounts for the entry and exit from the $O(\delta)$-neighborhood, which depends on the initial distance from the quasi-equilibrium surface.

At Wright's Law semiconductor improvement rates ($\varepsilon_{\text{drift}} \approx 0.15$):
$\Delta t \approx \pi/\sqrt{0.15} \approx 8.1$ years.

In both cases, the crisis duration scales as $\varepsilon_{\text{drift}}^{-1/2}$: halving the drift rate increases the transition time by $\sqrt{2} \approx 41\%$.


\section{The Welfare Loss Decomposition}\label{app:welfare}

\begin{proof}[Proof of Proposition~\ref{prop:welfare}]
From Theorem~\ref{thm:lyapunov}, $c_n = \Pcycle/k_{n,n-1}$ where $k_{n,n-1} = \phi_n'(\bar{F}_{n-1})\bar{F}_{n-1}/|\sigma_{n-1}|$.
For power-law $\phi_n(z) = a_n z^{\beta_n}$: $\phi_n'(\bar{F}_{n-1})\cdot\bar{F}_{n-1} = \beta_n\,\phi_n(\bar{F}_{n-1}) = \beta_n\,\sigma_n\,J\,\bar{F}_n$ (using the equilibrium condition $\phi_n(\bar{F}_{n-1}) = \sigma_n J \bar{F}_n$).
Thus $k_{n,n-1} = \beta_n\,\sigma_n\,J\,\bar{F}_n/\sigma_{n-1}$, giving $c_n = \Pcycle\,\sigma_{n-1}/(\beta_n\,\sigma_n\,J\,\bar{F}_n)$.
On the quasi-equilibrium surface, $c_n D_{KL} = c_n \bar{F}_n g(F_n/\bar{F}_n) = \Pcycle\sigma_{n-1}/(\beta_n\sigma_n J)\,g(F_n/\bar{F}_n)$.
\end{proof}

\begin{proof}[Proof of Proposition~\ref{prop:damping}]
(i)~The eigenvalue of the reduced Jacobian is $-\sigma_n/\varepsilon_n$ (Equation~\ref{eq:jacobian} restricted to the aggregate mode).
(ii)~Direct from the equilibrium condition $\bar{F}_n = \phi_n(\bar{F}_{n-1})/(\sigma_n J)$.
(iii)~$\dot{V}_n = -c_n\sigma_n(F_n - \bar{F}_n)^2/F_n \approx -c_n\sigma_n(\delta F_n)^2/\bar{F}_n$.
Substituting $c_n$ from Proposition~\ref{prop:welfare}:
$c_n\sigma_n = \Pcycle\sigma_{n-1}/(\beta_n J\bar{F}_n)$,
which is independent of $\sigma_n$.
The $\sigma_n$ in $c_n$ exactly cancels the $\sigma_n$ in the dissipation formula.
\end{proof}

\begin{proof}[Proof of Corollary~\ref{cor:ordering}]
(i)~$W_{nn} = \Pcycle/(\beta_n|\sigma_n|)$ under power-law gains, strictly decreasing in $\beta_n$.
Higher $\beta_n$ means lower $W_{nn}$, meaning tighter institutional quality.
(ii)~From $c_n \propto 1/\beta_n$, higher $\beta_n$ gives lower weight $c_n$, hence lower $V$ at any non-equilibrium state.
\end{proof}

\begin{proof}[Proof of Proposition~\ref{prop:logistic}]
For logistic gain $\phi_n(z) = r_n z(1 - z/K_n)$:
\[
\phi_n'(z) = r_n(1 - 2z/K_n), \qquad \phi_n(z) = r_n z(1 - z/K_n).
\]
The elasticity at $z = \bar{F}_{n-1}$ with utilization $u_n = \bar{F}_{n-1}/K_n$ is:
\[
\varepsilon_{T_n} = \frac{\phi_n'(\bar{F}_{n-1})\bar{F}_{n-1}}{\phi_n(\bar{F}_{n-1})} = \frac{r_n(1-2u_n)\bar{F}_{n-1}}{r_n\bar{F}_{n-1}(1-u_n)} = \frac{1 - 2u_n}{1 - u_n}.
\]
This has a pole at $u_n = 1/2$ (the logistic inflection point) and changes sign there.

The tree coefficient is $c_n = \Pcycle/k_{n,n-1}$ with $k_{n,n-1} = \varepsilon_{T_n}\sigma_n J\bar{F}_n/\sigma_{n-1}$.
Substituting $\varepsilon_{T_n}$:
\[
c_n = \frac{\Pcycle\,\sigma_{n-1}(1-u_n)}{\sigma_n J\bar{F}_n(1-2u_n)}.
\]
For $u_n > 1/2$: $\varepsilon_{T_n} < 0$, so $c_n < 0$, and $V = \sum c_n \bar{F}_n g(F_n/\bar{F}_n)$ ceases to be positive semidefinite.
The Lyapunov argument fails, and local stability is no longer guaranteed by the graph-theoretic construction.
\end{proof}


\end{document}
