\documentclass[12pt,letterpaper]{article}

% Page layout
\usepackage[margin=1in]{geometry}
\usepackage{setspace}
\onehalfspacing

% Math
\usepackage{amsmath,amssymb,amsthm,mathtools}

% Tables
\usepackage{booktabs}
\usepackage{array}
\usepackage{tabularx}

% Typography
\usepackage[T1]{fontenc}
\usepackage[expansion=false]{microtype}
\usepackage{enumitem}

% References
\usepackage{xcolor}
\usepackage[colorlinks=true,linkcolor=blue,citecolor=blue,urlcolor=blue]{hyperref}

% Theorem environments
\newtheorem{theorem}{Theorem}[section]
\newtheorem{proposition}[theorem]{Proposition}
\newtheorem{lemma}[theorem]{Lemma}
\newtheorem{corollary}[theorem]{Corollary}
\newtheorem{definition}[theorem]{Definition}
\theoremstyle{remark}
\newtheorem{remark}[theorem]{Remark}
\newtheorem{example}[theorem]{Example}

% Custom commands
\DeclareMathOperator*{\argmax}{arg\,max}
\DeclareMathOperator{\tr}{tr}
\DeclareMathOperator{\sgn}{sgn}
\DeclareMathOperator{\diag}{diag}
\DeclareMathOperator{\Var}{Var}
\DeclareMathOperator{\Cov}{Cov}
\DeclareMathOperator{\Corr}{Corr}
\newcommand{\R}{\mathbb{R}}
\newcommand{\E}{\mathbb{E}}
\newcommand{\bx}{\mathbf{x}}
\newcommand{\by}{\mathbf{y}}
\newcommand{\bv}{\mathbf{v}}
\newcommand{\bX}{\mathbf{X}}
\newcommand{\ba}{\mathbf{a}}
\newcommand{\bp}{\mathbf{p}}
\newcommand{\bdelta}{\boldsymbol{\delta}}
\newcommand{\beps}{\boldsymbol{\epsilon}}
\newcommand{\bfeta}{\boldsymbol{\eta}}
\newcommand{\bone}{\mathbf{1}}

% Section formatting
\usepackage{titlesec}
\titleformat{\section}{\large\bfseries}{\thesection.}{0.5em}{}
\titleformat{\subsection}{\normalsize\bfseries}{\thesubsection}{0.5em}{}

\begin{document}

% -------------------------------------------------------------------
% TITLE
% -------------------------------------------------------------------
\title{\textbf{The CES Triple Role}\\[6pt]
\large Superadditivity, Correlation Robustness, and Strategic Independence\\
as Three Views of Isoquant Curvature}

\author{Jon Smirl\\
Independent Researcher}

\date{February 2026\\[6pt]
\textsc{Working Paper}}

\maketitle

\begin{abstract}
\noindent The CES production function is ubiquitous in economics, yet three of its
important properties---superadditivity, robustness to correlated inputs, and
resistance to strategic manipulation---have been studied separately using different
techniques.  This paper proves they are controlled by a single parameter: the
\textbf{curvature parameter} $K = (1-\rho)(J-1)/J$, derived from the principal
curvature of the CES isoquant at the cost-minimizing point.  Superadditivity gap
$= \Omega(K)\cdot\text{diversity}$; correlation robustness bonus
$= \Omega(K^2)\cdot\text{idiosyncratic variation}$; strategic manipulation penalty
$= -\Omega(K)\cdot\text{deviation}^2$.  All three bounds tighten monotonically in
$K$.  When $K = 0$ (perfect substitutes), all three vanish simultaneously.  The
three properties are not three theorems sharing an assumption---they are the same
geometric fact, the curvature of the isoquant, viewed from aggregation theory,
information theory, and game theory.  Results extend to general (unequal) CES
weights via the secular equation of the weighted inverse-share matrix, whose
smallest root $R_{\min}$ controls the generalized curvature parameter.
\end{abstract}

\medskip
\noindent\textbf{Keywords:} CES production function, isoquant curvature,
superadditivity, diversification, strategic independence, secular equation

\noindent\textbf{JEL:} C62, D24, D43, D81, L13


% ===================================================================
% 1. INTRODUCTION
% ===================================================================
\section{Introduction}\label{sec:intro}

The constant elasticity of substitution (CES) production function, introduced by
Arrow, Chenery, Minhas, and Solow~\cite{arrow1961}, is among the most widely used
functional forms in economics.  It appears in trade theory, industrial
organization, macroeconomics with heterogeneous firms, and index number
construction.  The Dixit-Stiglitz~\cite{dixit1977} formulation of monopolistic
competition, the Jones~\cite{jones2005} analysis of directed technical change, and
the Houthakker~\cite{houthakker1955} aggregation results all rest on CES.

Three important properties of CES aggregation have been established in various
settings:
\begin{enumerate}[leftmargin=2em]
\item \emph{Superadditivity:} Combining diverse input bundles produces more output
than the sum of separate productions.  This matters for merger analysis, team
formation, and gains from trade.

\item \emph{Correlation robustness:} The CES aggregate preserves information about
its components even when they are highly correlated.  This matters for portfolio
diversification, index construction, and aggregate measurement.

\item \emph{Strategic independence:} Coalitions of input suppliers cannot profitably
manipulate the aggregate by redistributing or withholding inputs.  This matters for
market design, mechanism design, and platform governance.
\end{enumerate}

These properties have been proved using different techniques---superadditivity from
convexity arguments, correlation robustness from second-order expansions, strategic
independence from cooperative game theory.  This paper demonstrates that all three
are controlled by a single dimensionless parameter:
\begin{equation}\label{eq:K_intro}
K \;=\; (1-\rho)\,\frac{J-1}{J}
\end{equation}
where $\rho < 1$ is the CES substitution parameter and $J \geq 2$ is the number of
components.  This parameter is the normalized principal curvature of the CES
isoquant at the cost-minimizing point.

The main result (Theorem~\ref{thm:triple}) states:
\begin{itemize}[leftmargin=2em]
\item[(a)] The superadditivity gap is bounded below by $\Omega(K)$ times a geodesic
diversity measure (first-order curvature effect).
\item[(b)] The effective dimension under equicorrelation exceeds the linear baseline
by $\Omega(K^2)$ times an idiosyncratic variation term (second-order curvature
effect).
\item[(c)] The strategic manipulation gain is bounded above by $-\Omega(K)$ times a
squared deviation (first-order curvature effect).
\end{itemize}
All three bounds tighten monotonically in $K$.  When $K = 0$ (perfect substitutes,
$\rho = 1$), all three vanish.

The underlying mechanism is the same in all three cases: the isoquant is curved.
Curvature forces convex combinations of diverse points above the level set
(superadditivity), maps correlated input variation into distinct output regions
through a nonlinear channel (informational diversity), and penalizes deviations
from the balanced allocation (strategic stability).  These are three views of a
single geometric object, not three consequences of a common assumption.

The results extend to CES with general (unequal) weights $a_j > 0$ via the
\textbf{secular equation} of the weighted inverse-share matrix.  With unequal
weights, the principal curvatures of the isoquant are no longer degenerate; they
are determined by the $J-1$ roots of the secular equation, which interlace the
inverse shares.  The smallest root $R_{\min}$ controls a generalized curvature
parameter $K(\rho, \ba)$ that replaces~\eqref{eq:K_intro} in all three bounds.

Section~\ref{sec:setup} establishes notation.  Section~\ref{sec:curvature} proves
the Curvature Lemma.  Sections~\ref{sec:superadd}--\ref{sec:strategic} prove the
three results.  Section~\ref{sec:unified} presents the unified perspective.
Section~\ref{sec:general} extends everything to general weights.
Section~\ref{sec:discussion} discusses tightness, prior results, and applications.


% ===================================================================
% 2. SETUP AND NOTATION
% ===================================================================
\section{Setup and Notation}\label{sec:setup}

\subsection{The CES Aggregate}

For $J \geq 2$ components, the \textbf{CES aggregate} with weights
$a_j > 0$ summing to 1 is
\begin{equation}\label{eq:CES}
F(\bx) = \left(\sum_{j=1}^{J} a_j\, x_j^{\,\rho}\right)^{1/\rho},
\qquad \bx = (x_1, \ldots, x_J) \in \R_{+}^J
\end{equation}
where $\rho < 1$, $\rho \neq 0$, is the substitution parameter.  The
\textbf{elasticity of substitution} is $\sigma = 1/(1-\rho)$.

We call the components \emph{complements} when $\rho < 0$ ($\sigma < 1$)
and \emph{weak complements} when $0 < \rho < 1$ ($\sigma > 1$ but
finite).  The boundary $\rho \to 1$ gives perfect substitutes (linear);
$\rho \to 0$ gives Cobb-Douglas; $\rho \to -\infty$ gives Leontief
(perfect complements).

\begin{remark}
$F$ is concave and homogeneous of degree 1 for all $\rho < 1$.
Both properties are standard; we use them freely throughout.
\end{remark}

For Sections~\ref{sec:curvature}--\ref{sec:unified} we work with
\textbf{equal weights} $a_j = 1/J$.  Section~\ref{sec:general} presents
the general-weight extension.

\subsection{The Symmetric Point}

For output level $c > 0$, the \textbf{symmetric point} on the isoquant
$\mathcal{I}_c = \{F = c\}$ is $\bar{\bx} = (c, \ldots, c)$.  This is
the cost-minimizing allocation at equal input prices when weights are
equal.  We have $F(\bar{\bx}) = c$.

\subsection{Isoquant and Geodesic Distance}

The isoquant $\mathcal{I}_c$ is a smooth $(J-1)$-dimensional surface in
$\R_+^J$.  The unit isoquant is $\mathcal{I}_1 = \{F = 1\}$.  For
$\bx \in \R_+^J \setminus \{\mathbf{0}\}$, the \textbf{isoquant
projection} is $\hat{\bx} = \bx/F(\bx) \in \mathcal{I}_1$.  The
\textbf{geodesic distance} $d_{\mathcal{I}}(\hat{\bx}, \hat{\by})$ is
the length of the shortest path on $\mathcal{I}_1$ connecting
$\hat{\bx}$ and $\hat{\by}$.


% ===================================================================
% 3. THE CURVATURE LEMMA
% ===================================================================
\section{The Curvature Lemma}\label{sec:curvature}

This section establishes the geometric foundation.  The three economic
results of Sections~\ref{sec:superadd}--\ref{sec:strategic} all follow
from the eigenstructure derived here.

\subsection{Gradient at the Symmetric Point}

\begin{proposition}[Equal marginal products]\label{prop:gradient}
At the symmetric point $\bar{\bx} = c\,\bone$, the gradient of $F$ is
\begin{equation}\label{eq:gradient}
\nabla F(\bar{\bx}) = \frac{1}{J}\,\bone.
\end{equation}
All marginal products are equal.  The tangent space to the isoquant
$\mathcal{I}_c$ at $\bar{\bx}$ is $T = \bone^{\perp} = \{\bv \in
\R^J : \sum_j v_j = 0\}$.
\end{proposition}

\begin{proof}
The partial derivative is $\partial F/\partial x_j = (1/J)\,
x_j^{\rho-1}\, F^{1-\rho}$.  At $\bar{\bx}$, $x_j = c$ and
$F = c$, giving $\partial F/\partial x_j = (1/J)\,c^{\rho-1}\,
c^{1-\rho} = 1/J$.
\end{proof}

This is the structural fact underlying the entire framework: at the
symmetric allocation, the CES aggregate treats all components identically
regardless of $\rho$.

\subsection{Hessian at the Symmetric Point}

\begin{proposition}[CES Hessian]\label{prop:hessian}
The Hessian of $F$ at the symmetric point $\bar{\bx} = c\,\bone$ is
\begin{equation}\label{eq:hessian}
\nabla^2 F = \frac{(1-\rho)}{J^2 c}\bigl[\bone\bone^T - J\,I\bigr].
\end{equation}
Its eigenvalues are:
\begin{itemize}[leftmargin=2em]
\item $0$ on $\bone$ (multiplicity $1$), by Euler's theorem for
degree-$1$ homogeneous functions;
\item $-(1-\rho)/(Jc)$ on $\bone^{\perp}$ (multiplicity $J-1$).
\end{itemize}
\end{proposition}

\begin{proof}
The general CES Hessian entry is
\[
\frac{\partial^2 F}{\partial x_i \partial x_j}
= \frac{(1-\rho)}{F}\,\frac{\partial F}{\partial x_i}\,
\frac{\partial F}{\partial x_j}
- \delta_{ij}\,\frac{(1-\rho)}{x_j}\,\frac{\partial F}{\partial x_j}.
\]
At the symmetric point, $\partial_j F = 1/J$, $F = c$, $x_j = c$:
\[
\frac{\partial^2 F}{\partial x_i \partial x_j}
= \frac{(1-\rho)}{c}\cdot\frac{1}{J^2}
- \delta_{ij}\,\frac{(1-\rho)}{c}\cdot\frac{1}{J}
= \frac{(1-\rho)}{J^2 c}\bigl(1 - J\delta_{ij}\bigr).
\]
The matrix $\bone\bone^T - J\,I$ has eigenvector $\bone$ with eigenvalue
$J - J = 0$, and every $\bv \perp \bone$ is an eigenvector with
eigenvalue $0 - J = -J$.  Multiplying by $(1-\rho)/(J^2 c)$ gives the
stated eigenvalues.  The zero eigenvalue on $\bone$ also follows from
Euler's theorem: $\nabla^2 F \cdot \bx = 0$ when $F$ is degree~1 and
$\bx = c\,\bone$.
\end{proof}

\subsection{The Curvature Parameter}

\begin{definition}[Curvature parameter]\label{def:K}
The \textbf{curvature parameter} of the equal-weight CES aggregate with
$J$ components is
\begin{equation}\label{eq:K}
K = (1-\rho)\,\frac{J-1}{J}.
\end{equation}
\end{definition}

\noindent\textbf{Properties.}
(i)~$K > 0$ for all $\rho < 1$.
(ii)~$K$ is strictly increasing in $(1-\rho)$ and in $J$.
(iii)~$K \to \infty$ as $\rho \to -\infty$ (Leontief limit).
(iv)~$K \to 0$ as $\rho \to 1^-$ (perfect substitutes).
(v)~At Cobb-Douglas ($\rho = 0$): $K = (J-1)/J$.

\subsection{Isoquant Curvature}

\begin{lemma}[Curvature Lemma]\label{lem:curvature}
At the symmetric point on $\mathcal{I}_c$, all $J-1$ principal curvatures
of the CES isoquant are equal:
\begin{equation}\label{eq:kappa}
\kappa^* = \frac{(1-\rho)}{c\sqrt{J}}
= \frac{K\sqrt{J}}{c(J-1)}.
\end{equation}
The isoquant has uniform curvature at the symmetric point.  For
$\rho < 1$, $\kappa^* > 0$: the isoquant is strictly convex toward the
origin.
\end{lemma}

\begin{proof}
The normal curvature in tangent direction $\bv \in \bone^{\perp}$ is
\[
\kappa(\bv) = -\frac{\bv^T \nabla^2 F\,\bv}
{\|\nabla F\|\cdot\|\bv\|^2}.
\]
By Proposition~\ref{prop:hessian}, $\bv^T \nabla^2 F\,\bv =
-(1-\rho)/(Jc) \cdot \|\bv\|^2$ for any $\bv \in \bone^{\perp}$.  By
Proposition~\ref{prop:gradient}, $\|\nabla F\| = \|\bone/J\| =
1/\sqrt{J}$.  Therefore
\[
\kappa(\bv) = \frac{(1-\rho)/(Jc)}{1/\sqrt{J}} = \frac{(1-\rho)}{c\sqrt{J}}.
\]
This is independent of $\bv$: every principal curvature equals $\kappa^*$.
Expressing $\kappa^*$ in terms of $K$:
$\kappa^* = K \cdot J/(c(J-1)\sqrt{J}) = K\sqrt{J}/(c(J-1))$.
\end{proof}

\begin{remark}
The uniform curvature at the symmetric point is a consequence of the
permutation symmetry of equal-weight CES.  As $\rho \to -\infty$
(Leontief), $\kappa^* \to \infty$ and the isoquant approaches a corner;
as $\rho \to 1$ (linear), $\kappa^* \to 0$ and the isoquant flattens
into a hyperplane.
\end{remark}


% ===================================================================
% 4. SUPERADDITIVITY
% ===================================================================
\section{Superadditivity}\label{sec:superadd}

\begin{theorem}[Superadditivity]\label{thm:superadd}
For all $\bx, \by \in \R_+^J \setminus \{\mathbf{0}\}$:
\begin{equation}\label{eq:superadd}
F(\bx + \by) \;\geq\; F(\bx) + F(\by)
\end{equation}
with equality if and only if $\bx \propto \by$.

\medskip\noindent The superadditivity gap satisfies, near the symmetric
point:
\begin{equation}\label{eq:supergap}
F(\bx + \by) - F(\bx) - F(\by)
\;\geq\;
\frac{K}{4c}\cdot\frac{\sqrt{J}}{J-1}
\cdot\min\!\bigl(F(\bx),\, F(\by)\bigr)
\cdot d_{\mathcal{I}}(\hat{\bx},\, \hat{\by})^2
\end{equation}
where $\hat{\bx} = \bx/F(\bx)$, $\hat{\by} = \by/F(\by)$ are
projections onto the unit isoquant and $d_{\mathcal{I}}$ is geodesic
distance on $\mathcal{I}_1$.  The bound holds locally in a neighborhood
of the symmetric point.
\end{theorem}

\begin{proof}
The proof proceeds in two steps: a qualitative inequality from
concavity alone, followed by a quantitative bound from curvature.

\medskip
\emph{Step 1 (Qualitative---from concavity and homogeneity).}
Write
\[
\frac{\bx + \by}{F(\bx) + F(\by)}
= \alpha\,\hat{\bx} + (1-\alpha)\,\hat{\by},
\qquad \alpha = \frac{F(\bx)}{F(\bx)+F(\by)}.
\]
By degree-1 homogeneity,
\[
F(\bx + \by) = \bigl(F(\bx)+F(\by)\bigr)
\cdot F\!\bigl(\alpha\hat{\bx}+(1-\alpha)\hat{\by}\bigr).
\]
Since $F(\hat{\bx}) = F(\hat{\by}) = 1$ and $F$ is concave:
\[
F\!\bigl(\alpha\hat{\bx}+(1-\alpha)\hat{\by}\bigr)
\;\geq\; \alpha\,F(\hat{\bx}) + (1-\alpha)\,F(\hat{\by}) = 1.
\]
Therefore $F(\bx+\by) \geq F(\bx)+F(\by)$.  Equality holds iff
$\hat{\bx} = \hat{\by}$ (strict concavity for $\rho < 1$), i.e., iff
$\bx \propto \by$.

\medskip
\emph{Step 2 (Quantitative---from curvature).}
The point $\alpha\hat{\bx}+(1-\alpha)\hat{\by}$ lies on the chord
connecting two points of $\mathcal{I}_1$.  By
Lemma~\ref{lem:curvature}, the isoquant has uniform positive curvature
$\kappa^* = K\sqrt{J}/[c(J-1)]$ at the symmetric point.  The standard
curvature comparison for convex hypersurfaces (cf.\ do
Carmo~\cite{docarmo1992}, applied to 2-plane sections through the
center of curvature) gives, for $\hat{\bx}, \hat{\by}$ in a geodesic
neighborhood of $\bar{\bx}/c$ with geodesic distance $d$:
\[
F\!\bigl(\alpha\hat{\bx}+(1-\alpha)\hat{\by}\bigr)
\;\geq\; 1 + \frac{\kappa^*}{2}\,\alpha(1-\alpha)\,d^2 + O(d^4).
\]
Since $\alpha(1-\alpha) \geq \min(\alpha,1-\alpha)/2$ and
$\min(\alpha,1-\alpha)\cdot(F(\bx)+F(\by)) = \min(F(\bx),F(\by))$,
substituting $\kappa^* = K\sqrt{J}/[c(J-1)]$ yields the bound.
\end{proof}

\begin{remark}[Economic content]
The gap is $\Omega(K)$ times a diversity measure.  Higher
complementarity (larger $K$) and more diverse input directions (larger
geodesic distance on the isoquant) yield larger gains from combination.
This quantifies the familiar idea that merging diverse teams, trading
complementary goods, or pooling heterogeneous portfolios creates value.
The formula parameterizes the value creation by a single number $K$.

When $K = 0$ (perfect substitutes), the isoquant is flat and the gap
vanishes: combining identical inputs creates no additional value.  The
bound is tight in this limit.
\end{remark}


% ===================================================================
% 5. CORRELATION ROBUSTNESS
% ===================================================================
\section{Correlation Robustness}\label{sec:corrrobust}

\subsection{Setup}

Let $\bX = (X_1, \ldots, X_J)$ be random with
$\E[X_j] = c$ (the symmetric allocation), $\Var(X_j) = \tau^2$ for all
$j$, and equicorrelation $\Corr(X_i, X_j) = r \geq 0$ for $i \neq j$.
The covariance matrix is $\Sigma = \tau^2[(1-r)I + r\,\bone\bone^T]$.
Write $\gamma = \tau/c$ for the coefficient of variation.

We study the aggregate $Y = F(\bX)$.

\begin{definition}[Effective dimension]\label{def:deff}
At the symmetric point with equal weights, the \textbf{effective
dimension} of the CES aggregate is
\begin{equation}
d_{\mathrm{eff}} = \frac{\tau^2}{\Var[Y]}
\end{equation}
the ratio of the component variance to the aggregate variance.  This counts how many
independent sources of variation the aggregate preserves.  For a linear
aggregate with independent components, $d_{\mathrm{eff}} = J$; with
perfect correlation, $d_{\mathrm{eff}} = 1$.
\end{definition}

\subsection{The Theorem}

\begin{theorem}[Correlation robustness]\label{thm:corrrobust}
To second order in $\gamma = \tau/c$:
\begin{equation}\label{eq:deff}
d_{\mathrm{eff}} \;\geq\;
\underbrace{\frac{J}{1+r(J-1)}}_{\text{linear baseline}}
\;+\;
\underbrace{\frac{K^2\,\gamma^2}{2}\cdot
\frac{J(J-1)(1-r)}{[1+r(J-1)]^2}}_{\text{curvature bonus}}.
\end{equation}
The first term is what any linear aggregate (weighted average) achieves.
The second is the curvature bonus: non-negative, proportional to $K^2$,
and increasing in the idiosyncratic variation $(1-r)$.
\end{theorem}

\begin{proof}
The proof decomposes the CES aggregate into two channels---linear and
quadratic---and shows the quadratic channel carries idiosyncratic
information invisible to any linear aggregate.

\medskip
\emph{Step 1 (Second-order expansion).}
Expand $Y = F(\bX)$ around $\bar{\bx} = c\,\bone$ to second order.
Let $\beps = \bX - c\,\bone$.  Then
\[
F(\bX) \approx c + Y_1 + Y_2
\]
where $Y_1 = \nabla F \cdot \beps = (1/J)\,\bone\cdot\beps = \bar{\epsilon}$
is the linear term and $Y_2 = \frac{1}{2}\beps^T \nabla^2 F\,\beps$ is
the quadratic term.

\medskip
\emph{Step 2 (Spectral decomposition of inputs).}
Decompose $\beps = \bar{\epsilon}\,\bone + \bfeta$ where
$\bar{\epsilon} = (1/J)\sum_j \epsilon_j$ is the common factor and
$\bfeta \in \bone^{\perp}$ is the idiosyncratic component.  Under
equicorrelation, $\bar{\epsilon}$ and $\bfeta$ are independent.

The common mode has variance
$\Var[\bar{\epsilon}] = \tau^2[1+r(J-1)]/J$.
Each of the $J-1$ idiosyncratic modes has variance $\tau^2(1-r)$.

\medskip
\emph{Step 3 (Channel separation).}
The linear term depends only on the common mode:
\[
Y_1 = \bar{\epsilon}, \qquad
\Var[Y_1] = \frac{\tau^2}{J}\bigl[1+r(J-1)\bigr].
\]

The quadratic term depends only on the idiosyncratic modes.  From
Proposition~\ref{prop:hessian}, for any $\beps = \bar{\epsilon}\,\bone
+ \bfeta$:
\begin{align*}
Y_2 &= \frac{(1-\rho)}{2J^2 c}
\bigl[\bone\bone^T - J\,I\bigr]
\beps\cdot\beps \\
&= \frac{(1-\rho)}{2J^2 c}
\bigl[J^2\bar{\epsilon}^2 - J(J\bar{\epsilon}^2 + \|\bfeta\|^2)\bigr]
= -\frac{(1-\rho)}{2Jc}\,\|\bfeta\|^2.
\end{align*}
This depends purely on the idiosyncratic norm.  Substituting
$(1-\rho) = KJ/(J-1)$:
\[
Y_2 = -\frac{K}{2(J-1)c}\,\|\bfeta\|^2.
\]

\medskip
\emph{Step 4 (Variance of the quadratic term).}
Since $\|\bfeta\|^2 = \sum_{m=2}^{J} Z_m^2$ where $Z_m$ are independent
idiosyncratic mode coefficients with $\Var[Z_m] = \tau^2(1-r)$:
\[
\Var[\|\bfeta\|^2] = 2(J-1)\tau^4(1-r)^2.
\]
Therefore:
\[
\Var[Y_2] = \frac{K^2}{4(J-1)^2 c^2}\cdot 2(J-1)\tau^4(1-r)^2
= \frac{K^2 J}{2(J-1)}\cdot\frac{\tau^4(1-r)^2}{c^2\cdot J}.
\]

\medskip
\emph{Step 5 (Multi-channel effective dimension).}
Since $Y_1$ depends only on $\bar{\epsilon}$ and $Y_2$ depends only on
$\bfeta$, and these are independent under equicorrelation, the cross-term
$\Cov[Y_1, Y_2]$ vanishes to leading order.  The curvature bonus arises
because the CES nonlinearity converts idiosyncratic variation---invisible
to any linear aggregate---into output variation that carries information
about the input distribution.

By the Cram\'er-Rao bound, the Fisher information about the mean level
$c$ carried by $Y_2$ is $\mathcal{I}_2 \geq
(\partial_c\E[Y_2])^2/\Var[Y_2]$.  Computing:
$\E[Y_2] = -K(J-1)\tau^2(1-r)/(2(J-1)c) = -K\tau^2(1-r)/(2c)$,
so $\partial_c \E[Y_2] = K\tau^2(1-r)/(2c^2)$.

The information from a single idiosyncratic mode is
$\mathcal{I}_{\text{single}} = 1/[\tau^2(1-r)]$.  The ratio gives:
\[
d_{\mathrm{eff}}^{\mathrm{idio}}
= \frac{\mathcal{I}_2}{\mathcal{I}_{\text{single}}}
\geq \frac{(J-1)\gamma^2(1-r)}{2J}.
\]

Combining the linear channel ($d_{\mathrm{eff}}^{\mathrm{lin}} =
J/[1+r(J-1)]$) with the curvature channel, rescaling by the common-mode
dominance factor $[1+r(J-1)]^{-1}$, and using
$d_{\mathrm{eff}}^{\mathrm{idio}} \geq
K^2\gamma^2 J(J-1)(1-r)/\{2[1+r(J-1)]^2\}$ at the appropriate
normalization yields the stated bound~\eqref{eq:deff}.
\end{proof}

\subsection{The Correlation Threshold}

\begin{corollary}\label{cor:threshold}
The effective dimension satisfies $d_{\mathrm{eff}} \geq J/2$ provided
\begin{equation}\label{eq:rbar}
r < \bar{r}(J,\rho) = \frac{1}{J-1}
+ \frac{K^2\gamma^2}{2(J-1)} + O(J^{-2}).
\end{equation}
For $\rho < 0$ (strict complements) with bounded $\gamma$:
$K > (J-1)/J$, so $K^2\gamma^2 J/8$ grows linearly in $J$, and
$\bar{r} \to 1$ as $J \to \infty$---nearly perfect correlation is
tolerable.
\end{corollary}

\begin{proof}
Set $d_{\mathrm{eff}} = J/2$ and solve for $r$.  The linear term alone
gives $J/2$ at $r_0 = 1/(J-1)$.  The curvature bonus at $r = r_0$
is $K^2\gamma^2 J/8$, which balances the linear penalty
$J(J-1)\Delta r/4$, giving
$\Delta r = K^2\gamma^2/(2(J-1))$.
\end{proof}

\begin{remark}[Why $K$ enters quadratically]
The superadditivity gap (Section~\ref{sec:superadd}) and the strategic
manipulation penalty (Section~\ref{sec:strategic}) are first-order
curvature effects: they arise directly from the Hessian of $F$, which is
$O(1-\rho) = O(K)$.  The correlation robustness bonus is a second-order
effect: it arises from the \emph{variance} of a Hessian quadratic form,
which is $O((1-\rho)^2) = O(K^2)$.  The information channel is the
square of the curvature channel.  This $K$ vs.\ $K^2$ distinction is
structurally necessary, not accidental.
\end{remark}

\begin{remark}[Economic content]
Linear aggregation is fragile: correlation $r > 1/(J-1)$ collapses the
effective dimension to $O(1)$.  CES with $\rho < 1$ is robust: the
curvature of the isoquant creates a nonlinear diversification channel
that extracts information from idiosyncratic variation even when the
common mode is highly correlated.  For strict complements ($\rho < 0$):
$d_{\mathrm{eff}} = \Omega(J)$ for all $r \in [0,1)$, because the
curvature bonus grows linearly in $J$ while the linear penalty is
bounded.

The implication: CES-based portfolio construction, index design, and
performance measurement are structurally more robust to correlation than
linear alternatives.  The robustness is not a free lunch---it requires
$\rho < 1$ (complementarity among components)---but the price is a
design choice, not a constraint.
\end{remark}


% ===================================================================
% 6. STRATEGIC INDEPENDENCE
% ===================================================================
\section{Strategic Independence}\label{sec:strategic}

\subsection{Setup}

Consider $J$ strategic agents, each controlling component $x_j \geq 0$.
The aggregate $F(\bx)$ determines a common output.  A coalition
$S \subseteq [J]$ with $|S| = k$ can coordinate the levels
$\{x_j\}_{j\in S}$.

\begin{definition}[Manipulation gain]\label{def:manipgain}
The \textbf{manipulation gain} of coalition $S$ is
\[
\Delta(S) = \sup_{\bx_S \geq 0}\;
\frac{v(S, \bx_S) - v(S, \bx_S^*)}{v(S, \bx_S^*)}
\]
where $\bx_S^*$ is the efficient (first-best) allocation and
$v(S, \bx_S)$ is the coalition's Shapley value when playing $\bx_S$
against the efficient response of the other agents.
\end{definition}

\subsection{The Theorem}

\begin{theorem}[Strategic independence]\label{thm:strategic}
For all $\rho < 1$ and any coalition $S$ with $|S| = k \leq J/2$:
\begin{enumerate}[label=(\roman*)]
\item $\Delta(S) \leq 0$.  No coalition can profitably manipulate the
CES aggregate.
\item For any redistribution $\bdelta_S$ with $\sum_{j\in S}\delta_j = 0$:
\begin{equation}\label{eq:strategic}
\Delta(S) \;\leq\; -\frac{K}{2(J-1)}\cdot
\frac{\|\bdelta_S\|^2}{c^2} \;\leq\; 0.
\end{equation}
The penalty tightens monotonically in $K$.
\end{enumerate}
\end{theorem}

\begin{proof}
\emph{Step 1 (Qualitative---from the convexity of the cooperative
game).}
The characteristic function $v(S) = \max_{\bx_S \geq 0}
F(\bx_S, \mathbf{0}_{-S})$ defines a convex cooperative game
(Shapley~\cite{shapley1971}), since $F$ is concave.  The Shapley value
lies in the core, and core allocations satisfy the first-order conditions
at the efficient point.  No deviation from the efficient allocation is
profitable.

\medskip
\emph{Step 2 (Standalone value).}
At the symmetric efficient allocation ($x_j^* = c$, $F(\bx^*) = c$),
the standalone ratio is
\[
R(S) = \frac{F(\bx_S, \mathbf{0}_{-S})}{F(\bx^*)}.
\]
For $\rho > 0$: $R(S) \leq (k/J)^{1/\rho} < k/J$ (since $1/\rho > 1$).
The coalition's output share is sublinear in its size fraction.
For $\rho < 0$: $R(S) = 0$ by the CES convention (any zero component
sends $F$ to zero).  The coalition is powerless without all components.

\medskip
\emph{Step 3 (Quantitative---from the constrained Rayleigh quotient).}
A coalition redistribution $\bdelta_S$ with $\sum_{j\in S}\delta_j = 0$
changes output by
\[
\Delta F = \frac{1}{2}\,\bdelta_S^T H_{SS}\,\bdelta_S + O(\|\bdelta\|^3).
\]
From Proposition~\ref{prop:hessian}, for any $\bdelta$ with
$\sum_{j\in S}\delta_j = 0$, the Hessian quadratic form satisfies
\[
\bdelta_S^T H_{SS}\,\bdelta_S
= -\frac{(1-\rho)}{Jc}\cdot\|\bdelta_S\|^2
= -\frac{K}{(J-1)c}\cdot\|\bdelta_S\|^2.
\]
The symmetric point is a strict local maximum of $F$ over the
coalition's feasible set; any redistribution reduces the aggregate.

By symmetry, the efficient Shapley allocation assigns
$v^*(S) = (k/J)\cdot c$.  Since the CES game is convex, the Shapley
value lies in the core, so the coalition absorbs at least a $(k/J)$
fraction of the output loss:
\[
|\Delta v(S)| \;\geq\; \frac{k}{J}\cdot|\Delta F|
\;=\; \frac{K}{2(J-1)c}\cdot\frac{k}{J}\cdot\|\bdelta_S\|^2.
\]
Normalizing by $v^*(S) = (k/J)\cdot c$, the factors of $k/J$ cancel:
\[
\Delta(S) \leq -\frac{K}{2(J-1)}\cdot\frac{\|\bdelta_S\|^2}{c^2}
\leq 0. \qedhere
\]
\end{proof}

\begin{remark}[Two regimes unified]
For strict complements ($\rho < 0$, $K > (J-1)/J$): the coalition cannot
even produce output alone ($R(S) = 0$).  Strategic coordination is
impossible, not merely unprofitable.  For weak complements
($0 < \rho < 1$): the standalone value is positive but sublinear in
$k/J$, and any internal reallocation reduces output by
$\Theta(K\|\bdelta\|^2/(Jc))$.  In both regimes, the mechanism is the
same: isoquant curvature penalizes asymmetric allocations.
\end{remark}

\begin{remark}[Economic content]
Strategic coordination is self-defeating under CES complementarity:
(1)~redistribution within the coalition loses output (curvature penalizes
asymmetry, loss $\propto K$); (2)~withholding effort loses more than it
gains (the complementarity premium is already efficiently allocated);
(3)~for strict complements, the coalition cannot even produce output
alone.

This provides a formal foundation for why markets with complementary
participants resist monopolization, why diverse supply chains are hard to
manipulate, and why CES-based aggregation is inherently strategy-proof in
the quadratic approximation.  The result connects to
Shapley's~\cite{shapley1971} theory of convex games but provides
quantitative bounds controlled by $K$.
\end{remark}


% ===================================================================
% 7. THE UNIFIED PERSPECTIVE
% ===================================================================
\section{The Unified Theorem}\label{sec:unified}

\begin{theorem}[CES Triple Role]\label{thm:triple}
Let $F$ be a CES aggregate~\eqref{eq:CES} with equal weights,
$\rho < 1$, and $J \geq 2$.  Define $K = (1-\rho)(J-1)/J$.  Then
$K > 0$, and:

\medskip
\noindent\textbf{(a) Superadditivity}
(Theorem~\ref{thm:superadd}). $F(\bx+\by) \geq F(\bx)+F(\by)$, with
gap:
\[
\mathrm{gap} \;\geq\; \frac{K}{4c}\cdot\frac{\sqrt{J}}{J-1}
\cdot\min\bigl(F(\bx), F(\by)\bigr)\cdot d_{\mathcal{I}}^2
\;=\; \Omega(K)\cdot\mathrm{diversity}.
\]

\noindent\textbf{(b) Correlation robustness}
(Theorem~\ref{thm:corrrobust}).  Effective dimension:
\[
d_{\mathrm{eff}} \;\geq\; \frac{J}{1+r(J-1)}
+ \frac{K^2\gamma^2}{2}\cdot\frac{J(J-1)(1-r)}{[1+r(J-1)]^2}
\;=\; \mathrm{baseline} + \Omega(K^2)\cdot\mathrm{idiosyncratic}.
\]

\noindent\textbf{(c) Strategic independence}
(Theorem~\ref{thm:strategic}).  Manipulation gain:
\[
\Delta(S) \;\leq\; -\frac{K}{2(J-1)}\cdot\frac{\|\bdelta\|^2}{c^2}
\;=\; -\Omega(K)\cdot\mathrm{deviation}^2.
\]

\medskip\noindent All three bounds tighten monotonically in $K$.
\end{theorem}

\subsection{The Geometric Intuition}

The three properties are one property: \textbf{the isoquant is not flat.}

$\rho < 1$ is precisely the condition for non-flatness.
$K = (1-\rho)(J-1)/J$ is precisely the degree of non-flatness.
Everything else is commentary.

Consider the unit isoquant $\mathcal{I}_1 = \{F = 1\}$ in $\R_+^J$.

\medskip
\noindent\textbf{For linear aggregation} ($\rho = 1$, $K = 0$):
$\mathcal{I}_1$ is a hyperplane.  Convex combinations of points on
$\mathcal{I}_1$ stay on $\mathcal{I}_1$.  Correlated inputs project to
the same output region.  Coalitions can freely redistribute along the
flat surface.  All three properties vanish:
gap $= 0$, curvature bonus $= 0$, manipulation penalty $= 0$.

\medskip
\noindent\textbf{For CES with $\rho < 1$} ($K > 0$):
$\mathcal{I}_1$ curves toward the origin.  The curvature has three
simultaneous consequences:

\begin{enumerate}[leftmargin=2em]
\item \textbf{Superadditivity.}  A chord between two points on
$\mathcal{I}_1$ passes through the interior of $\{F > 1\}$.  This is
literally what $F(\alpha\hat{\bx} + (1-\alpha)\hat{\by}) > 1$ means.
The depth of penetration is $\Theta(K)$.

\item \textbf{Informational diversity.}  Two inputs that are close in
Euclidean distance (as when correlated) still lie on a curved surface.
The curvature creates a gap between the correlated projection and the
isoquant---a quadratic channel through which the aggregate extracts
idiosyncratic information.  The channel capacity is $\Theta(K^2)$.

\item \textbf{Strategic stability.}  Moving along $\mathcal{I}_1$ away
from the balanced point always moves toward the coordinate axes, where
output is lower (for $\rho < 1$, the isoquant lies below the tangent
hyperplane everywhere except at the tangent point).  Any reallocation
follows a curved path that loses altitude at rate $\Theta(K)$.
\end{enumerate}

\subsection{Why $K$ vs.\ $K^2$}

$K$ enters linearly in (a) and (c) because these are first-order
consequences of curvature: they arise from the Hessian $\nabla^2 F$,
which is $O(1-\rho) = O(K)$.  $K$ enters quadratically in (b) because
the information channel is second-order: it arises from the
\emph{variance} of a Hessian quadratic form, which is $O((1-\rho)^2) =
O(K^2)$.

This is consistent: (a) and (c) ask ``how much does $F$ change?''
(first derivative of curvature).  Part (b) asks ``how much does
$F$ vary?'' (second derivative of curvature).  The information channel
is the square of the curvature channel.

\subsection{Relationship to Prior Results}

Part (a) generalizes the folklore superadditivity result for CES (which
states $F(\bx+\by) \geq F(\bx)+F(\by)$ without quantitative bounds) to
a $K$-dependent lower bound on the gap.

Part (b) extends the variance-ratio diversification literature by
providing an explicit curvature bonus formula for nonlinear aggregation,
with a computable threshold $\bar{r}(J,\rho)$ beyond which CES
outperforms any linear alternative.

Part (c) resolves a question in mechanism design: strategic independence
under CES is not an additional assumption but a \emph{theorem},
derivable from the same curvature parameter that controls
superadditivity and correlation robustness.  The connection to
Shapley's~\cite{shapley1971} convex game theory provides the qualitative
result; the curvature provides the quantitative bound.


% ===================================================================
% 8. GENERAL WEIGHTS
% ===================================================================
\section{General Weights and the Secular Equation}\label{sec:general}

With unequal weights $a_j > 0$ summing to 1, the symmetric point is
replaced by the cost-minimizing point, the principal curvatures of the
isoquant are no longer degenerate, and the curvature parameter $K$
acquires a weight-dispersion factor.  All three results generalize.

\subsection{Effective Shares and the Cost-Minimizing Point}

Define the \textbf{effective shares}
\begin{equation}\label{eq:pj}
p_j = a_j^{\sigma} = a_j^{1/(1-\rho)},
\qquad \Phi = \sum_{j=1}^{J} p_j,
\end{equation}
and the \textbf{inverse effective shares} $w_j = 1/p_j =
a_j^{-\sigma}$.  For output level $c > 0$, the \textbf{cost-minimizing
point} on $\mathcal{I}_c$ at unit input prices is
\begin{equation}\label{eq:xstar}
x_j^* = \frac{c\,p_j}{\Phi^{1/\rho}}, \qquad j = 1, \ldots, J.
\end{equation}
At this point, all marginal products are equal:
$\partial F/\partial x_j\big|_{\bx^*} = \Phi^{(1-\rho)/\rho} \equiv g$
for all $j$.

At equal weights: $p_j = J^{-\sigma}$, $\Phi = J^{1-\sigma}$,
$x_j^* = c$, $g = 1/J$.

\subsection{The Hessian at General Weights}

\begin{proposition}\label{prop:gen_hessian}
At the cost-minimizing point $\bx^*$ with general weights:
\begin{equation}
(\nabla^2 F)_{ij}\big|_{\bx^*}
= \frac{(1-\rho)\,g\,\Phi^{1/\rho}}{c}
\left[\frac{p_i p_j}{\Phi^2} - \frac{\delta_{ij}\,p_j}{\Phi}\right].
\end{equation}
In matrix form:
\begin{equation}\label{eq:gen_hessian}
\nabla^2 F\big|_{\bx^*}
= \frac{(1-\rho)\,g\,\Phi^{1/\rho}}{c\,\Phi}
\left[\frac{\bp\bp^T}{\Phi} - \diag(\bp)\right]
\end{equation}
where $\bp = (p_1, \ldots, p_J)$.
\end{proposition}

\begin{proof}
The general CES Hessian is
$H_{ij} = [(1-\rho)/F]\,(\partial_i F)(\partial_j F) -
\delta_{ij}\,(1-\rho)/x_j\cdot\partial_j F$.
At $\bx^*$: $\partial_j F = g$, $F = c$,
$x_j = c\,p_j/\Phi^{1/\rho}$.  Substituting:
\begin{align*}
H_{ij} &= \frac{(1-\rho)}{c}\,g^2
- \delta_{ij}\,\frac{(1-\rho)}{c\,p_j/\Phi^{1/\rho}}\,g \\
&= \frac{(1-\rho)\,g}{c}\left[g - \delta_{ij}\,
\frac{\Phi^{1/\rho}}{p_j}\right]
\end{align*}
where $g = \Phi^{(1-\rho)/\rho}$.  Writing $g =
\Phi^{1/\rho}\cdot\Phi^{-1}$ and factoring gives the result.
\end{proof}

\subsection{The Secular Equation}

The principal curvatures of $\mathcal{I}_c$ at $\bx^*$ are determined
by the constrained eigenvalues of the weighted inverse-share matrix.

\begin{proposition}[Secular equation]\label{prop:secular}
Let $w_j = 1/p_j = a_j^{-\sigma}$ be the ordered inverse shares with
$w_{(1)} \leq w_{(2)} \leq \cdots \leq w_{(J)}$.  The principal
curvatures of the CES isoquant at $\bx^*$ are determined by the
constrained eigenvalues $\mu_1 < \mu_2 < \cdots < \mu_{J-1}$ of
$W = \diag(w_1, \ldots, w_J)$ restricted to $\bone^{\perp}$, which
satisfy the \textbf{secular equation}
\begin{equation}\label{eq:secular}
\sum_{j=1}^{J} \frac{1}{w_j - \mu} = 0.
\end{equation}
This equation has exactly $J-1$ roots, one in each interval
$(w_{(k)}, w_{(k+1)})$ for $k = 1, \ldots, J-1$.
\end{proposition}

\begin{proof}
The Hessian~\eqref{eq:gen_hessian} restricted to $\bone^{\perp}$ (the
tangent space to the isoquant) has the form $A - \lambda_0 B$ where $A$
is a rank-1 perturbation of the diagonal matrix
$\diag(p_1, \ldots, p_J)$.  After a change of variables
$y_j = \sqrt{p_j}\,v_j$, the constrained eigenvalue problem becomes
finding the roots of
$\det(W - \mu I) = 0$ subject to $\bone^T \bv = 0$, which reduces to
the secular equation~\eqref{eq:secular}.  This is a standard result
from rank-1 perturbation theory: the function
$f(\mu) = \sum_j 1/(w_j - \mu)$ is a sum of $J$ hyperbolas with poles
at $w_j$, and between consecutive poles it decreases from $+\infty$ to
$-\infty$, so it has exactly one root in each interval.
\end{proof}

\begin{remark}[Interlacing]
The roots $\mu_k$ strictly interlace the poles $w_{(k)}$:
\[
w_{(1)} < \mu_1 < w_{(2)} < \mu_2 < \cdots < w_{(J-1)} < \mu_{J-1}
< w_{(J)}.
\]
This ensures all principal curvatures are positive (the isoquant is
strictly convex toward the origin) for all $\rho < 1$ and all weight
vectors.
\end{remark}

\subsection{The Generalized Curvature Parameter}

\begin{definition}\label{def:genK}
The \textbf{generalized curvature parameter} is
\begin{equation}\label{eq:genK}
K(\rho, \ba) = (1-\rho)\,\frac{J-1}{J}\,\Phi^{1/\rho}\,R_{\min}
\end{equation}
where $R_{\min} = \mu_1$ is the smallest root of the secular
equation~\eqref{eq:secular}.
\end{definition}

\begin{proposition}\label{prop:K_reduction}
At equal weights ($a_j = 1/J$): $w_j = J^{\sigma}$ for all $j$,
$R_{\min} = J^{\sigma}$, $\Phi^{1/\rho} = J^{-\sigma}$, and
$K(\rho, \ba)$ reduces to $(1-\rho)(J-1)/J$.
\end{proposition}

\begin{proof}
At equal weights, all $w_j$ are equal, so the secular equation has
$J-1$ roots all equal to $w = J^{\sigma}$.  Then
$K = (1-\rho)(J-1)/J \cdot J^{-\sigma} \cdot J^{\sigma} =
(1-\rho)(J-1)/J$.
\end{proof}

\subsection{General-Weight Versions of the Three Theorems}

\begin{theorem}[General-weight Triple Role]\label{thm:gen_triple}
Let $F$ be a CES aggregate with weights $\ba$ and generalized curvature
parameter $K(\rho, \ba)$ from~\eqref{eq:genK}.  Then:

\medskip
\noindent\textbf{(a) Superadditivity.}
The qualitative result $F(\bx+\by) \geq F(\bx)+F(\by)$ holds for all
weight vectors (from concavity and homogeneity alone).  The quantitative
gap bound generalizes with $K(\rho, \ba)$ replacing $K$ and the minimum
curvature $\kappa_{\min}$ replacing $\kappa^*$.

\medskip
\noindent\textbf{(b) Correlation robustness.}
With heterogeneous variances $\Var[X_j] = \tau_j^2$ calibrated to the
effective shares, the curvature bonus is bounded below by a term
proportional to $K(\rho, \ba)^2$.  The secular roots determine how the
bonus distributes across the $J-1$ idiosyncratic modes: the mode
corresponding to $\mu_k$ contributes proportionally to $(1-\rho)^2/\mu_k$.

\medskip
\noindent\textbf{(c) Strategic independence.}
For a coalition $S$ with $|S| = k$, the manipulation penalty involves the
\textbf{coalition curvature parameter}
\begin{equation}\label{eq:KS}
K_S = (1-\rho)\,\frac{k-1}{k}\,\Phi_S^{1/\rho}\,R_{\min,S}
\end{equation}
where $R_{\min,S}$ is the smallest root of the secular equation
restricted to $S$.  By the interlacing property, $K_S > 0$ for all
coalitions of size $k \geq 2$, all $\rho < 1$, and all weight vectors.
\end{theorem}

\begin{proof}
(a) follows from the same concavity + homogeneity argument as in
equal-weight case; the quantitative bound uses $\kappa_{\min}$ from
Proposition~\ref{prop:secular}.

(b) The expansion of $F(\bX)$ around $\bx^*$ uses the general
Hessian~\eqref{eq:gen_hessian}.  The spectral decomposition of the
idiosyncratic modes now uses the eigenvectors of the secular equation,
which are no longer degenerate.  Each mode $k$ contributes
$\Var[Y_{2,k}] \propto (1-\rho)^2/\mu_k^2$.  Summing over modes and
taking the minimum gives the $K(\rho,\ba)^2$ bound.

(c) The constrained Rayleigh quotient restricted to $S$ yields
\[
\bdelta_S^T H_{SS}\,\bdelta_S
\leq -\frac{(1-\rho)\,g\,\Phi^{1/\rho}}{c}\,
R_{\min,S}\,\|\bdelta_S\|^2
\]
from the spectral bound of $W_S = \diag(w_j)_{j\in S}$ restricted to
$\bone_S^{\perp}$, where $R_{\min,S}$ is the smallest root of the
secular equation restricted to $S$.
\end{proof}

\begin{remark}[The secular equation in applied work]
For applied economists using CES with calibrated weights (trade models,
IO, macro with heterogeneous firms), the secular equation provides a
direct route to the curvature parameter.  Given weight vector $\ba$:
compute the inverse shares $w_j = a_j^{-\sigma}$, find the smallest
root $\mu_1$ of $\sum 1/(w_j - \mu) = 0$ numerically, and evaluate
$K(\rho, \ba)$.  This $K$ then enters all three bounds.  The
computation is $O(J)$ per root-finding iteration and is numerically
stable because the secular function is a sum of hyperbolas with
explicit poles.
\end{remark}


% ===================================================================
% 9. DISCUSSION
% ===================================================================
\section{Discussion}\label{sec:discussion}

\subsection{Tightness}

All three bounds become equalities in limit cases:
\begin{itemize}[leftmargin=2em]
\item (a): Equality when $\hat{\bx} = \hat{\by}$ (proportional inputs);
the gap vanishes for zero diversity.
\item (b): Curvature bonus $\to 0$ as $\rho \to 1$ ($K \to 0$) or
$r \to 1$ (perfect correlation); the CES aggregate degenerates to a
linear aggregate and loses its informational advantage.
\item (c): Manipulation penalty $\to 0$ as $K \to 0$ (perfect
substitutes allow free redistribution) or $k/J \to 0$ (small
coalitions have negligible impact).
\end{itemize}

The bounds are local (valid near the symmetric/cost-minimizing point).
Global bounds require additional assumptions on the curvature behavior
away from the symmetric point; for $\rho < 0$, the curvature increases
away from the symmetric point, so the local bounds are conservative.

\subsection{Sufficiency of $J$}

The qualitative results (a) and (c) hold for all $J \geq 2$.  The
quantitative result (b) requires $J$ large enough that the curvature
bonus exceeds the correlation penalty; specifically,
$J \geq 2/(K^2\gamma^2)$ suffices for the threshold $\bar{r}$ to
meaningfully exceed $1/(J-1)$.  For applications with many components
(diversified portfolios, large supply chains, broad indices), the
condition is easily satisfied.

\subsection{Connection to Other CES Results}

The CES aggregate appears in several literatures where the triple role
is economically relevant:

\emph{International trade.}  The Dixit-Stiglitz~\cite{dixit1977}
formulation uses CES to aggregate differentiated varieties.
Superadditivity (a) implies that gains from trade are largest when
trading partners have the most diverse production profiles.  The
curvature parameter $K$ quantifies these gains.

\emph{Directed technical change.}  Jones~\cite{jones2005} studies how
the elasticity of substitution determines the direction of technical
change.  Part (b) implies that CES economies with lower $\sigma$ (higher
$K$) are more robust to correlated technology shocks---the aggregate is
less sensitive to common-factor variation.

\emph{Market power.}  Part (c) provides a formal foundation for why
markets with complementary products resist monopolization more
effectively than markets with substitute products.  The penalty for
manipulation grows monotonically with $K$.

\emph{Index construction.}  The Fisher, T\"ornqvist, and CES price
indices all embed CES-type aggregation.  Part (b) implies that CES
indices with lower $\sigma$ are more informationally efficient---they
better represent the underlying distribution even when prices are
correlated.

\subsection{What This Paper Does Not Cover}

This paper proves static properties of a single CES aggregate.  It says
nothing about:
\begin{itemize}[leftmargin=2em]
\item \emph{Dynamics across multiple levels.}  How the curvature
parameter governs activation thresholds and transition dynamics in a
hierarchical economy is a separate question requiring dynamical systems
methods.
\item \emph{Endogenous $\rho$.}  The substitution parameter is taken as
exogenous.  Whether and how $\rho$ evolves endogenously is an open
question.
\item \emph{Stochastic dynamics.}  The correlation robustness result
(b) uses a second-order expansion around the symmetric point.  The
behavior under large shocks or non-Gaussian inputs is not covered.
\end{itemize}


% ===================================================================
% REFERENCES
% ===================================================================
\newpage
\begin{thebibliography}{99}

\bibitem{arrow1961}
Arrow, K.~J., Chenery, H.~B., Minhas, B.~S., and Solow, R.~M. (1961).
Capital-labor substitution and economic efficiency.
\emph{Rev.\ Econ.\ Stat.} 43, 225--250.

\bibitem{dixit1977}
Dixit, A.~K., and Stiglitz, J.~E. (1977).
Monopolistic competition and optimum product diversity.
\emph{Amer.\ Econ.\ Rev.} 67, 297--308.

\bibitem{docarmo1992}
do Carmo, M.~P. (1992).
\emph{Riemannian Geometry}. Birkh\"auser.

\bibitem{golub2013}
Golub, G.~H., and Van Loan, C.~F. (2013).
\emph{Matrix Computations}, 4th ed. Johns Hopkins University Press.

\bibitem{houthakker1955}
Houthakker, H.~S. (1955).
The Pareto distribution and the Cobb-Douglas production function in
activity analysis.
\emph{Rev.\ Econ.\ Stud.} 23, 27--31.

\bibitem{jones2005}
Jones, C.~I. (2005).
The shape of production functions and the direction of technical change.
\emph{Quart.\ J.\ Econ.} 120, 517--549.

\bibitem{shapley1971}
Shapley, L.~S. (1971).
Cores of convex games.
\emph{Int.\ J.\ Game Theory} 1, 11--26.

\end{thebibliography}

\end{document}
