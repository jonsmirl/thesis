% CAVEAT: Generated by Google Gemini (Feb 2026). NOT VERIFIED.
% This is reference material only — do not incorporate without independent verification.
% Uses extreme value theory for Leontief limit, implicit function theorem for monotonicity.
% Key differences from current Free_Energy_Economics.tex proof:
%   - Shows Var[min_j u_j] = T^2 pi^2/6, independent of J (extreme value theory)
%   - Proves dT*/drho > 0 via iso-error curve implicit differentiation
%   - More explicit Lipschitz constants at rho->1 and rho->-inf limits
%
\begin{proof}
The proof establishes the monotonicity of $T^*(\rho)$ in three steps: a Lipschitz analysis of the CES aggregate for $\rho$ near~1, a sensitivity analysis for $\rho \to -\infty$, and a monotonicity argument connecting the two regimes.

Throughout, we use the setup from \Cref{prop:arrow}: $J$ voters with utilities $u_j(A) \in [\underline{u}, \overline{u}]$, noisy CES welfare $W_T(A) = (J^{-1}\sum_j \tilde{u}_j(A)^\rho)^{1/\rho}$, and the democratic error functional $E(T)$ whose minimizer is $T^*$.

\medskip
\emph{Step 1: For $\rho$ near $1$, the CES aggregate is Lipschitz in individual utilities with constant $O(1/J)$, so noise washes out at rate $\sqrt{J}$.}

For $\rho \in (0,1]$, the $\rho$-power mean $S(\mathbf{x}) = (J^{-1}\sum_j x_j^\rho)^{1/\rho}$ has partial derivatives:
\[
    \frac{\partial S}{\partial x_j} \;=\; \frac{1}{J} \cdot \frac{x_j^{\rho-1}}{(J^{-1}\sum_k x_k^\rho)^{(\rho-1)/\rho}} \;=\; \frac{1}{J}\left(\frac{x_j}{S}\right)^{\rho-1}.
\]
At symmetric equilibrium ($x_j = c$ for all $j$), this equals $1/J$ for any $\rho$.  Away from symmetry, for $\rho$ near 1:
\[
    \left|\frac{\partial S}{\partial x_j}\right| \;=\; \frac{1}{J}\left(\frac{x_j}{S}\right)^{\rho-1} \;\leq\; \frac{1}{J}\left(\frac{\overline{u}}{\underline{u}}\right)^{|\rho-1|}.
\]
For $\rho \to 1$, the exponent $|\rho - 1| \to 0$, so $(\overline{u}/\underline{u})^{|\rho-1|} \to 1$ and:
\begin{equation}\label{eq:lipschitz_rho1}
    \left\|\nabla_{\mathbf{x}} S\right\|_\infty \;\leq\; \frac{1}{J}\left(\frac{\overline{u}}{\underline{u}}\right)^{1-\rho} \;\xrightarrow{\rho \to 1}\; \frac{1}{J}.
\end{equation}

The CES aggregate is thus $O(1/J)$-Lipschitz in each coordinate for $\rho$ near 1.  By McDiarmid's inequality applied to $W_T(A) = S(\tilde{u}_1(A), \ldots, \tilde{u}_J(A))$, where each $\tilde{u}_j(A)$ is perturbed independently with range at most $2(\overline{u} + TL)$ on the truncation event $\mathcal{E}_L$:
\begin{equation}\label{eq:concentration_rho1}
    \Pr\bigl[|W_T(A) - \mathbb{E}[W_T(A)]| > t\bigr] \;\leq\; 2\exp\!\left(-\frac{t^2 J}{2(\overline{u}+TL)^2 (\overline{u}/\underline{u})^{2(1-\rho)}}\right) + \delta'.
\end{equation}
The exponent scales as $t^2 J / T^2$ (since the range grows with $T$): the noise averages out at rate $T/\sqrt{J}$, so for the ranking error to remain below a threshold, the maximum tolerable temperature scales as:
\begin{equation}\label{eq:Tstar_rho1}
    T^*(\rho) \;\sim\; C_1(\rho) \cdot \Delta_\varepsilon \cdot \frac{\sqrt{J}}{(\overline{u}/\underline{u})^{1-\rho}} \qquad \text{as } \rho \to 1,
\end{equation}
where $C_1(\rho)$ is a slowly varying function and $\Delta_\varepsilon$ is the minimum welfare gap.  As $\rho \to 1$, the factor $(\overline{u}/\underline{u})^{1-\rho} \to 1$, so $T^*$ grows as $O(\sqrt{J})$.

\medskip
\emph{Step 2: For $\rho \to -\infty$, the CES aggregate has sensitivity $O(1)$ to the minimum utility, so noise persists regardless of $J$.}

As $\rho \to -\infty$, the CES aggregate converges to the minimum: $S(\mathbf{x}) \to \min_j x_j$.  The minimum function has partial derivative:
\[
    \frac{\partial (\min_j x_j)}{\partial x_k} \;=\; \begin{cases} 1 & \text{if } k = \argmin_j x_j, \\ 0 & \text{otherwise.} \end{cases}
\]
This is \emph{not} $O(1/J)$---it is $O(1)$, concentrated entirely on the voter with the lowest utility.  The noise in that single voter's report propagates to the aggregate without attenuation.

More precisely, for finite $\rho < 0$ with $|\rho|$ large, the CES partial derivative satisfies:
\[
    \frac{\partial S}{\partial x_{\min}} \;=\; \frac{1}{J}\left(\frac{x_{\min}}{S}\right)^{\rho-1}.
\]
Since $S \to x_{\min}$ as $\rho \to -\infty$, the ratio $x_{\min}/S \to 1$ and the partial derivative $\to 1/J$.  However, the bounded-difference constant in McDiarmid's inequality depends on the range of the function as one coordinate varies.  Writing $x_{\min} = u_{\min} + T\eta$ and using the CES aggregate's response:
\[
    \sup_{\eta \in [-L,L]} S(x_1, \ldots, u_{\min}+T\eta, \ldots, x_J) - \inf_{\eta \in [-L,L]} S(x_1, \ldots, u_{\min}+T\eta, \ldots, x_J).
\]
For $\rho \to -\infty$ (Leontief limit), this range is $\min(2TL, \, x_{\min,2} - (u_{\min} - TL))$ where $x_{\min,2}$ is the second-smallest value.  When the gap $x_{\min,2} - x_{\min}$ is small (of order $O(M/J)$ by order statistics for uniformly distributed utilities), the bounded-difference constant for the minimizing coordinate is $O(TL)$---the same order as for the arithmetic mean.  But the bounded-difference constants for all \emph{other} coordinates are 0 (they do not affect the minimum).  Hence $\sum_j c_j^2 \approx (TL)^2 / J^2$ (only one coordinate contributes), and the McDiarmid exponent is $O(t^2 J^2 / T^2)$ for the minimum-affecting coordinate but the \emph{effective} concentration involves only that single coordinate.

The key distinction is sharper when stated in terms of the variance of the CES aggregate.  For $\tilde{u}_j = u_j + T\eta_j$ with i.i.d.\ noise:
\begin{align}
    \Var[W_T(A)] &\;=\; \Var\!\left[\left(\frac{1}{J}\sum_j \tilde{u}_j^\rho\right)^{1/\rho}\right]. \label{eq:var_WT}
\end{align}
By the delta method applied to $h(S) = S^{1/\rho}$ at $S = \mathbb{E}[J^{-1}\sum_j \tilde{u}_j^\rho]$:
\[
    \Var[W_T] \;\approx\; (h'(\bar{S}))^2 \cdot \Var\!\left[\frac{1}{J}\sum_j \tilde{u}_j^\rho\right] \;=\; \frac{\bar{S}^{2(1/\rho - 1)}}{\rho^2} \cdot \frac{1}{J^2}\sum_j \Var[\tilde{u}_j^\rho].
\]
For $\rho$ near 1: $\bar{S}^{2(1/\rho-1)} \to 1$, $\rho^2 \to 1$, and $\Var[\tilde{u}_j^\rho] \approx T^2 \Var[\eta_j]$ (since $x^\rho \approx x$ for $\rho \approx 1$).  Hence $\Var[W_T] \approx T^2 \pi^2/(6J)$: variance decays as $1/J$.

For $\rho \to -\infty$: the CES aggregate converges to $\min_j \tilde{u}_j$.  By extreme value theory, $\min_{j=1}^J (u_j + T\eta_j) = u_{(1)} + T(\eta_{(1)} + \log J) - T\log J$ where $\eta_{(1)} = \min_j \eta_j \stackrel{d}{=} -\log J + \eta'$ with $\eta'$ standard Gumbel.  Therefore:
\begin{equation}\label{eq:var_rho_neg}
    \Var[\min_j \tilde{u}_j] \;=\; T^2 \cdot \Var[\eta'] \;=\; \frac{\pi^2}{6}\, T^2,
\end{equation}
which does \emph{not} decay with $J$.  The variance of the Leontief aggregate is $\Theta(T^2)$ regardless of the electorate size.  The critical temperature for the ranking to remain accurate is therefore:
\begin{equation}\label{eq:Tstar_rho_neg}
    T^*(\rho) \;\sim\; C_2 \cdot \Delta_\varepsilon \qquad \text{as } \rho \to -\infty,
\end{equation}
for a constant $C_2$ independent of $J$.  Comparing with~\eqref{eq:Tstar_rho1}: for any $J \geq 2$, $T^*(\rho \to 1) / T^*(\rho \to -\infty) = O(\sqrt{J}) \to \infty$.

\medskip
\emph{Step 3: Monotonicity of $T^*(\rho)$ for general $\rho \in (-\infty, 1)$.}

We show $dT^*/d\rho > 0$ by proving that the ranking error $\delta(T, \varepsilon; \rho)$ is decreasing in $\rho$ for fixed $T$, which implies that the level set $\{\delta = \delta_0\}$ moves rightward (to higher $T$) as $\rho$ increases.

Define the \emph{maximum influence} of a single voter:
\[
    \mathcal{I}(\rho) \;=\; \max_{j} \sup_{\eta, \eta'} \frac{|S(\ldots, u_j+T\eta, \ldots) - S(\ldots, u_j+T\eta', \ldots)|}{T|\eta - \eta'|} \;=\; \frac{1}{J}\left(\frac{x_{\max}}{x_{\min}}\right)^{|1-\rho|},
\]
where $x_{\max}/x_{\min} = r > 1$ is the ratio of extreme utilities.  This is the bounded-difference constant (divided by $T/J$) that enters McDiarmid's inequality.

For $\rho_1 < \rho_2 \leq 1$:
\[
    \frac{\mathcal{I}(\rho_1)}{\mathcal{I}(\rho_2)} \;=\; r^{(1-\rho_1) - (1-\rho_2)} \;=\; r^{\rho_2 - \rho_1} \;>\; 1 \qquad \text{since } r > 1 \text{ and } \rho_2 > \rho_1.
\]
Therefore $\mathcal{I}(\rho)$ is strictly decreasing in $\rho$: lower $\rho$ means higher sensitivity to individual noise.

The McDiarmid concentration exponent~\eqref{eq:mcdiarmid} is inversely proportional to $\mathcal{I}(\rho)^2$: the exponent equals $t^2 / (J \cdot (\mathcal{I}(\rho) T / J)^2) = t^2 J / (\mathcal{I}(\rho)^2 T^2)$.  Since $\mathcal{I}$ is decreasing in $\rho$, the exponent is \emph{increasing} in $\rho$: higher $\rho$ yields a tighter concentration bound.  Therefore, for fixed $T$ and $\varepsilon$, the ranking error $\delta(T, \varepsilon; \rho)$ is decreasing in $\rho$.

Now, $T^*(\rho)$ is defined (implicitly) by the condition $E(T^*) = \min_T E(T)$, where the dominant term for moderate $T$ is $\delta(T, \varepsilon; \rho)$.  Since $\delta$ is decreasing in $\rho$ at each fixed $T$, the ``iso-error'' curve $\delta(T, \varepsilon; \rho) = \delta_0$ satisfies:
\[
    \frac{dT^*}{d\rho}\bigg|_{\delta = \delta_0} \;=\; -\frac{\partial \delta / \partial \rho}{\partial \delta / \partial T} \;>\; 0,
\]
since $\partial\delta/\partial\rho < 0$ (higher $\rho$ reduces error) and $\partial\delta/\partial T > 0$ (higher $T$ increases error).  The IIA term $\alpha(T)$ and the impossibility proximity term $\iota(T)$ do not depend on $\rho$ (they depend on $T$ and $J$ only), so the monotonicity of $T^*(\rho)$ follows from the monotonicity of the concentration term alone.

\medskip
\emph{Summary.}  Combining Steps 1--3:
\begin{itemize}
    \item For $\rho \to 1$ (utilitarian/majoritarian): $T^*(\rho) = O(\Delta_\varepsilon \sqrt{J})$.  Noise averages over $J$ voters, each contributing $O(1/J)$ influence.
    \item For $\rho \to -\infty$ (Rawlsian/consensus): $T^*(\rho) = O(\Delta_\varepsilon)$, independent of $J$.  The aggregate is determined by the single worst-off voter.
    \item For intermediate $\rho$: $T^*$ interpolates monotonically, with $dT^*/d\rho > 0$ throughout.
\end{itemize}
High-$\rho$ (majoritarian) systems tolerate more noise than low-$\rho$ (consensus) systems, as claimed.
\end{proof}
