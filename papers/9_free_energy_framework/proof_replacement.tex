% CAVEAT: Generated by Google Gemini (Feb 2026). NOT VERIFIED.
% This is reference material only — do not incorporate without independent verification.
% Uses Gumbel noise (vs Logistic in current proof). Claims explicit Hoeffding/McDiarmid bounds.
% Key differences from current Free_Energy_Economics.tex proof:
%   - Gumbel noise connects to multinomial logit directly
%   - Separates cardinal (exact IIA) vs ordinal (approximate IIA) channels
%   - Adds bias correction step (rho(rho-1) term)
%   - Introduces "impossibility proximity cost" iota(T) in T* optimization
%
\begin{proof}
The proof proceeds in five steps: explicit construction of~$W_T$, verification of Pareto efficiency, a precise bound on the IIA violation, concentration of the noisy ranking around the true CES ranking, and existence of the optimal temperature~$T^*$.

\medskip
\emph{Step 1: Construction of $W_T$.}
Fix $T > 0$ and $\rho \in (0,1)$.  There are $J \geq 2$ voters and a finite set $\mathcal{A}$ of alternatives with $|\mathcal{A}| = N \geq 3$.  Each voter $j$ has a cardinal utility function $u_j : \mathcal{A} \to [\underline{u}, \overline{u}]$ with $0 < \underline{u} < \overline{u} < \infty$.  Write $M = \overline{u} - \underline{u}$ for the utility range.

Define the \emph{noisy reported utility} of voter $j$ for alternative $A$:
\[
    \tilde{u}_j(A) \;=\; u_j(A) \;+\; T \cdot \eta_j(A),
\]
where $\{\eta_j(A)\}_{j \in \{1,\ldots,J\},\, A \in \mathcal{A}}$ are i.i.d.\ standard Gumbel random variables.  By the classical result of \citet{mckelvey1995}, this is equivalent to each voter drawing a ranking from the multinomial logit model with choice probabilities $P_j(A) = e^{u_j(A)/T}\big/\sum_{B} e^{u_j(B)/T}$; the Gumbel shock $\eta_j(A)$ is the latent utility perturbation consistent with that draw.

Define the \emph{noisy CES social welfare}:
\begin{equation}\label{eq:WT_def}
    W_T(A) \;=\; \left(\frac{1}{J}\sum_{j=1}^{J} \tilde{u}_j(A)^\rho\right)^{1/\rho}.
\end{equation}
The social ranking is: $A \succsim_T B$ iff $W_T(A) \geq W_T(B)$.  Equal weights $1/J$ ensure non-dictatorship: no single voter determines the aggregate.

(For $\rho < 0$, restrict the domain so that $\tilde{u}_j(A) > 0$ a.s.\ by choosing $\underline{u}$ sufficiently large relative to $T$; the argument carries through with modified constants.  For $\rho \in (0,1)$ as assumed, $\tilde{u}_j(A) > 0$ a.s.\ when $\underline{u} > 0$, since the Gumbel left tail decays as $\exp(-e^{-x})$ and $\Pr[\eta < -\underline{u}/T] = \exp(-e^{\underline{u}/T}) \to 0$ exponentially.)

\medskip
\emph{Step 2: Pareto efficiency holds exactly.}
Suppose all voters unanimously prefer $A$ to $B$: $u_j(A) > u_j(B)$ for all $j$.  Let $\Delta_{\min} = \min_{j} [u_j(A) - u_j(B)] > 0$.

Write $D_j = \tilde{u}_j(A) - \tilde{u}_j(B) = [u_j(A) - u_j(B)] + T[\eta_j(A) - \eta_j(B)]$.  The difference $\eta_j(A) - \eta_j(B)$ of two independent standard Gumbels follows the standard logistic distribution with mean~$0$ and variance~$\pi^2/3$.  Hence $\mathbb{E}[D_j] = u_j(A) - u_j(B) \geq \Delta_{\min} > 0$.

Define $S_T(A) = J^{-1}\sum_j \tilde{u}_j(A)^\rho$ and similarly for $B$.  The social ranking $W_T(A) > W_T(B)$ holds iff $S_T(A) > S_T(B)$ (since $x \mapsto x^{1/\rho}$ is strictly increasing for $\rho > 0$).  For each $j$, define $V_j = \tilde{u}_j(A)^\rho - \tilde{u}_j(B)^\rho$.  Since $\tilde{u}_j(A)$ first-order stochastically dominates $\tilde{u}_j(B)$ (the distributions are identical-shape Gumbel translates, with the $A$-translate shifted rightward by $u_j(A) - u_j(B) > 0$) and $x^\rho$ is strictly increasing for $x > 0$, we have $\mathbb{E}[V_j] > 0$ for each $j$.  Write $\bar{\mu}_V = J^{-1}\sum_j \mathbb{E}[V_j] > 0$.

To obtain a finite-sample bound, we truncate.  Fix $\delta' > 0$ and set $L = \underline{u}/(2T) + \log(4JN/\delta')$.  On the event $\mathcal{E}_L = \{\max_{j,A}|\eta_j(A)| \leq L\}$, which satisfies $\Pr[\mathcal{E}_L^c] \leq 2JN \exp(-e^{L/2}) \leq \delta'$, we have $\tilde{u}_j(A) \in [\underline{u} - TL,\, \overline{u} + TL]$ and each $|V_j|$ is bounded by $R_L = 2(\overline{u} + TL)^\rho$.  Conditioned on $\mathcal{E}_L$, the $\{V_j\}$ are independent with $|V_j| \leq R_L$, so by Hoeffding's inequality:
\begin{equation}\label{eq:pareto_bound}
    \Pr\!\left[S_T(A) \leq S_T(B) \;\Big|\; \mathcal{E}_L\right] \;=\; \Pr\!\left[\frac{1}{J}\sum_j V_j \leq 0 \;\Big|\; \mathcal{E}_L\right] \;\leq\; \exp\!\left(-\frac{J \bar{\mu}_V^2}{2 R_L^2}\right).
\end{equation}
Therefore:
\[
    \Pr[W_T(A) \leq W_T(B)] \;\leq\; \exp\!\left(-\frac{J \bar{\mu}_V^2}{2 R_L^2}\right) + \delta'.
\]
For fixed $T > 0$ and $J \to \infty$, this vanishes exponentially.  For fixed $J$ and $\Delta_{\min}/T \to \infty$, we have $\bar{\mu}_V \to J^{-1}\sum_j[u_j(A)^\rho - u_j(B)^\rho] > 0$ while $R_L$ remains bounded, so the bound again vanishes.  In either regime, Pareto efficiency holds with probability approaching~1.

For fixed finite $J$ and $T$, the Pareto violation probability is exponentially small in $J$ but strictly positive.  This is the honest cost of operating at $T > 0$: deterministic Pareto compliance is a $T = 0$ property.  At $T > 0$, we have probabilistic Pareto compliance with failure probability controlled by~\eqref{eq:pareto_bound}.

\medskip
\emph{Step 3: Approximate IIA with violation bounded by $O(T)$.}

\textbf{Definition} ($\alpha$-approximate IIA).  The social welfare function $W_T$ satisfies \emph{$\alpha$-approximate IIA} if for any $A, B \in \mathcal{A}$ and any two utility profiles $\mathbf{u}, \mathbf{u}'$ agreeing on $\{A,B\}$ (i.e., $u_j(A) = u_j'(A)$ and $u_j(B) = u_j'(B)$ for all $j$) but possibly differing on alternatives $C \notin \{A,B\}$:
\[
    \bigl|\Pr_{\mathbf{u}}[A \succsim_T B] - \Pr_{\mathbf{u}'}[A \succsim_T B]\bigr| \;\leq\; \alpha.
\]

\textbf{Direct channel (cardinal reports).}  Observe that $W_T(A)$ in~\eqref{eq:WT_def} depends only on $\{\tilde{u}_j(A)\}_{j=1}^J$, and $\tilde{u}_j(A) = u_j(A) + T\eta_j(A)$ is a function of $u_j(A)$ and the noise $\eta_j(A)$ alone---\emph{not} of $u_j(C)$ for any $C \neq A$.  The noise terms $\eta_j(A)$ are independent across alternatives.  Therefore, if voters report noisy cardinal utilities, $W_T$ satisfies \emph{exact} IIA: the ranking of $A$ vs.\ $B$ depends only on utilities for $A$ and $B$.

\textbf{Indirect channel (ordinal reports via multinomial logit).}  If voters instead report ordinal rankings drawn from the full multinomial logit over $\mathcal{A}$, then the probability that voter $j$ ranks $A$ above $B$ is:
\[
    P_j(A \succ B \mid \mathcal{A}) \;=\; \frac{e^{u_j(A)/T}}{\sum_{A' \in \mathcal{A}} e^{u_j(A')/T}},
\]
which depends on $u_j(C)$ for all $C \in \mathcal{A}$ through the denominator.  The corresponding binary comparison is $P_j(A \succ B \mid \{A,B\}) = e^{u_j(A)/T}/(e^{u_j(A)/T} + e^{u_j(B)/T})$.  The per-voter IIA deviation is:
\begin{align}
    \bigl|P_j(A \succ B \mid \mathcal{A}) &- P_j(A \succ B \mid \{A,B\})\bigr| \notag \\
    &= P_j(A \succ B \mid \{A,B\}) \cdot \left(1 - \frac{e^{u_j(A)/T} + e^{u_j(B)/T}}{\sum_{A'} e^{u_j(A')/T}}\right) \notag \\
    &\leq \frac{(N-2)\, e^{\overline{u}/T}}{2\, e^{\underline{u}/T} + (N-2)\, e^{\overline{u}/T}}. \label{eq:iia_per_voter}
\end{align}
For small $T$ (i.e., $T \ll M$), the bound~\eqref{eq:iia_per_voter} is exponentially small: $O(e^{-M/T})$.  For large $T$ (i.e., $T \gg M$), expand $e^{u/T} = 1 + u/T + O(u^2/T^2)$ to get the bound tending to $(N-2)/N$, which is the uniform-choice baseline---all alternatives are treated symmetrically, so IIA holds trivially in the sense that no alternative is ``relevant'' or ``irrelevant.''

For the aggregate social ranking, the IIA violation probability---that including irrelevant alternatives flips the social ordering of $A$ vs.\ $B$---requires the aggregate shift to exceed the CES gap.  Each voter's effective contribution to $S_T(A) - S_T(B)$ shifts by at most $\Delta_C = \rho\, T\, \overline{u}^{\rho-1} \cdot \log(N/2) / J$ per irrelevant alternative (by the mean value theorem applied to $x^\rho$ composed with the logistic shift).  By Hoeffding's inequality applied to the $J$ independent shifts:
\begin{equation}\label{eq:iia_bound}
    \alpha(T) \;\leq\; 2\exp\!\left(-\frac{J\, (\Delta_{AB}^\rho)^2}{2\, (N-2)^2\, \rho^2\, T^2\, \overline{u}^{2(\rho-1)}\, \log^2(N/2)}\right)
\end{equation}
where $\Delta_{AB}^\rho = |S^*(A) - S^*(B)| = |J^{-1}\sum_j [u_j(A)^\rho - u_j(B)^\rho]|$ is the true CES gap in $\rho$-power space.  For fixed $\Delta_{AB}^\rho > 0$, this is exponentially small in $J$ and increasing in $T$ (the exponent's denominator grows with $T^2$), confirming that the IIA violation is $O(T)$ in the sense that $\alpha(T)$ increases with $T$ and is exponentially small for moderate~$T$.

\medskip
\emph{Step 4: Concentration of the noisy CES ranking around the true ranking.}

Let $W^*(A) = (J^{-1}\sum_j u_j(A)^\rho)^{1/\rho}$ be the true (noiseless) CES welfare, and $S^*(A) = J^{-1}\sum_j u_j(A)^\rho$ the corresponding $\rho$-power mean.  We bound $\delta(T,\varepsilon) = \Pr[\exists\, A,B:\; W^*(A) - W^*(B) > \varepsilon \text{ but } W_T(A) \leq W_T(B)]$.

\textbf{Bias bound.}  For each voter $j$ and alternative $A$, define $g_{j,A}(\eta) = (u_j(A) + T\eta)^\rho$.  A second-order Taylor expansion around $\eta = 0$:
\[
    \mathbb{E}[g_{j,A}(\eta)] = u_j(A)^\rho + \frac{\rho(\rho-1)}{2}\, u_j(A)^{\rho-2}\, T^2 \,\mathbb{E}[\eta^2] + O(T^3\, \underline{u}^{\,\rho-3}),
\]
where $\mathbb{E}[\eta^2] = \pi^2/6 + \gamma_{\textsc{em}}^2$ for the standard Gumbel ($\gamma_{\textsc{em}} \approx 0.5772$ is the Euler--Mascheroni constant).  Since $\rho \in (0,1)$, the coefficient $\rho(\rho-1) < 0$, so the bias $b_{j,A} := \mathbb{E}[g_{j,A}] - u_j(A)^\rho$ satisfies:
\[
    |b_{j,A}| \;\leq\; \frac{(1-\rho)\rho}{2}\, \underline{u}^{\,\rho-2}\, T^2 \, (\pi^2/6 + \gamma_{\textsc{em}}^2) \;=:\; B_\rho\, T^2.
\]
The aggregate bias: $|\mathbb{E}[S_T(A)] - S^*(A)| \leq B_\rho T^2$.

\textbf{Concentration via McDiarmid's inequality.}  On the truncation event $\mathcal{E}_L$ (from Step~2), each $g_{j,A}$ is a bounded function of $\eta_j(A)$ alone, and changing $\eta_j(A)$ by $\Delta\eta$ changes $g_{j,A}$ by at most $|\rho\, T\, (\underline{u} - TL)^{\rho-1}\, \Delta\eta|$ (mean value theorem, since $|g_{j,A}'(\eta)| = \rho T (u_j(A)+T\eta)^{\rho-1} \leq \rho T (\underline{u}-TL)^{\rho-1}$ on $\mathcal{E}_L$).  The range of $\eta_j(A)$ on $\mathcal{E}_L$ is $2L$, so the bounded-difference constant for the $j$-th summand of $S_T(A) = J^{-1}\sum_j g_{j,A}$ is $c_j = 2\rho TL(\underline{u}-TL)^{\rho-1}/J$.

McDiarmid's inequality applied to $S_T(A)$ as a function of the $J$ independent variables $\{\eta_j(A)\}$:
\begin{equation}\label{eq:mcdiarmid}
    \Pr\!\left[\bigl|S_T(A) - \mathbb{E}[S_T(A)]\bigr| > t \;\Big|\; \mathcal{E}_L\right] \;\leq\; 2\exp\!\left(-\frac{2\,t^2}{\sum_j c_j^2}\right) = 2\exp\!\left(-\frac{t^2\, J}{2\,\rho^2\, T^2\, L^2\, (\underline{u}-TL)^{2(\rho-1)}}\right).
\end{equation}

\textbf{Ranking agreement.}  For two alternatives with true CES gap $\Delta^\rho = S^*(A) - S^*(B) > 0$ (corresponding to $W^*(A) - W^*(B) > \varepsilon$), the noisy ranking agrees provided:
\[
    |S_T(A) - \mathbb{E}[S_T(A)]| + |S_T(B) - \mathbb{E}[S_T(B)]| + 2B_\rho T^2 \;<\; \Delta^\rho.
\]
Set $t = (\Delta^\rho - 2B_\rho T^2)/4$ (valid when $T^2 < \Delta^\rho/(2B_\rho)$).  Applying~\eqref{eq:mcdiarmid} to both $S_T(A)$ and $S_T(B)$, then taking a union bound over all $\binom{N}{2}$ pairs:
\begin{equation}\label{eq:delta_bound}
    \delta(T, \varepsilon) \;\leq\; N^2 \exp\!\left(-\frac{(\Delta_\varepsilon^\rho - 2B_\rho T^2)^2\, J}{32\,\rho^2\, T^2\, L^2\, (\underline{u}-TL)^{2(\rho-1)}}\right) + \delta',
\end{equation}
where $\Delta_\varepsilon^\rho = \min\{S^*(A) - S^*(B) : W^*(A) - W^*(B) > \varepsilon\} > 0$ is the minimum CES gap (in $\rho$-power) among pairs separated by at least~$\varepsilon$ in welfare.

\textbf{Limiting behavior.}
\begin{itemize}
    \item \emph{$T \to 0^+$ with $J$ fixed:} The numerator in the exponent of~\eqref{eq:delta_bound} tends to $(\Delta_\varepsilon^\rho)^2 > 0$ while the denominator tends to $0$ (since $T^2 L^2 \to 0$ for fixed $L = O(\log(1/\delta'))$).  Hence the exponent diverges to $+\infty$ and $\delta \to 0$: the noisy ranking converges to the true CES ranking.  However, this limit recovers Arrow's impossibility: as $T \to 0$, the logit noise vanishes, preferences become deterministic, and Arrow's theorem requires that any rule satisfying exact IIA and Pareto be dictatorial.  The equal-weight CES rule $W_T$ remains non-dictatorial but violates exact IIA.  Thus $T = 0$ is the phase transition.
    \item \emph{$T \to \infty$:} The bias $B_\rho T^2$ eventually exceeds $\Delta_\varepsilon^\rho / 2$, making the bound~\eqref{eq:delta_bound} vacuous.  The variance of $S_T(A)$ grows as $O(T^2/J)$, overwhelming the signal.  Hence $\delta \to 1$: the ranking becomes effectively random.
\end{itemize}

\medskip
\emph{Step 5: Existence of optimal $T^*$.}

Define the \emph{democratic error functional}:
\[
    E(T) \;=\; \delta(T, \varepsilon) \;+\; \alpha(T) \;+\; \iota(T),
\]
where $\delta$ is the ranking-error bound~\eqref{eq:delta_bound}, $\alpha$ is the IIA violation~\eqref{eq:iia_bound}, and $\iota(T) = \max\bigl(0,\; 1 - T/T_{\textsc{arr}}\bigr)$ is the \emph{impossibility proximity cost}: for $T < T_{\textsc{arr}}$ (a threshold below which the noise is insufficient to escape the Arrow constraint), the system is effectively dictatorial.  Specifically, set $T_{\textsc{arr}} = M / (2\log J)$, so that for $T < T_{\textsc{arr}}$ the logit places probability $> 1 - 1/J$ on each voter's true top choice, making one voter's preference dominate with high probability.

All three terms are continuous in $T$ on $(0, \infty)$.  Their combined behavior:
\begin{itemize}
    \item $E(T) \to 1$ as $T \to 0^+$: $\iota(T) \to 1$ (impossibility dominates), while $\delta, \alpha \to 0$.
    \item $E(T) \to 1$ as $T \to \infty$: $\delta(T,\varepsilon) \to 1$ (ranking randomizes), while $\iota(T) = 0$.
    \item At the intermediate temperature $T_0 = \Delta_\varepsilon^\rho / (4 B_\rho \sqrt{J})$: the bias term $2 B_\rho T_0^2 = (\Delta_\varepsilon^\rho)^2 / (8 B_\rho J)$ is negligible relative to $\Delta_\varepsilon^\rho$ for large $J$; the McDiarmid exponent in~\eqref{eq:delta_bound} is of order $J$; the IIA violation~\eqref{eq:iia_bound} is exponentially small; and $\iota(T_0) = 0$ for $J$ sufficiently large.  Therefore $E(T_0) < 1$.
\end{itemize}

Since $E$ is continuous on $(0,\infty)$, satisfies $\liminf_{T \to 0^+} E(T) \geq 1$ and $\liminf_{T \to \infty} E(T) \geq 1$, and achieves $E(T_0) < 1$ at an interior point, the extreme value theorem on any compact interval $[T_0/2,\, 2T_0] \subset (0,\infty)$ containing $T_0$ guarantees a minimizer:
\[
    T^* \;=\; \argmin_{T > 0}\; E(T) \;\in\; (0, \infty). \qedhere
\]
\end{proof}
