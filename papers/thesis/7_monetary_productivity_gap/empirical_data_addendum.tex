\textbf{EMPIRICAL DATA GATHERING ADDENDUM}

\emph{A Practical Guide for Implementing the Empirical Strategy}

Companion to: "Taxing Concentration, Not Transfer"

1. OVERVIEW

This addendum provides step-by-step instructions for gathering the empirical data needed to implement the estimation strategy in the main paper. The goal is to move from calibrated projections to actual empirical estimates.

The empirical strategy requires data for three purposes:

1. Structural estimation of behavioral parameters (bequest elasticity, planning effectiveness, dispersion preference)

2. Cross-country comparison (UK vs. US inheritance tax systems)

3. Within-US analysis (state inheritance tax variation)

\textbf{Estimated Timeline:}

• Public data assembly: 2-4 weeks

• Restricted data applications: 3-6 months for approval

• Data cleaning and preparation: 4-8 weeks

• Estimation and analysis: 8-12 weeks

• Total: 6-12 months for full implementation

2. US DATA SOURCES

2.1 IRS Statistics of Income (SOI) --- Estate Tax Data

\textbf{What it contains:}

Estate tax return data including gross estate value, deductions, taxable estate, tax liability, asset composition, and beneficiary information. Available at varying levels of detail.

\textbf{Public-Use Data (Free, Immediate Access):}

URL: https://www.irs.gov/statistics/soi-tax-stats-estate-tax-statistics

Contains: Aggregate statistics by estate size, year, state. Published tables going back to 1916.

Limitations: No microdata, no beneficiary-level detail.

\textbf{SOI Public-Use File (PUF):}

URL: https://www.irs.gov/statistics/soi-tax-stats-estate-tax-study-public-use-file

Contains: Individual estate-level data with identifying information removed. Available for selected years.

Variables: Gross estate, net estate, deductions by type, tax before/after credits, marital status, state.

Access: Free download after registration.

\textbf{Restricted Microdata (Research Access Required):}

Application: Submit proposal to IRS Statistics of Income Division

URL: https://www.irs.gov/statistics/soi-tax-stats-statistics-of-income-research-program

Contains: Full microdata including beneficiary information, trust details, specific asset values.

Requirements: Must be affiliated with research institution, proposal review takes 3-6 months.

Alternative: Access through Federal Statistical Research Data Centers (FSRDCs).

\textbf{Key Variables to Extract:}

\begin{longtable}[]{@{}
  >{\raggedright\arraybackslash}p{(\columnwidth - 4\tabcolsep) * \real{0.3333}}
  >{\raggedright\arraybackslash}p{(\columnwidth - 4\tabcolsep) * \real{0.3333}}
  >{\raggedright\arraybackslash}p{(\columnwidth - 4\tabcolsep) * \real{0.3333}}@{}}
\toprule\noalign{}
\endhead
\bottomrule\noalign{}
\endlastfoot
\textbf{Variable} & \textbf{SOI Field} & \textbf{Use in Analysis} \\
Gross estate & GRSS\_EST & Estate size distribution \\
Net estate tax & NET\_EST\_TX & Effective tax rate \\
Charitable deduction & CHAR\_DED & Charitable response \\
Marital deduction & MARITAL\_DED & Spousal transfers \\
State & STATE & State-level analysis \\
Year of death & YR\_DEATH & Time series analysis \\
\end{longtable}

2.2 Survey of Consumer Finances (SCF)

\textbf{What it contains:}

Triennial survey of household wealth, income, and financial behavior. Includes questions on inheritances received and expected.

\textbf{Access:}

URL: https://www.federalreserve.gov/econres/scfindex.htm

Public-use data freely downloadable in Stata, SAS, and ASCII formats.

\textbf{Key Variables:}

• X5804-X5809: Inheritances received (amount, year, from whom)

• X5821-X5826: Expected inheritances

• NETWORTH: Total household net worth

• INCOME: Total household income

• Various demographic and financial variables

\textbf{Sample Code (Stata):}

use scf2022.dta, clear

* Identify households that received inheritance

gen received\_inheritance = (x5804 \textgreater{} 0)

* Amount of largest inheritance

gen inheritance\_amount = x5804

* Tabulate by wealth decile

xtile wealth\_decile = networth, nq(10)

tab wealth\_decile received\_inheritance

2.3 Panel Study of Income Dynamics (PSID)

\textbf{What it contains:}

Longitudinal survey following families since 1968. Wealth supplements (1984, 1989, 1994, 1999, 2001, 2003, and biennially since) include inheritance questions.

\textbf{Access:}

URL: https://psidonline.isr.umich.edu/

Free registration required. Custom data extracts available through online tool.

\textbf{Advantage over SCF:}

Panel structure allows tracking wealth before and after inheritance receipt---essential for estimating causal effects on labor supply and consumption.

\textbf{Key Variables:}

• ER71485 (2019): Received inheritance/gift since last interview

• ER71486: Amount of inheritance

• ER71487: Year received

• ER71488: Relationship to person who left inheritance

2.4 Forbes 400 Data

\textbf{What it contains:}

Annual list of 400 wealthiest Americans with wealth estimates, source of wealth, and biographical information.

\textbf{Access:}

Historical data: https://www.forbes.com/forbes-400/

Academic compilations: Kaplan and Rauh (2013) provide cleaned panel data.

Peterson Institute database: https://www.piie.com/publications/working-papers/origins-superrich-billionaire-characteristics-database

\textbf{Use in Analysis:}

Track persistence of dynastic wealth at the very top. Identify inherited vs. self-made fortunes. Calculate implied bequest patterns.

3. STATE-LEVEL INHERITANCE TAX DATA

3.1 Current State Inheritance Taxes

Six states currently levy inheritance taxes with rates varying by beneficiary class:

\begin{longtable}[]{@{}
  >{\raggedright\arraybackslash}p{(\columnwidth - 8\tabcolsep) * \real{0.2000}}
  >{\raggedright\arraybackslash}p{(\columnwidth - 8\tabcolsep) * \real{0.2000}}
  >{\raggedright\arraybackslash}p{(\columnwidth - 8\tabcolsep) * \real{0.2000}}
  >{\raggedright\arraybackslash}p{(\columnwidth - 8\tabcolsep) * \real{0.2000}}
  >{\raggedright\arraybackslash}p{(\columnwidth - 8\tabcolsep) * \real{0.2000}}@{}}
\toprule\noalign{}
\endhead
\bottomrule\noalign{}
\endlastfoot
\textbf{State} & \textbf{Spouse Rate} & \textbf{Child Rate} & \textbf{Sibling Rate} & \textbf{Other Rate} \\
Iowa & Exempt & Exempt & 5-10\% & 10-15\% \\
Kentucky & Exempt & Exempt & 4-16\% & 6-16\% \\
Maryland & Exempt & Exempt & 10\% & 10\% \\
Nebraska & Exempt & 1\% & 11\% & 15\% \\
New Jersey & Exempt & Exempt & 11-16\% & 15-16\% \\
Pennsylvania & Exempt & 4.5\% & 12\% & 15\% \\
\end{longtable}

3.2 Historical State Data

Before EGTRRA (2001) phased out the federal credit for state death taxes, most states had inheritance or estate taxes. Compiling historical tax parameters is essential for the diff-in-diff analysis.

\textbf{Primary Source:}

Commerce Clearing House (CCH) State Tax Handbook (annual editions)

Available in law libraries and through Wolters Kluwer subscription.

\textbf{Secondary Sources:}

• McIntyre, Michael J. "State Death Taxes After EGRTRA: A Long Day\textquotesingle s Journey into Night." State Tax Notes (2002).

• Conway, Karen Smith, and Jonathan Rork. "State \textquotesingle Death\textquotesingle{} Taxes and Elderly Migration." National Tax Journal (2004).

• NCSL State Tax Actions Database: https://www.ncsl.org/research/fiscal-policy/state-tax-actions-database.aspx

\textbf{Data to Compile:}

For each state-year (1990-2025):

• Type of tax (inheritance, estate, both, neither)

• Tax rates by beneficiary class

• Exemption amounts by beneficiary class

• Special provisions (family business exemptions, etc.)

3.3 State Revenue Data

\textbf{Census Bureau State Government Finances:}

URL: https://www.census.gov/programs-surveys/state/data/tables.html

Contains: Annual state tax revenue by type, including "death and gift taxes."

Years: 1992-present in consistent format.

\textbf{Individual State Revenue Departments:}

Many states publish detailed revenue statistics. Contact information:

• Iowa: tax.iowa.gov

• Kentucky: revenue.ky.gov

• Maryland: marylandtaxes.gov

• Nebraska: revenue.nebraska.gov

• New Jersey: nj.gov/treasury

• Pennsylvania: revenue.pa.gov

4. UK DATA FOR CROSS-COUNTRY ANALYSIS

4.1 HMRC Inheritance Tax Statistics

\textbf{What it contains:}

Annual statistics on inheritance tax including estates chargeable, tax paid, reliefs claimed, and distributions by estate size.

\textbf{Access:}

URL: https://www.gov.uk/government/statistics/inheritance-tax-statistics

Free download of published tables and commentary.

\textbf{Key Tables:}

• Table 12.1: Estates notified and tax due by estate size

• Table 12.2: Estates by type of asset

• Table 12.3: Reliefs and exemptions claimed

• Table 12.4: Time series of IHT statistics

4.2 Wealth and Assets Survey (WAS)

\textbf{What it contains:}

Biennial survey of household wealth in Great Britain, analogous to US SCF. Includes inheritance questions.

\textbf{Access:}

URL: https://www.ons.gov.uk/peoplepopulationandcommunity/personalandhouseholdfinances/debt/methodologies/wealthandassetssurveyqmi

Public-use data through UK Data Service (registration required).

UK Data Service: https://ukdataservice.ac.uk/

4.3 Understanding Society (UKHLS)

\textbf{What it contains:}

Longitudinal household survey continuing the British Household Panel Survey. Includes wealth modules with inheritance data.

\textbf{Access:}

URL: https://www.understandingsociety.ac.uk/

Free registration required. Special license for sensitive variables.

\textbf{Key Variables:}

• w\_inherit: Whether received inheritance

• w\_inheramt: Amount inherited

• w\_inherwho: Relationship to deceased

5. INTERNATIONAL COMPARISON DATA

5.1 OECD Data

\textbf{Inheritance/Estate/Gift Tax Revenue:}

URL: https://stats.oecd.org/ → Tax → Revenue Statistics

Variable: Tax on property → Recurrent taxes on net wealth → Estate, inheritance and gift taxes (4300)

\textbf{Wealth Distribution:}

URL: https://stats.oecd.org/ → Wealth Distribution Database

Contains: Gini coefficients, wealth shares by percentile, for 25+ countries.

5.2 Luxembourg Wealth Study (LWS)

\textbf{What it contains:}

Harmonized microdata on household wealth from 15+ countries. Enables cross-country comparison with consistent definitions.

\textbf{Access:}

URL: https://www.lisdatacenter.org/our-data/lws-database/

Free for academic researchers. Remote execution system (no direct data download).

5.3 World Inequality Database

\textbf{What it contains:}

Long-run data on income and wealth inequality for most countries. Includes top wealth shares from estate tax data where available.

\textbf{Access:}

URL: https://wid.world/

Free download of all data and methodology documentation.

6. STEP-BY-STEP ANALYSIS GUIDE

6.1 Structural Estimation: Bequest Elasticity

\textbf{Step 1: Compile estate tax policy variation}

• Create state-year panel of estate/inheritance tax parameters (1980-2025)

• Note federal exemption changes: 1997 (\$600K), 2001 (\$675K), 2002-2009 (phased to \$3.5M), 2010 (repealed), 2011-2012 (\$5M), 2013-2017 (\$5.25M indexed), 2018+ (\$11M+)

\textbf{Step 2: Merge with estate tax return data}

• Use IRS SOI public-use files for state-year aggregates

• Restricted access provides individual-level data for finer analysis

\textbf{Step 3: Estimate elasticity}

Specification (state-year level):

ln(estates\_st) = β·ln(1 - τ\_st) + X\_st\textquotesingle γ + μ\_s + δ\_t + ε\_st

where τ\_st is the combined federal-state marginal tax rate.

The coefficient β estimates the elasticity of reported estates with respect to the net-of-tax rate.

\textbf{Step 4: Robustness checks}

• Bunching analysis around exemption thresholds

• Event studies around major tax changes (EGTRRA 2001, TCJA 2017)

• Placebo tests using non-taxable estates

6.2 Cross-Country Analysis: UK vs. US

\textbf{Step 1: Construct comparable samples}

• US: SCF households with wealth \textgreater{} \$1M

• UK: WAS households with wealth \textgreater{} £800K (PPP-adjusted)

• Match on demographics, income, wealth level

\textbf{Step 2: Estimate outcome differences}

y\_it = α + β·UK\_i + X\_it\textquotesingle γ + δ\_t + ε\_it

Outcomes: number of beneficiaries, concentration of bequests, charitable giving, effective tax rate.

\textbf{Step 3: Within-country time series}

• Examine changes around UK IHT reforms (2006 transferable nil-rate band, 2017 residence nil-rate band)

• Compare to US changes in same period

6.3 State-Level Diff-in-Diff

\textbf{Step 1: Define treatment and control}

Treatment: States that repealed inheritance tax after 2001

Control: States that retained inheritance tax (IA, KY, MD, NE, NJ, PA)

\textbf{Step 2: Construct outcome variables}

• Bequest concentration (share to top beneficiary)

• Number of beneficiaries per estate

• Charitable bequests as share of estate

• Wealth migration (in/out of state)

\textbf{Step 3: Estimate effect}

y\_ist = α + β·(Post\_t × Repeal\_s) + μ\_s + δ\_t + X\_ist\textquotesingle γ + ε\_ist

\textbf{Step 4: Event study specification}

y\_ist = α + Σ\_k β\_k·(Year\_t = k) × Repeal\_s + μ\_s + δ\_t + ε\_ist

Plot β\_k coefficients to visualize pre-trends and treatment effects.

7. SAMPLE CODE TEMPLATES

7.1 Stata: Bequest Elasticity Estimation

* Load IRS SOI data

use soi\_estate\_tax\_panel.dta, clear

* Generate variables

gen ln\_estates = ln(num\_estates)

gen net\_of\_tax = 1 - marginal\_rate

gen ln\_net\_of\_tax = ln(net\_of\_tax)

* Basic elasticity regression

reghdfe ln\_estates ln\_net\_of\_tax gdp\_per\_capita pop\_over65, ///

absorb(state year) cluster(state)

* Store coefficient

local elasticity = \_b{[}ln\_net\_of\_tax{]}

di "Bequest elasticity: " `elasticity\textquotesingle{}

7.2 Stata: State-Level Diff-in-Diff

* Load merged state-level data

use state\_inheritance\_panel.dta, clear

* Define treatment

gen repeal\_state = inlist(state, "IN", "OH", "NC", ...) // states that repealed

gen post = (year \textgreater= 2002)

gen treat = repeal\_state * post

* Diff-in-diff

reghdfe bequest\_concentration treat, absorb(state year) cluster(state)

* Event study

gen rel\_year = year - 2001

forvalues k = -5/10 \{

gen rel\_`k\textquotesingle{} = (rel\_year == `k\textquotesingle) * repeal\_state

\}

reghdfe bequest\_concentration rel\_* (omit: rel\_-1), absorb(state year) cluster(state)

coefplot, vertical

7.3 R: Cross-Country Analysis

\# Load packages

library(tidyverse)

library(fixest)

library(haven)

\# Load US SCF data

scf \textless- read\_dta("scf2022.dta") \%\textgreater\%

filter(networth \textgreater{} 1000000) \%\textgreater\%

mutate(country = "US")

\# Load UK WAS data

was \textless- read\_dta("was\_wave7.dta") \%\textgreater\%

filter(totwlth \textgreater{} 800000) \%\textgreater\%

mutate(country = "UK")

\# Combine and estimate

combined \textless- bind\_rows(scf, was)

\# Regression

model \textless- feols(num\_beneficiaries \textasciitilde{} UK + age + income + wealth \textbar{} year,

data = combined, cluster = \textasciitilde country\_year)

summary(model)

8. PROJECT TIMELINE AND CHECKLIST

8.1 Phase 1: Data Assembly (Weeks 1-8)

□ Download IRS SOI public-use estate tax files

□ Download SCF data (2019, 2022 waves)

□ Register for PSID access and create custom extract

□ Download HMRC inheritance tax statistics

□ Register for UK Data Service and download WAS

□ Compile state inheritance tax parameters (1990-2025)

□ Download Census state government finance data

□ Download OECD tax revenue and wealth distribution data

□ Submit application for IRS restricted data access (if pursuing)

8.2 Phase 2: Data Cleaning (Weeks 9-16)

□ Harmonize US and UK wealth definitions

□ Create state-year panel of tax parameters

□ Link estate tax data to state characteristics

□ Construct inheritance receipt variables from SCF/PSID

□ Create consistent beneficiary classification across datasets

□ Document all variable definitions and coding decisions

8.3 Phase 3: Estimation (Weeks 17-28)

□ Estimate bequest elasticity (Section 6.1)

□ Estimate planning effectiveness parameter

□ Estimate dispersion preference parameter

□ Run cross-country UK vs. US analysis (Section 6.2)

□ Run state-level diff-in-diff (Section 6.3)

□ Conduct robustness checks and sensitivity analysis

□ Bootstrap standard errors

8.4 Phase 4: Write-Up (Weeks 29-36)

□ Update paper with empirical results

□ Create publication-quality tables and figures

□ Write detailed data appendix

□ Prepare replication package

□ Submit to journal

9. KEY CONTACTS AND RESOURCES

9.1 Data Access Contacts

\textbf{IRS Statistics of Income:}

Email: sis@irs.gov

Phone: (202) 317-3100

For research proposals: SOI.Research@irs.gov

\textbf{Federal Statistical Research Data Centers:}

URL: https://www.census.gov/about/adrm/fsrdc.html

Provides access to restricted IRS, Census, and other federal microdata.

\textbf{UK Data Service:}

Email: help@ukdataservice.ac.uk

Phone: +44 (0)1206 872143

9.2 Expert Contacts

Researchers with expertise in estate/inheritance taxation who have used relevant data:

• Wojciech Kopczuk (Columbia) --- Estate tax behavioral responses

• David Joulfaian (Treasury, retired) --- Estate tax data

• Lily Batchelder (NYU Law) --- Inheritance tax policy

• Emmanuel Saez (Berkeley) --- Wealth inequality

• Gabriel Zucman (Berkeley) --- Wealth data and taxation

9.3 Software Requirements

• Stata 17+ (for reghdfe, coefplot packages)

• R 4.0+ with tidyverse, fixest, haven packages

• Python 3.8+ with pandas, statsmodels (optional)

• LaTeX for paper compilation

9.4 Estimated Costs

\begin{longtable}[]{@{}
  >{\raggedright\arraybackslash}p{(\columnwidth - 2\tabcolsep) * \real{0.6667}}
  >{\raggedright\arraybackslash}p{(\columnwidth - 2\tabcolsep) * \real{0.3333}}@{}}
\toprule\noalign{}
\endhead
\bottomrule\noalign{}
\endlastfoot
\textbf{Item} & \textbf{Cost} \\
IRS SOI public data & Free \\
SCF, PSID data & Free \\
UK Data Service access & Free (academic) \\
FSRDC access (if needed) & \$20,000-50,000/year \\
CCH State Tax Handbook (historical) & \textasciitilde\$200/year \\
Stata license & \textasciitilde\$600 (academic) \\
Research assistant (optional) & \$15-25/hour \\
\end{longtable}

Note: Most analyses can be completed with freely available public data. FSRDC access is only necessary for individual-level estate tax return analysis and is optional for the core findings.
