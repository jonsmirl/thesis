\begin{table}[htbp]
\centering
\caption{Laffer Curve Empirical Tests: CES Complementarity Predictions}
\label{tab:laffer}
\small
\begin{tabular}{llcrr}
\toprule
Test & Metric & Statistic & Value & $p$-value \\
\midrule
Revenue Elasticity & mann_whitney_U & $U$ & $+6.0000$ & 0.3500 \\
 & interaction & $coeff$ & $+0.0123$ & 0.0019 \\
Event Study & event_consistency & $frac$ & $+0.5714$ & 0.3953 \\
 & weighted_DID & $value$ & $-0.0174$ & -- \\
Displacement & migration_elasticity & $beta$ & $+0.0136$ & 0.0000 \\
Asymmetry & F_test_LOW & $F$ & $+0.3044$ & 0.5838 \\
 & cubic_d_LOW & $d$ & $+0.0000$ & -- \\
 & F_test_HIGH & $F$ & $+0.0000$ & 0.9999 \\
 & cubic_d_HIGH & $d$ & $-0.0000$ & -- \\
\bottomrule
\end{tabular}
\begin{minipage}{0.95\textwidth}
\vspace{0.5em}
\footnotesize\textit{Notes:} Tests predictions from the CES Laffer derivation (Smirl 2026). Test A: interaction of tax rate with sectoral complementarity ($\rho$ class) on GDP share; Mann-Whitney U compares $|\beta|$ across $\rho$ classes. Test B: DID around major tax changes (binomial test for consistency). Test C: IRS SOI migration AGI flow vs pairwise tax differential. Test D: cubic vs quadratic F-test for asymmetric decline by $\rho$ class. Data: FRED state GDP by industry, Tax Foundation rates, IRS SOI migration.
\end{minipage}
\end{table}